\theoremstyle{plain}
\newtheorem{thm}{Theorem}[chapter] %Reset counter elk hoofdstuk
\theoremstyle{definition}
\newmdtheoremenv{de}[thm]{Definitie} % Definitie met frame
\newtheorem{ei}[thm]{Eigenschap}
\newtheorem{st}[thm]{Stelling}
\newtheorem{gev}[thm]{Gevolg}
\newtheorem{pr}[thm]{Propositie}
\newtheorem{opm}[thm]{Opmerking}
\newtheorem{vb}[thm]{Voorbeeld}

\newcommand{\deref}[1]{\footnote{Zie definitie \ref{#1} op pagina \pageref{#1}.}}
\newcommand{\stref}[1]{\footnote{Zie stelling \ref{#1} op pagina \pageref{#1}.}}
\newcommand{\eiref}[1]{\footnote{Zie eigenschap \ref{#1} op pagina \pageref{#1}.}}
\newcommand{\gevref}[1]{\footnote{Zie gevolg \ref{#1} op pagina \pageref{#1}.}}
\newcommand{\prref}[1]{\footnote{Zie propositie \ref{#1} op pagina \pageref{#1}.}}
\newcommand{\opmref}[1]{\footnote{Zie opmerking \ref{#1} op pagina \pageref{#1}.}}
\newcommand{\vbmref}[1]{\footnote{Zie voorbeeld \ref{#1} op pagina \pageref{#1}.}}


% Mooiere TODO's
\newcommand{\TODO}[1]{\todo[color=red,inline,size=\small]{TODO: #1}}
\newcommand{\extra}[1]{\todo[color=orange,inline,size=\small]{EXTRA: #1}}
\newcommand{\clarify}[1]{\todo[color=yellow,inline,size=\small]{CLARIFY: #1}}
\newcommand{\question}[1]{\todo[color=green,inline,size=\small]{QUESTION: #1}}

\newcommand{\waarom}[0]{\clarify{waarom?}}

\newcommand{\term}[1]{\index{#1}\textbf{#1}}

%badges
\newcommand{\commj}[0]{\framebox[1.1\width]{Commutatief: Ja}}
\newcommand{\commn}[0]{\framebox[1.1\width]{Commutatief: Nee}}
\newcommand{\eenhj}[1]{\framebox[1.1\width]{Eenheidselement: #1}}
\newcommand{\eenhn}[0]{\framebox[1.1\width]{Eenheidselement: Nee}}
