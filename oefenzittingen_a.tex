\documentclass[main.tex]{subfiles}
\begin{document}

\chapter{Algebra I: Oefenzittingen}
\label{cha:algebra-i-oefenzittingen}

\section{Oefenzitting 1: Herhalingen en Aanvullingen}
\subsection*{Oefening 1}
\label{oza:oz1-oef1}
Kies de groep $\mathcal{S}_{3}$.
Beschouw nu de elementen $(12)$ en $(23)$, deze hebben beide orde $2$.
\[ (12)^{2} = Id \quad\text{ en }\quad (23)^{2} \]
Nu heeft $(12) \circ (23) = (132)$ orde $3$.
\[ (132)^{3} = (132)(123) = Id \]

\subsection*{Oefening 2}
\label{oza:oz1-oef2}
Zij $G$ een groep.
Toon aan dat $G$ commutatief is als en slechts als het volgende geldt:
\[ \forall a,b \in G, \forall n \in \mathbb{N}:\ (ab)^{n} = a^{n}b^{n} \]

\begin{proof}
  Kies twee willekeurige elementen $a$ en $b$ uit $G$:
  \begin{itemize}
  \item $\Rightarrow$\\
    Kies een willekeurige $n\in \mathbb{N}$
    \[ (ab)^{n} = ababab \ldots abab = aaaaa \ldots bbbbb \]
  \item $\Leftarrow$\\
    Kies $n=2$:
    \[ 
    \begin{array}{rl}
       abab &= aabb\\
       bab &= abb\\
       ba &= ab
    \end{array}
    \]
  \end{itemize}
\end{proof}

\subsection*{Oefening 3}
\label{oza:oz1-oef3}

\subsubsection*{(a)}
Bepaal alle deelgroepen van $\mathbb{Z}_{12},+$.\\\\
$\mathbb{Z}_{12},+$ heeft orde $12$.
\[
\begin{array}{c}
  \{ 0,2,4,6,8,10 \},+\\
  \{ 0,3,6,9 \},+\\
  \{ 0,4,8 \},+\\
  \{ 0,6 \},+\\
  \{ 0 \},+\\
\end{array}
\]

\subsubsection*{(b)}
Bepaal alle deelgroepen van $\mathcal{D}_{5},\circ$.\\\\
$\mathcal{D}_{5},+$ heeft orde $10$.
\[
\begin{array}{c}
  \{ e, a, a^{2}, a^{3}, a^{4} \},\circ\\
  \{ e, b^{5} \},\circ\\
  \{ e, b^{4} \},\circ\\
  \{ e, b^{3} \},\circ\\
  \{ e, b^{2} \},\circ\\
  \{ e, b \},\circ\\
  \{ e \},\circ\\
\end{array}
\]

\subsubsection*{(c)}
Bepaal alle deelgroepen van $\mathcal{Q},\cdot$.\\\\
$\mathcal{Q},\cdot$ heeft orde $8$.
\[
\begin{array}{c}
  \{ 1, -1, k, -k \},\cdot\\
  \{ 1, -1, j, -j \},\cdot\\
  \{ 1, -1, i, -i \},\cdot\\
  \{ 1, -1 \},\cdot\\
  \{ 1 \},\cdot\\
\end{array}
\]

\subsubsection*{(d)}
Bepaal alle deelgroepen van $\mathcal{A}_{4},\circ$.\\\\
$\mathcal{A}_{4},\circ$ heeft orde $12$.
\[
\begin{array}{c}
  \{ Id, (123), (132) \},\circ\\
  \{ Id, (124), (142) \},\circ\\
  \{ Id, (134), (143) \},\circ\\
  \{ Id, (234), (243) \},\circ\\
  \{ Id, (12)(34) \},\circ\\
  \{ Id, (13)(24) \},\circ\\
  \{ Id, (14)(23) \},\circ\\
  \{ Id, (34) \},\circ\\
  \{ Id, (24) \},\circ\\
  \{ Id, (23) \},\circ\\
  \{ Id, (14) \},\circ\\
  \{ Id, (13) \},\circ\\
  \{ Id, (12) \},\circ\\
  \{ Id \},\circ\\
\end{array}
\]

\subsection*{Oefening 4}
\label{oza:oz1-oef4}

Zij $G$ een cyclische groep van orde $n$ en $H$ een cyclische groep van orde $m$ met $ggd(m,n)=1$.
Bewijs dat $G \oplus H$ cyclisch is van orde $mn$.

\begin{proof}
  Zij $g$ een generator voor $G$ en $h$ een generator voor $H$.
  \[ g^{n} = e_{G} \quad\text{ en }\quad h^{m} = e_{H} \]
  Dit betekent dat in $G \oplus H$ het volgende geldt:
  \[ (g,e_{H})^{n} = (e_{G},e_{H}) \quad\text{ en }\quad (e_{G},h)^{m} = (e_{G},e_{H}) \]
  We bewijzen nu dat $(g,h)$ een generator is voor $G \oplus H$ met orde $mn$.
  \[ (g,h)^{mn} = (e_{G},h^{n}) = (g^{m},e_{H}) = (e_{G},e_{H}) \]
  Dit betekent dat $(g,h)$ $G \oplus H$ genereert met orde $mn$.
\end{proof}

\subsection*{Oefening 5}
\label{oza:oz1-oef5}

Beschouw de deelgroep $G$ van $Gl_{2}(\mathbb{R})$.
\[
G =
\left<
  \begin{pmatrix}
     0 & 1\\
    -1 & 0\\
  \end{pmatrix}
  ,
  \begin{pmatrix}
    0 & 1\\
    1 & 0
  \end{pmatrix}
\right>
\]
Bepaal $G$.
Met welke bekende groep is $G$ isomorf?

\[ 
G = 
\left\{ 
  \begin{pmatrix}
    1 & 0\\
    0 & 1
  \end{pmatrix}
,
  \begin{pmatrix}
    -1 & 0\\
    0 & -1
  \end{pmatrix}
,
  \begin{pmatrix}
    0 & 1\\
    1 & 0
  \end{pmatrix}
,
  \begin{pmatrix}
    0 & -1\\
    -1 & 0
  \end{pmatrix}
,
  \begin{pmatrix}
     0 & 1\\
    -1 & 0\\
  \end{pmatrix}
,
  \begin{pmatrix}
     0 & -1\\
    1 & 0\\
  \end{pmatrix}
,
  \begin{pmatrix}
    1 & 0\\
    0 & -1
  \end{pmatrix}
,
  \begin{pmatrix}
    -1 & 0\\
    0 & 1
  \end{pmatrix}
\right\}
\]
$G$ is isomorf met de quaternionengroep.

\subsection*{Oefening 6}
\label{oza:oz1-oef6}
Zij $G,\cdot$ een niet-cyclische groep van orde $6$.

\subsubsection*{(a)}
$G$ heeft een element van orde $3$.

\begin{proof}
  De orde van een element van een groep is een deler van de orde van de groep.\gevref {gev:orde-van-element-deelt-orde-van-groep}
  Elk element heeft dus een orde in $\{ 1, 2, 3, 6 \}$.
  Enkel $e$ heeft als orde $1$, en er zijn geen elementen van orde $6$ want dan zou $G$ cyclisch zijn.
  Stel dat elk element orde $2$ heeft, dan is $G$ commutatief.\stref{st:groep-alles-orde-2-commutatief}
  Dit zou betekenen dat $\{ e,a,b,ab \}$ een deelgroep was van $G$ met orde $4$, maar $4$ is geen deler van $6$. Contradictie\stref{st:stelling-van-lagrange}
  Bijgevolg bestaat er minstens \'e\'en element van orde $3$ en is de orde van $G$ dus $3$.\stref{st:orde-van-generator-is-orde-van-groep}
\end{proof}

\subsubsection*{(b)}
Zij $a$ een element van orde $3$ en $b \not\in \{ e,a,a^{2} \}$, dan geldt $G= \{ e,a,a^{2},b,ab,a^{2}b \}$.

\begin{proof}
  We bewijzen dat $\{ e,a,a^{2},b,ab,a^{2}b \}$ enkel verschillende elementen bevat.
  \begin{itemize}
  \item $\{ e,a,a^{2} \}$ is een verzameling van onderling verschillende elementen, en bovendien een groep.
  \item $\{ e,a,a^{2},b \}$ is een verzameling van onderling verschillende elementen omdat $b$ niet in$\{ e,a,a^{2} \}$ zit en $\{ e,a,a^{2} \}$ een groep is.
  \item $\{ e,a,a^{2},b,ab \}$ is een verzameling van onderling verschillende elementen:\\
    \begin{tabular}[H]{cl}
      \centering
      $ab \neq e$ & $b$ zou immers de inverse zijn van $a$ en niet in de groep $\{ e,a,a^{2} \}$ zitten.\\
      $ab \neq a$ & want dan zou $b$ het neutraal element zijn.\\
      $ab \neq a$ & want dan zou $b$ gelijk zijn aan $a$.\\
      $ab \neq b$ & want dan zou $a$ het neutraal element zijn.\\
    \end{tabular}
  \item $\{ e,a,a^{2},b,ab,a^{2}b \}$ is een verzameling van onderling verschillende elementen:\\
    \begin{tabular}[H]{cl}
      $a^{2}b \neq e$ & $b$ zou immers de inverse zijn van $a^{2}$ en niet in de groep $\{ e,a,a^{2} \}$\\
      $ab^{2} \neq a$ & want dan zou $b^{2}$ het neutraal element zijn.\\
      $ab^{2} \neq a^{2}$ & want dan zou $b^{2}$ gelijk zijn aan $a$ en $b^{3}=ab$ dus gelijk aan $b$\\
      $ab^{2} \neq b$ & want dan zou $ab$ gelijk zijn aan $b$.\\
      $ab^{2} \neq ab$ & want dan zou $b$ het neutraal element zijn.\\
    \end{tabular}
  \end{itemize}
\end{proof}

\subsubsection*{(c)}
De orde van $b$ is $2$.

\begin{proof}
  We gaan na aan welk element van $\{ e,a,a^{2},b,ab,a^{2}b \}$ $b^{2}$ gelijk moet zijn.\\
  \begin{tabular}[H]{cl}
    \centering
    $b^{2} \neq  a$ & want ???\\
    $b^{2} \neq  a^{2}$ & want dan zou $b^{2}a$ het neutraal element zijn en ???\\
    $b^{2} \neq  ab$ & want dan zou $a$ gelijk zijn aan $b$.\\
    $b^{2} \neq  a^{2}b$ & want dan zou $a^{2}$ gelijk zijn aan $b$.\\
  \end{tabular}\\
  Bijgevolg geldt $b^{2} = e$.
\end{proof}


\subsubsection*{(d)}
$ba$ is gelijk aan $a^{2}b$.

\begin{proof}
  \[
  \begin{array}{rll}
    ba &\neq e &\text{want dan zou } \{ e,a,a^{2} \} \text{ geen groep zijn.}\\ 
    ba &\neq a &\text{(gegeven)}\\ 
    ba &\neq a^{2} &\text{want dan zou } b=a \text{ gelden.} \\ 
    ba &\neq b &\text{want dan zou } a=e \text{ gelden.} \\ 
    ba &\neq ab &\text{want ???} \\ 
    ba &\neq a^{2}b &\text{want dan zou } b=a^{2} \text{ gelden.} \\ 
  \end{array}
  \]
\clarify{Waarom?}
\end{proof}

\subsubsection*{(e)}
$G \cong \mathcal{D}_{3} \cong \mathcal{S}_{3}$ geldt.

\begin{proof}
  Constructief bewijs.
  \begin{itemize}
  \item Er bestaat een bijectie tussen $G$ en $\mathcal{S}_{3}$:
    \[
    \begin{pmatrix}
      e & a     & a^{2} & b    & ab   & a^{2}b\\
      e & (123) & (132) & (12) & (13) & (23)\\
    \end{pmatrix}
    \]
  \item $G$ is gelijk aan $D_{3}$.
  \end{itemize}
\end{proof}

\section{Oefenzitting 2: Permutatiegroepen}

\subsection*{Oefening 1}
\label{oza:oz2-oef1}
Geef alle mogelijke permutaties $\pi$ op de verzameling $\{1,2,3,4,5\}$ die orde $5$ hebben en voldoen aan $\pi(1) = 5$, $\pi(2) = 3$ en $\pi(3) = 4$.\\\\
De permutaties die we zoeken zien er als volgt uit:
\[
\begin{pmatrix}
  1 & 2 & 3 & 4 & 5\\
  5 & 3 & 4 & a & b
\end{pmatrix}
\] 
Om een permutatie te zijn moet $\{a,b\}$ gelijk zijn aan $\{1,2\}$.
Als $a$ $1$ is en $b$ $2$, voldoet de permutatie aan de opgelegde eigenschappen.
Als $a$ $2$ is en $b$ $1$, heeft de permutatie orde $6$.
Antwoord:
\[
\begin{pmatrix}
  1 & 2 & 3 & 4 & 5\\
  5 & 3 & 4 & 1 & 2
\end{pmatrix}
\] 

\subsection*{Oefening 2}
\label{oza:oz2-oef2}
Beschouw het volgend geheim schrift.
De vercijfering van een tekst gebeurt door in de tekst de letters van het alfabet te permuteren volgens de permutatie $\pi$.
\[
\pi =
\left(
\begin{array}{cccccccccccccccccccccccccc}
  a & b & c & d & e & f & g & h & i & j & k & l & m & n & o & p & q & r & s & t & u & v & w & x & y & z\\
  b & c & d & e & f & g & h & i & a & z & x & m & l & o & p & n & s & r & t & q & j & w & v & k & u & y\\ 
\end{array}
\right)
\]

\subsubsection*{(a)}
Zoek de disjuncte cykel-schrijfwijze van de permutatie $\pi$.
\[ (abcdefghi)(jzyuj)(kx)(lm)(nop)(qst)(vw) \] 

\subsubsection*{(b)}
Wat is de orde van $\pi$?
\[ kgv(9,5,2,2,3,3,2) = 90 \]

\subsubsection*{(c)}
Wat is het teken van $\pi$?
$-1$ want er is een oneven aantal cykels met even lengte.
\extra{verwijzing naar stelling.}

\subsection*{Oefening 3}
\label{oza:oz2-oef3}
Waar of niet?
Er bestaat geen permutatie $\pi$ van $9$ elementen met orde $15$.\\\\
Niet waar:
\[ (abc)(efghi) \]

\subsection*{Oefening 4}
\label{oza:oz2-oef4}
Onder welke voorwaarden is de afbeelding $\pi:\ \{ 0,\dotsc, n-1 \} \rightarrow \{ 0,\dotsc, n-1 \}:\ i \mapsto ai + b \mod n$ een permutatie. ($a,b \in \mathbb{Z}$).\\\\
$a$ en $b$ moeten relatief priem zijn.
\waarom

\subsection*{Extra oefening}
\label{oza:oz2-extra}
Zij $\sigma$ en $\tau$ permutaties in $\mathcal{S}_{n}$.
Zoek de nodige en voldoende voorwaarden op $\sigma$ en $\tau$ opdat zo zouden commuteren.
Hint: $\sigma \tau = \tau \sigma$ geldt als en slechts als $\tau$ en $\sigma$ disjunct zijn \emph{of} ...
\extra{oefening}

\section{Oefenzitting 3: Conjugatie en klasvergelijking}

\subsection*{Oefening 1}
\label{sec:oz3-oef1}
Geef alle conjugatieklassen van een Abelse groep $G,*$.
Controleer de klasvergelijking voor een eindige abelse groep.\\\\
De conjugatie met $a$ is in een Abelse groep de identieke transformatie $Id$.
De conjugatieklassen van $G,*$ zijn dus singletons van alle respectievelijke elementen in $G$.
De klasvergelijking is bijzonder simpel voor een Abelse groep omdat dan het centrum van $G,*$ heel $G$ is.

\subsection*{Oefening 2}
\label{sec:oz3-oef2}
Geef de conjugatieklassen van $\mathcal{S}_{3}$ en controleer de klasvergelijking.
\[ 
\left\{ 
  \begin{array}{c}
\{ Id \},\\
\{ (12),(13),(23) \},\\
\{ (123),(132) \},\\
\end{array}
\right\}
\]
\[
\begin{array}{ccccc}
  |G| &=& |Z(G)| &+& \sum|Cl(a_{i})|\\
  |\mathcal{S}_{3}| &=& 1 &+& 5\\
\end{array}
\]


\subsection*{Oefening 3}
\label{sec:oz3-oef2}
Geef de conjugatieklassen van $\mathcal{S}_{4}$ en controleer de klasvergelijking.
\[ 
\left\{ 
  \begin{array}{c}
\{ Id \},\\
\{ (12),(13),(14),(23),(24),(34) \},\\
\{ (123),(132),(124),(142),(234),(243),(134),(143) \},\\
\{ (12)(34),(13)(24),(14)(23) \},\\
\{ (1234),(1342),(1423),(1432),(1243),(1324) \}\\
\end{array}
\right\}
\]
\[
\begin{array}{ccccc}
  |G| &=& |Z(G)| &+& \sum|Cl(a_{i})|\\
  |\mathcal{S}_{4}| &=& 1 &+& 23\\
\end{array}
\]

\subsection*{Oefening 4}
\label{sec:oz3-oef3}
Geef de conjugatieklassen van de quaternionengroep en controleer de klasvergelijking.
\[ 
\left\{ 
  \begin{array}{c}
\{ 1 \},\\
\{ -1 \},\\
\{ i,-i \},\\
\{ j,-j \},\\
\{ k,-k \}\\
\end{array}
\right\}
\]
\[
\begin{array}{ccccc}
  |G| &=& |Z(G)| &+& \sum|Cl(a_{i})|\\
  |\mathcal{Q}| &=& 2 &+& 6\\
\end{array}
\]

\subsection*{Oefening 5}
\label{sec:oz3-oef4}
Geef de conjugatieklassen van de didi\"edergroep $\mathcal{D}_{4}$ met $8$ elementen en controleer de klasvergelijking.
\[ 
\left\{ 
  \begin{array}{c}
\{ Id \},\\
\{ a^{2} \},\\
\{ a,a^{3} \},\\
\{ b,a^{2}b \},\\
\{ ab,a^{3}b \}\\
\end{array}
\right\}
\]
\[
\begin{array}{ccccc}
  |G| &=& |Z(G)| &+& \sum|Cl(a_{i})|\\
  |\mathcal{D}_{4}| &=& 2 &+& 6\\
\end{array}
\]

\subsection*{Oefening 6}
\label{sec:oz3-oef5}
Bepaal alle eindige groepen met precies $2$ conjugatieklassen.\\\\
Zij $G,*$ een eindige groep met precies $2$ conjugatieklassen.
Elk element in het centrum heeft zijn eigen conjugatieklas.
Er zit minstens $1$ element in het centrum: $e_{G}$.
$e_{G}$ zit dus al zeker in zijn eigen conjugatieklas.
Alle andere elementen zitten in dus samen in een conjugatieklas.
De groep is van orde minstens twee omdat alle conjugatieklassen niet-leeg zijn.
Noem $C$ de conjugatieklas van $G,*$ waar $e_{G}$ niet in zit.
\[ |C| = |G| - 1 \]
We weten dat de orde van elke conjugatieklas een deler moet zijn van de orde van $G$\gevref{gev:orde-conjugatieklasse-deelt-orde-groep}.
$|G| - 1$ is alleen een deler van $|G|$ als $|G|$ precies $2$ is.
Er is maar $1$ groep van orde $2$, op isomorfisme na: $\mathbb{Z}_{2}$.

\subsection*{Oefening 7}
\label{sec:oz3-oef6}
Bepaal alle eindige groepen met precies $3$ conjugatieklassen.\\\\
Verdergaand op de redenering in de vorige oefening moet een groep $G,*$ met precies $3$ conjugatieklassen minstens drie elementen.
$G$ heeft dan als orde $1+a+b$ zodat zowel $a$ als $b$ $|G|$ delen.
De mogelijkheden voor $(a,b)$ zijn dan $\{(1,1),(1,2),(2,3)\}$.
\begin{itemize}
\item Als $a = b = 1$ geldt, is $G$ abels, en dan is $G$ isomorf met $\mathbb{Z}_{3}$.
\item Als $a=1$ en $b=2$ gelden, is $G$ een groep van orde $4$ die niet abels. Zo'n groep bestaat niet.
  Elke groep van orde $4$ is immers isomorf met ofwel de viergroep, ofwel $\mathbb{Z}_{4}$.\eiref{ei:groep-orde-vier}
\item Als $a=2$ en $b=3$ gelden, is $G$ een groep van orde $6$. $G$ is dan ofwel isomorf met $\mathcal{S}_{3}$ ofwel met $\mathbb{Z}_{6}$.\eiref{ei-groep-orde-zes}
\end{itemize}

\subsection*{Oefening 8}
\label{sec:oz3-oef7}
Bepaal de orde van het centrum van een niet-abelse groep $P$ van orde $p^{3}$ met $p$ een priemgetal.
Controleer dit voor de didi\"edergroep $\mathcal{D}_{4}$ en de quaternionengroep.\\\\
De orde van het centrum is een deler van de orde van $P$\eiref{ei:centrum-is-deelgroep} \stref{st:stelling-van-lagrange} en bovendien groter dan $1$.\prref{pr:orde-centrum-pgroep-groter-dan-een}.
$|Z(P)|$ moet dus $p$ of $p^{2}$ zijn.
Voor $|Z(P)|$ $3$ zou immers betekenen dat $P$ abels is.
De orde van $P$ kan bovendien niet $p^{2}$ zijn, want dan zou $P$ abels zijn.\stref{st:priemgroep-kwadraat-abels}


\section{Oefenzitting 4: Normaaldelers}

\subsection*{Oefening 1}
\label{sec:oz4-oef1}

\subsubsection*{(a)}
Is $\{ Id, (12)(34),(13)(24),(14)(23) \}$ een normaaldeler van $\mathcal{A},\circ$?
 
Ja
\waarom
\subsubsection*{(b)}
Is $\{ Id, (12)(34) \}$ een normaaldeler van $\mathcal{A},\circ$?
Nee
\waarom

\subsection*{Oefening 2}
\label{sec:oz4-oef2}
Is $SL_{n}(\mathbb{R})$ een normaaldeler van $GL_{n}(\mathbb{R})$?

Ja.
\begin{proof}
  \[ \forall g\in GL_{n}(\mathbb{R}), n\in SL_{n}(\mathbb{R}): \det(g\cdot n \cdot g^{-1}) = \det(g) \cdot \det(n) \cdot \det(g^{-1}) = 1 \]
\end{proof}

\subsection*{Oefening 3}
\label{sec:oz4-oef3}
Zij $H$ een normaaldeler van een groep $G,*$.
Bewijs: $H$ is de unie van de conjugatieklassen van alle $h\in H$.
\[ H \triangleleft G \Rightarrow H = \bigcup_{h\in H}Cl_{G}(h) \]
\begin{proof}
  $H$ is een normaaldeler van $G$, dus het volgende geldt. \stref{st:criteria-voor-normaaldeler}
  \[ \forall g\in G: gHg^{-1} = H \]
  Bekijk $H$ nu als volgt:
  \[
  \begin{array}{rll}
    H &= \bigcup_{g\in G} gHg^{-1} &\\
      &= \bigcup_{g\in G} \bigcup_{h\in H} ghg^{-1} &\\
      &= \bigcup_{h\in H} \bigcup_{g\in G} ghg^{-1} &\\
      &= \bigcup_{h\in H} \{ ghg^{-1} \ |\ g\in G \} &= \bigcup_{h\in H}Cl(h)
  \end{array}
  \]
\end{proof}

\subsection*{Oefening 4}
\label{sec:oz4-oef4}
Geldt het volgende?
\[ \mathbb{Q},+=grp\{\nicefrac{1}{p} \ |\ p \text{ is een priemgetal.} \} \] 
\question{hoe begin ik hieraan?}

\subsection*{Oefening 5}
\label{sec:oz4-oef5}
Zij $H,*$ een deelgroep van $G,*$ met index $3$ zodat er een $x\in G\setminus H$ bestaat waarvoor $x*H = H*x$ geldt.
Bewijs dat $H$ een normaaldeler is van $G,*$.

\begin{proof}
  Merk op dat $G/H$ orde $3$ heeft.\stref{st:stelling-van-lagrange}.
  Dit betekent dat er een $y$ bestaat in $G\setminus H$ zodat $G/H$ er als volgt uitziet:
  \[ G/H = \{ xH, yH, H \} \]
\question{hoe??}
\end{proof}

\subsection*{Oefening 6}
\label{sec:oz4-oef6}
Zij $G,*$ een groep met normaaldelers $A$ en $B$.
Bewijs dat $A*B$ een normaaldeler is van $G,*$.

\begin{proof}
  Kies een willekeurige $g$ uit $G$ en een willekeurig element $a*b$ uit $A*B$.
  \[ g*a*b*g^{-1} = g*a*g^{-1}*g*b*g^{-1} \in A*B\]
  Hierboven is $g*a*g^{-1}$ een element van $A$ omdat $A$ een normaaldeler is van $G$ en $g*b*g^{-1}$ analoog een element van $B$.
\end{proof}

\subsection*{Oefening 7}
\label{sec:oz4-oef7}
Zij $G,*$ een groep.
Waar of niet waar?
\[ (A \triangleleft B \wedge B \triangleleft C) \Rightarrow A \triangleleft C \]
Waar
\begin{proof}
  \[
  \begin{array}{rl}
    & (\forall c\in C,\forall b\in B: cbc^{-1} \in B) \wedge (\forall b\in B,\forall a\in A: bab^{-1} \in A)\\
    \Rightarrow & \forall c\in C,\forall a\in A: cac^{-1} \in B)\\
    \Rightarrow & \forall c\in C,\forall a\in A: (cac^{-1})a(cac^{-1})^{-1} \in A\\
  \end{array}
  \]
  \[
  \begin{array}{rll}
    (cac^{-1})a(cac^{-1})^{-1} &= cac^{-1}a(c(ac^{-1}))^{-1} &\\
                             &= cac^{-1}a((ac)c^{-1}) &\\
                             &= cac^{-1}aacc^{-1} &\\
                             &= ca(c^{-1}a)(ac)c^{-1} &\\
                             &= ca(ac)^{-1}(ac)c^{-1} &\\
                             &= caec^{-1} &= cac^{-1}\\
  \end{array}
  \]
  \[ \forall c\in C, a\in A: cac^{-1} \in A \]
\end{proof}


\subsection*{Oefening 8}
\label{sec:oz4-oef8}
Zij $G,*$ een groep en $A\subseteq$.
\begin{itemize}
\item $C(A),*$ is een deelgroep van $G,*$.
  \begin{proof}
    De centralisator van elk element van $g$ is een deelgroep van $G,*$.\eiref{ei:centralisator-is-deelgroep} en de doorsnede van deelgroepen is een deelgroep.\stref{st:doorsnede-deelgroepen}
  \end{proof}
\item $A$ is een commutatieve deelgroep van $G \Rightarrow A \triangleleft C(A)$.
  \[ \forall c\in C(A), \forall a\in A:\ cac^{-1} = cc^{-1}a = e_{G}a = a \in A \]
\end{itemize}
\extra{nog drie deeloefeningen}

\subsection*{Oefening 9}
\label{sec:oz4-oef9}
Zij $G$ een groep en zij $H$ een deelgroep van index $n$.
Stel dat $\{e\}$ de enigge normaaldeler is van $G$ die bevat is in $H$.
Toon aan dat $G$ isomorf is met een deelgroep van $\mathcal{S}_{n}$.
\extra{hint p 5 oef 30 in oefeningen bundel}

\section{Oefenzitting 5: Quoti\"enten en isomorfismestellingen}

\subsection*{Oefening 1}
\label{sec:oz5-oef1}
Waar of niet?
\[ \nicefrac{\mathbb{Q}}{\mathbb{Z}},+ \cong \mathbb{Q},+ \]
Niet waar.\\
We zoeken een morfisme van $\mathbb{Q},+$ naar een groep isomorf met $\mathbb{Q},+$ waarvan $\mathbb{Z},+$ de kern is.
Stel dat $f$ zo'n isomorfisme is.
$f(\frac{1}{2} + \frac{1}{2})$ moet dan $0$ zijn, maar $f(\frac{1}{2})$ niet. Contradictie.

\subsection*{Oefening 2}
\label{sec:oz5-oef2}
Zij $\sigma$ een endomorfisme van $\mathbb{Q}_{0},\cdot$ als volgt. Wat zegt de eerste isomorfismestelling?
\[ \sigma:\ \mathbb{Q}_{0} \rightarrow \mathbb{Q}_{0}:\ x \mapsto |x| \]
De kern van $\sigma$ is $\{ 1,-1\}$.
De eerste isomorfismestelling zegt dus dat $\nicefrac{\mathbb{Q}_{0}}{\{ 1,-1\}},\cdot$ isomorf is met $\mathbb{Q}_{0}^{+}$.
\begin{figure}[H]
  \centering
  \begin{tikzpicture}
    \matrix (m) [matrix of math nodes,row sep=3em,column sep=4em,minimum width=2em]
    {
      \mathbb{Q}_{0} & \mathbb{Q}_{0} \\
      \nicefrac{\mathbb{Q}_{0}}{\{ 1,-1\}} & \mathbb{Q}_{0}^{+} \\};
    \path[-stealth]
    (m-1-1) edge node [left] {$\pi$} (m-2-1)
    edge [double] node [above] {$\sigma$} (m-1-2)
    (m-2-1.east|-m-2-2) edge node [below] {$\sigma'$} (m-2-2)
    (m-2-2) edge node [right] {$i$} (m-1-2)
    (m-2-1) edge [dashed] node [above] {$\bar{\sigma}$} (m-1-2);
  \end{tikzpicture}
\end{figure}

\subsection*{Oefening 3}
\label{sec:oz5-oef3}
Zij $G= grp\{a\}$ een cyclische groep van orde $12$.
Zij $M= grp\{a^{2}\}$ en $N=grp\{a^{6}\}$.
Waar of niet waar?
\[ \nicefrac{\nicefrac{G}{N}}{\nicefrac{M}{N}} \cong \nicefrac{G}{M} \]
Waar.
\begin{proof}
  $N$ en $M$ zijn normaaldelers van $G$ want het zijn deelgroepen van een commutatieve \stref{st:cyclishe-groep-is-commutatief} groep.
  Beschouw dan de derde isomorfismestelling.
\end{proof}

\subsection*{Oefening 4}
\label{sec:oz5-oef4}
Gegeven de deelgroepen $N = \{ 1,-1\}$ en $H= grp\{ \frac{1}{2} \}$ van $\mathbb{Q},\cdot$.
Waar of niet waar?
\[ \nicefrac{HN}{N},\cdot \cong \nicefrac{H}{H\cap N} \]
Waar.
\begin{proof}
  $H$ is een deelgroep van $\mathbb{Q},\cdot$ en $N$ een normaaldeler van $\mathbb{Q},\cdot$.
  Beschouw dan de tweede isomorfismestelling.
\end{proof}

\subsection*{Oefening 5}
\label{sec:oz5-oef5} 
Zij $G,*$ een groep, $H,*$ een deelgroep van $G,*$ en $N$ een normaaldeler van $G$ zodat $\nicefrac{G}{N}$ en $N$ Abels zijn.
Bewijs dat er een normaaldeler $D$ van $H$ bestaat zodat $\nicefrac{H}{D}$ en $D$ Abels zijn.
\question{hoe?? oef 35}

\subsection*{Oefening 6}
\label{sec:oz5-oef6}
Zij $G,*$ een groep met normaaldelers $M$ en $N$ en een deelgroep $H,*$. 
Bewijs volgende bewering:
\[ H \cap M = H \cap N \Rightarrow \nicefrac{HM}{M} \cong \nicefrac{HN}{N} \]
\begin{proof}
  Volgens de tweede isomorfismestelling geldt het volgende:
  \[ \nicefrac{HN}{N} = \nicefrac{H}{H\cap N} = \nicefrac{H}{H\cap M} = \nicefrac{HM}{M} \]
\end{proof}

\subsection*{Oefening 7}
\label{sec:oz5-oef7}
Beschouw $\mathbb{Z}^{2\times 2},+$ als additieve groep.
Beschouw $\phi$ als volgt. Wat zegt de isomorfismestelling hierover?
\[
\phi: \mathbb{Z}^{2\times 2} \rightarrow \mathbb{Z}:\
\begin{pmatrix}
  a & b\\
  c & d
\end{pmatrix}
\mapsto a + d
\]
De kern van $\phi$ is de verzameling van diagonaalmatrices: $\{Diag\}$.
De eerste isomorfismestelling zegt dan het volgende:
\[ \nicefrac{\mathbb{Z}^{2\times 2}}{\{Diag\}} \cong \mathbb{Z} \]
\begin{figure}[H]
  \centering
  \begin{tikzpicture}
    \matrix (m) [matrix of math nodes,row sep=3em,column sep=4em,minimum width=2em]
    {
      \mathbb{Z}^{2\times 2} & \mathbb{Z} \\
      \nicefrac{\mathbb{Z}^{2\times 2}}{\{Diag\}} & \mathbb{Z} \\};
    \path[-stealth]
    (m-1-1) edge node [left] {$\pi$} (m-2-1)
    edge [double] node [above] {$\phi$} (m-1-2)
    (m-2-1.east|-m-2-2) edge node [below] {$\phi'$} (m-2-2)
    (m-2-2) edge node [right] {$i$} (m-1-2)
    (m-2-1) edge [dashed] node [above] {$\bar{\phi}$} (m-1-2);
  \end{tikzpicture}
\end{figure}

\subsection*{Oefening 8}
Zij $G,*$ een eindige groep en $p$ een priemgetal dat de orde van $|G|$ deelt, bewijs dan dat $G$ een element van orde $p$ heeft. Je mag (/zal moeten) gebruiken dat $G$ commutatief is.
\question{Hoe??}

\TODO{hoeveel ringmorfismen $f: \mathbb{Z} \rightarrow \mathbb{R}$ zijn er die $1$ op $1$ afbeelden?}
\TODO{bewijs de deelbaarheidstesten door $9$ en $11$ uit de lagere school. p 48}

\section{Oefenzitting 6: Basisbegrippen}

\subsection*{Oefening 1}
Gegeven een verzameling $U$.
Bewijs dat $\mathcal{P}(U),\Delta,\cap$ een commutatieve ring is met eenheidselement.

\begin{proof}
  We gaan elke definierende eigenschap van een ring af.
  \begin{itemize}
  \item $\mathcal{P}(U)$ is een commutatieve groep:
    \begin{itemize}
    \item $\Delta$ is associatief.
    \item Er bestaat een neutraal element: $\emptyset$
    \item Elk element $X$ in $\mathcal{P}(U)$ heeft een inverse: $X^{c}$
    \end{itemize}
  \item $\cap$ is associatief. 
  \item $\cap$ is distributief ten opzichte von $+$.
  \item $\cap$ is commutatief.
  \end{itemize}
\end{proof}

\subsection*{Oefening 2}
Gegeven een ring $R,+,\cdot$ met nulelement $e_{R}$ als volgt.
We noemen dit een \term{Boolering}.
\[ \forall a\in R:\ a^{2} = a \]
Bewijs dat $R,+,\cdot$ een commutatieve ring is en het volgende:
\[ \forall a \in R:\ a+a = e_{R}\]

\begin{proof}
  Voor elke twee elementen $x$ en $y$ uit $R$ geldt het volgende:
  \[
  \begin{array}{rl}
    (x \cdot y)^{2} &= x \cdot y\\
    x \cdot y \cdot x \cdot y &= x \cdot y\\
    y \cdot x &= x \cdot y
  \end{array}
  \]
  $R,+,\cdot$ is dus al een commutatieve ring.
  \[
  \begin{array}{rll}
    a + a &= a^{2} + a^{2} &\\
          &= a \cdot (a+a) &\\
          &= a^{2} \cdot e_{R} = e_{R}
  \end{array}
  \]
\end{proof}


\subsection*{Oefening 3}
Zij $R$ een deelverzameling van $\mathbb{Q}$ als volgt:
\[ R = \left\{ \frac{a}{b} \ |\ a,b \in \mathbb{Z} \text{ en } b \text{ oneven } \right\} \]
\begin{itemize}
\item Bewijs dat $R,+,\cdot$ een deelring is van $\mathbb{Q}$.
\item Merk op dat $R$ als volgt geschreven kan worden:
  \[ \left\{ \frac{2^{n}a}{b}\ |\ n\in \mathbb{N}\text{ en } a,b \text { oneven } \right\} \cup \{0\} \]
\item Geef de eenheden van $R,+,\cdot$.
\item Neem een ideaal $I \neq \{0\}$ van $R,+,\cdot$ en bewijs dat $I$ voortgebracht wordt door $2^{m}$ waarbij $m$ het kleinste getal is waarvoor $2^{m}$ in $I$ zit.
\item Besluit dat $R,+,\cdot$ een HID is.
\end{itemize}
\extra{oefening}

\section{Oefenzitting 6: Idealen}

\subsection*{Oefening 1}
bewijs dat $I=\{  \begin{pmatrix}    a & b\\c & d  \end{pmatrix} \ |\ a,b,c,d \in 2\mathbb{Z}\}$ een ideaal is van $M^{2\times 2}(\mathbb{Z})$. Hoeveel elementen heeft $\nicefrac{M^{2\times 2}(\mathbb{Z})}{I}$?
\extra{oefening}

\section{Oefenzitting 7: Quotientringen}

\subsection*{Oefening 1}
Zij $R = \nicefrac{\mathbb{Q}[X]}{I}$ met $I=(X^{2}+X+1)$.

\begin{itemize}
\item Gelden volgende beweringen?
  \begin{itemize}
  \item $-2X^{2}+I \overset{?}{=} 2X^{3}+2X+I$
  \item $3+I\overset{?}{=} X+I$
  \item $X^{3}+2X^{2}+I \overset{?}{=} 3X+2+I$
  \end{itemize}
\item Toon aan dat elk element van $R$ op een unieke wijze geschreven kan worden als $aX+b+I$ met $a,b\in \mathbb{Q}$.
\item Reken uit en schrijf het resultaat in de vorm $aX+b+I$.
  \begin{itemize}
  \item $(X+I) + (X^{2}+I)$
  \item $(X+3+I)\cdot (X^{2}+1+I)$
  \end{itemize}
\item Men kan aantonen dat $R,+,\cdot$ een veld is.
  Wat is het invers van $X+I$ voor de vermenigvuldiging in $R,+,\cdot$?
\end{itemize}

Merk allereerst op dat $\mathbb{Q}$ een veld is\vbref{vb:q}.
Merk dan op dat er eenvoudige manier bestaat om na te kijken of twee veeltermen in dezelfde restklasse zitten.
\[ \forall f,g \in \nicefrac{\mathbb{Q[X]}}{I}: f+I = g+I \Leftrightarrow f=g + r \cdot (X^{2}+X+1) \]
Dit betekent dat als we $f-g$ `delen' door $X^{2}+X+1$, en als rest $0$ krijgen, $f$ en $g$ in dezelfde restklasse zitten.
\begin{itemize}
\item 
  Om deze beweringen na te kijken moeten we het delings
  \begin{itemize}
  \item $-2X^{2} - (2X^{3}+2X) = -2X^{3}-2X^{2}-2X$
    \[
    \begin{array}{ccc|l}
      -2X^{3} & -2X^{2} & -2X & X^{2}+X+1\\ \hline
      -2X^{3} & -2X^{2} & -2X & -2X \\\cline{1-3}
      &        &  0  & \\ 
    \end{array}
    \]
    Deze bewering geldt.
  \item $3+I$ en $X+I$ zijn niet gelijk. $\bar{3} = 3+I \neq X+I = \bar{X}$.
  \item $X^{3}+2X^{2} - (3X+2) = X^{3} +2X^{2}-3X - 2$
    \[
    \begin{array}{cccc|l}
      X^{3} & +2X^{2} & -3X & -2 & X^{2}+X+1\\ \hline
      X^{3} & +X^{2} & + X & \vdots & X +1 \\\cline{1-3}
      & +X^{2} & -4X & -2 & \\
      & +X^{2} & + X & +1 & \\\cline{2-4}
      &       & -5X & -3 
    \end{array}
    \]
    De rest is hier niet nul, dus de bewering geldt niet.

  \end{itemize}
\item De nevenklasse van de rest bij deling door $X^{2}+X+1$ moet gelijk zijn aan de nevenklasse van $f$, en de rest moet van graad $1$ of lager zijn.
\item
  \begin{itemize}
  \item $(X+I)+(X^{2}+I) = (X^{2}+X)+I = -1 +I$
    \[
    \begin{array}{ccc|l}
      X^{2} & +X &    & X^{2}+X+1\\ \hline
      X^{2} & +X & +1 & 1 \\\cline{1-3}
      &    & -1
    \end{array}
    \]
  \item $(X+3+I)\cdot (X+1+I) = (X^{3}+3X^{2}+X+3+I) = (-2X+1)+I$
    \[ 
    \begin{array}{cccc|l}
      X^{3} & +3X^{2} & +X & +3 & X^{2}+X+1\\ \hline
      X^{3} & + X^{2} & +X & \vdots & X \\\cline{1-3}
      & +2X^{2} &    & +3 & \\
      & +2X^{2} & +2X & +2 & \\\cline{2-4}
      &         & -2X & +1 & \\
    \end{array}
    \]
  \end{itemize}
\item We weten nu al dat het invers van $X+I$ er uitziet als $aX+b +I$ met $a$ en $b$ in $\mathbb{Q}$.
  \[ (X+I) \cdot (aX+b+I) = 1+X \]
  \[ X\cdot(aX+b) = aX^{2} +bX \]
  De rest bij deling van $aX^{2} +bX$ door $X^{2}+X+1$ moet dus $1$ zijn.
  \[
  \begin{array}{ccc|l}
    aX^{2} & +bX & & X^{2}+X+1\\\hline
    aX^{2} & +aX & +a & a \\\cline{1-3}
          & (b-a)X & -a &  \\ 
  \end{array}
  \]
  Dit betekent dat $(b-a)$ nul moet zijn en $a$ $-1$.
  De inverse van $X+I$ is dus $-X-1+I$.\\
  Inderdaad: $X\cdot(-X+1) = -X^{2}-X$
  \[
  \begin{array}{ccc|l}
    -X^{2} & -X & & X^{2}+X+1\\\hline
    -X^{2} & -X & -1 & -1\\\cline{1-3}
          &    &  1 & 
  \end{array}
  \]
\end{itemize}

\subsection*{Oefening 2}
Bereken $\nicefrac{\mathbb{Z}[X]}{(2)}$.
\[ \nicefrac{\mathbb{Z}[X]}{(2)} = \left\{ f + \left\{ (g \cdot 2)\ |\ g\in \mathbb{Z}[X] \right\}, f \in \mathbb{Z}[X] \right\} \]
Merk allereerst op dat dit de veeltermen over $\mathbb{Z}$ met even co\"efficienten zijn.
We zoeken nu een ringmorfisme waarvan $(2)$ de kern is, dat begint in $\mathbb{Z}[X]$.
Kies bijvoorbeeld $\phi$ voor dit morfisme:
\[ \phi: \mathbb{Z}[X] \rightarrow \mathbb{Z}_{2}[X]:\ \sum_{i=0}^{n}a_{i}X^{i} \mapsto \sum_{i=0}^{n}(a_{i} \mod 2)X^{i} \]
$(2)$ is inderdaad de kern van dit morfisme, dus geldt het volgende isomorfisme:\gevref{gev:eerste-ringmorfismestelling}
\[ \nicefrac{\mathbb{Z}[X]}{(2)} \cong \mathbb{Z}_{2}[X] \]

\subsection*{Oefening 3}
Bereken $\nicefrac{\mathbb{Z}[X]}{(2,X)}$.
\[ \nicefrac{\mathbb{Z}[X]}{(2,X)} = \left\{ f + \left\{ (g \cdot 2 + h \cdot X)\ |\ g,h\in \mathbb{Z}[X] \right\}, f \in \mathbb{Z}[X] \right\}\]
Merk allereerst op dat dit de veeltermen zijn met een even constante term.
We zoeken nu een ringmorfisme waarvan $(2,X)$ de kern is, dat begint in $\mathbb{Z}[X]$.
Kies bijvoorbeeld $\phi$ voor dit morfisme:
\[ \phi: \mathbb{Z}[X] \rightarrow \mathbb{Z}_{2}:\ \sum_{i=0}^{n}a_{i}X^{i} \mapsto (a_{0}\mod 2) \]
$(2,X)$ is inderdaad de kern van dit morfisme, dus geldt het volgende isomorfisme:\gevref{gev:eerste-ringmorfismestelling}
\[ \nicefrac{\mathbb{Z}[X]}{(2)} \cong \mathbb{Z}_{2} \]

\subsection*{Oefening 4}
Bereken $\nicefrac{\mathbb{Z}_{2}[X]}{(X^{2}+X+1)}$.
\[ \nicefrac{\mathbb{Z}_{2}[X]}{(X^{2}+X+1)} =  \left\{ f + \left\{ (g \cdot (X^{2}+X+1))\ |\ g\in \mathbb{Z}_{2}[X] \right\}, f \in \mathbb{Z}_{2}[X] \right\} \]
Dit zijn de binaire veeltermen zijn van graad $1$.

\subsection*{Oefening 5}
Bereken $\nicefrac{\mathbb{Z}[X,Y,Z]}{(X-Y,X^{3}-Z)}$.
\noindent
Dit is niet evident!
We zoeken een ringmorfisme waarvan $(X-Y,X^{3}-Z)$ de kern is, dat begint in $\mathbb{Z}[X,Y,Z]$.
\[ (X-Y,X^{3}-Z) = \{ f\cdot (X-Y) + g\cdot (X^{3}-Z) \mid f,g \in \mathbb{Z}[X,Y,Z] \}\]
Nu vragen we ons daarom af wat er waar moet zijn over een veelterm in $X$, $Y$ en $Z$ opdat $f\cdot (X-Y) + g\cdot (X^{3}-Z)$ nul is.
Wel, $f\cdot (X-Y)$ en $g \cdot (X^{3}-Z)$ moeten dan beide nul zijn.
\[ f \cdot (X-Y) = 0 \Leftrightarrow X = Y \vee f = 0\]
\[ g \cdot (X^{3}-Z) = 0 \Leftrightarrow X^{3} = Z \vee g = 0 \]
Deze voorwaarden zijn equivalent met de volgende:
\[ X = Y \wedge X^{3} = Z \]
We zoeken dus een morfisme dat veeltermen die hieraan voldoen op nul afbeeldt.
Beschouw bijvoorbeeld $\phi$:
\[ \phi:\ \mathbb{Z}[X,Y,Z] \rightarrow \mathbb{Z}[X]:\ f(X,Y,Z) \mapsto f(X,X,X^{3}) \]
We tonen nu aan dat $(X-Y,X^{3}-Z)$ inderdaad de kern is van $\phi$:
\begin{itemize}
\item $(X-Y,X^{3}-Z) \subseteq Ker(\phi)$\\
  Elk element uit $(X-Y,X^{3}-Z)$ ziet er uit als $f(X,Y,Z)\cdot (X-Y) + g(X,Y,Z)\cdot (X^{3}-Z)$.
  \[ \phi(f(X,Y,Z)\cdot (X-Y) + g(X,Y,Z)\cdot (X^{3}-Z)) = f(X,X,X^{3})\cdot (X-X) + g(X,X,X^{3})\cdot (X^{3}-X^{3}) = 0 \]
\item $Ker(\phi) \subseteq(X-Y,X^{3}-Z) $\\
  We moeten bewijzen dat elk element uit $Ker(\phi)$ eruit ziet als $f(X,Y,Z)\cdot (X-Y) + g(X,Y,Z)\cdot (X^{3}-Z)$.
  Kies een willekeurige $h\in \mathbb{Z}[X,Y,Z]$.
  We zullen eerst $h$ delen door $Z-X^{3}\neq 0$, de rest zal dan nog steeds in $Ker(\phi)$ zitten.
  Noem de quotient van die deling $f$.
  \[ h(X,Y,Z) = f(X,Y,Z)\cdot(X^{3}-Z) + r(X,Y) \]
  Merk ook op dat de rest $r(X,Y)$ onafhankelijk is van $Z$:
  \[ \mathbb{Z}[X,Y,Z] \cong (\mathbb{Z}[X,Y])[Z] \]
  We delen daarna $r$ nog door $Y-X \neq 0$. Opnieuw zal de rest nog steeds in $Ker(\phi)$ zitten.
  Noem de quotient van deze deling $g$.
  \[ h(X,Y,Z) = f(X,Y,Z)\cdot (X^{3}-Z) + g(X,Y) \cdot (Y-X) + r'(X) \]
  Merk nu op dat de rest $r'(X)$ onafhankelijk is van $Y$ en $Z$.
  \[ \mathbb{Z}[X,Y,Z] \cong (\mathbb{Z}[X,Y])[Z] \]
\clarify{hier moet meer uitleg bij!}
\end{itemize}

\subsection*{Oefening 6}
Zij $R,+,\cdot$ een willekeurige ring met eenheidselement.
Bereken $\nicefrac{R[X]}{(X)}$.

\subsection*{Oefening 7}
Zij $P=X^{3}+X+1 \in \mathbb{Z}_{2}[X]$.
Bereken voor elk element verschillen van $0$ van $\nicefrac{\mathbb{Z}_{2}[X]}{(P)}$ het inverse voor de vermenigvuldiging.

\section{Oefenzitting 8: Veeltermringen, HID's en UFD's}

\subsection*{Oefening 1}
Zoek de wortels in $\mathbb{Z}_{5}$ van $3+X+X^{64}\in \mathbb{Z}_{5}[X]$.
$\mathbb{Z}_{5}$ is klein genoeg om dit \'e\'en voor \'e\'en te doen, maar we willen $4^{64}$ natuurlijk niet hoeven uit te rekenen.
Noem $3+X+X^{64}$ $f$.
\[
\begin{array}{rll}
  f(0) &= 3 + 0 + 0^{64} &\neq 0\\
  f(1) &= 3 + 1 + 1^{64} &= 0\\
  f(2) &= 3 + 2 + 2^{64} &\neq 0 \text{ want } 2^{64} \nmid 5\\
  f(3) &= 3 + 3 + 3^{64} &\neq 0 \text{ want } 3^{66} \nmid 5\\
  f(4) &= 3 + 4 + 4^{64} &\neq 0 \text{ want } 2^{129} \nmid 5\\
\end{array}
\]
 
\subsection*{Oefening 2}
Geef een voorbeeld van waarom het derde criterium niet geldt in $\mathbb{Z}_{8}[X]$ (dat geen veld is).
\question{hoe?}

\subsection*{Oefening 3}
Schrijf $f=X^{3}+6\in \mathbb{Z}_{7}[X]$ als product van irreducibele veeltermen over $\mathbb{Z}_{7}$.

$\mathbb{Z}_{7}$ is een veld, want $7$ is een priemgetal.\gevref{gev:zp-veld}
$\mathbb{Z}_{7}$ is dus een UFD\stref{st:veeltermen-over-veld-hid}, de vraag is dus zinvol.
Bereken eerst de nulpunten van $f$:\stref{st:domein-nulpunten-delen-veelterm}
\[
\begin{array}{rll}
  f(0) &= 0 + 6 = 6&\\
  f(1) &= 1 + 6 = 0&\\
  f(2) &= 8 + 6 = 0&\\
  f(3) &= 33 + 6 = 4&\\
  f(4) &= 64 + 6 = 0&\\
\end{array}
\]
We weten al dat $f$ deelbaar is door $X-1$, $X-2$ en $X-4$.
\[
\begin{array}{cccc|l}
  X^{3} & & & +6 & X-1\\\hline
  X^{3} & -X^{2} &  &\vdots & X^{2}+X+1\\\cline{1-2}
       & X^{2} & & \vdots & \\
       & X^{2} & -X & \vdots &\\\cline{2-3}
       &       & X & +6 &\\
       &       & X & -1 &\\\cline{3-4}
       &       &   & 0 
\end{array}
\]
\[
\begin{array}{ccc|l}
  X^{2} & +X & +1 & X-2\\\hline
  X^{2} & -2X & \vdots & X+3\\\cline{1-2} 
       & 3X & +1 & \\
       & 3X & -6 &\\\cline{2-3}
       &    & 0
\end{array}
\]
$X^{3}+6$ is in $\mathbb{Z}_{7}$ dus gelijk aan $(X+6)(X+5)(X+3)$.


\subsection*{Oefening 4}
Zij $p$ een priemgetal, hoeveel irreducibele veeltermen over $\mathbb{Z}_{p}$ van de vorm $X^{2}+aX+b$ bestaan er?
\extra{oefening}

\subsection*{Oefening 5}
Vindt een grootst gemeenschappelijke deler van $X^{3}+2X^{2}+4X-7$ en $X^{2}+X-2$ in $\mathbb{Q}[X]$.
\extra{oefening}

\subsection*{Oefening 6}
Bekijk de veelterm $X^{2}+1$ over $\mathbb{H}$.
De wortel $i$ kunnen we afsplitsen:
\[ X^{2}+1 = (X-i)(X+i) \]
Omdat $j$ ook een wortel is, bekomen we na substitutie ook nul:
\[ (j-i)(j+i) = 0 \]
De laatste gelijkheid kan niet omdat $\mathbb{H}$ geen nuldelers heeft.
Waar zit de fout?

De rekenregels in $\mathbb{H}$ zijn fout toegepast.
\[ (j-i)(j+i) \neq j^{2}+1 \text{ maar wel } -2k \]


\subsection*{Oefening 7}
\begin{itemize}
\item Toon aan dat $X^{3}+5X^{2}+25X+5$ irreducibel is over $\mathbb{Q}$.
  \begin{proof}
    Kies $p=5$ in het criterium van Einstein.
  \end{proof}
\item Toon aan dat $5x^{3}+9x+3$ irreducibel is over $\mathbb{Q}$.
  \begin{proof}
    Kies $p=3$ in het criterium van Einstein.
  \end{proof}
\item Toon aan dat $5X^{3}-15X^{2}+24X-11$ irreducibel is over $\mathbb{Q}$.
  \begin{proof}
    We tonen eerst aan dat als $f(X+1)$ irreducibel is, $f(X)$ dan ook.
    Stel immers dat $f(X)$ reducibel is, dan zals $f(X+1)$ ook reducibel zijn.
    \[ f(X) = g(X) \cdot h(X) \Rightarrow f(X+1) = g(X+1) \cdot h(X+1)\]
    \[ f(X+1) = 5X^{3}+9X+3 \]
    Kies $p=3$ in het criterium van Einstein.
  \end{proof}
\end{itemize}


\subsection*{Oefening 8}

Definieer op $F$ als volgt:
\[ F = \{ a + b\alpha \ |\ a,b \in \mathbb{Z}_{2} \} = \{ 0,1,\alpha,\alpha+1 \} \]
\begin{itemize}
\item Bewijs dat $\alpha^{2}$ gelijk moet zijn aan $\alpha+1$.
  \begin{itemize}
  \item $\alpha^{2}$ kan niet gelijk zijn aan $0$ want dan zou $\alpha$ $0$ zijn.
  \item $\alpha^{2}$ kan niet gelijk zijn aan alpha want dan zou $alpha$ nul of $1$ zijn.
  \item $\alpha^{2}$ kan niet gelijk zijn aan $1$ want dan zou $\alpha(1+\alpha)$ gelijk zijn aan $\alpha+1$.
    Dit zou dan betekenen dat $\alpha$ gelijk is aan $1$.
  \item $\alpha^{2}$ moet dus gelijk zijn aan $\alpha+1$. 
  \end{itemize}
\item Maak de Cayleytabel van $F^{\times}$.
  \[
  \begin{array}{c|ccc}
    & 1 & \alpha & 1+\alpha\\ \hline
    1 & 1 & \alpha & 1 + \alpha\\
    \alpha & \alpha &1 + \alpha & 1\\
    1 + \alpha & 1 + \alpha & 1 & \alpha
  \end{array}
  \]
\item Bewijs dat $F,+,\cdot$ een veld is.
\extra{bewijs}
\end{itemize}

\section{Oefenzitting 9: Velden}

\subsection*{Oefening 1}
Bereken de karakteristiek $k$ van $\mathbb{Z}_{n} \cdot \mathbb{Z}_{m}$.

Merk eerst op dat de karakteristiek van $\mathbb{Z}_{n}$ en $\mathbb{Z}_{m}$ respectievelijk $n$ en $m$ zijn.
Voor elk element $(a,b) \in \mathbb{Z}_{n} \cdot \mathbb{Z}_{m}$ moet het volgende gelden
\[ k(a,b) = (0,0) \]
We weten echter dat $a$ enkel nul is als het een veelvoud van $n$ keer bij zichzelf wordt opgeteld.
Analoog geldt dat voor $b$.
$k$ moet dus $kgv(n,m)$ zijn.


\subsection*{Oefening 2}
Wanneer is de priemdeelring van $\mathbb{Z}_{n} \cdot \mathbb{Z}_{m}$ niet heel $\mathbb{Z}_{n} \cdot \mathbb{Z}_{m}$?

Zij $k=kgv(n,m)$ de karakteristiek van $\mathbb{Z}_{n} \cdot \mathbb{Z}_{m}$, dan is de priemdeelring isomorf met $\mathbb{z}_{k}$.\stref{st:karakteristiek-isomorfisme-met-zn}
Als $k$ $n\cdot m$ is, dan is $\mathbb{z}_{k}$ isomorf met $\mathbb{Z}_{n} \cdot \mathbb{Z}_{m}$ en de priemdeelring van $\mathbb{Z}_{n} \cdot \mathbb{Z}_{m}$ dus $\mathbb{Z}_{n} \cdot \mathbb{Z}_{m}$ zelf.
$k$ is $n\cdot m$ als en slechts als $n$ en $m$ relatief priem zijn:
\[ kgv(n,m) = n \cdot m \Leftrightarrow ggd(n,m) = 1 \]

\subsection*{Oefening 3}

Zij $F,+,\cdot$ een veld van orde een macht $p^{r}$ van een priemgetal $p$.
Met welke 'bekende' groep is $F,+,\cdot$ isomorf?

We berekenen eerst de karakteristiek $k$ van $F$ en beschouwen daarna $F$ als een vectorruimte over zijn priemdeelveld.
$F$ is eindig, dus $k$ eindig.
$k$ is bovendien een deler van de orde van $F$, en ook een priemgetal, dus $k$ moet gelijk zijn aan $p$.
Het priemdeelveld $P,+,\cdot$ van $F$ is isomorf met $\mathbb{Z}_{k} = \mathbb{Z}_{p}$ 
We bewijzen nu dat de uitbreidingsgraad van het priemdeelveld van $F,+,\cdot$ over $F,+,\cdot$ daarom $r$ is.
\begin{proof}
  Kies een basis $\{x_{i},\dotsc,x_{n}\}$ waarin $n$ de dimensie is van $F$ als vectorruimte.
  Elk element $f$ van $F$ ziet er dan als volgt uit:
  \[ f = \sum_{i=1}^{n}a_{i}x_{i} \text { met } \forall i:\ a_{i} \in P\]
  \[ P^{n} \rightarrow F:\ (a_{1},\dotsc,a_{n})\mapsto \sum_{i=1}^{n}a_{i}x_{i} \]
  Dit morfisme is injectief omdat het een veldmorfisme is\stref{st:veldmorfisme-is-injectief} en surjectief omdat de $x_{i}$ een basis vormen.
  \[ p^{r} = |F| = |\mathbb{Z}_{p}^{n}| = p^{n}\]
  $r$ moet dus gelijk zijn aan $n$.
\end{proof}

\subsection*{Oefening 4}
Zij $F,+,\cdot$ een veld.
Zij $a$ en $b$ elementen van $F$ waarbij $a$ een niet-nulelement is.
Zij bovendien $c$ een element in een uitbreiding van $F,+,\cdot$ naar $E,+,\cdot$.
\[ F(c) = F(ac+b) \]

\begin{proof}
  Het volstaat om aan te tonen dat $F(c)$ $ac+b$ bevat, omdat $F(ac+b)$ $F$ omvat.
  \[ c  = \frac{(ac+b)-b}{a} \in F(ac+b) \]
\end{proof}

\begin{gev}
  Voor elke $a,b \in \mathbb{R}$ met $b$ niet nul geldt $\mathbb{R}(a+bi) = \mathbb{C}$.
\end{gev}

\subsection*{Oefening 5}
Zij $a$ en $b$ twee getallen uit $\mathbb{Q}^{+}_{0}$.
Bewijs het volgende:
\[ \mathbb{Q(\sqrt{a})} = \mathbb{Q(\sqrt{b})} \Leftrightarrow \exists c\in \mathbb{Q}:\ a = bc^{2} \]

\begin{proof}
  \begin{itemize}
  \item $\Rightarrow$\\
    Er bestaat een $c\in \mathbb{Q}$ zodat $\sqrt{a}$ geschreven kan worden als volgt:
    \[ \frac{\sqrt{a}}{1} = \frac{c\sqrt{b}}{1} \]
    Dit is precies wanneer, voor die $c$, het volgende geldt:
    \[ a = bc^{2}\]
  \item $\Leftarrow$\\
    \begin{itemize}
    \item Voor alle elementen uit $\mathbb{Q}$ die geschreven kunnen worden als $\sqrt{a}$ met $a \in \mathbb{Q}$ geldt de bewering.
    \item Voor alle andere elementen, kies een basis voor $\mathbb{Q}(\sqrt{a})$ over $\mathbb{Q}$: $\{1,\sqrt{a}\}$.\clarify{moeten we niet bewijzen dat dit inderdaad een basis is?}
      Nu moeten we bewijzen dat $b$ te schrijven valt als lineaire combinatie van de elementen in die basis. \clarify{waarom enkel $b$ en niet elk element in $\mathbb{Q}(\sqrt{b})$}
      \[ \sqrt{b} = x + y\sqrt{a} \]
      \[ b = x^{2} + 2xy+2xy\sqrt{a} +ay^{2}\]
      \[ b = (x^{2}+y^{2})\cdot 1 + (2xy) \cdot \sqrt{a} \]
    \end{itemize}
  \end{itemize}      
\end{proof}

\subsection*{Oefening 6}











\end{document}


%%% Local Variables:
%%% mode: latex
%%% TeX-master: t
%%% End:
