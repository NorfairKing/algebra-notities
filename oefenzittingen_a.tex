\documentclass[main.tex]{subfiles}
\begin{document}

\chapter{Algebra I: Oefenzittingen}
\label{cha:algebra-i-oefenzittingen}

\section{Oefenzitting 1: Herhalingen en Aanvullingen}
\subsection*{Oefening 1}
\label{oza:oz1-oef1}
Kies de groep $\mathcal{S}_{3}$.
Beschouw nu de elementen $(12)$ en $(23)$, deze hebben beide orde $2$.
\[ (12)^{2} = Id \quad\text{ en }\quad (23)^{2} \]
Nu heeft $(12) \circ (23) = (132)$ orde $3$.
\[ (132)^{3} = (132)(123) = Id \]

\subsection*{Oefening 2}
\label{oza:oz1-oef2}
Zij $G,*$ een groep.
Stel dat $\forall x \in G: x^{2} = e$ geldt.
Toon aan dat $G,*$ commutatief is.

\begin{proof}
  Inderdaad\stref{st:groep-alles-orde-2-commutatief}
\end{proof}

\subsection*{Oefening 3}
\label{oza:oz1-oef3}
Zij $G$ een groep.
Toon aan dat $G$ commutatief is als en slechts als het volgende geldt:
\[ \forall a,b \in G, \forall n \in \mathbb{N}:\ (ab)^{n} = a^{n}b^{n} \]

\begin{proof}
  Kies twee willekeurige elementen $a$ en $b$ uit $G$:
  \begin{itemize}
  \item $\Rightarrow$\\
    Kies een willekeurige $n\in \mathbb{N}$
    \[ (ab)^{n} = ababab \ldots abab = aaaaa \ldots bbbbb \]
  \item $\Leftarrow$\\
    Kies $n=2$:
    \[ 
    \begin{array}{rl}
       abab &= aabb\\
       bab &= abb\\
       ba &= ab
    \end{array}
    \]
  \end{itemize}
\end{proof}

\subsection*{Oefening 4}
\label{oza:oz1-oef4}

\subsubsection*{(a)}
Bepaal alle deelgroepen van $\mathbb{Z}_{12},+$.\\\\
$\mathbb{Z}_{12},+$ heeft orde $12$.
\[
\begin{array}{c}
  \{ 0,2,4,6,8,10 \},+\\
  \{ 0,3,6,9 \},+\\
  \{ 0,4,8 \},+\\
  \{ 0,6 \},+\\
  \{ 0 \},+\\
\end{array}
\]

\subsubsection*{(b)}
Bepaal alle deelgroepen van $\mathcal{D}_{5},\circ$.\\\\
$\mathcal{D}_{5},+$ heeft orde $10$.
\[
\begin{array}{c}
  \{ e, a, a^{2}, a^{3}, a^{4} \},\circ\\
  \{ e, b^{5} \},\circ\\
  \{ e, b^{4} \},\circ\\
  \{ e, b^{3} \},\circ\\
  \{ e, b^{2} \},\circ\\
  \{ e, b \},\circ\\
  \{ e \},\circ\\
\end{array}
\]

\subsubsection*{(c)}
Bepaal alle deelgroepen van $\mathcal{Q},\cdot$.\\\\
$\mathcal{Q},\cdot$ heeft orde $8$.
\[
\begin{array}{c}
  \{ 1, -1, k, -k \},\cdot\\
  \{ 1, -1, j, -j \},\cdot\\
  \{ 1, -1, i, -i \},\cdot\\
  \{ 1, -1 \},\cdot\\
  \{ 1 \},\cdot\\
\end{array}
\]

\subsubsection*{(d)}
Bepaal alle deelgroepen van $\mathcal{A}_{4},\circ$.\\\\
$\mathcal{A}_{4},\circ$ heeft orde $12$.
\[
\begin{array}{c}
  \{ Id, (123), (132) \},\circ\\
  \{ Id, (124), (142) \},\circ\\
  \{ Id, (134), (143) \},\circ\\
  \{ Id, (234), (243) \},\circ\\
  \{ Id, (12)(34) \},\circ\\
  \{ Id, (13)(24) \},\circ\\
  \{ Id, (14)(23) \},\circ\\
  \{ Id, (34) \},\circ\\
  \{ Id, (24) \},\circ\\
  \{ Id, (23) \},\circ\\
  \{ Id, (14) \},\circ\\
  \{ Id, (13) \},\circ\\
  \{ Id, (12) \},\circ\\
  \{ Id \},\circ\\
\end{array}
\]

\subsection*{Oefening 5}
\label{oza:oz1-oef5}

Zij $G$ een cyclische groep van orde $n$ en $H$ een cyclische groep van orde $m$ met $ggd(m,n)=1$.
Bewijs dat $G \oplus H$ cyclisch is van orde $mn$.

\begin{proof}
  Zij $g$ een generator voor $G$ en $h$ een generator voor $H$.
  \[ g^{n} = e_{G} \quad\text{ en }\quad h^{m} = e_{H} \]
  Dit betekent dat in $G \oplus H$ het volgende geldt:
  \[ (g,e_{H})^{n} = (e_{G},e_{H}) \quad\text{ en }\quad (e_{G},h)^{m} = (e_{G},e_{H}) \]
  We bewijzen nu dat $(g,h)$ een generator is voor $G \oplus H$ met orde $mn$.
  \[ (g,h)^{mn} = (e_{G},h^{n}) = (g^{m},e_{H}) = (e_{G},e_{H}) \]
  Dit betekent dat $(g,h)$ $G \oplus H$ genereert met orde $mn$.
\end{proof}

\subsection*{Oefening 6}
\label{oza:oz1-oef6}

Beschouw de deelgroep $G$ van $Gl_{2}(\mathbb{R})$.
\[
G =
\left<
  \begin{pmatrix}
     0 & 1\\
    -1 & 0\\
  \end{pmatrix}
  ,
  \begin{pmatrix}
    0 & 1\\
    1 & 0
  \end{pmatrix}
\right>
\]
Bepaal $G$.
Met welke bekende groep is $G$ isomorf?

\[ 
G = 
\left\{ 
  \begin{pmatrix}
    1 & 0\\
    0 & 1
  \end{pmatrix}
,
  \begin{pmatrix}
    -1 & 0\\
    0 & -1
  \end{pmatrix}
,
  \begin{pmatrix}
    0 & 1\\
    1 & 0
  \end{pmatrix}
,
  \begin{pmatrix}
    0 & -1\\
    -1 & 0
  \end{pmatrix}
,
  \begin{pmatrix}
     0 & 1\\
    -1 & 0\\
  \end{pmatrix}
,
  \begin{pmatrix}
     0 & -1\\
    1 & 0\\
  \end{pmatrix}
,
  \begin{pmatrix}
    1 & 0\\
    0 & -1
  \end{pmatrix}
,
  \begin{pmatrix}
    -1 & 0\\
    0 & 1
  \end{pmatrix}
\right\}
\]
$G$ is isomorf met de quaternionengroep.


\subsection*{Oefening 7}
\label{oza:oz1-oef7}
\subsubsection*{(a)}
Zij $G,*$ en $H,\Box$ groepen met een morfisme $f: G \rightarrow H$.
Als een verzameling $A$ een deelgroep is van $G$, dan is $f(A)$ een deelgroep van $H$.\\\\
Inderdaad,\stref{st:fa-deelgroep-h}
\subsubsection*{(b)}
Zij $G,*$ en $H,\Box$ groepen met een morfisme $f: G \rightarrow H$.
Als een verzameling $B$ een deelgroep is van $H$, dan is $f^{-1}(B)$ een deelgroep van $G$.\\\\
Inderdaad\stref{st:fbm-deelgroep-g}

\subsection*{Oefening 8}
\label{oza:oz1-oef8}
Zij $G,\cdot$ een niet-cyclische groep van orde $6$.

\subsubsection*{(a)}
$G$ heeft een element van orde $3$.

\begin{proof}
De orde van een element van een groep is een deler van de orde van de groep.\gevref {gev:orde-van-element-deelt-orde-van-groep}
Elk element heeft dus een orde in $\{ 1, 2, 3, 6 \}$.
Enkel $e$ heeft als orde $1$, en er zijn geen elementen van orde $6$ want dan zou $G$ cyclisch zijn.
Stel dat elk element orde $2$ heeft, dan is $G$ commutatief.\stref{st:groep-alles-orde-2-commutatief}
Dit zou betekenen dat $\{ e,a,b,ab \}$ een deelgroep was van $G$ met orde $4$, maar $4$ is geen deler van $6$. Contradictie\stref{st:stelling-van-lagrange}
Bijgevolg bestaat er minstens \'e\'en element van orde $3$ en is de orde van $G$ dus $3$.\stref{st:orde-van-generator-is-orde-van-groep}

\end{proof}

\subsubsection*{(b)}
Zij $a$ een element van orde $3$ en $b \not\in \{ e,a,a^{2} \}$, dan geldt $G= \{ e,a,a^{2},b,ab,a^{2}b \}$.

\begin{proof}
  We bewijzen dat $\{ e,a,a^{2},b,ab,a^{2}b \}$ enkel verschillende elementen bevat.
  \begin{itemize}
  \item $\{ e,a,a^{2} \}$ is een verzameling van onderling verschillende elementen, en bovendien een groep.
  \item $\{ e,a,a^{2},b \}$ is een verzameling van onderling verschillende elementen omdat $b$ niet in$\{ e,a,a^{2} \}$ zit en $\{ e,a,a^{2} \}$ een groep is.
  \item $\{ e,a,a^{2},b,ab \}$ is een verzameling van onderling verschillende elementen:\\
    \begin{tabular}[H]{cl}
      \centering
      $ab \neq e$ & $b$ zou immers de inverse zijn van $a$ en niet in de groep $\{ e,a,a^{2} \}$ zitten.\\
      $ab \neq a$ & want dan zou $b$ het neutraal element zijn.\\
      $ab \neq a$ & want dan zou $b$ gelijk zijn aan $a$.\\
      $ab \neq b$ & want dan zou $a$ het neutraal element zijn.\\
    \end{tabular}
  \item $\{ e,a,a^{2},b,ab,a^{2}b \}$ is een verzameling van onderling verschillende elementen:\\
    \begin{tabular}[H]{cl}
      $a^{2}b \neq e$ & $b$ zou immers de inverse zijn van $a^{2}$ en niet in de groep $\{ e,a,a^{2} \}$\\
      $ab^{2} \neq a$ & want dan zou $b^{2}$ het neutraal element zijn.\\
      $ab^{2} \neq a^{2}$ & want dan zou $b^{2}$ gelijk zijn aan $a$ en $b^{3}=ab$ dus gelijk aan $b$\\
      $ab^{2} \neq b$ & want dan zou $ab$ gelijk zijn aan $b$.\\
      $ab^{2} \neq ab$ & want dan zou $b$ het neutraal element zijn.\\
    \end{tabular}
  \end{itemize}
\end{proof}

\subsubsection*{(c)}
De orde van $b$ is $2$.

\begin{proof}
  We gaan na aan welk element van $\{ e,a,a^{2},b,ab,a^{2}b \}$ $b^{2}$ gelijk moet zijn.\\
  \begin{tabular}[H]{cl}
    \centering
    $b^{2} \neq  a$ & want ???\\
    $b^{2} \neq  a^{2}$ & want dan zou $b^{2}a$ het neutraal element zijn en ???\\
    $b^{2} \neq  ab$ & want dan zou $a$ gelijk zijn aan $b$.\\
    $b^{2} \neq  a^{2}b$ & want dan zou $a^{2}$ gelijk zijn aan $b$.\\
  \end{tabular}\\
  Bijgevolg geldt $b^{2} = e$.
\end{proof}


\subsubsection*{(d)}
$ba$ is gelijk aan $a^{2}b$.

\begin{proof}
  \[
  \begin{array}{rll}
    ba &\neq e &\text{want dan zou } \{ e,a,a^{2} \} \text{ geen groep zijn.}\\ 
    ba &\neq a &\text{(gegeven)}\\ 
    ba &\neq a^{2} &\text{want dan zou } b=a \text{ gelden.} \\ 
    ba &\neq b &\text{want dan zou } a=e \text{ gelden.} \\ 
    ba &\neq ab &\text{want ???} \\ 
    ba &\neq a^{2}b &\text{want dan zou } b=a^{2} \text{ gelden.} \\ 
  \end{array}
  \]
\clarify{Waarom?}
\end{proof}

\subsubsection*{(e)}
$G \cong \mathcal{D}_{3} \cong \mathcal{S}_{3}$ geldt.

\begin{proof}
  Constructief bewijs.
  \begin{itemize}
  \item Er bestaat een bijectie tussen $G$ en $\mathcal{S}_{3}$:
    \[
    \begin{pmatrix}
      e & a     & a^{2} & b    & ab   & a^{2}b\\
      e & (123) & (132) & (12) & (13) & (23)\\
    \end{pmatrix}
    \]
  \item $G$ is gelijk aan $D_{3}$.
  \end{itemize}
\end{proof}

\section{Oefenzitting 2: Permutatiegroepen}

\subsection*{Oefening 1}
\label{oza:oz2-oef1}
Geef alle mogelijke permutaties $\pi$ op de verzameling $\{1,2,3,4,5\}$ die orde $5$ hebben en voldoen aan $\pi(1) = 5$, $\pi(2) = 3$ en $\pi(3) = 4$.\\\\
De permutaties die we zoeken zien er als volgt uit:
\[
\begin{pmatrix}
  1 & 2 & 3 & 4 & 5\\
  5 & 3 & 4 & a & b
\end{pmatrix}
\] 
Om een permutatie te zijn moet $\{a,b\}$ gelijk zijn aan $\{1,2\}$.
Als $a$ $1$ is en $b$ $2$, voldoet de permutatie aan de opgelegde eigenschappen.
Als $a$ $2$ is en $b$ $1$, heeft de permutatie orde $6$.
Antwoord:
\[
\begin{pmatrix}
  1 & 2 & 3 & 4 & 5\\
  5 & 3 & 4 & 1 & 2
\end{pmatrix}
\] 

\subsection*{Oefening 2}
\label{oza:oz2-oef2}
Beschouw het volgend geheim schrift.
De vercijfering van een tekst gebeurt door in de tekst de letters van het alfabet te permuteren volgens de permutatie $\pi$.
\[
\pi =
\left(
\begin{array}{cccccccccccccccccccccccccc}
  a & b & c & d & e & f & g & h & i & j & k & l & m & n & o & p & q & r & s & t & u & v & w & x & y & z\\
  b & c & d & e & f & g & h & i & a & z & x & m & l & o & p & n & s & r & t & q & j & w & v & k & u & y\\ 
\end{array}
\right)
\]

\subsubsection*{(a)}
Zoek de disjuncte cykel-schrijfwijze van de permutatie $\pi$.
\[ (abcdefghi)(jzyuj)(kx)(lm)(nop)(qst)(vw) \] 

\subsubsection*{(b)}
Wat is de orde van $\pi$?
\[ kgv(9,5,2,2,3,3,2) = 90 \]

\subsubsection*{(c)}
Wat is het teken van $\pi$?
$-1$ want er is een oneven aantal cykels met even lengte.
\extra{verwijzing naar stelling.}

\subsection*{Oefening 3}
\label{oza:oz2-oef3}
Waar of niet?
Er bestaat geen permutatie $\pi$ van $9$ elementen met orde $15$.\\\\
Niet waar:
\[ (abc)(efghi) \]

\subsection*{Oefening 4}
\label{oza:oz2-oef4}
Onder welke voorwaarden is de afbeelding $\pi:\ \{ 0,\dotsc, n-1 \} \rightarrow \{ 0,\dotsc, n-1 \}:\ i \mapsto ai + b \mod n$ een permutatie. ($a,b \in \mathbb{Z}$).\\\\
$a$ en $b$ moeten relatief priem zijn.
\waarom

\subsection*{Extra oefening}
\label{oza:oz2-extra}
Zij $\sigma$ en $\tau$ permutaties in $\mathcal{S}_{n}$.
Zoek de nodige en voldoende voorwaarden op $\sigma$ en $\tau$ opdat zo zouden commuteren.
Hint: $\sigma \tau = \tau \sigma$ geldt als en slechts als $\tau$ en $\sigma$ disjunct zijn \emph{of} ...
\extra{oefening}

\section{Oefenzitting 3: Conjugatie en klasvergelijking}

\subsection*{Oefening 1}
\label{sec:oz3-oef1}
Geef alle conjugatieklassen van een Abelse groep $G,*$.
Controleer de klasvergelijking voor een eindige abelse groep.\\\\
De conjugatie met $a$ is in een Abelse groep de identieke transformatie $Id$.
De conjugatieklassen van $G,*$ zijn dus singletons van alle respectievelijke elementen in $G$.
De klasvergelijking is bijzonder simpel voor een Abelse groep omdat dan het centrum van $G,*$ heel $G$ is.

\subsection*{Oefening 2}
\label{sec:oz3-oef2}
Geef de conjugatieklassen van $\mathcal{S}_{3}$ en controleer de klasvergelijking.
\[ 
\left\{ 
  \begin{array}{c}
\{ Id \},\\
\{ (12),(13),(23) \},\\
\{ (123),(132) \},\\
\end{array}
\right\}
\]
\[
\begin{array}{ccccc}
  |G| &=& |Z(G)| &+& \sum|Cl(a_{i})|\\
  |\mathcal{S}_{3}| &=& 1 &+& 5\\
\end{array}
\]


\subsection*{Oefening 3}
\label{sec:oz3-oef2}
Geef de conjugatieklassen van $\mathcal{S}_{4}$ en controleer de klasvergelijking.
\[ 
\left\{ 
  \begin{array}{c}
\{ Id \},\\
\{ (12),(13),(14),(23),(24),(34) \},\\
\{ (123),(132),(124),(142),(234),(243),(134),(143) \},\\
\{ (12)(34),(13)(24),(14)(23) \},\\
\{ (1234),(1342),(1423),(1432),(1243),(1324) \}\\
\end{array}
\right\}
\]
\[
\begin{array}{ccccc}
  |G| &=& |Z(G)| &+& \sum|Cl(a_{i})|\\
  |\mathcal{S}_{4}| &=& 1 &+& 23\\
\end{array}
\]

\subsection*{Oefening 4}
\label{sec:oz3-oef3}
Geef de conjugatieklassen van de quaternionengroep en controleer de klasvergelijking.
\[ 
\left\{ 
  \begin{array}{c}
\{ 1 \},\\
\{ -1 \},\\
\{ i,-i \},\\
\{ j,-j \},\\
\{ k,-k \}\\
\end{array}
\right\}
\]
\[
\begin{array}{ccccc}
  |G| &=& |Z(G)| &+& \sum|Cl(a_{i})|\\
  |\mathcal{Q}| &=& 2 &+& 6\\
\end{array}
\]

\subsection*{Oefening 5}
\label{sec:oz3-oef4}
Geef de conjugatieklassen van de didi\"edergroep $\mathcal{D}_{4}$ met $8$ elementen en controleer de klasvergelijking.
\[ 
\left\{ 
  \begin{array}{c}
\{ Id \},\\
\{ a^{2} \},\\
\{ a,a^{3} \},\\
\{ b,a^{2}b \},\\
\{ ab,a^{3}b \}\\
\end{array}
\right\}
\]
\[
\begin{array}{ccccc}
  |G| &=& |Z(G)| &+& \sum|Cl(a_{i})|\\
  |\mathcal{D}_{4}| &=& 2 &+& 6\\
\end{array}
\]

\subsection*{Oefening 6}
\label{sec:oz3-oef5}
Bepaal alle eindige groepen met precies $2$ conjugatieklassen.\\\\
Zij $G,*$ een eindige groep met precies $2$ conjugatieklassen.
Elk element in het centrum heeft zijn eigen conjugatieklas.
Er zit minstens $1$ element in het centrum: $e_{G}$.
$e_{G}$ zit dus al zeker in zijn eigen conjugatieklas.
Alle andere elementen zitten in dus samen in een conjugatieklas.
De groep is van orde minstens twee omdat alle conjugatieklassen niet-leeg zijn.
Noem $C$ de conjugatieklas van $G,*$ waar $e_{G}$ niet in zit.
\[ |C| = |G| - 1 \]
We weten dat de orde van elke conjugatieklas een deler moet zijn van de orde van $G$\gevref{gev:orde-conjugatieklasse-deelt-orde-groep}.
$|G| - 1$ is alleen een deler van $|G|$ als $|G|$ precies $2$ is.
Er is maar $1$ groep van orde $2$, op isomorfisme na: $\mathbb{Z}_{2}$.

\subsection*{Oefening 7}
\label{sec:oz3-oef6}
Bepaal alle eindige groepen met precies $3$ conjugatieklassen.\\\\
Verdergaand op de redenering in de vorige oefening moet een groep $G,*$ met precies $3$ conjugatieklassen minstens drie elementen.
$G$ heeft dan als orde $1+a+b$ zodat zowel $a$ als $b$ $|G|$ delen.
De mogelijkheden voor $(a,b)$ zijn dan $\{(1,1),(1,2),(2,3)\}$.
\begin{itemize}
\item Als $a = b = 1$ geldt, is $G$ abels, en dan is $G$ isomorf met $\mathbb{Z}_{3}$.
\item Als $a=1$ en $b=2$ gelden, is $G$ een groep van orde $4$ die niet abels. Zo'n groep bestaat niet.
  Elke groep van orde $4$ is immers isomorf met ofwel de viergroep, ofwel $\mathbb{Z}_{4}$.\eiref{ei:groep-orde-vier}
\item Als $a=2$ en $b=3$ gelden, is $G$ een groep van orde $6$. $G$ is dan ofwel isomorf met $\mathcal{S}_{3}$ ofwel met $\mathbb{Z}_{6}$.\eiref{ei-groep-orde-zes}
\end{itemize}

\subsection*{Oefening 8}
\label{sec:oz3-oef7}
Bepaal de orde van het centrum van een niet-abelse groep $P$ van orde $p^{3}$ met $p$ een priemgetal.
Controleer dit voor de didi\"edergroep $\mathcal{D}_{4}$ en de quaternionengroep.\\\\
De orde van het centrum is een deler van de orde van $P$\eiref{ei:centrum-is-deelgroep} \stref{st:stelling-van-lagrange} en bovendien groter dan $1$.\prref{pr:orde-centrum-pgroep-groter-dan-een}.
$|Z(P)|$ moet dus $p$ of $p^{2}$ zijn.
Voor $|Z(P)|$ $3$ zou immers betekenen dat $P$ abels is.
De orde van $P$ kan bovendien niet $p^{2}$ zijn, want dan zou $P$ abels zijn.\stref{st:priemgroep-kwadraat-abels}







\end{document}
