\documentclass[main.tex]{subfiles}
\begin{document}

\chapter{Algebra I: Oefenzittingen}
\label{cha:algebra-i-oefenzittingen}

\section{Oefenzitting 1}

\TODO{Oefening 1}

\subsection{Oefening 2}
\label{oz1-oef2}
Zij $G,*$ een groep.
Stel dat $\forall x \in G: x^{2} = e$ geldt.
Toon aan dat $G,*$ commutatief is.

\begin{proof}
  Kies twee willekeurige elementen $x$ en $y$ uit $G$:
  \[ 
  \begin{array}{rl}
  (x*y)^{2} &= e\\
  (x*y*x*y) &= e\\
  x*(x*y*x*y)*y &= x*e*y\\
  (x*x)*y*x*(y*y) &= x*y\\
  y*x &= x*y\\
  \end{array}
  \]
\end{proof}

\subsection{Oefening 3}
Zij $G$ een groep.
Toon aan dat $G$ commutatief is als en slechts als het volgende geldt:
\[ \forall a,b \in G, \forall n \in \mathbb{N}:\ (ab)^{n} = a^{n}b^{n} \]

\begin{proof}
  Kies twee willekeurige elementen $a$ en $b$ uit $G$:
  \begin{itemize}
  \item $\Rightarrow$\\
    Kies een willekeurige $n\in \mathbb{N}$
    \[ (ab)^{n} = ababab \ldots abab = aaaaa \ldots bbbbb \]
  \item $\Leftarrow$\\
    Kies $n=2$:
    \[ 
    \begin{array}{rl}
       abab &= aabb\\
       bab &= abb\\
       ba &= ab
    \end{array}
    \]
  \end{itemize}
\end{proof}

\TODO{Oefening 3}


\subsection{Oefening 4}

\subsubsection{(a)}
Bepaal alle deelgroepen van $\mathbb{Z}_{12},+$.
\[
\begin{array}{c}
  \{ 0,2,4,6,8,10 \},+\\
  \{ 0,3,6,9 \},+\\
  \{ 0,4,8 \},+\\
  \{ 0,6 \},+\\
  \{ 0 \},+\\
\end{array}
\]


\TODO{Oefening 4b}
\TODO{Oefening 4c}
\TODO{Oefening 4d}


\TODO{Oefening 5}
\TODO{Oefening 6}

\subsection{Oefening 7}
\subsubsection{(a)}
Zij $G,*$ en $H,\Box$ groepen met een morfisme $f: G \rightarrow H$.
Als een verzameling $A$ een deelgroep is van $G$, dan is $f(A)$ een deelgroep van $H$.

Inderdaad, zie \ref{st:fa-deelgroep-h}
\subsubsection{(b)}
Zij $G,*$ en $H,\Box$ groepen met een morfisme $f: G \rightarrow H$.
Als een verzameling $B$ een deelgroep is van $H$, dan is $f^{-1}(B)$ een deelgroep van $G$.

Inderdaad, zie \ref{st:fbm-deelgroep-g}

\subsection{Oefening 8}
Zij $G,\cdot$ een niet-cyclische groep van orde $6$.

\subsubsection{(a)}
$G$ heeft een element van orde $3$.

\begin{proof}
De orde van een element van een groep is een deler van de orde van de groep.\footnote{Zie gevolg \ref {gev:orde-van-element-deelt-orde-van-groep}.}
Elk element heeft dus een orde in $\{ 1, 2, 3, 6 \}$.
Enkel $e$ heeft als orde $1$, en er zijn geen elementen van orde $6$ want dan zou $G$ cyclisch zijn.
Stel dat elk element orde $2$ is, dan is $G$ commutatief.\footnote{Zie oefening 2. (\ref{oz1-oef2})}
Dit zou betekenen dat $\{ e,a,b,ab \}$ een deelgroep was van $G$ met orde $4$, maar $4$ is geen deler van $6$. Contradictie\footnote{Zie stelling \ref{st:stelling-van-lagrange}.}
Bijgevolg bestaat er minstens \'e\'en element van orde 3.
\end{proof}

\subsubsection{(b)}
Zij $a$ een element van orde $3$ en $b \not\in \{ e,a,a^{2} \}$, dan geldt $G= \{ e,a,a^{2},b,ab,a^{2}b \}$.

\begin{proof}
  We bewijzen dat $\{ e,a,a^{2},b,ab,a^{2}b \}$ enkel verschillende elementen bevat.
  \begin{itemize}
  \item $\{ e,a,a^{2} \}$ is een verzameling van onderling verschillende elementen, en bovendien een groep.
  \item $\{ e,a,a^{2},b \}$ is een verzameling van onderling verschillende elementen omdat $b$ niet in$\{ e,a,a^{2} \}$ zit en $\{ e,a,a^{2} \}$ een groep is.
\clarify{Waarom?}
  \item $\{ e,a,a^{2},b,ab \}$ is een verzameling van onderling verschillende elementen:
    \[ ab \neq e \text{ want dan zou } \{ e,a,a^{2} \} \text{ geen groep zijn.} \]
    \[ ab \neq a \text{ want dan zou } b = e \text{ gelden. } \]
    \[ ab \neq a^{2} \text{ want dan zou } b = a \text{ gelden. } \]
    \[ ab \neq b \text{ want dan zou } a = b \text{ gelden. } \]
  \item $\{ e,a,a^{2},b,ab,a^{2}b \}$ is een verzameling van onderling verschillende elementen:
    \[ a^{2}b \neq e \text{ want dan zou } \{ e,a,a^{2},b \} \text{ geen groep zijn.} \]
\clarify{Waarom?}
    \[ ab^{2} \neq a \text{ want dan zou } b^{2} = e \text{ gelden. } \]
\clarify{Waarom?}
    \[ ab^{2} \neq a^{2} \text{ want dan zou }  \]
\clarify{Waarom?}
    \[ ab^{2} \neq b \text{ want dan zou } ab = b \text{ gelden. } \]
    \[ ab^{2} \neq ab \text{ want dan zou } b = e \text{ gelden. } \]
  \end{itemize}
  
\end{proof}

\subsubsection{(c)}
De orde van $b$ is $2$.

\begin{proof}
  \[
  \begin{array}{rll}
    b^{2} &\neq a &\text{want} ???\\ 
    b^{2} &\neq a^{2} &\text{want} ??? \\ 
    b^{2} &\neq ab &\text{want dan zou } b=a \text{ gelden.} \\ 
    b^{2} &\neq a^{2}b &\text{want dan zou } b=a^{2} \text{ gelden.} \\ 
  \end{array}
  \]
\clarify{Waarom?}
  Bijgevolg geldt $b^{2} = e$.
\end{proof}


\subsubsection{(d)}
$ba$ is gelijk aan $a^{2}b$.

\begin{proof}
  \[
  \begin{array}{rll}
    ba &\neq e &\text{want dan zou } \{ e,a,a^{2} \} \text{ geen groep zijn.}\\ 
    ba &\neq a &\text{(gegeven)}\\ 
    ba &\neq a^{2} &\text{want dan zou } b=a \text{ gelden.} \\ 
    ba &\neq b &\text{want dan zou } a=e \text{ gelden.} \\ 
    ba &\neq ab &\text{want ???} \\ 
    ba &\neq a^{2}b &\text{want dan zou } b=a^{2} \text{ gelden.} \\ 
  \end{array}
  \]
\clarify{Waarom?}
\end{proof}

\subsubsection{(e)}
$G \cong \mathcal{D}_{3} \cong \mathcal{S}_{3}$ geldt.

\begin{proof}
  Constructief bewijs.
  \begin{itemize}
  \item Er bestaat een bijectie tussen $G$ en $\mathcal{S}_{3}$:
    \[
    \begin{pmatrix}
      e & a     & a^{2} & b    & ab   & a^{2}b\\
      e & (123) & (132) & (12) & (13) & (23)\\
    \end{pmatrix}
    \]
  \item $G$ is gelijk aan $D_{4}$.
  \end{itemize}
\end{proof}

\end{document}
