\documentclass[main.tex]{subfiles}
\begin{document}

\chapter{Voorbeelden}
\label{cha:voorbeelden}

\section{Groepen}
\label{sec:groepen}

\begin{de}
  De \term{triviale groep} is de groep met enkel een neutraal element.
  \[ G,* = \{ e_{G} \},* \]
\commutatief
\end{de}

\begin{vb}
  De getallen, uitgerust met de optelling, vormen groepen met $0$ als neutraal element.
  \[ \mathbb{Z} \subseteq \mathbb{Q} \subseteq \mathbb{R} \subseteq \mathbb{C} \]
\commutatief
\end{vb}

\begin{vb}
  De complexe en re\"ele getallen, uitgerust met de vermeningvuldiging, vormen groepen met $1$ als neutraal element.
  \[ \mathbb{R} \subseteq \mathbb{C} \]
\commutatief  
\end{vb}

\begin{vb}
  De $n$-tallen, uitgerust met de optelling, vormen groepen met $\vec{0}$ als neutraal element.
  \[ \mathbb{Z}^{n} \subseteq \mathbb{Q}^{n} \subseteq \mathbb{R}^{n} \subseteq \mathbb{C}^{n} \]
\commutatief  
\end{vb}

\begin{vb}
  De $n$-de eenheidswortels $\{ z \in \mathbb{C}\ |\ z^{n} = 1 \}$, uitgerust met de vermeningvuldiging is een groep met neutraal element $1$.
\end{vb}

\begin{de}
  De \term{cirkelgroep} $S^{1}$ is de verzameling van complexe getallen met modulus $1$, uitgerust met de vermenigvuldiging.
\end{de}

\begin{vb}
  De matrixes van dezelfde vorm, uitgerust met de optelling, vormen groepen met $[0]$ als neutraal element.
  \[ \mathbb{Z}^{m \times n} \subseteq \mathbb{Q}^{m \times n} \subseteq \mathbb{R}^{m \times n} \subseteq \mathbb{C}^{m \times n} \]
\commutatief  
\end{vb}

\begin{vb}
  De verzameling $GL_{n}(\mathbb{R})$ van inverteerbare re\"ele $n\times n$-matrices, uitgerust met de matrix vermeningvuldiging, vormen een groep met $\mathbb{I}_{n}$ als neutraal element.
\end{vb}

\begin{vb}
  De verzameling $\mathbb{R}[x]$ van re\"ele veeltermen in de veranderlijke $x$, uitgeruist met de optelling, is een groep met neutraal element $0$.\\
\commutatief  
\end{vb}

\begin{de}
  De \term{viergroep} of \term{groep van Klein} $V,\cdot$ heeft als neutraal element $e$.
  Er geldt een eenvoudige regel: $ab = c$, $bc = a$, $ca = b$
  \[
  \begin{array}{c|cccc}
    \circ & e & a & b & c \\
    \hline
    e & e & a & b & c \\
    a & a & e & c & b \\
    b & b & c & b & a \\
    c & c & b & a & e \\
  \end{array}
  \]
\commutatief
\end{de}

\begin{vb}
  De re\"ele getallen $\mathbb{Z}$, uitgerust met de optelling modulo $n\in \mathbb{N}_{0}$, vormen een groep $\mathbb{Z}_{n}$ met neutraal element $[0]_{n}$. Zo een groep heet de \term{restklassengroep} van graad $n$.
\commutatief       
\end{vb}

\begin{de}
  De \term{symmetrische groep} van graad $n$: $\mathcal{S}_{n},\circ$ is de groep van permutaties van $\{ 1,\dotsc,n \}$.
  \[ |\mathcal{D}_{n}| = n! \]
  Voor $n \le 2$ is $\mathcal{S}_{n}$ commutatief.
\end{de}

\begin{ei}
  $\mathcal{S}_{n}$ is een deelgroep van $\mathcal{S}_{n+1}$.
  \TODO{bewijs}
\end{ei}

\begin{vb}
  $\mathcal{S}_{1},\circ$: De groep van permutaties van $\{ 1 \}$.\\
  \[
  \begin{array}{c|c}
    \circ & Id \\
    \hline
    Id & Id\\
  \end{array}
  \]
\commutatief  
\end{vb}

\begin{vb}
  $\mathcal{S}_{2},\circ$: De groep van permutaties van $\{ 1,2 \}$.\\
  \[
  \begin{array}{c|cc}
    \circ & Id & (12) \\
    \hline
    Id & Id & (12)\\
    (12) & (12) & Id
  \end{array}
  \]
\commutatief  
\end{vb}

\begin{vb}
  $\mathcal{S}_{3},\circ$: De groep van permutaties van $\{ 1,2,3 \}$.\\
  \[
  \begin{array}{c|cccccc}
    \circ & Id & (12) & (13) & (23) & (123) & (132) \\
    \hline
    Id & Id & (12) & (13) & (23) & (123) & (132)\\
    (12) & (12) & Id & (132) & (123) & (23) & (13)\\
    (13) & (13) & (123) & Id & (132) & (12) & (23)\\
    (23) & (23) & (132) & (123) & Id & (13) & (12)\\
    (123) & (123) & (13) & (23) & (12) & (132) & Id\\
    (132) & (132) & (23) & (12) & (13) & Id & (123)\\
  \end{array}
  \]
\end{vb}

\begin{vb}
  $\mathcal{S}_{4},\circ$: De groep van permutaties van $\{ 1,2,3,4 \}$.\\
  \[ 
  \mathcal{S}_{4} = 
  \begin{array}{l}
     \{\\
     Id,\\
     (12), (13), (14), (23), (24), (34),\\
     (123), (124), (132), (134), (142), (143), (234), (243)\\
     (12)(34), (13)(42), (14)(23),\\
     (1234), (1243), (1324), (1342), (1423), (1432)\\
     \}
  \end{array}
  \]
\end{vb}

\begin{de}
  $Q,\cdot$,: De \term{quaternionengroep}.\\
  \[ Q = \{ 1,-1,i,-i,j,-j,k,-k \} \]
  Er gelden vier eenvoudige regels:
  \begin{itemize}
  \item $i^{2}=j^{2}=k^{2}= -1$
  \item $ij=k$
  \item $jk=i$
  \item $ki=j$
  \end{itemize}
  \[ 
  \begin{array}{r|rrrrrrrr}
\cdot & 1 & -1 & i & -i & j & -j & k & -k\\
\hline
1 & 1 & -1 & i & -i & j & -j & k & -k\\
-1 & -1 & 1 & -i & i & -j & j & -k & k\\
i & i & -i & -1 & 1 & k & -k & -j & j\\
-i & -i & i & 1 & -1 & -k & k & j & -j\\
j & j & -j & -k & k & -1 & 1 & i & -i\\
-j & -j & j & k & -k & 1 & -1 & -i & i\\
k & k & -k & j & -j & -i & i & -1 & 1\\
-k & -k & k & -j & j & i & -i & 1 & -1
  \end{array}
\]
\end{de}

\begin{de}
  De \term{Di\"edergroep} of \term{Dihedrale groep} van graad $n$: $\mathcal{D}_{n},\circ$ is de groep van starre bewegingen die een regelmatige $n$-hoek op zichzelf afbeelden.
  $\mathcal{D}_{n}$ bevat $n$ rotaties $Id, a, \dotsc, a^{n-1}$ en $n$ spiegelingen $b,ab,\dotsc,a^{n-1}b$.
  \[ |\mathcal{D}_{n}| = 2n \]
  Er gelden drie eenvoudige regels:
  \begin{itemize}
  \item $a^{n} = Id$
  \item $b^{2} = Id$
  \item $ba = a^{-1}b$             
  \end{itemize}
\end{de}

\begin{vb}
  $\mathcal{D}_{1},\circ$: De groep van starre bewegingen die een regelmatige $1$-hoek op zichzelf afbeelden.
  \[
  \begin{array}{c|cccccc}
    \circ & Id & b\\
    \hline
    Id & Id & b\\
    b & b & Id\\
  \end{array}
  \]
\commutatief  
\end{vb}

\begin{vb}
  $\mathcal{D}_{2},\circ$: De groep van starre bewegingen die een regelmatige $2$-hoek op zichzelf afbeelden.
  \[
  \begin{array}{c|cccccc}
    \circ & Id & a & b & ab\\
    \hline
    Id & Id & a & b & ab\\
    a & a & Id & ab & b\\
    b & b & ab & Id & a\\
    ab & ab & b & a & Id
  \end{array}
  \]
\commutatief  
\end{vb}

\begin{vb}
  $\mathcal{D}_{3},\circ$: De groep van starre bewegingen die een regelmatige $3$-hoek op zichzelf afbeelden.
  \[
  \begin{array}{c|cccccc}
    \circ & Id & a & a^{2} & b & ab & a^{2}b\\
    \hline
    Id & Id & a & a^{2} & b & ab & a^{2}b \\
    a & a & a^{2} & Id & ab & a^{2}b & b \\
    a^{2} & a^{2} & Id & a & a^{2}b & b & ab \\
    b & b & a^{2}b & ab & Id & a^{2} & a \\
    ab & ab & b & a^{2}b & a & Id & a^{2} \\
    a^{2}b & a^{2}b & ab & b & a^{2} & a & Id \\
  \end{array}
  \]
\end{vb}

\begin{vb}
  $\mathcal{D}_{4},\circ$: De groep van starre bewegingen die een regelmatige $4$-hoek op zichzelf afbeelden.
  \[
  \begin{array}{c|cccccccc}
    \circ & Id & a & a^{2} & a^{3} & b & ab & a^{2}b & a^{3}b \\
    \hline
    Id & Id & a  & a^{2} & a^{3} & b & ab & a^{2}b & a^{3}b \\
    a & a & a^{2} & a^{3} & Id & ab & a^{2} & a^{3}b & b \\
    a^{2} & a^{2} & a^{3} & Id & a & a^{2}b & a^{3}b & b & ab \\
    a^{3} & a^{3} & Id & a & a^{2} & a^{3}b & b & ab & a^{2}b \\
    b & b & a^{3}b & a^{2}b & ab & Id & a^{3} & a^{2} & a \\
    ab & ab & b & a^{3}b & a^{2}b & a & Id & a^{3} & a^{2} \\
    a^{2}b & a^{2}b & ab & b & a^{3}b & a^{2} & a & Id & a^{3} \\
    a^{3}b & a^{3}b & a^{2}b & ab & b & a^{3} & a^{2} & a & Id \\
  \end{array}
  \]
\end{vb}

\section{Ringen}
\label{sec:ringen}

\begin{de}
  De \term{nulring} $\boldsymbol{0}$ is de ring met \'e\'en enkel element zodat zowel $\{0\},+$ als $\{0\},\cdot$ een groep zijn met hetzelfde neutraal element.
  \[ \boldsymbol{0} = \{0\},+,\cdot \]
  \commutatief
\end{de}

\begin{vb}
  De gehele getallen, uitgeruist met de optelling en de vermenigvuldiging, vormen een commutatieve ring $\mathbb{Z},+,\cdot$.
  \commutatief
\end{vb}

\begin{vb}
  De rationale getallen, uitgerust met de optelling en de vermenigvuldiging, vormen velden.
  \[ \mathbb{Q} \subseteq \mathbb{R} \subseteq \mathbb{C} \]
  \commutatief
\end{vb}

\begin{vb}
  De reele veeltermen, uitgerust met de optelling en de vermenigvuldiging, vormen een commutatieve ring $\mathbb{R},+,\cdot$.\\
  \commutatief
\end{vb}

\begin{vb}
  De verzameling van re\"ele vierkante matrices $\mathbb{R}^{n\times n}$, uitgerust met de optelling en de matrixvermenigvuldiging vormen een ring. 
\end{vb}

\subsection{Deelringen}
\label{sec:deelringen}

\begin{vb}
  De diagonaalmatrices vormen een deelring van de vierkante matrices.
\end{vb}


\TODO{nog voorbelden op pagina 42 A}
\TODO{quaternionen $\mathbb{H}$ p 45 A}
\TODO{deelringen p 46 A}
\TODO{ringmorfismen p 46 A}
\TODO{breukenvelden p 49 A}


\subsection{Eenhedengroepen}
\label{sec:eenhedengroepen}

\begin{vb}
  De eenhedengroep $\mathbb{Z}^{\times},\cdot$ van $\mathbb{Z},+,\cdot$ is de groep $\{1,-1\},\cdot$.
  \[ \mathbb{Z}^{\times} = \{1,-1\}\]
\end{vb}

\begin{vb}
  De eenhedengroep $(\mathbb{R}^{n\times n})^{\times},\cdot$ van $\mathbb{R}^{n\times n},+,\cdot$ is de groep $GL_{n}(\mathbb{R}),\cdot$ van inverteerbare matrices.
  \[ (\mathbb{R}^{n\times n})^{\times} = GL_{n}(\mathbb{R}) \]
\end{vb}

\begin{vb}
  De eenhedengroep van $\mathbb{R}[x],+,\cdot$ en die van $\mathbb{R},+,\cdot$ zijn beide gelijk aan  $\mathbb{R}_{0},\cdot$.
\end{vb}


\end{document}
