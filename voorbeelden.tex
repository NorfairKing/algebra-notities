\documentclass[main.tex]{subfiles}
\begin{document}

\chapter{Voorbeelden}
\label{cha:voorbeelden}

\section{Groepen}
\label{sec:groepen}

\begin{de}
  De \term{symmetrische groep} van graad $n$: $S_{n},\circ$ is de groep van permutaties van $\{ 1,\dotsc,n \}$.
\end{de}

\begin{vb}
  $\mathcal{S}_{1},\circ$: De groep van permutaties van $\{ 1 \}$.\\
  \[
  \begin{array}{c|c}
    \circ & Id \\
    \hline
    Id & Id\\
  \end{array}
  \]
\commutatief  
\end{vb}

\begin{vb}
  $\mathcal{S}_{2},\circ$: De groep van permutaties van $\{ 1,2 \}$.\\
  \[
  \begin{array}{c|cc}
    \circ & Id & (12) \\
    \hline
    Id & Id & (12)\\
    (12) & (12) & Id
  \end{array}
  \]
\commutatief  
\end{vb}

\begin{vb}
  $\mathcal{S}_{3},\circ$: De groep van permutaties van $\{ 1,2,3 \}$.\\
  \[
  \begin{array}{c|cccccc}
    \circ & Id & (12) & (13) & (23) & (123) & (132) \\
    \hline
    Id & Id & (12) & (13) & (23) & (123) & (132)\\
    (12) & (12) & Id & (132) & (123) & (23) & (13)\\
    (13) & (13) & (123) & Id & (132) & (12) & (23)\\
    (23) & (23) & (132) & (123) & Id & (13) & (12)\\
    (123) & (123) & (13) & (23) & (12) & (132) & Id\\
    (132) & (132) & (23) & (12) & (13) & Id & (123)\\
  \end{array}
  \]
\end{vb}

\begin{vb}
  $\mathcal{S}_{4},\circ$: De groep van permutaties van $\{ 1,2,3,4 \}$.\\
  De Cayley-tabel kan eenvoudig uitgerekend worden.
  \[ 
  \mathcal{S}_{4} = 
  \begin{array}{l}
     \{\\
     Id,\\
     (12), (13), (14), (23), (24), (34),\\
     (123), (124), (132), (134), (142), (143), (234), (243)\\
     (12)(34), (13)(42), (14)(23),\\
     (1234), (1243), (1324), (1342), (1423), (1432)\\
     \}
  \end{array}
  \]
\end{vb}

\end{document}
