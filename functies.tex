\documentclass[main.tex]{subfiles}
\begin{document}

\chapter{Functies en afbeeldingen}
\label{cha:functies-en-afbeeldingen}

\begin{de}
  Een \emph{functie} $f$ van $A$ naar $B$: $f: A \rightarrow B$ is een relatie tussen $A$ en $B$ die 1-waardig is.
  \begin{enumerate}
  \item $f \in A \subseteq B$ ($f$ is een relatie van $A$ naar $B$.)
  \item $(x,y_{1}) \in f \wedge (x,y_{2}) \in f \Rightarrow y_{1} = y_{2}$ ($f$ is 1-waardig.)
  \end{enumerate}
\end{de}

\begin{de}
  De \emph{definitie van een functie} ziet er als volgt uit:
  \[ f: A \rightarrow B: a \mapsto b \]
  Hier noemen we $B$ het codomein van $f$.
  We lezen: ``$f$ is een functie van $A$ naar $B$ die $a$ afbeeldt op $b$.''
\end{de}

\begin{de}
  Wanneer er geen koppel $(x,y)$ in $f$ bestaat zeggen we dat de functie $f$ ongedefini\"eerd is in $x$.
\end{de}

\begin{de}
  Een \emph{afbeelding} $f$ van $A$ naar $B$: $f: A \rightarrow B$ is een functie die overal gedefini\"eerd is.
  \[ \forall x \in A,\ \exists y \in B: (x,y) \in f \]
\end{de}

\begin{de}
  Wanneer we over fucties spreken noteren we vaak $f(x) = y$ in plaats van $x f y$ of $(x,y) \in f$.
\end{de}

\begin{de}
  In $y = f(x)$ noemen we $x$ het argument en $y$ het beeld van $x$ onder $f$.
\end{de}

\begin{st}
  Zij $f$ en $g$ functies van $A$ naar $B$: $f: A \rightarrow B$, dan geldt:
  \[ \forall x \in A: f = g \Leftrightarrow f(x) = g(x) \]
  
  \begin{proof}
    Bewijs van een equivalentie.
    \begin{itemize}
    \item $\Rightarrow$\\
      Als de verzamelingen $f$ en $g$ gelijk zijn is het beeld van elke $x$ inderdaad hetzelfde onder $f$ als onder $g$.
    \item $\Leftarrow$\\
      Geldt er voor koppel $(x,y_{1}) \in f$ en $(x,y_{2})$ dat $y_{1}$ gelijk is aan $y_{2}$, dan moeten $f$ en $g$ wel dezelfde verzameling zij.
    \end{itemize}
  \end{proof}
\end{st}

\begin{de}
  Zij $f: A \rightarrow A$ een functie van $A$ naar zichzelf.
  We noemen $x$ een \emph{vast punt} van $f$ als $f$ $x$ op zichzelf afbeeldt.
  \[ f(x) = x \]
\end{de}

\begin{de}
  Zij $f: A \rightarrow A$ een functie van $A$ naar zichzelf.
  We noemen een deelverzameling $X$ van $A$ een \emph{invariante of stabiele deelverzameling} voor $f$ als $f$ alle elementen van $X$ op zichzelf afbeeldt.
  \[ f(X) \subseteq X \]
\end{de}

\section{Jecties}
\label{sec:jecties}

\begin{de}
  Een functie $f: A \rightarrow B$ is \emph{injectief} als ze voor verschillende argumenten nooit hetzelfde beeld geeft.
  \[ \forall x_{1},x_{2} \in A:\ f(x_{1}) = f(x_{2}) \Rightarrow x_{1}= x_{2} \]
\clarify{ is dit enkel gedefinieerd voor afbeeldingen of ook voor functies? }
\end{de}

\begin{de}
  Een functie $f: A \rightarrow B$ is \emph{surjectief} als elk element in het codomein $B$ een beeld is van een element uit $A$.
  \[ \forall y \in B:\ \exists x \in A:\ y = f(x) \]
  \clarify{ is dit enkel gedefinieerd voor afbeeldingen of ook voor functies? }
\end{de}

\begin{de}
  Een functie $f: A \rightarrow B$ is \emph{bijectief} als het een injectie en een surjectie is.
\end{de}

\begin{de}
  Een functie $f: A \rightarrow A$ van een verzameling naar zichzelf noemen we een transformatie.
  \clarify{ functie of afbeelding ? }
\end{de}

\begin{de}
  Een bijectieve functie van een eindige verzameling naar zichzelf noemen we een permutatie.
  \clarify{ functie of afbeelding ? }
\end{de}

\begin{de}
  Een bijectieve functie van een eindige verzameling $A$ naar een eindige verzameling $B$ noemen we een substitutie. 
  \clarify{ functie of afbeelding ? }
\end{de}

\begin{de}
  De \emph{identieke transformatie} $id_{A}$ van een verzameling $A$ is de afbeelding die elk element op zichzelf afbeeldt.
  \[ id_{A}: A \rightarrow A \text{ met } \forall x \in A: f(x) = x \]
\end{de}

\begin{de}
  De \emph{inlassing} $i_{AB}$ van $A$ in $B$ met $A \subseteq B$ beeldt elk element ook op zichzelf af, maar heeft een ander codomein.
  \[ i_{AB}: A \rightarrow B \text{ met } \forall x \in A: f(x) = x \]
\end{de}

\begin{de}
  Een \emph{constante functie} $f$ beeldt elk argument af op \'e\'enzelfde beeld $b$.
  \[ f: A \rightarrow B \text{ met } \forall x \in dom f: f(x) = b \]
  \clarify{ functie of afbeelding ? }
\end{de}

\end{document}
