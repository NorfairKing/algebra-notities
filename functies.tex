\documentclass[main.tex]{subfiles}
\begin{document}

\chapter{Functies en afbeeldingen}
\label{cha:functies-en-afbeeldingen}

\begin{de}
  Een \emph{functie} $f$ van $A$ naar $B$: $f: A \rightarrow B$ is een relatie tussen $A$ en $B$ die 1-waardig is.
  \begin{enumerate}
  \item $f \in A \subseteq B$ ($f$ is een relatie van $A$ naar $B$.)
  \item $(x,y_{1}) \in f \wedge (x,y_{2}) \in f \Rightarrow y_{1} = y_{2}$ ($f$ is 1-waardig.)
  \end{enumerate}
\end{de}

\begin{de}
  Wanneer er geen koppel $(x,y)$ in $f$ bestaat zeggen we dat de functie $f$ ongedefini\"eerd is in $x$.
\end{de}

\begin{de}
  Een \emph{afbeelding} $f$ van $A$ naar $B$: $f: A \rightarrow B$ is een functie die overal gedefini\"eerd is.
  \[ \forall x \in A,\ \exists y \in B: (x,y) \in f \]
\end{de}

\begin{de}
  Wanneer we over fucties spreken noteren we vaak $f(x) = y$ in plaats van $x f y$ of $(x,y) \in f$.
\end{de}

\begin{de}
  In $y = f(x)$ noemen we $x$ het argument en $y$ het beeld van $x$ onder $f$.
\end{de}

\begin{st}
  Zij $f$ een functie van $A$ naar $B$: $f: A \rightarrow B$, dan geldt:
  \[ \forall x \in A\]
  \TODO{bewijs}
\end{st}

\begin{de}
  Zij $f: A \rightarrow A$ een functie van $A$ naar zichzelf.
  We noemen $x$ een \emph{vast punt} van $f$ als $f$ $x$ op zichzelf afbeeldt.
  \[ f(x) = x \]
\end{de}

\begin{de}
  Zij $f: A \rightarrow A$ een functie van $A$ naar zichzelf.
  We noemen een deelverzameling $X$ van $A$ een \emph{invariante of stabiele deelverzameling} voor $f$ als $f$ alle elementen van $X$ op zichzelf afbeeldt.
  \[ f(X) \subseteq X \]
\end{de}

\section{Jecties}
\label{sec:jecties}

\begin{de}
  Een afbeelding $f: A \rightarrow B$ is $injectief$ als ze voor verschillende argumenten nooit hetzelfde beeld geeft.
  \[ f(x_{1}) = f(x_{2}) \Rightarrow x_{1}= x_{2} \]
\end{de}

\end{document}
