\documentclass[main.tex]{subfiles}
\begin{document}

\chapter{Functies en afbeeldingen}
\label{cha:functies-en-afbeeldingen}

\begin{de}
  Een (partiele) \emph{functie} $f$ van $A$ naar $B$: $f: A \rightarrow B$ is een relatie tussen $A$ en $B$ die 1-waardig is.
  \begin{enumerate}
  \item $f \in A \times B$ ($f$ is een relatie van $A$ naar $B$.)
  \item $(x,y_{1}) \in f \wedge (x,y_{2}) \in f \Rightarrow y_{1} = y_{2}$ ($f$ is 1-waardig.)
  \end{enumerate}
  Vaak worden $A$ en $B$ opgenomen in de definitie van een functie om over surjecties te kunnen spreken. Een functie $f: A \rightarrow B$ is dan het drietal $(f,A,B)$. Soms wordt een functie en een afbeelding als hetzelfde gezien, waarbij men dan bedoelt wat hier een afbeelding wordt genoemd. (Zie later.) In dat geval valt $A$ samen met het domein van de functie.
\end{de}

\begin{de}
  De \emph{definitie van een functie} ziet er als volgt uit:
  \[ f: A \rightarrow B: a \mapsto b \]
  Hier noemen we $B$ het codomein van $f$ en er geldt $dom f \subseteq A$.
  We lezen: ``$f$ is een functie van $A$ naar $B$ die $a$ afbeeldt op $b$.''
\end{de}

\begin{de}
  Wanneer er geen koppel $(x,y)$ in $f$ bestaat zeggen we dat de functie $f$ \emph{ongedefinieerd} is in $x$.
\end{de}

\begin{de}
  Wanneer we over functies spreken gebruiken we soms de volgende afkorting. Zij $f$ een functie $f: A \rightarrow B$ en $C \subseteq A$ een verzameling.
  \[ f(C) = \{ f(c)\ |\ c \in C \} \] 
\end{de}

\begin{de}
  Zij $f$ een functie: $f: A \rightarrow B$. 
  In $y = f(x)$ noemen we $x$ het \emph{argument} en $y$ het \emph{beeld} van $x$ onder $f$.
\end{de}

\begin{de}
  Het \emph{beeld $f(A)$ van een verzameling} $A$ onder een functie $f$ is de verzameling van alle beelden van de elementen van $A$.
  \[ f(A) = \{ y \in Y \ |\ \exists x \in A:\ f(x) = y \} \]
\end{de}

\begin{de}
  Het \emph{invers beeld $f^{-1}(A)$ van een verzameling} $B$ onder een functie $f$ is de verzameling van alle elementen uit $X$ die op een element in $B$ afgebeeldt worden.
  \[ f^{-1}(B) = \{ x \in X \ |\ f(x) \in B \} \]
\end{de}

\begin{de}
  Een \emph{afbeelding} (of volledige functie) $f$ van $A$ naar $B$: $f: A \rightarrow B$ is een functie die overal gedefini\"eerd is.
  \[ \forall x \in A,\ \exists y \in B: (x,y) \in f \]
\end{de} 
 
\begin{de}
  De \emph{definitie van een afbeelding} ziet er als volgt uit:
  \[ f: A \rightarrow B: a \mapsto b \]
  Hier noemen we $B$ het codomein van $f$ en $A$ het domein van $f$.
  We lezen: ``$f$ is een functie afbeelding van $A$ op $B$ die $a$ afbeeldt op $b$.''
\end{de} 
 
\begin{de}
  Wanneer we over afbeeldingen spreken noteren we vaak $f(x) = y$ in plaats van $x f y$ of $(x,y) \in f$.
\end{de}


\begin{st}
  Zij $f$ en $g$ functies van $A$ naar $B$: $f: A \rightarrow B$, dan geldt:
  \[ \forall x \in A: f = g \Leftrightarrow f(x) = g(x) \]  
  \begin{proof}
    Bewijs van een equivalentie.
    \begin{itemize}
    \item $\Rightarrow$\\
      Als de verzamelingen $f$ en $g$ gelijk zijn is het beeld van elke $x$ inderdaad hetzelfde onder $f$ als onder $g$.
    \item $\Leftarrow$\\
      Geldt er voor koppel $(x,y_{1}) \in f$ en $(x,y_{2})$ dat $y_{1}$ gelijk is aan $y_{2}$, dan moeten $f$ en $g$ wel dezelfde verzameling zij.
    \end{itemize}
  \end{proof}
\end{st}

\begin{de}
  Zij $f: A \rightarrow A$ een functie van $A$ naar zichzelf.
  We noemen $x$ een \emph{vast punt} van $f$ als $f$ $x$ op zichzelf afbeeldt.
  \[ f(x) = x \]
\end{de}

\begin{de}
  Zij $f: A \rightarrow A$ een functie van $A$ naar zichzelf.
  We noemen een deelverzameling $X$ van $A$ een \emph{invariante of stabiele deelverzameling} voor $f$ als $f$ $X$ op een deelverzameling van zichzelf afbeeldt.
  \[ f(X) \subseteq X \]
\end{de}

\section{Jecties}
\label{sec:jecties}

\begin{de}
  \label{de:afbeelding-injectief}
  Een afbeelding $f: A \rightarrow B$ is \emph{injectief} (een injectie) als ze voor verschillende argumenten nooit hetzelfde beeld geeft.
  \[ \forall x_{1},x_{2} \in A:\ f(x_{1}) = f(x_{2}) \Rightarrow x_{1}= x_{2} \]
\end{de}

\begin{de}
  \label{de:afbeelding-surjectief}
  Een afbeelding $f: A \rightarrow B$ is \emph{surjectief} (een surjectie) als elk element in het codomein $B$ een beeld is van een element uit $A$.
  \[ \forall y \in B:\ \exists x \in A:\ y = f(x) \]
\end{de}

\begin{de}
  \label{de:afbeelding-bijectief}
  Een afbeelding $f: A \rightarrow B$ is \emph{bijectief} (een bijectie) als het een injectie en een surjectie is.
\end{de}

\begin{de}
  We noemen twee verzamelingen $X$ en $Y$ equipotent als er een bijectie $f: X \rightarrow Y$ bestaat.
\end{de}

\begin{de}
  Een afbeelding $f: A \rightarrow A$ van een verzameling op zichzelf noemen we een transformatie.
\end{de}

\begin{de}
  Een bijectieve afbeelding van een eindige verzameling naar zichzelf noemen we een permutatie.
\end{de}

\begin{de}
  Een bijectieve functie van een eindige verzameling $A$ naar een eindige verzameling $B$ noemen we een substitutie. 
\end{de}

\begin{de}
  De \emph{identieke transformatie} $id_{A}$ van een verzameling $A$ is de (bijectieve) afbeelding die elk element op zichzelf afbeeldt.
  \[ id_{A}: A \rightarrow A \text{ met } \forall x \in A: f(x) = x \]
\end{de}

\begin{de}
  De \emph{inlassing} $i_{AB}$ van $A$ in $B$ met $A \subseteq B$ beeldt elk element ook op zichzelf af, maar heeft een ander codomein.
  \[ i_{AB}: A \rightarrow B \text{ met } \forall x \in A: f(x) = x \]
\end{de}

\begin{de}
  Een \emph{constante afbeelding} $f$ beeldt elk argument af op \'e\'enzelfde beeld $b$.
  \[ f: A \rightarrow B \text{ met } \forall x \in dom f: f(x) = b \]
\end{de}

\begin{de}
  De karacteristieke afbeelding van $A$ in $C$ is gedefinieerd voor $A \subseteq C$ in $B = \{0,1\}$ als volgt:
  \[
  E_{A}: C \rightarrow B:
  \left \lbrace
    \begin{array}{cc}
      x \mapsto 1 &\text{ als } x \in A\\
      x \mapsto 0 &\text{ als } x \not\in A\\
    \end{array}
  \right \rbrace
  \]
\end{de}

\begin{de}
  De \emph{beperking} $f|_{C}$ van $f$ tot $C$ is gedefinieerd voor $f: A \rightarrow B$ en $C \in A$ als volgt:
  \[ f|_{C}: C \rightarrow B: x \mapsto f(x), \forall x \in C\]
  Een functie, beperkt tot haar domein is een afbeelding.
\end{de}

\begin{st}
  Zij $f: A \rightarrow B$ en $g: B \rightarrow C$ twee afbeeldingen, dan is de samenstelling $g \circ f$ ervan ook een afbeelding.

  \begin{proof}
    Inderdaad, $z = (g \circ f)(x) = g(f(x))$ en zowel $f(x)$ en $g(f(x))$ zijn goed gedefinieerd omdat $f$ en $g$ afbeeldingen zijn. 
  \end{proof}
\end{st}

\begin{st}
  Zij $f$ en $g$ injecties, dan is hun samenstelling $g \circ f$ ook een injectie.
\TODO{bewijs}
\end{st}

\begin{st}
  Zij $f$ en $g$ surjectie, dan is hun samenstelling $g \circ f$ ook een surjectie.
\TODO{bewijs}
\end{st}

\begin{st}
  Zij $f$ en $g$ bijectie, dan is hun samenstelling $g \circ f$ ook een bijectie.
\TODO{bewijs}
\end{st}

\begin{de}
  Een functie $f: A \rightarrow B$ is \emph{inverteerbaar} als en slechts als $f^{-1}: B \rightarrow A$ ook een functie is. $f^{-1}$ noemen we dan de inverse functie.
\end{de}

\begin{de}
  Een afbeelding $f: A \rightarrow B$ is \emph{inverteerbaar} als en slechts als $f^{-1}: B \rightarrow A$ ook een afbeelding is. $f^{-1}$ noemen we dan de inverse afbeelding.
\end{de}

\begin{st}
  Een afbeelding is inverteerbaar als en slechts als ze bijectief is.
\TODO{bewijs}
\end{st}


\section{Kardinaliteit}
\label{sec:kardinaliteit}

\begin{de}
  Definieer $\mathbb{E}_{n}$ als de verzameling met de $n$ eerste elementen uit $\mathbb{N}$.
  \[ \mathbb{E}_{n} = \{ i \in \mathbb{N}\ |\ 1 \le i \le n \}\] 
\end{de}

\begin{de}
  Zij $X$ een verzameling. We zeggen dat $n$ de \emph{kardinaliteit} is van $X$ als er een bijectie bestaat tussen $X$ en $\mathbb{E}_{n}$.
  \[ |X| = \#X = n \]
\end{de}

\begin{de}
  Een verzameling is \emph{aftelbaar oneindig} als $X$ equipotent is met $\mathbb{N}_{0}$.
  \[ |X| = \aleph_{0} \]
\end{de}

\begin{de}
  We noemen een verzameling \emph{aftelbaar} als ze eindig of aftelbaar oneindig is. 
\end{de}

\begin{de}
  Een verzameling is \emph{overaftelbaar} als ze nie aftelbaar is.
\end{de}

\end{document}
