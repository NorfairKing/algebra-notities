\documentclass[main.tex]{subfiles}
\begin{document}

\chapter{Algoritmes}
\label{cha:algoritmes}

\section{Jordanmatrix berekenen}

\subsection*{Abstract}

\subsubsection*{Vraag}
Gegeven is de lineaire afbeelding $\mathcal{A}:\ \mathbb{C}^{n} \rightarrow \mathbb{C}^{n}$ met $A$ als matrix ten opzichte van de standaarbasis van $\mathbb{C}$.
Bepaal de jordanvorm $J$ van $A$ en geef de $P$ zodat $J=P^{-1}AP$ geldt.
Geef ook de karakteristieke veelterm en de minimale veelterm van $A$.
\subsubsection*{Antwoord}
\begin{itemize}
\item Bereken de karakteristieke veelterm van $A$.
  \[ f_{A}(X) = \prod_{i=1}^{q}(X-c_{i})^{n_{i}} \]
\item Componenten van de vectorruimten bepalen:
  \[ V_{i} = Ker(A-c_{i}I)^{p_{i}} \]
  We bepalen de $p_{i}$ als volgt:
  Bepaal de dimensie van $A-c_{i}I$, $(A-c_{i}I)^{2}$,...
  \[ dim(Ker(A-c_{i}I)^{j}) = n-dim((A-c_{i}I)^{j}) \]
  Ga door met $j$ te verhogen tot $dim(Ker(A-c_{i}I)^{j})$ $dim(V_{i})$ is.
  $p_{i}$ is dan $j$.
\item Diagram voor elke $V_{i}$
  Definieer nu $d_{i}$ als volgt:
  \[ d_{1} = dim(Ker(A-c_{i}I)) \text{ en } d_{j} = dim(Ker(A-c_{i}I)^{j}-dim(Ker(A-c_{i}I)^{j-1} \]
  Maak voor elke $V_{i}$ een diagram van rijen `doosjes' met op de $j$-de rij $d_{j}$ doosjes.
  Vul dan het diagram in als volgt: (begin onderaan links)
  \begin{itemize}
  \item Vul de dozen in rij $k$ op met lineair onafhankelijke vectoren uit $Ker(A-c_{i}I)^{k}\setminus Ker(A-c_{i}I)^{k-1}$.
  \item Vul de doos boven een gevulde doos met vector $v$ op met de vector $(A-c_{i}I)v$.
  \item Als de huidige doos de onderste uit een rij is, vul deze dan in met een vector die onafhankelijk is van de vectoren links ervan.
  \end{itemize}
\item Maak een matrix $P$ door de collecties vectoren als de kolommen in een matrix te zetten.
\item Maak de Jordanmatrix door de eigenwaarden op de diagonaal de zetten, en een $1$ er links van als de overeenkomstige vector in een doos zit waar nog een vector onder staat.
\end{itemize}

\subsection*{Voorbeeld}

\subsubsection*{Vraag}
Gegeven is de lineaire afbeelding $\mathcal{A}:\ \mathbb{C}^{5} \rightarrow \mathbb{C}^{5}$ met $A$ als matrix ten opzichte van de standaarbasis van $\mathbb{C}$.
\[
\begin{pmatrix}
  1 & 1 & -2 & -3 & 0\\
  0 & 3 & 0 & -2 & 0\\
  0 & -1 & 3 & 3 & 0\\
  0 & 0 & 0 & 1 & 0\\
  -1 & 2 & -1 & -2 & 1
\end{pmatrix}
\]
Bepaal de jordanvorm $J$ van $A$ en geef de $P$ zodat $J=P^{-1}AP$ geldt.
Geef ook de karakteristieke veelterm en de minimale veelterm van $A$.
\subsubsection*{Antwoord}
\begin{itemize}
\item 
  \[
  \begin{array}{rll}
    \begin{vmatrix}
      X-1 & -1 & 2 & 3 & 0\\
      0 & X-3 & 0 & 2 & 0\\
      0 & 1 & X-3 & -3 & 0\\
      0 & 0 & 0 & X-1 & 0\\
      1 & -2 & 1 & 2 & X-1
    \end{vmatrix}
    &= 
    (X-1) 
    \begin{vmatrix}
      X-1 & -1 & 2 & 3 \\
      0 & X-3 & 0 & 2 \\
      0 & 1 & X-3 & -3 \\
      0 & 0 & 0 & X-1 
    \end{vmatrix} &\\
    &=
    (X-1)^{2}
    \begin{vmatrix}
      X-1 & -1 & 2 \\
      0 & X-3 & 0\\
      0 & 1 & X-3
    \end{vmatrix} &\\
    &= 
    (X-1)^{3}
    \begin{vmatrix}
      X-3 & 0\\
      1 & X-3
    \end{vmatrix} &\\
    &=
    (X-1)^{3}(X-3)^{2}
  \end{array}
  \]
\item De vectorruimte valt uit elkaar in twee componenten:
  \[ \mathbb{C}^{5} = Ker(A-I)^{p_{1}} \oplus Ker(A-3I )^{p_{2}} \]
  De dimensies zijn als volgt:
  \[ dim(V_{1})= 3 \text{ en } dim(V_{2}) = 2\]
  \begin{itemize}
  \item $V_{1}$
    \begin{itemize}
    \item
      \[ dim(A-I) = dim
      \begin{pmatrix}
        0 & 1 & -2 & -3 & 0\\
        0 & 1 & 0 & -2 & 0\\
        0 & -1 & 2 & 3 & 0\\
        0 & 0 & 0 & 0 & 0\\
        -1 & 2 & -1 & -2 & 0
      \end{pmatrix}
      = 3 \Rightarrow dim(Ker(A-I)) = 2
      \]
    \item
      \[ dim(A-I)^{2} = dim
      \begin{pmatrix}
        0 & 4 & -4 & -8 & 0\\
        0 & 4 & 0 & -4 & 0\\
        0 & -4 & 4 & 0 & 0\\
        0 & 0 & 0 & 0 & 0\\
        0 & 4 & 0 & -4 & 0
      \end{pmatrix}
      = 2 \Rightarrow dim(Ker(A-I)^{2}) = 3 \overset{!}{=} dim(V_{1})
      \Rightarrow p_{1} = 2
      \]
    \end{itemize}
  \item $V_{2}$
    \begin{itemize}
    \item
      \[
      dim(A-3I) = dim
      \begin{pmatrix}
        -2 & 1 & -2 & -3 & 0\\
        0 & 0 & 0  & -2 & 0\\
        0 & -1 & 0 & 3 & 0\\
        0 & 0 & 0 & -2 & 0\\
        -1 & 2 & -1 & -2 & -2
      \end{pmatrix}
      =  4 \Rightarrow dim(Ker(A-3I)) = 1
      \]
    \item 
      \[ 
      dim((A-3I)^{2}) = dim
      \begin{pmatrix}
        4 & 0 & 4 & 4 & 0\\
        0 & 0 & 0 & 4 & 0\\
        0 & 0 & 0 & -4& 0\\
        0 & 0 & 0 & 4 & 0\\
        4 & -4 & 4 & 4 & 4
      \end{pmatrix}
      = 3 \Rightarrow dim(Ker(A-3I)) = 2 \overset{!}{=} dim(V_{2}) \Rightarrow p_{2} = 2
      \]
    \end{itemize}
  \end{itemize}
\item diagram
  \begin{itemize}
  \item $V_{1}$\\
    $d_{1} = 2$ en $d_{2} = 1$, dus het diagram ziet er al volgt uit:
    \[
    \begin{array}{cc}
      \boxed{v_{2}} & \boxed{v_{3}}\\
      \boxed{v_{1}}
    \end{array}
    \] 
    $v_{1}$ moet in $Ker(A-I)^{2}\setminus Ker(A-I)$ zitten.
    Kies bijvoorbeeld $v_{1} = (1,0,0,0,0)$.
    $v_{2}$ staat boven $v_{1}$, dus $v_{2}$ moet $(A-I)v_{1}= (0,0,0,0,-1)$ zijn.
    Kies nu een vector $v_{3}$, lineair onafhankelijk van $v_{2}$, uit $Ker(A-I)$.
    Bijvoorbeeld $v_{3} = (1,1,-1,1,0)$.
    
  \item $V_{2}$\\
    $d_{1} = 1$ en $d_{2} = 1$, dus het diagram ziet er als volgt uit:
    \[
    \begin{array}{c}
      \boxed{v_{2}}\\
      \boxed{v_{1}}
    \end{array}
    \]
    $v_{1}$ moet in $Ker(A-3I)^{2}\setminus Ker(A-3I)$ zitten.
    Kies bijvoorbeeld $v_{1}= (0,1,0,0,1)$.
    $v_{2}$ staat boven $v_{1}$, dus $v_{2}$ moet $(A-3I)v_{1}= (1,0,-1,0,0)$ zijn.

  \end{itemize}
\item 
  \[ P = 
  \left(
    \begin{array}{ccc|cc}
      1 & 0 & 1 & 0 & 1\\
      0 & 0 & 1 & 1 & 0\\
      0 & 0 & -1& 0 &-1\\
      0 & 0 & 1 & 0 & 0\\
      0 & -1& 0 & 1 & 0\\
    \end{array}
  \right)
  \]
\item 
  \[ J =
  \left(
    \begin{array}{ccc|cc}
      1 & 0 & 0 & 0 & 0\\
      1 & 1 & 0 & 0 & 0\\
      0 & 0 & 1 & 0 & 0\\\hline
      0 & 0 & 0 & 3 & 0\\
      0 & 0 & 0 & 1 & 3\\
    \end{array}
  \right)
  \]
\end{itemize}

\newpage
\section{Veelterm over veld ontbinden in priemfactoren}

\subsection*{Abstract}
\subsubsection*{Vraag}
Ontbind een veelterm $g(X)$ in priemfactoren over een veld $\mathbb{Z}_{p^{n}}$.
(De orde van het veld zal relatief klein zijn.)
\subsubsection*{Antwoord}
\begin{itemize}
\item Bepaal de nulpunten van $g(X)$.
\item Deel $g(X)$ achtereenvolgens door $(X-a)$ voor elk nulpunt $a$.
\item Herhaal tot $g(X)$ helemaal ontbonden is. (In deze stap kunnen er meervoudige nulpunten tevoorschijn komen.)\end{itemize}

\subsection*{Voorbeeld}
\subsubsection*{Vraag}
Ontbind $g(X) = X^{4} + 2X^{3} + 2X^{2} + 2X + 1$ in priemfactoren over $\mathbb{Z}_{5}$.
\subsubsection*{Antwoord}
\begin{itemize}
\item Bepaal de nulpunten:
  \[
  \begin{array}{|c|c|}
    \hline
    a & g(a)\\
    \hline
    \hline
    0 & 1\\ \hline
    1 & 3\\ \hline
    2 & 0\\ \hline
    3 & 0\\ \hline
    4 & 0\\ \hline
  \end{array}
  \]
\item We delen $g(X)$ achtereenvolgens door $(X-2)$, $(X-3)$ en $(X-4)$.
  Merk echter eerst op dat $(X-2)$ in $\mathbb{Z}_{5}$ eigenlijk $(X+3)$ is.
  Zo zijn $(X-3)$ en $(X-4)$ eigenlijk $(X+2)$ en $(X+1)$.
  \begin{itemize}
  \item 
    \[
    \begin{array}{ccccc|c}
      X^{4} &+2X^{3} &+2X^{2} &+2X    &+1     & X+3\\\hline
      X^{4} &+3X^{3} &\vdots &\vdots &\vdots & X^{3}+4X^{2}+0X+2\\\cline{1-2}
      &+4X^{3} & +2X^{2} &\vdots &\vdots &\\
      &+4X^{3} & +2X^{2} &\vdots &\vdots &\\\cline{2-3}
      &       & +0X^{2} & 2X    &\vdots &\\
      &       & +0X^{2} & 0X    &\vdots &\\\cline{3-4}
      &       &        & 2X    &+1     &\\
      &       &        & 2X    &+1     &\\\cline{4-5}
      &       &        &       & 0     &
    \end{array}
    \]
    Er blijft $X^{3}+4X^{2}+0X+2$ over.
  \item 
    \[
    \begin{array}{cccc|c}
      X^{3} &+4X^{2} &+0X    &+2       & X+2\\\hline
      X^{3} &+2X^{2} &\vdots &\vdots   & X^{2}+2X+1\\\cline{1-2}
           &+2X^{2} &+0X    &\vdots   &\\
           &+2X^{2} &+4X    &\vdots   &\\\cline{2-3}
           &       &+X     &+2       &\\
           &       &+X     &+2       &\\\cline{3-4}
           &       &       &0        &\\
    \end{array}
    \]
    Er blijft nog $X^{2}+2X+1$ over.
  \item 
    \[
    \begin{array}{ccc|c}
      X^{2} &+2X &+1     & X+1\\\hline
      X^{2} &+X  &\vdots & X+1\\\cline{1-2}
           & X  &+1     &\\
           & X  &+1     &\\\cline{2-3}
           &    &0      &\\
    \end{array}
    \]
    Hier blijft nog $X+1$ over.
  \end{itemize}
\item Het resultaat is $X^{4} + 2X^{3} + 2X^{2} + 2X + 1 = (X+1)^{2}(X+2)(X+3)$.
\end{itemize}


\end{document}
