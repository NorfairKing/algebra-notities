\documentclass[main.tex]{subfiles}
\begin{document}

\chapter{Algoritmes}
\label{cha:algoritmes}

\section{Jordanmatrix berekenen}

\subsection*{Abstract}

\subsubsection*{Vraag}
Gegeven is de lineaire afbeelding $\mathcal{A}:\ \mathbb{C}^{n} \rightarrow \mathbb{C}^{n}$ met $A$ als matrix ten opzichte van de standaarbasis van $\mathbb{C}$.
Bepaal de jordanvorm $J$ van $A$ en geef de $P$ zodat $J=P^{-1}AP$ geldt.
Geef ook de karakteristieke veelterm en de minimale veelterm van $A$.
\subsubsection*{Antwoord}
\begin{itemize}
\item Bereken de karakteristieke veelterm van $A$.
  \[ f_{A}(X) = \prod_{i=1}^{q}(X-c_{i})^{n_{i}} \]
\item Componenten van de vectorruimten bepalen:
  \[ V_{i} = Ker(A-c_{i}I)^{p_{i}} \]
  We bepalen de $p_{i}$ als volgt:
  Bepaal de dimensie van $A-c_{i}I$, $(A-c_{i}I)^{2}$,...
  \[ dim(Ker(A-c_{i}I)^{j}) = n-dim((A-c_{i}I)^{j}) \]
  Ga door met $j$ te verhogen tot $dim(Ker(A-c_{i}I)^{j})$ $dim(V_{i})$ is.
  $p_{i}$ is dan $j$.
\item Diagram voor elke $V_{i}$
  Definieer nu $d_{i}$ als volgt:
  \[ d_{1} = dim(Ker(A-c_{i}I)) \text{ en } d_{j} = dim(Ker(A-c_{i}I)^{j}-dim(Ker(A-c_{i}I)^{j-1} \]
  Maak voor elke $V_{i}$ een diagram van rijen `doosjes' met op de $j$-de rij $d_{j}$ doosjes.
  Vul dan het diagram in als volgt: (begin onderaan links)
  \begin{itemize}
  \item Vul de dozen in rij $k$ op met lineair onafhankelijke vectoren uit $Ker(A-c_{i}I)^{k}\setminus Ker(A-c_{i}I)^{k-1}$.
  \item Vul de doos boven een gevulde doos met vector $v$ op met de vector $(A-c_{i}I)v$.
  \item Als de huidige doos de onderste uit een rij is, vul deze dan in met een vector die onafhankelijk is van de vectoren links ervan.
  \end{itemize}
\item Maak een matrix $P$ door de collecties vectoren als de kolommen in een matrix te zetten.
\item Maak de Jordanmatrix door de eigenwaarden op de diagonaal de zetten, en een $1$ er links van als de overeenkomstige vector in een doos zit waar nog een vector onder staat.
\end{itemize}

\subsection*{Voorbeeld}

\subsubsection*{Vraag}
Gegeven is de lineaire afbeelding $\mathcal{A}:\ \mathbb{C}^{5} \rightarrow \mathbb{C}^{5}$ met $A$ als matrix ten opzichte van de standaarbasis van $\mathbb{C}$.
\[
\begin{pmatrix}
  1 & 1 & -2 & -3 & 0\\
  0 & 3 & 0 & -2 & 0\\
  0 & -1 & 3 & 3 & 0\\
  0 & 0 & 0 & 1 & 0\\
  -1 & 2 & -1 & -2 & 1
\end{pmatrix}
\]
Bepaal de jordanvorm $J$ van $A$ en geef de $P$ zodat $J=P^{-1}AP$ geldt.
Geef ook de karakteristieke veelterm en de minimale veelterm van $A$.
\subsubsection*{Antwoord}
\begin{itemize}
\item 
  \[
  \begin{array}{rll}
    \begin{vmatrix}
      X-1 & -1 & 2 & 3 & 0\\
      0 & X-3 & 0 & 2 & 0\\
      0 & 1 & X-3 & -3 & 0\\
      0 & 0 & 0 & X-1 & 0\\
      1 & -2 & 1 & 2 & X-1
    \end{vmatrix}
    &= 
    (X-1) 
    \begin{vmatrix}
      X-1 & -1 & 2 & 3 \\
      0 & X-3 & 0 & 2 \\
      0 & 1 & X-3 & -3 \\
      0 & 0 & 0 & X-1 
    \end{vmatrix} &\\
    &=
    (X-1)^{2}
    \begin{vmatrix}
      X-1 & -1 & 2 \\
      0 & X-3 & 0\\
      0 & 1 & X-3
    \end{vmatrix} &\\
    &= 
    (X-1)^{3}
    \begin{vmatrix}
      X-3 & 0\\
      1 & X-3
    \end{vmatrix} &\\
    &=
    (X-1)^{3}(X-3)^{2}
  \end{array}
  \]
\item De vectorruimte valt uit elkaar in twee componenten:
  \[ \mathbb{C}^{5} = Ker(A-I)^{p_{1}} \oplus Ker(A-3I )^{p_{2}} \]
  De dimensies zijn als volgt:
  \[ dim(V_{1})= 3 \text{ en } dim(V_{2}) = 2\]
  \begin{itemize}
  \item $V_{1}$
    \begin{itemize}
    \item
      \[ dim(A-I) = dim
      \begin{pmatrix}
        0 & 1 & -2 & -3 & 0\\
        0 & 1 & 0 & -2 & 0\\
        0 & -1 & 2 & 3 & 0\\
        0 & 0 & 0 & 0 & 0\\
        -1 & 2 & -1 & -2 & 0
      \end{pmatrix}
      = 3 \Rightarrow dim(Ker(A-I)) = 2
      \]
    \item
      \[ dim(A-I)^{2} = dim
      \begin{pmatrix}
        0 & 4 & -4 & -8 & 0\\
        0 & 4 & 0 & -4 & 0\\
        0 & -4 & 4 & 0 & 0\\
        0 & 0 & 0 & 0 & 0\\
        0 & 4 & 0 & -4 & 0
      \end{pmatrix}
      = 2 \Rightarrow dim(Ker(A-I)^{2}) = 3 \overset{!}{=} dim(V_{1})
      \Rightarrow p_{1} = 2
      \]
    \end{itemize}
  \item $V_{2}$
    \begin{itemize}
    \item
      \[
      dim(A-3I) = dim
      \begin{vmatrix}
        -2 & 1 & -2 & -3 & 0\\
        0 & 0 & 0  & -2 & 0\\
        0 & -1 & 0 & 3 & 0\\
        0 & 0 & 0 & -2 & 0\\
        -1 & 2 & -1 & -2 & -2
      \end{vmatrix}
      =  4 \Rightarrow dim(Ker(A-3I)) = 1
      \]
    \item 
      \[ 
      dim((A-3I)^{2}) = dim
      \begin{vmatrix}
        4 & 0 & 4 & 4 & 0\\
        0 & 0 & 0 & 4 & 0\\
        0 & 0 & 0 & -4& 0\\
        0 & 0 & 0 & 4 & 0\\
        4 & -4 & 4 & 4 & 4
      \end{vmatrix}
      = 3 \Rightarrow dim(Ker(A-3I)) = 2 \overset{!}{=} dim(V_{2}) \Rightarrow p_{2} = 2
      \]
    \end{itemize}
  \end{itemize}
\item diagram
  \begin{itemize}
  \item $V_{1}$\\
    $d_{1} = 2$ en $d_{2} = 1$, dus het diagram ziet er al volgt uit:
    \[
    \begin{array}{cc}
      \boxed{v_{2}} & \boxed{v_{3}}\\
      \boxed{v_{1}}
    \end{array}
    \] 
    $v_{1}$ moet in $Ker(A-I)^{2}\setminus Ker(A-I)$ zitten.
    Kies bijvoorbeeld $v_{1} = (1,0,0,0,0)$.
    $v_{2}$ staat boven $v_{1}$, dus $v_{2}$ moet $(A-I)v_{1}= (0,0,0,0,-1)$ zijn.
    Kies nu een vector $v_{3}$, lineair onafhankelijk van $v_{2}$, uit $Ker(A-I)$.
    Bijvoorbeeld $v_{3} = (1,1,-1,1,0)$.
    
  \item $V_{2}$\\
    $d_{1} = 1$ en $d_{2} = 1$, dus het diagram ziet er als volgt uit:
    \[
    \begin{array}{c}
      \boxed{v_{2}}\\
      \boxed{v_{1}}
    \end{array}
    \]
    $v_{1}$ moet in $Ker(A-3I)^{2}\setminus Ker(A-3I)$ zitten.
    Kies bijvoorbeeld $v_{1}= (0,1,0,0,1)$.
    $v_{2}$ staat boven $v_{1}$, dus $v_{2}$ moet $(A-3I)v_{1}= (1,0,-1,0,0)$ zijn.

  \end{itemize}
\item 
  \[ P = 
  \left(
    \begin{array}{ccc|cc}
      1 & 0 & 1 & 0 & 1\\
      0 & 0 & 1 & 1 & 0\\
      0 & 0 & -1& 0 &-1\\
      0 & 0 & 1 & 0 & 0\\
      0 & -1& 0 & 1 & 0\\
    \end{array}
  \right)
  \]
\item 
  \[ J =
  \left(
    \begin{array}{ccc|cc}
      1 & 0 & 0 & 0 & 0\\
      1 & 1 & 0 & 0 & 0\\
      0 & 0 & 1 & 0 & 0\\\hline
      0 & 0 & 0 & 3 & 0\\
      0 & 0 & 0 & 1 & 3\\
    \end{array}
  \right)
  \]
\end{itemize}



\end{document}
