\documentclass[main.tex]{subfiles}
\begin{document}

\chapter{Toepassingen van Algebra: Oefenzittingen}
\label{cha:tai-oefenzittingen}

\section{Oefenzitting 1}

\subsection*{Oefening 1}
Op $\mathbb{R}$ definieren we de samenstellingswet $\tau:\ a\tau b = a + b + a^{2}b^{2}$
\subsubsection*{(a)}
Geef het neutraal element van deze wet.\\\\
Het neutraal element is hier $0$:
\[
a\tau 0=a+0+a^{2}0^{2}=a=0+a+0^{2}a^2
\]

\subsubsection*{(b)}
ze is niet associatief. Ga na!\\\\
\[
c \tau (a \tau b) = c \tau (a+b+a^2b^2)= c+ (a+b+a^2b^2)+c^2+(a+b+a^2b^2)^2
\]
\[
\neq (c \tau a) \tau b= (c+a+c^2a^2) \tau b= (c+a+c^2a^2)+b+(c+a+c^2a^2)^2+b
\]

\subsubsection*{(c)}
Ze is commutatief. Waarom?\\
De samenstellingswet is commutatief omdat zowel de optelling als het product voor $a$ en $b$ commutatief zijn.

\subsection*{Oefening 2}
Bewijs dat in $R^2 \times R^2$ volgende relaties equivalantierelaties zijn:
\[
G= \{((a,b),(c,d)|a^2+b^2=c^2+d^2\}
\]
\[
H=\{((a,b),(c,d))|b-a=d-c\}
\]
\[
J=\{((a,b),(c,d))|b+a=d+c\}
\]
Deze relaties zijn inderdaad reflexief, transitief en symmetrisch.\\
welke zijn de partities die door deze relaties bepaald worden?\\\\
\[
G_{(a,b)}=\{(x,y) \in \mathbb{R}^2|x^2+y^2=a^2+b^2\}
\]
Dit zijn de concentrische cirkels met de oorsprong als middelpunt en straal $\sqrt{a^2+b^2}$.
\[
H_{(a,b)}=\{(x,y) \varepsilon : y=x+b-a\}
\]
Dit zijn alle punten op dezelfde evenwijdige rechte met de identieke.
\[
J_{(a,b)}=\{(x,y) \varepsilon R^2: y=-x+b+a\}
\]
Dit zijn alle punten op dezelfde evenwijdige rechte met de tegengestelde van deidentieke.

welke zijn de partities die hierdoor gedefinieerd worden?
\[
(H \cap J)_{(a,b)} =\{(x,y) \in \mathbb{R}^2| x=a,y=b\}=\{(a,b)\}
\]

\subsection*{Oefening 3}
los het volgende stelsel op:

\[
\left\{
\begin{array}{c}
  3x_1-2x_2+6x_3=4\ mod\ 7\\
  4x_1+x_2+x_3=0\ mod\ 7\\
  2x_1+x_2+2x_3=-1\ mod\ 7\\
\end{array}
\right.
\]
\[
\longrightarrow
\left\{
\begin{array}{c}
  x_1-3x_2+2x_3=6\ mod\ 7\\
  4x_1+x_2+x_3=0\ mod\ 7\\
  2x_1+x_2+2x_3=-1\ mod\ 7\\
\end{array}
\right.
\]
\[
\longrightarrow
\left\{
\begin{array}{c}
  x_1-3x_2+2x_3=6\ mod\ 7\\
  0x_1-1x_2+0x_3=4\ mod\ 7\\
  0x_1+0x_2+5x_3=1\ mod\ 7\\
\end{array}
\right.
\]
De oplossingsverzameling van dit stelsel is $\{(2,3,3)\}$

\subsection*{Oefening 4}
Bepaal de isometrieen van een gelijkzijdige driehoek.\\\\
Stel voor deze ismetrieen de bewerkingstabel op, onder de samenstellingswet $\circ$.
Benoem de volgende isometrieen:
\begin{itemize}
\item[1]
identieke
\item[$r_1$]
rotatie over $120$ graden
\item[$r_2$]
rotatie over $240$ graden
\item[$s_1$]
spiegeling over do hoogtelijn door hoek $1$
\item[$s_2$]
spiegeling over do hoogtelijn door hoek $2$
\item[$s_3$]
spiegeling over do hoogtelijn door hoek $3$
\end{itemize}
\[
\]
De bijhoordende bewerkingstabel is dan de volgende:
\[
\begin{array}{c|ccccccc}
  \circ& 1     &  r_1  &  r_2  &  s_1  &  s_2  &  s_3 \\
  \hline
  1    & 1     &  r_1  &  r_2  &  s_1  &  s_2  &  s_3 \\
  r_1  &  r_1  &  r_2  &  1    &  s_3  &  s_1  &  s_2 \\
  r_2  &  r_2  &  1    &  r_1  &  s_2  &  s_3  &  s_1 \\
  s_1  &  s_1  &  s_2  &  s_3  &  1    &  r_1  &  r_2 \\
  s_2  &  s_2  &  s_3  &  s_1  &  r_2  &  1    &  r_1 \\
  s_3  &  s_3  &  s_1  &  s_2  &  r_1  &  r_2  & 1
\end{array}
\]

\subsection*{Oefening 5}
Een latijns vierkant in een n x n tabel waarin slechts n verschillende elementen voorkomen.
In elke rij en elke kolom komt namelijk elk element juist eenmaal voor.
\begin{itemize}
\item[(a)] Bewijs dat de bewerkingstabel voor een eindige groep steeds een latijns vierkant is
\item[(b)] Is dit ook een voldoende voorwaarde om een groep te hebben? Bepaal of volgend latijns vierkant de bewerkingstabel van een groep is
\end{itemize} 
\[ 
\begin{array}{c|cccccc}
   \tau & a & b & c & d & e & f\\
   \hline
   a & c & e & a & b & f & d\\
   b & f & c & b & a & d & e\\
   c & a & b & c & d & e & f\\
   d & e & a & d & f & c & b\\
   e & d & f & e & c & b & a\\
   f & b & d & f & e & a & c\\
\end{array}
\]

\subsubsection*{(a)}
Elke rij en kolom van de bewerkings tabel zijn verschillende elementen.
Stel immers dat er op een rij of kolom twee keer hetzelfde element voorkomt, dat is de bewerking niet injetief.
Bijgevolg is de bewerking niet injectief en dus niet inverteerbaar.
Er is dan geen uniek invers element voor elk element van de groep.
Dit is in contradictie met de definitie van een bewerking op een groep.

\subsubsection*{(b)}
De bewerking is niet associatief:
\[
a \tau (a \tau b)=a \tau e=f
\]
\[
(a \tau a) \tau b)=c \tau b=b
\]


\end{document}
