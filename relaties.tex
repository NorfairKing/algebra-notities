\documentclass[main.tex]{subfiles}
\begin{document}

\chapter{Relaties}
\label{cha:relaties}

\begin{de}
  Een (binaire) \emph{relatie} $R$ is een verzameling koppels $(x,y)$.
  Wanneer $(x,y)$ een koppel is in $R$ noteren we $x R y$.
\end{de}

\begin{de}
  De \emph{inverse $R^{-1}$ van een relatie} $R$ is de volgende relatie:
  \[ R^{-1} = \left\{ (x,y)\ |\ (y,x) \in R \right\} \]
\end{de}

\begin{st}
  De inverse van de inverse van een relatie is opnieuw de originele verzameling.
  \[ R^{-1^{-1}} = R \]
  \TODO{ bewijs }
\end{st}

\begin{de}
  De \emph{samenstelling $S \circ R$ van twee relaties} $R$ en $S$ (lees: ``$S$ na $R$'') is de volgende relatie.
  \[ \left\{ (x,y) \ |\ (\exists z) ((x,z) \in R \wedge (z,y) \in S) \right\} \]
\end{de}

\begin{st}
  De \emph{samenstelling van relaties is associatief}.
  \[ (T \circ S) \circ R = T \circ (S \circ R) \]
  \TODO{ bewijs }
\end{st}

\begin{st}
  De \emph{inverse van een verzameling is distributief ten opzichte van de samenstelling van relaties}.
  \[ (S \circ R)^{-1} = R^{-1} \circ S^{-1} \]
  \TODO{ bewijs }
\end{st}

\begin{de}
  Zij $R$ een relatie. Het \emph{domein} (domain) is als volgt gedefinieerd.
  \[ dom R = \left\{ x \ |\ (\exists y)(x,y) \in R \right\} \]
\end{de}

\begin{de}
  Zij $R$ een relatie. Het \emph{beeld} (range) is als volgt gedefinieerd.
  \[ ran R = \left\{ y \ |\ (\exists xi)(x,y) \in R \right\} \]
\end{de}

\begin{st}
  Het domein van een relatie is het beeld van zijn inverse.
  \[ dom R = ran R^{-1} \]
  \TODO{ bewijs }
\end{st}

\begin{st}
  Het beeld van een relatie is het domein van zijn inverse.
  \[ ran R = dom R^{-1} \]
  \TODO{ bewijs }
\end{st}

\begin{st}
  Domein na samenstelling:
  \[ dom (R \circ S) \subseteq dom S \]
  \TODO{ bewijs }
\end{st}
 
\begin{st}
  Beeld na samenstelling:
  \[ ran (R \circ S) \subseteq ran R \]
  \TODO{ bewijs }
\end{st}

\begin{st}
  Domein na samenstelling (2):
  \[ ran S \subseteq dom R \Rightarrow dom(R \circ S) = dom S \]
  \TODO{ bewijs }
\end{st}

\begin{de}
  Een $n$-aire relatie is, analoog aan een binaire relatie, een verzameling $n$-tallen.
\end{de}

\end{document}
