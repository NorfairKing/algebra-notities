\documentclass[main.tex]{subfiles}
\begin{document}

\chapter{Samenstellingswetten}
\label{cha:samenstellingswetten}

\begin{de}
  Een (inwendige) samenstellingswet of \emph{bewerking} $\top$ onder de elementen van een verzameling $A$ is een partiele functie:
  \[ f:\ A\times A \rightarrow A:\ (x,y) \rightarrow \top((x,y)) \]
  De enige voorwaarde voor een bewerking is dat ze intern is.
  Met andere woorden:
  \[ \forall x,y \in A: \top((x,y)) \in A \]
  Vaker dan de functienotatie gebruiken we de infixnotatie:
  \[ x \top y \Leftrightarrow  ((x,y),\top((x,y))) \in \top \]
\end{de}

\begin{de}
  We noemen een bewerking overal bepaald als het een afbeelding is.
\end{de}

\begin{de}
  Zij $\top:\ A\times A\rightarrow A$ een bewerking gedefinieerd op een verzameling $A$.
  We noemen een deelverzameling $B\subseteq A$ stabiel of gesloten onder $\top$ als de bewerking intern is binnen $B$.
  \[ \forall x,y \in E\subseteq A: x \top y \in E \]
\end{de}

\begin{de}
  \emph{Associativiteit}\\
  We noemen een bewerking $\top:\ A\times A\rightarrow A$ \emph{associatief} als de haakjes niet uit maken.
  \[ \forall x,y,z \in A:\ x\top(y\top z) = (x\top y) \top z \]
\end{de}

\begin{de}
  \emph{Commutativiteit}\\
  We noemen een bewerking $\top:\ A\times A\rightarrow A$ \emph{associatief} als de volgorde van de argumenten niet uit maakt.
  \[ \forall x,y \in A:\ x \top y \Leftrightarrow y \top x \]
\end{de}

\begin{de}
  Zij $\top:\ A\times A\rightarrow A$ een bewerking gedefinieerd op een verzameling $A$.
  We noemen $e\in A$ het neutraal element van $\top$ in $A$ als de volgende gelijkheden gelden.
  \[ \forall a \in A: a\top e = e = e\top a \]
\end{de}

\begin{st}
  \label{st:neutraal-element-uniek}
  Als er een neutraal element $e$ bestaat voor een bewerking $\top$ in een verzameling $A$ is dat neutraal element uniek.
  \begin{proof}
    Bewijs uit het ongerijmde\\
    Stel dat er twee verschillende neutrale elementen $e_{1}$ en $e_{2}$ bestaan, dan gelden volgende gelijkheden:
    \[ e_{2}\top e_{1} = e_{1}  =e_{1}\top e_{2}\]
    \[ e_{1}\top e_{2} = e_{2}  =e_{2}\top e_{1}\]
    Bijgevolg zijn deze neutrale elementen gelijk. Contradictie.
  \end{proof}
\end{st}

\begin{de}
  Zij $\top:\ A\times A\rightarrow A$ een bewerking gedefinieerd op een verzameling $A$.
  Zij $l$ een element van $A$.\\
  $l$ is \emph{links-regulier} of \emph{links-schrapbaar} als het links geschrapt kan worden.
  \[ \forall x,y \in A:\ l \top x = l \top y \Rightarrow x = y \]
  $r$ is \emph{rechts-regulier} of \emph{rechtss-schrapbaar} als het rechtse geschrapt kan worden.
  \[ \forall x,y \in A:\ x \top r = y \top r \Rightarrow x = y \]
  Een element is \emph{regulier} of \emph{schrapbaar} als het zowel links- als rechts-regulier is. 
\end{de}

\begin{opm}
  Als een element links/rechts schrapbaar is, is de afbeelding $x \mapsto l\top x$ / $x \mapsto x\top r$ een injectie. Het ``schrappen'' van dat element is dan de linker/linker inverse afbeelding van deze afbeelding.
\end{opm}

\TODO{symmetriseerbaar p 14}


\end{document}
