\documentclass[main.tex]{subfiles}
\begin{document}
\chapter{Velden}
\label{cha:velden}

\section{Karakteristiek van een ring}
\label{sec:karakt-van-een}

\begin{de}
  Zij $R,+,\cdot$ een ring met nulelement $e$.
  De \term{karakteristiek} $char(R)$ van een ring $R,+,\cdot$ is ...
  \begin{itemize}
  \item ... de kleinste $n\in \mathbb{N}_{0}$ zodat $nx=e$ geldt voor alle elementen $x\in R$.
  \item ... $0$ indien er niet zo'n $n$ bestaat.
  \end{itemize}
\end{de} 

\begin{ei}
  Zij $R,+,\cdot$ een ring  en karakteristiek $char(R) = n \neq 0$, dan is de (additieve) orde van elk element in $R,+$ een deler van $n$.

  \begin{proof}
    Zij $s$ orde van een willekeurig element $x$ uit $R$.
    Omdat $nx$ het nulelement is, is $n$ een kandidaat voor de orde, maar groter dan $s$, dus is $s$ een deler van $n$.\eiref{ei:groep-eindige-orde-deelbaarheid}
  \end{proof}
\end{ei}

\begin{st}
  \label{st:eenheidselement-orde-karakteristiek}
  Zij $R,+,\cdot$ een ring met eenheidselement $i$.
  \[ char(R) = n \neq 0 \Leftrightarrow i \text{ heeft orde } n \]

  \begin{proof}
    Bewijs van een equivalentie.
    \begin{itemize}
    \item $\Rightarrow$: Dit volgt rechtstreeks uit de definitie van de karakteristiek.
    \item $\Leftarrow$\\
      Voor een gegeven $k\in \mathbb{N}_{0}$ geldt het volgende:
      \[ \forall x \in R: kx = k(i\cdot x) = (ki) \cdot x \]
      Dit betekent precies dat $kx$ het nulelement is als en slechts als $k$ de orde is van $i$ in $R,+,\cdot$.
      $k$ is dus ook de charakteristiek van $R,+,\cdot$.
    \end{itemize}
  \end{proof}
\end{st}

\begin{gev}
  Zij $R,+,\cdot$ een ring met eenheidselement $i$.
  \[ char(R) = 0 \Leftrightarrow i \text{ heeft orde } \infty \]

  \begin{proof}
    Dit volgt rechtstreeks.
    Als er immers geen enkele $k\in \mathbb{N}_{0}$ bestaat zodat $ki$ het nulelement is, kan er ook geen $k$ bestaan zodat voor elke $x\in R$ $kx$ het nuleleement is.
  \end{proof}
\end{gev}

\begin{gev}
  De karakteristiek van een eindig veld $F,+,\cdot$ is een deler van $|F|$.
  
  \begin{proof}
    De karakteristiek van $F,+,\cdot$ is de orde van het eenheidselement in $F$ en de orde van het eenheidselement is een deler van $|F|$.\stref{st:stelling-van-lagrange}
  \end{proof}
\end{gev}

\begin{st}
  Zij $R,+,\cdot$ een niet-triviale ring zonder nuldelers met nulelement $e$, dan hebben alle elementen uit $R_{e}$ hebben dezelfde orde in $R,+$

  \begin{proof}
    Als alle elementen in $R_{e}$ als orde $\infty$ hebben is de stelling evident.
    Stel daarom dat er een element $a\in R_{e}$ bestaat met eindige orde $n$.
    We moeten dan aantonen dat elk ander element $x\in R_{e}$ ook orde $n$ heeft.
    Beschouw nu $a \cdot (nx)$:
    \[ a \cdot (nx) = (na) \cdot x = e \cdot x = e \]
    Omdat $R$ geen nulderes heeft volgt hieruit dat $nx$ het nulelement is
    \clarify{waarom is $n$ minimaal?}
  \end{proof}
\extra{bewijs}
\end{st}

\begin{st}
 Zij $R,+,\cdot$ een niet-triviale ring zonder nuldelers, dan is de karakteristiek van $R,+,\cdot$ nul of een priemgetal.
 
 \begin{proof}
   Stel dat de karakteristiek $n\neq 0$ van $R,+,\cdot$ valt te schrijven als $r\cdot s$ met $r, s \in \mathbb{N}_{0}$.
   Voor elk niet-nul-element $a$ uit $R$ geldt nu het volgende:
   \[ ra \cdot sa = (rs)a^{2} = na^{2} = 0 \]
   Dit betekent dat ofwel $ra$, ofwel $sa$ het nulelement is.
   $r$ of $s$ moet dan gelijk zijn aan $n$ (en de andere aan $1$).
   Dit houdt precies in dat $n$ een priemgetal is.
 \end{proof}
\end{st}

\begin{gev}
  \label{gev:karakteristiek-van-domein-is-priem}
  De karakteristiek van een integriteitsdomein (of veld) is nul of een priemgetal.
\end{gev}

\begin{st}
  \label{st:rekenregel-in-integriteitsdomein}
  Een \term{rekenregel in integriteitsdomein}\\
  Zij $R,+,\cdot$ een integriteitsdomein van karakteristiek $p$.
  \[ \forall a,b \in R:\ (a+b)^{p} = a^{p}+b^{p} \]

  \begin{proof}
    De (normale) binomiaalontwikkeling van $(a+b)^{p}$ ziet er als volgt uit:
    \[ (a+b)^{p} = \sum_{i=0}^{p}\frac{p!}{i!(p-i)!} a^{i}b^{p-i} \]
    Merk nu op dat $p$ een deler is van $binom{p}{i}$ voor alle $i$ van $1$ tot en met $p-1$.
    Voor elk van die $i$'s wordt de term dan het nulelement.
    Er blijft dan enkel nog $a^{p} + b^{p}$ over in het rechterlid.
  \end{proof}
\end{st}

\begin{st}
  Zij $R,+,\cdot$ een integriteitsdomein van karakteristiek $p$.
  \[ \forall a,b \in R:\ (a-b)^{p} = a^{p}-b^{p} \]

  \begin{proof}
    \[ (a-b)^{p} = (a+(-b))^{p} = a^{p} + (-b)^{p} = a^{p} - b^{p} \]
    De derde gelijkheid geldt omwille van de rekenregel in integriteitsdomeinen.\stref{st:rekenregel-in-integriteitsdomein}
    De laatste gelijkheid geldt enkel omdat $p$ een priemgetal is.
\clarify{waarom mag dit ook voor $p=2$??}
  \end{proof}
\end{st}

\begin{st}
  Zij $R,+,\cdot$ een ring met priemkarakteristiek $p$.
  \[ \forall a,b \in R, \forall k \in \mathbb{N}:\ (a+b)^{p^{k}} = a^{p^{k}}+b^{p^{k}} \]
\extra{bewijs}
\end{st}

\begin{st}
  Zij $R,+,\cdot$ een ring met priemkarakteristiek $p$.
  \[ \forall a,b \in R, \forall k \in \mathbb{N}:\ (a-b)^{p^{k}} = a^{p^{k}}-b^{p^{k}} \]
\extra{bewijs}
\end{st}

\begin{st}
  Het \term{Frobeniusmorfisme}\\
  \begin{itemize}
  \item Zij $R,+,\cdot$ een integriteitsdomein van karakteristiek $p\neq 0$.
    De afbeelding $\phi$ is een injectief ringmorfisme.
    We noemen het het \term{Frobeniusmorfisme}.
    \[ \phi:\ R \rightarrow R:\ x \mapsto x^{p} \]
  \item Als $|R|$ eindig is (en dus een veld\stref{st:eindig-integriteitsdomein-is-veld}), dan is $\phi$ een isomorfisme.
  \end{itemize}

  \begin{proof}
    in delen.
    \begin{itemize}
    \item Zij $e$ het nulelement van $R,+,\cdot$.  $\phi$ is een
      ringmorfisme vanwege de rekenregel in
      integriteitsdomeinen\stref{st:rekenregel-in-integriteitsdomein}.
      Uit de implicatie $x^{p} = e \Rightarrow x = 0$ volgt dat $\phi$
      injectief is.  \clarify{vanwaar komt die implicatie?}
    \item Een injectieve afbeelding van een verzameling naar zichzelf is surjectief en bijgevolg ook een bijectie. \clarify{referentie naar bewijs?}
    \end{itemize}
  \end{proof}
\end{st}


\subsection{Deelvelden}
\label{sec:deelvelden}

\begin{de}
  Zij $F,+,\cdot$ een veld en $E$ een niet-lege deelverzameling van $F$.
  We noemen $E,+,\cdot$ een \term{deelveld} van $F,+,\cdot$ als $E,+,\cdot$ zelf een veld is (voor dezelfde bewerkingen).
\end{de}

\begin{st}
  \label{st:criterium-deelveld}
  Het \term{criterium voor deelvelden}\\
  Zij $F,+,\cdot$ een veld met eenheidselement $i$ en $E$ een deelverzameling van $F$, dan is $E,+,\cdot$ een deelveld van $f$ als en slechts als aan volgende voorwaarden voldaan is.
  \begin{itemize}
  \item $E$ is een deelring van $F$ met hetzelfde eenheidselement.
  \item $\forall x\in E:\ \exists y\in E:\ xy =i = yx$ (Elk element heeft een interne inverse.)
  \item $\forall x,y\in E:\ x\cdot y \in E$ (De multiplicatieve bewerking is intern.)
  \end{itemize}

  \begin{proof}
    Bewijs van een equivalentie
    \begin{itemize}
    \item $\Rightarrow$\\
      Dit volgt bijna rechtstreeks uit de definitie.
      We bewijzen nog dat $E,+,\cdot$ en $F,+,\cdot$ hetzelfde eenheidselement hebben.
      Noem voorlopig het eenheidselement van $E,+,\cdot$ en $F,+,\cdot$ respectievelijk $i_{E}$ en $i_{F}$.
      Noteer de inverse in $F,+,\cdot$ als $^{-1}$ en de inverse in $E,+,\cdot$ als $\bar{}$.
      \[
      \begin{array}{rll}
        i_{F} &= i_{E}\cdot i_{E}^{-1} &\\
              &= i_{E} \cdot i_{E} \cdot i_{E}^{-1} &\\
              &= i_{E} \cdot i_{F} &= i_{E}
      \end{array}
      \]
      Merk goed op dat alle bewerkingen in $F,+,\cdot$ gebeuren.
      Enkel in de tweede gelijkheid gebruiken we dat $i_{E}$ het eenheidselement is in $E,+,\cdot$.
      Voor de rest beschouwen we $i_{E}$ gewoon als een element van $F$.
      Merk ook op dat we niet zomaar het volgende mogen zeggen:
      \[ i_{F} = i_{E} \cdot i_{F} = i_{E} \]
      De linker gelijkheid geldt immers niet zomaar omdat we a priori niet weten of $i_{F}$ wel een element is van $E$.
    \item $\Leftarrow$\\
      Omdat $E$ een deelring is die $i$ bevat, volgt uit de twede voorwaarde dat $E$ een lichaam is.
      $E,+,\cdot$ is bovendien commutatief want het is een deelring van een commutatieve ring. \needed
      $E,+,\cdot$ is een commutatief lichaam en $E$ een deelverzameling van $F$, dus een deelveld.
    \end{itemize}
  \end{proof}
\end{st}

\subsection{Priemdeelring en Priemdeelveld}
\label{sec:priemd-en-priemd}

\begin{de}
  Zij $R,+,\cdot$ een ring met eenheidselement $i$.
  De doorsnede van alle deelringen van $R,+,\cdot$ die $i$ bevatten is zelf een deelring van $R,+,\cdot$ (die $i$ bevat).
  Deze `kleinste deelring die $i$ bevat' de \term{priemdeelring} van $R,+,\cdot$.
\end{de}

\begin{ei}
  \label{ei:priemdeelring-voortgebracht-door-eenheidselement}
  De priemdeelring van een ring $R,+,\cdot$ met eenheidselement $i$ is $<1>$:
  \[ <1> = \{ zi \ |\ z \in \mathbb{Z} \} \]

  \begin{proof}
    Kies $I$ zodat $I,+,\cdot$ een deelring is van $R,+,\cdot$.
    $I$ moet dan $<i>$ omvatten.
\question{en nu?}
  \end{proof}
\end{ei}

\begin{st}
  \label{st:karakteristiek-isomorfisme-met-zn}
  Zij $R,+,\cdot$ een ring met eenheidselement $i$.
  \begin{itemize}
  \item Als $char(R) = n \neq 0$ geldt, dan is de priemdeelring van $R$ isomorf met $\mathbb{Z}_{n}$.
  \item Als $char(R) = 0$ geldt, dan is de priemdeelring van $R$ isomorf met $\mathbb{Z}$.
  \end{itemize}

  \begin{proof}
    Beschouw de afbeelding $\phi$:
    \[ \phi:\ \mathbb{Z} \rightarrow \mathbb{R}: z \mapsto zi \]
    $\phi$ is nu een ringmorfisme:
    \[ \forall x,y\in \mathbb{Z}: \phi(x+y) = (x+y) i = (xi) + (yi) = \phi(x) + \phi(y) \]
    \[ \forall x,y\in \mathbb{Z}: \phi(x\cdot y) = (x\cdot y)i = (xi) \cdot (yi) = \phi(x) \cdot \phi(y) \]
    Merk op dat $\phi(\mathbb{Z}),+,\cdot$ de priemdeelring is van $R,+,\cdot$.\eiref{ei:priemdeelring-voortgebracht-door-eenheidselement}
    \begin{itemize}
    \item Als de karakteristiek van $R,+,\cdot$ $n\neq 0$ is, dan wordt de kern van $\phi$ voortgebracht door $n$ in $\mathbb{Z}$ want $n$ is de orde van $i$ in $R,+$.
      \[ Ker(\phi) = (n) \]
      De priemdeelring is dan isomorf met $\nicefrac{\mathbb{Z}}{(n)} \cong \mathbb{Z}_{n}$.\needed
    \item Als de karakteristiek van $R,+,\cdot$ nul is,dan is $\phi$ injectief \waarom en is de priemdeelring van $R,+,\cdot$ dus isomorf met $\mathbb{Z}$.
    \end{itemize}
  \end{proof}
\end{st}


\begin{de}
  Zij $F,+,\cdot$ een veld.
  De doorsnede van alle deelvelden van $F$ is zelf een deelveld van $F$.
  Di kleinste deelveld heet het \term{priemdeelveld} van $F,+,\cdot$.
\end{de}

\begin{st}
  \label{st:char-priem-priemdeelveld-zp}
  Zij $F,+,\cdot$ een veld.
  Als $char(F)$ een priemgetal is, dan is het priemdeelveld van $F,+,\cdot$ isomorf met $\mathbb{Z}_{p}$.
\end{st}

\begin{st}
  Zij $F,+,\cdot$ een veld.
  Als $char(F)$ nul is, dan is het priemdeelveld van $F,+,\cdot$ isomorf met $\mathbb{Q}$.

  \begin{proof}
    \begin{itemize}
    \item Als $char(F)$ een priemgetal $p$ is, weten we al dat de priemdeelring van $F,+,\cdot$ isomorf is met het veld $\mathbb{Z}_{p}$.
      Die deelring is dan ook een deelveld.\stref{st:karakteristiek-isomorfisme-met-zn}
    \item Als $char(F)$ nul is, dan is $\phi$ een injectief ringmorfisme:
      \[ \phi:\ \mathbb{Z} \rightarrow \mathbb{R}: z \mapsto z\cdot i \]
      Beschouw nu de afbeelding $\phi'$:
      \[ \phi':\ \mathbb{Q} \rightarrow F:\ \frac{a}{b} \mapsto \phi(a) \cdot (\phi(b))^{-1} = (ai) \cdot (bi)^{-1} \]
      Merk nu op dat $\phi'$ en ringmorfisme is:
      \[
      \forall \left(\frac{a}{b}\right),\left(\frac{c}{d}\right) \in
      \mathbb{Q}:\
      \begin{array}{rll}
        \phi'(\nicefrac{a}{b} + \nicefrac{c}{d}) &=  \phi'\left(\frac{ad+bc}{bd}\right) &\\
                                                 &= \phi(ad+bc) \cdot (\phi(bd))^{-1} &\\
                                                 &= ((ad+bc)i) \cdot ((bd)i)^{-1} &\\
                                                 &= ((ad)i) + ((bc)i)) \cdot ((bd)i)^{-1} &\\
                                                 &= ((ad)i)\cdot ((bd)i)^{-1}) +  ((bc)i) \cdot ((bd)\cdot i)^{-1}) &\\
                                                 &= \phi'(\nicefrac{ad}{bd}) + \phi'(\nicefrac{bc}{bd}) &= \phi'(\nicefrac{a}{b}) + \phi'(\nicefrac{c}{d}) \\
      \end{array}
      \]
      \[
      \forall \left(\frac{a}{b}\right),\left(\frac{c}{d}\right) \in
      \mathbb{Q}:\
      \begin{array}{rll}
        \phi'(\nicefrac{a}{b} \cdot \nicefrac{c}{d}) &= \phi'(\nicefrac{ac}{bd}) &\\
                                                     &= \phi(ac) \cdot (\phi(bd))^{-1} &\\
                                                     &= ((ac)i) \cdot ((bd) \cdot i)^{-1} &\\
                                                     &= ((ai) \cdot (ci)) \cdot ((bi) \cdot (di))^{-1} &\\
                                                     &= (ai) \cdot (bi)^{-1} \cdot (ci) \cdot (di)^{-1} &\\
                                                     &= \phi(a) \cdot (\phi(b))^{-1} \cdot \phi(c) \cdot (phi(d))^{-1} &= \phi'(\nicefrac{a}{b}) \cdot \phi'(\nicefrac{c}{d})
      \end{array}
      \]
      $\phi'$ is bovendien injectief.  Kies immers twee verschillende
      elementen $\frac{a}{b}$ en $\frac{c}{d}$ uit $\mathbb{Q}$ die op
      hetzelfde element van $F$ afgebeeld worden onder $\phi'$, dan
      geldt het volgende:
      \[ (ai) \cdot (bi)^{-1} = (ci) \cdot (di)^{-1} \]
      \[ \phi(a) \cdot \phi(d) = \phi(b) \cdot \phi(c) \]
      Omdat $\phi$ injectief is geldt $ad = bc$ en bijgevolg
      $\frac{a}{b} = \frac{c}{d}$.
      Het priemdeelveld van $F$ is nu precies het beeld van $\mathbb{Q}$ onder $\phi'$. \waarom
      De kern van $\phi$ is bovendien $\{0\}$, dus het priemdeelveld is isomorf met $\nicefrac{\mathbb{Q}}{\{0\}} \cong \mathbb{Q}$.
    \end{itemize}
  \end{proof}
  \begin{proof}
    Zij $F,+,\cdot$ een veld van karakteristiek $0$ en zij $P,+,\cdot$ het priemdeelveld van $F,+,\cdot$.
    $P$ omvat dan de priemdeelring van $F,+,\cdot$.
    We weten al dat die priemdeelring isomorf is met $\mathbb{Z}$ volgens $\phi$.\stref{st:karakteristiek-isomorfisme-met-zn}
    \[ \phi::\ \mathbb{Z} \Rightarrow F:\ z \mapsto (zi)\]
    Vermits het beeld tot $P$ behoort, kunnen we $\phi$ ook als volgt bekijken:
    \[ \mathbb{Z} \rightarrow P \]
    Er bestaat dan ook een ringmorfisme $\phi'$ zodat $\phi$ factoriseert als $\phi' \circ i$ waarbij $i$ de inclusie is van $\mathbb{Z}$ in $\mathbb{Q}$. \eiref{ei:factorisatie-breukenveld}
    \[ \phi': \mathbb{Q} \rightarrow P \]
    \begin{figure}[H]
      \centering
      \begin{tikzpicture}
        \matrix (m) [matrix of math nodes,row sep=3em,column sep=4em,minimum width=2em]
        {
          \mathbb{Z} & \mathbb{Q} \\
          P & \\};
        \path[-stealth]
        (m-1-1) edge node [left] {$\phi$} (m-2-1)
        edge node [above] {$i$} (m-1-2)
        (m-1-2) edge node [right] {$\phi'$} (m-2-1);
      \end{tikzpicture}
    \end{figure}
    Nu beeldt $\phi'$ $1$ af op het neutraal element van $P$.
    $\phi'$ is dus een veldmorfisme.
    Omdat $\phi'$ ook injectief is, is het beeld van $\mathbb{Q}$ onder $\phi'$ een deelveld van $P,+,\cdot$ en van $F,+,\cdot$.
    $P$ is echter het priemdeelveld van $F$, dus $\phi'$ moet ook surjectief zijn en bijgevolg een isomorfisme.
  \end{proof}
\end{st}

\begin{st}
  Zij $K,+,\cdot$ een deelveld van een veld $L,+,\cdot$.
  \[ char(K) = char(L) \]

  \begin{proof}
    Noem de karakteristiek van $L,+,\cdot$ $n$.
    De orde van het eenheidselement in $L$ is dan ook $n$.\stref{st:eenheidselement-orde-karakteristiek}
    Omdat $K,+,\cdot$ een deelveld is van $L,+,\cdot$ is $n$ ook de orde van het eenheidselement in $K,+,\cdot$.
  \end{proof}
\end{st}

\section{Veldmorfismen}
\label{sec:veldmorfismen}

\begin{de}
  Zij $F,+,\cdot$ en $E,\star,*$ velden, dan is een afbeelding $\phi: F \rightarrow E$ een \term{veld(homo)morfisme} als $\phi$ een ringmorfisme is dat het eenheidselement van $F$ op het eenheidselement van $E$ afbeeldt.
\end{de}

\begin{de}
  Een bijectief veldmorfisme noemen we een \term{veldisomorfisme}.
\end{de}

\begin{st}
  \label{st:veldmorfisme-is-injectief}
  Een veldmorfisme $\phi: F \rightarrow E$ is steeds injectief.
  In het bijzonder is $\phi(F)$ een deelveld van $E$ dat isomorf is met $F$.
\extra{bewijs}
\end{st}

\begin{opm}
  \label{opm:veldmorfisme-is-injectief}
  We zullen daarom een veld vaak identificeren met zijn beeld onder een veldmorfisme.
\end{opm}

\section{Velduitbreidingen: basisbegrippen}
\label{sec:veld-basisb}

\begin{de}
  Zij $F,+,\cdot$ een deelveld van een veld $E,+,\cdot$.
  Men noemt $E,+,\cdot$ een \term{velduitbreiding} of \term{uitbreiding} van $F$.
\end{de}

\begin{st}
  Zij $F,+,\cdot$ een deelveld van een veld $E,+,\cdot$.
  $E$ is een vectorruimte over $F$.
  
\extra{bewijs}
\end{st}

\begin{de}
  \label{de:uitbreidingsgraad}
  Zij $F,+,\cdot$ een deelveld van een veld $E,+,\cdot$.
  De \term{uitbreidingsgraad} of \term{graad} $[E:F]$ van $E$ over $F$ is de dimensie van $E$ als $F$-vectorruimte.
\end{de}

\begin{de}
  Een uitbreiding $E,+,\cdot$ van een veld $F,+,\cdot$ met een eindige uitbreidingsgraad noemen we een \term{eindige uitbreiding}.
\end{de}

\begin{de}
  Binnen een veld $F,+,\cdot$ noteren we $a\cdot b^{-1}$ vaak als $\frac{a}{b}$.
\end{de}

\begin{st}
  \label{st:eindig-veld-orde-macht-van-priemgetal}
  Het aantal elementen van een eindig veld is een macht van een priemgetal.

  \begin{proof}
    Zij $F,+,\cdot$ eeneindig veld, dan moet het priemdeelveld $P,+,\cdot$ van $F,+,\cdot$ isomorf zijn met een $\mathbb{Z}_{p}$ waarbij $p$ een priemgetal is.
    $F$ is een noodzakelijk eindigdimensionale vectorruimte over $P$.
    Noem de uitbreidingsgraad van $F,+,\cdot$ over $P,+,\cdot$ $n$.
    \[ |F| = |P|^{n} = p^{n} \]
\clarify{hoe komen we aan dit laatste??}
  \end{proof}
\end{st}

\begin{st}
  \label{st:fundamentele-stelling-van-veldentheorie}
  \examen
  De \term{fundamentele stelling van de veldentheorie}\\
  Zij $K,+,\cdot$ een veld en $f$ een niet-constante veelterm in $K[X]$, dan bestaat er een uitbreiding $E$ van $K$ zodat $f$ een wortel heeft in $E$.

  \begin{proof}
    Kies een irreducibele factor $p$ van $f$ in het UFD\stref{gev:veeltermen-over-veld-ufd} $K[X]$.
    Beschouw nu de quoti\"entring $\nicefrac{K[X]}{(p)}$.
    We beweren dat $p$ een wortel zal hebben $\nicefrac{K[X]}{(p)}$.
    \begin{itemize}
    \item $\nicefrac{K[X]}{(p)}$ is een veld omdat $p$ irreducibel is\gevref{gev:veld-door-hoofdideaal-van-irreducibele-veelterm-veld} en $(p)$ daarom een maximaal ideaal\stref{st:hoofdidiaal-van-irreducibele-veelterm-maximaal} in het HID\stref{st:veeltermen-over-veld-hid} $K[X]$.
    \item De afbeelding $f$ is een veldmorfisme en bijgevolg injectief.\stref{st:veldmorfisme-is-injectief}
      \[ f:\ K \rightarrow \nicefrac{K[X]}{(p)}:\ a \mapsto \bar{a} \]
      We identificeren daarom $K$ met het deelveld $\{ \bar{a} \mid a\in K \}$ van $\nicefrac{K[X]}{(p)}$.
      We beschouwen $\nicefrac{K[X]}{(p)}$ met andere woorden als een uitbreiding van $K$.
    \item $\bar{X}$ is nu een wortel van $p$ in $\nicefrac{K[X]}{(p)}$.
      \[ p(\bar{X}) = \sum_{i}\overline{a_{i}}(\overline{X})^{i}= \overline{\sum_{i}a_{i}X^{i}} = \bar{p} = \bar{e_{K}} \]
      Merk op dat we de identificatie gebruiken.
      We beschouwen $p\in K[X]$ als $p = \sum_{i}\bar{a_{i}}X^{i}$.
    \end{itemize}
\clarify{meer uitleg nodig!}
  \end{proof}
\end{st}

\begin{opm}
  Deze stelling geldt ook voor integriteitsdomeinen
\extra{bewijs: elk domein kan worden uitgebreid tot een breukenveld}
  maar niet voor algemene commutatieve ringen met eenheidselement.
\extra{voorbeeld p 79}
\end{opm}

\begin{st}
  \label{st:productformule}
  De \term{productformule}\\
  Zij $F,+,\cdot$, $L,+,\cdot$ en $E,+,\cdot$ velden waarbij $L,+,\cdot$ een uitbreiding is van $F,+,\cdot$ en $E,+,\cdot$ een uitbreiding is van $L,+,\cdot$.
  Als zowel $[L:F]$ als $[E:L]$ eindig zijn, geldt het volgende:
  \[ [E:F] = [E:L] \cdot [L:F]\]

  \begin{proof}
    Kies een basis $\alpha = \{ u_{1},\dotsc,u_{n}\}$ van $E$ als vectorruimte over $L$ en een basis $\beta = \{ v_{1},\dotsc,v_{m}\}$.
    We zullen aantonen dat $\gamma = \{ u_{i}\cdot v_{i} \mid i\in \{1,\dotsc,n\}, j \in \{1,dotsc,m\} \}$ een basis vormt voor $E$ als vectorruimte over $F$.
  \begin{itemize}
  \item $\gamma$ is vrij.\\
    Stel dat $\gamma$ niet vrij zou zijn, dan bestaan er co\"efficienten $a_{ij}\in L$ als volgt:
    \[ \sum_{i=1}^{m}\sum_{j=1}^{n}a_{ij}u_{i}\cdot v_{j} = e_{F} \]
    Omdat $E,+,\cdot$ een veld is kunnen we dit herschrijven als volgt:
    \[ \sum_{i=1}^{m}\left(\sum_{j=1}^{n}(a_{ij}v_{i})\right)\cdot u_{j} = e_{F} \]
    Omdat $\alpha$ linear onafhankelijk is moeten de co\"efficienten $\sum_{j=1}^{m}a_{ij}v_{i}$ elk het nulelement zijn.
    Omdat $\beta$ lineair afankelijk is moeten ook nog eens alle $a_{ij}$ dan het nulelement zijn.
  \item $\gamma$ is voortbrengend.\\
\extra{bewijs}
  \end{itemize}
\end{proof}
\end{st}

\subsection{Algebraische en transcendente elementen}
\label{sec:algebr-en-transc}

\begin{de}
  Zij $F,+,\cdot$ een veld en $E,+,\cdot$ een uitbreiding van $F,+,\cdot$.
  Een element $u\in E$ is \term{algebra\"isch} over $F,+,\cdot$ als er een niet-nul-veelterm $f\in F[X]$ bestaat zodat $u$ een wortel is van $f$.
  Anders noemen we $u$ \term{transcendent}.
\end{de}

\begin{st}
  Elk element van een veld $F,+,\cdot$ is transcendent over $F,+,\cdot$.
  \extra{bewijs}
\end{st}

\begin{st}
  Elk element van de vorm $\sqrt[n]{q}$ met $n\in \mathbb{N}_{0}$ en $q\in \mathbb{Q}^{+}$ is algebra\"isch over $\mathbb{Q}$.
  \extra{bewijs}
\end{st}

\subsection{Enkelvoudige uitbreidingen}
\label{sec:enkelv-uitbr}

\begin{de}
  Zij $F,+,\cdot$ een veld en $E,+,\cdot$ een uitbreiding van $F,+,\cdot$.
  Zij $u\in E$ een element van $E$.
  Het veld bekomen door \term{adjunctie} of \term{toevoeging} van $u$ aan $F$: $F(u)$ is de doorsnede van alle velden $L$ die tussen $F$ en $E$ liggen en $u$ bevatten.
  Dit is dus het ``kleinste deelveld van $E$ dat $F$ en $u$ omvat''.
\end{de}

\begin{ei}
  \label{ei:toevoeging-formule}
  Zij $F,+,\cdot$ een veld en $E,+,\cdot$ een uitbreiding van $F,+,\cdot$.
  Zij $u\in E$ een element van $E$.
  \[
  F(u) = 
  \left\{
      \frac
        {a_{r}u^{r} + \dotsb + a_{1} u + a_{0}}
        {b_{s}u^{s} + \dotsb + b_{1} u + b_{0}}
  \mid r,s \in \mathbb{N},\ a_{i},b_{i} \in F
  \right\}
  \]
  In bovenstaande formule mogen natuurijk niet alle $b_{i}$ het nulelement zijn.
\extra{bewijs}
\end{ei}

\begin{opm}
  $F(X)$ kunnen we dan zien als de verzameling van rationale veeltermen over $F,+,\cdot$.
\end{opm}

\subsection{Minimale veelterm}
\label{sec:minimale-veelterm}

\begin{de}
  Zij $F,+,\cdot$ een veld en $E,+,\cdot$ een uitbreiding van $F,+,\cdot$.
  Zij $u$ een element van $E$, algebra\"isch over $F,+,\cdot$.
  Een \term{minimale veelterm} van $u$ over $F$ is een veelterm $f\in F[X]$ van minimale graad waarvoor $f\neq e$ en $f(u) = 0$ gelden.
\end{de}

\begin{pr}
  \label{pr:minimale-veelterm-uniek}
  Zi $F,+,\cdot$ een $E$-vectorruimte en $u$ algebra\"isch over $F$.
  De minimale veelterm van $u$ over $v$ is uniek bepaald op een constante factor (verschillend van het nulelement) na.

  \begin{proof}
    Beschouw $J$:
    \[ J = \{ g\in F[X] \mid g(u) = 0 \} \]
    $J$ is een ideaal van $F[X],+,\cdot$.
    Omdat $u$ algebra\"isch is over $F,+,\cdot$ bevat $g$ meer dan enkel het nulelement.
    De voortbrengers van een niet-triviaal-ideaal in een veeltermenring zijn precies de minimale veeltermen \lemref{lem:minimale-veelterm-brengt-ideaal-voort} en deze zijn allemaal gelijk op een constante factor na.\stref{st:minimale-veeltermen-bijna-uniek}
  \end{proof}
\end{pr}

\begin{ei}
  \label{ei:minimale-veelterm-uniek-en-irreducibel}
  Zij $F,+,\cdot$ een veld en $E,+,\cdot$ een uitbreiding van $F,+,\cdot$.
  Zij $u$ een element van $E$, algebra\"isch over $F,+,\cdot$.
  Zij $f$ een minimale veelterm van $u$ over $F,+,\cdot$.
  \begin{itemize}
  \item Als $g \in F[X]$ ook een veelterm is over $F,+,\cdot$ met waarde nul in $u$, dan is $f$ een deler van $g$.
  \item $f$ is irreducibel in $F[X]$.
  \end{itemize}

  \begin{proof}
    \begin{itemize}
    \item Inderdaad\prref{pr:minimale-veelterm-uniek}
    \item Stel dat $f$ reducibel zou zijn in $F[X]$, dan bestaan er veeltermen $g$ en $h$ over $F,+,\cdot$ van lagere graad zodat $f$ als volgt geschreven kan worden:
      \[ f = gh \]
      Omdat $u$ een nulpunt is van $f$ moet $u$ een nulpunt zijjn van $g$ of $h$, maar dat is in tegenspraak met de minimaliteit van $gr(f)$.
    \end{itemize}
  \end{proof}
\end{ei}

\begin{ei}
  Zij $F,+,\cdot$ een veld en $E,+,\cdot$ een uitbreiding van $F,+,\cdot$.
  Zij $u\in E$ een element van $E$.
  Zij $f\in F[X]$ een veelterm over $F[X]$ is met waarde nul in $u$.
  \[ f \text{ is irreducibel } \Rightarrow f \text{ is de minimale veelterm van } u \text{ over } F,+,\cdot \]

  \begin{proof}
    Bewijs via contrapositie:
    Stel dat $f$ niet de minimale veelterm is van $u$, maar $p$, is $p$ deelbaar door $f$\eiref{ei:minimale-veelterm-uniek-en-irreducibel}. En bestaat dan dus een veelterm $q$ zodat $f=gq$ geldt.
    Bijgevolg is $f$ reducibel.
  \end{proof}
\end{ei}

\begin{st}
  \label{st:uitbreiding-van-transcendente-u-aan-veld-graad-oneindig}
  Zij $F,+,\cdot$ een veld en $E,+,\cdot$ een uitbreiding van $F,+,\cdot$.
  Zij $u\in E$ een element van $E$, transcendent over $F,+,\cdot$.
  \begin{itemize}
  \item $F(u) \cong F(X)$
  \item $[F(u):F]$ is niet eindig.
  \end{itemize}

  \begin{proof}
    \begin{itemize}
    \item Beschouw het substitutie-ringmorfisme $\phi$:
      \[ \phi:\ F[X] \rightarrow F(u):\ g \mapsto g(u) \]
      $u$ is transcendent als en slechts als enkel de nulveelterm op het nelelement wordt afgebeeld.
      Dit is precies wanneer $\phi$ injectief is.\eiref{ei:ringmorfisme-injectief-asa-kern-nul}
      Omdat $\phi$ injectief is, kunnen we $\phi$
      uitbreiden tot een, nog steeds injectief, ringmorfisme $\phi'$.
      \[ \phi':\ F(X) \rightarrow F(u):\ \frac{g}{h} \mapsto \frac{g(u)}{h(u)} \]
      Kies immers twee elementen die op hetzelfde element uit $F(u)$ worden afgebeeldt onder $\phi'$.
      Die elementen moeten dan hetzelfde zijn (, maar enkel omdat $\phi$ al injectief is).
      \[ \forall g,h,i,j \in F(u): \frac{g(u)}{h(u)} = \frac{i(u)}{j(u)} \Rightarrow \frac{g}{h} = \frac{i}{j} \]
      $\phi'(F(X))$ omvat nu duidelijk zowel $F$ als $u$.
      \[ \phi'(X) = u \quad\text{ en }\quad F \subseteq \phi'(F(X)) \]
      Bovendien $\phi'F(X)$ is veld omdat $\phi'$ injectief is.\stref{st:veldmorfisme-injectief}
      We besluiten dat $\phi'(F(X))$ een deelveld is van $F(u)$ dat $F$ en $u$ omvat.
      \[ \phi'(F(X)) = F(u) \]
      $\phi'$ is nu ook surjectief en dus een isomorfisme.
      Dit betekent precies dat $F(u)$ isomorf is met $F(X)$.
    \item De dimensie van $F[X]$ is niet eindig, dus de dimensie van $F(X)$ zeker niet.
      Er bestaat bovendien een bijectie van $F[X]$ naar $F(X)$ (zie hierboven).
    \end{itemize}
  \end{proof}
\end{st}

\begin{st}
  \label{st:uitbreiding-van-algebraische-u-aan-veld-graad-minimale-veelterm}
  Zij $F,+,\cdot$ een veld met eenheidselement $i$ en $E,+,\cdot$ een uitbreiding van $F,+,\cdot$.
  Zij $u\in E$ een element van $E$, algebra\"isch over $F,+,\cdot$ en $f$ een minimale veelterm van $u$ over $F,+,\cdot$.
  \begin{itemize}
  \item $F(u) \cong F(X)/(f)$
  \item $[F(u):F] = deg(f)$
  \item De volgende verzameling vormt een basis van $F(u)$ over $F$.
  \[ \{ i,u,u^{2},\dotsc u^{deg(f)-1} \}\]
  \end{itemize}

  \begin{proof}
    Ingewikkeld, let op!
    \begin{itemize}
    \item Beschouw opnieuw het substitutie-ringmorfisme $\phi$:
      \[ \phi:\ F[X] \rightarrow F(u):\ g \mapsto g(u) \]
      De kern van $\phi$ is nu het (hoofd\stref{st:veeltermen-over-veld-hid})ideaal van $F[X]$ voortgebracht door $f$.\prref{pr:minimale-veelterm-uniek}
      Dit betekent precies dat $\phi(F[X])$ isomorf is met $\nicefrac{F[X]}{(f)}$.\gevref{gev:eerste-ringmorfismestelling}
      \[ \phi(F[X]) \cong \nicefrac{F[X]}{(f)} \]
      Omdat $f$ irreducibel is, is $(f)$ een maximaal ideaal in $F[X]$\stref{st:hoofdidiaal-van-irreducibele-veelterm-maximaal} en $\frac{F[X]}{(f)}$ bijgevolg een veld.\gevref{gev:veld-door-hoofdideaal-van-irreducibele-veelterm-veld}
      $\phi(F[X])$ is dan een deelveld van $F(u)$\stref{st:veldmorfisme-is-injectief} dat $F$ en $u$ omvat en bijgevolg is $\phi(F[X])$ gelijk aan $F(u)$.
    \item Vanwege het isomofirme tussen $\nicefrac{F[X]}{(f)}$ en $\phi(F[X]) = F(u)$ is het voldoende om te bewijzen dat de verzameling $\{ i,u,u^{2},\dotsc u^{deg(f)-1} \}$ een basis vormt van $\nicefrac{F[X]}{(f)}$ als vectorruimte over $F$.
      Er geldt dan het volgende\deref{de:uitbreidingsgraad}:
      \[ [F(u):F] = deg(f) \]
      \begin{itemize}
      \item Voortbrengend\\
        Kies een willekeurig element $\bar{g}$ uit $\nicefrac{F[X]}{(f)}$.
        $g$ zit dan in $F[X]$.
        Vanwege het delingsalgoritme in $F[X]$ kunnen we $g$ schrijven als $g=qf+r$ met $q$ en $r$ $\in F[X]$ en $gr(r) < gr(f)$.
        We noteren verder $d=gr(f)$ en $r$ als volgt:
        \[ r = \sum_{i=0}^{d-1}a_{i}X^{i} \]
        Er geld dan het volgende:\waarom
        \[ \bar{g} = \bar{q}\bar{f} + \bar{r} = \bar{r} = \overline{\sum_{i=0}^{d-1}a_{i}X^{i}} = \sum_{i=0}^{d-1}\bar{a}_{i}\bar{X}^{i} \]
        $\{ i,u,u^{2},\dotsc u^{deg(f)-1} \}$ is dus inderdaad een voortbrengende verzameling over $F,+,\cdot$.
        Merk op dat we hier de identificatie van $F$ met $\{ \bar{a}\mid a\in F\}$ gebruiken.
      \item Vrij\\
\extra{bewijs}
      \end{itemize}
    \end{itemize}
  \end{proof}
\end{st}

\begin{gev}
  Zij $F,+,\cdot$ een veld en $E,+,\cdot$ een uitbreiding van $F,+,\cdot$.
  \[ u \text{ is algebra\"isch over } F \Leftrightarrow [F(u):F] \text{ is eindig } \]

  \begin{proof}
    Dit is een combinatie van de vorige twee stellingen.\stref{st:uitbreiding-van-transcendente-u-aan-veld-graad-oneindig} \stref{st:uitbreiding-van-algebraische-u-aan-veld-graad-minimale-veelterm}
  \end{proof}
\end{gev}

\section{Algebra\"ische uitbreidingen en ontbindingsvelden}
\label{sec:algebr-uitbr-en}

\begin{de}
  Zij $F,+,\cdot$ en $S,+,\cdot$ velden en $E,+,\cdot$ een uitbreiding van zowal $F,+,\cdot$ als $S,+,\cdot$.
  Het veld verkregen door toevoeging van alle elementen van $S$ aan $F$: $F(S)$ is de doorsnede van alle velden $L$ die tussen $F,+,\cdot$ en $E,+,\cdot$ zitten en $S,+,\cdot$ uitbreiden.
  Dit is dus het ``kleinste deelveld van $E$ dat $F$ en $S$ omvat''.
  \[ S = \{ u_{1},\dotsc,u_{n} \}:\ F(u_{1},\dotsc,u_{n}) = F(S) \]
\end{de}

\begin{ei}
  Net als bij enkelvoudige uitbreidingen kan de uitbreiding van een veld met een verzameling beschreven worden met een formule.
  \[
  F(S) =
  \left\{
    \dfrac
      {
        \sum_{j=0}^{k}\sum_{i=0}^{r_{j}}a_{i,r_{j}}s_{j}^{i}
      }
      {
        \sum_{g=0}^{k}\sum_{f=0}^{f_{g}}b_{f,f_{g}}s_{g}^{f}
      }
    \mid a_{i,r_{j}},b_{f,f_{g}}, \in F \text{ en } s_{j}, s_{f} \in S 
  \right\}
  \]
  \zb
\end{ei}

\begin{st}
  Zij $F,+,\cdot$ en $S,+,\cdot$ velden en $E,+,\cdot$ een uitbreiding van zowal $F,+,\cdot$ als $S,+,\cdot$.
  \[ S = S_{1} \cup S_{2} \Rightarrow F(S) = F(S_{1})(S_{2}) \]
\extra{bewijs}
\end{st}

\begin{st}
  \label{st:rekenregel-uitbreidingen-basis}
  Zij $F,+,\cdot$ een veld en $E,+,\cdot$ een uitbreiding van $F,+,\cdot$.
  Zij $u_{1},u_{2}$ elementen van $E$.
  \[ F(u_{1},u_{2}) = F(u_{2})(u_{1})\]
\extra{bewijs}
\end{st}

\begin{st}
  \label{st:rekenregel-uitbreidingen}
  Zij $F,+,\cdot$ een veld en $E,+,\cdot$ een uitbreiding van $F,+,\cdot$.
  Zij $u_{1},\dotsc,u_{n}$ elementen van $E$.
  \[ F(u_{1},\dotsc,u_{n}) = F(u_{1},\dotsc,u_{n-1})(u_{n})\]
\extra{bewijs}
\end{st}

\begin{st}
  De \term{stelling van het primitieve element}\\
  Zij $F,+,\cdot$ een veld van karakteristiek $0$.
  Zij $a$ en $b$ elementen in een uitbreiding van $F$ die algebra\"isch zijn over $F$,
  dan bestaat er een element $c\in F(a,b)$ zodat het volgende geldt:
  \[ F(a,b) = F(c) \]
  \zb
\end{st}

\begin{opm}
  Elke eindige uitbreiding, in een veld van karakteristiek $0$, is dus in feite een enkelvoudige uitbreiding.
\end{opm}

\begin{de}
  Het element dat van een eindige uitbreiding een enkelvoudige uitbreiding maakt noemt men een \term{primitief element}.
\end{de}

\subsection{Algebra\"ische uitbreidingen}
\label{sec:algebr-uitbr}
\begin{de}
  Zij $F,+,\cdot$ een veld en $E,+,\cdot$ een uitbreiding van $F,+,\cdot$.
  $E$ is \term{algebra\"isch} over $F$ als elk element $u\in E$ algebra\"isch is over $F$.
  Anders noemen we $E,+,\cdot$ \term{transcendent}.
\end{de}

\begin{pr}
  \label{st:uitbreiding-eindig-dan-algebraisch}
  Zij $F,+,\cdot$ een veld en $E,+,\cdot$ een eindige uitbreiding van $F,+,\cdot$, dan is $E,+,\cdot$ algebra\"isch over $F,+,\cdot$.

  \begin{proof}
    Noem $i$ het eenheidselement van $F,+,\cdot$ (en dus ook van $E,+,\cdot$\stref{st:criterium-deelveld}).
    Zij $n$ de dimensie van $E,+,\cdot$ over $F,+,\cdot$:
    \[ n = [E:F] \]
    We kiezen nu een willekeurig element $u$ uit $E$ en tonen aan dat $u$ algebra\"isch is over $F,+,\cdot$.
    Beschouw de $n+1$ elementen $i,u,u^{2},\dotsc,u^{n}$.
    Deze zijn lineair onafhankelijk over $F,+,\cdot$.\waarom
    Er bestaan dus elementen $a_{i}\in F$ die niet allemaal nul zijn als volgt:
    \[ \sum_{i=0}^{n}a_{n}u^{n} = 0 \]
    Dit betekent dat de veelterm $\sum_{i=0}^{n}a_{n}X^{n}$, een niet-nul-veelterm in $F[X]$, $u$ als nulpunt heeft.
  \end{proof}
  \begin{proof}
    Bewijs uit het ongerijmde:\\
    Stel dat $E,+,\cdot$ niet algebra\"isch is over $F,+,\cdot$, dan bestaat er een element $u\in E$ dat transcendent is over $F,+,\cdot$.
    De velduibreiding $F(u)$ van $F,+,\cdot$ is dan zeker niet eindig\stref{st:uitbreiding-van-transcendente-u-aan-veld-graad-oneindig}, laat staan de uitbreiding $E,+,\cdot$ van $F,+,\cdot$.\stref{st:productformule}
    \[ F \subseteq F(u) \subseteq E \]
  \end{proof}
\end{pr}

\begin{tvb}
  Het omgekeerde van deze eigenschap geldt niet.
  
  \begin{proof}
    We moeten aantonen dat er een uitbreiding $E,+,\cdot$ van een veld $F,+,\cdot$ bestaat zodat $E,+,\cdot$ algebra\"isch is over $F,+,\cdot$, die niet eindig is.

    Beschouw de uitbreiding $E,+,\cdot$ van $\mathbb{Q},+,\cdot$:
    \[ E = \mathbb{Q}(\{\sqrt[n]{2} \mid n\in \mathbb{N}_{0}\}) \]
    \begin{itemize}
    \item $E,+,\cdot$ is algebra\"isch over $\mathbb{Q}$.
      $\sqrt[n]{k}$ is algebra\"isch over $\mathbb{Q}$ voor elke $k\in \mathbb{N}_{0}$.
      De veelterm $X^{n}-k$ heeft immers $\sqrt[n]{k}$ als wortel.
      Dit betekent dat $\mathbb{Q}(\sqrt[n]{k})$ algebra\"isch is over $\mathbb{Q}$ voor elke $n\in \mathbb{N}_{0}$. Omdat $\mathbb{Q}(\sqrt[n]{k})(\sqrt[m]{k})$ gelijk is aan  $\mathbb{Q}(\sqrt[n]{k},\sqrt[m]{k})$\stref{st:rekenregel-uitbreidingen} voor elke $n$ en $m$ en algebra\"isch over $\mathbb{Q}(\sqrt[n]{k})$, kunnen we besluiten dat $\mathbb{Q}(\{\sqrt[n]{2} \mid n\in \mathbb{N}_{0}\})$ algebra\"isch is over $\mathbb{Q}$. \clarify{waarom moeten we het enkel bewijzen voor deze en niet voor alle elementen?}
    \item $[E:\mathbb{Q}]$ is niet eindig.
      \extra{bewijs}
    \end{itemize}
  \end{proof}
\end{tvb}

\begin{st}
\label{st:element-asa-uitbreiding-algebraisch}
  Zij $F,+,\cdot$ een veld en $E,+,\cdot$ een uitbreiding van $F,+,\cdot$.
  Zij $u_{1},\dotsc,u_{n}$ elementen van $E$.
  \[ F(u) \text{ is algebra\"isch over } F \Leftrightarrow u \text{ is algebra\"isch over } F \]

  \begin{proof}
    Bewijs van een equivalentie
    \begin{itemize}
    \item $\Rightarrow$: dit staat in de definitie.
    \item $\Leftarrow$: De uitbreiding van $F$ tot $F(u)$ is eindig omdat $u$ algebra\"isch is over $F$\stref{st:uitbreiding-van-algebraische-u-aan-veld-graad-minimale-veelterm}, en bijgevolg eindig.\stref{st:uitbreiding-eindig-dan-algebraisch}
    \end{itemize}
  \end{proof}
\end{st}

\begin{st}
  Zij $F,+,\cdot$ een veld met nulelement $e$ en $E,+,\cdot$ een uitbreiding van $F,+,\cdot$.
  Zij $a$ en $b$ elementen van $E$.
  Als $a$ en $b$ algebra\"isch zijn over $F$, dan zijn ook de volgende elementen algebra\"isch over $F$:
  \begin{itemize}
  \item $a+b$
  \item $a-b$
  \item $a\cdot b$
  \item $\frac{a}{b}$ als $b \neq e$
  \end{itemize}

  \begin{proof}
    We beschouwen de volgende velduitbreidingen:
    \[
    F \subseteq F(a) \subseteq F(a)(b)
    \]
    Omdat $a$ algebra\"isch is over $F,+,\cdot$ is $[F(a):F]$ eindig.\stref{st:uitbreiding-van-algebraische-u-aan-veld-graad-minimale-veelterm}
    Emdat $b$ ook algebra\"isch is over $F,+,\cdot$ en dus zeker over $F(a)$, is $[f(a)(b):F(a)]$ eveneens eindig.
    Vanwege de productformule\stref{st:productformule} is $[F(a)(b):F]$ dan ook eindig.
    Als deelruimte van de eindigdimensionale vectorruimte $F(a)(b)$ is $F(a+b)$ ook eindigdimensionaal over $F,+,\cdot$. $F(a+b)$ is dus algebra\"isch over $F,+,\cdot$ {st:uitbreiding-eindig-dan-algebraisch}
    De andere beweringen zijn geheel analoog te bewijzen.
  \end{proof}
\end{st}

\begin{gev}
  Zij $F,+,\cdot$ een veld en $E,+,\cdot$ een uitbreiding van $F,+,\cdot$.
  De elementen van $E$ die algebra\"isch zijn over $F,+,\cdot$ vormen een deelveld van $E,+,\cdot$ en een uitbreiding van $F,+,\cdot$.
\extra{bewijs}
\end{gev}

\begin{de}
  Zij $F,+,\cdot$ een veld en $E,+,\cdot$ een uitbreiding van $F,+,\cdot$.
  Het deelveld van de elementen van $E$ die algebra\"isch zijn over $F,+,\cdot$ noemen we de algebra\"ische sluiting van $F,+,\cdot$ in $E,+,\cdot$.
\end{de}

\begin{st}
\label{st:transitiviteit-algebraisch-zijn}
  De \term{transitiviteit van het algebra\"isch zijn}.\\
  Zij $F,+,\cdot$ een veld, $L,+,\cdot$ een uitbreiding van $F,+,\cdot$ en $E,+,\cdot$ een uitbreiding van $L,+,\cdot$.
  Als $E,+,\cdot$ algebra\"isch is over $L,+,\cdot$ en $L,+,\cdot$ algebra\"isch over $F,+,\cdot$, dan is $E,+,\cdot$ algebra\"isch over $F,+,\cdot$.

  \begin{proof}
    Kies een willekeurig element $u\in E$ uit.
    We moeten dan aantonen dat $u$ algebra\"isch is over $F,+,\cdot$ en zullen dit doen via de eindigheid van $[F(u):F]$.\stref{st:uitbreiding-eindig-dan-algebraisch}
    \begin{itemize}
    \item Omdat $u$ algebra\"isch is over $L$, bestaat er een
      niet-nul-veelterm $p\in L[X]$ met $u$ als nulpunt.
      \[ p = \sum_{i=1}^{n}a_{i}X^{i} \] We beschouwen nu de volgende
      keten van velduitbreidingen:
      \[ F \subseteq F(a_{0}) \subseteq F(a_{0},a_{1}) \dotsb
      \subseteq F(a_{0},a_{1},\dotsc,a_{n}) \] Omdat $a_{0}$
      algebra\"isch is over $F,+,\cdot$ (want het is er een element
      van) is de eerste uitbreiding $F \subseteq F(a_{0})$ eindig.
      Ook $a_{1}$ is algebra\"isch over $F,+,\cdot$, dus $F(a_{0})
      \subseteq F(a_{0},a_{1})$ is ook eindig.  Bij wijze van
      voortzetting van deze redenering is heel de keten eindig,
      waardoor $F \subseteq F(a_{0},a_{1},\dotsc,a_{n})$ eindig is via
      de productformule.\stref{st:productformule}
    \item Kijk nu terug naar $p$.
      $u$ is algebra\"isch over het deelveld $F(a_{0},a_{1},\dotsc,a_{n})$ van $L,+,\cdot$.
      De uitbreiding van $F(a_{0},a_{1},\dotsc,a_{n})$ tot $F(a_{0},a_{1},\dotsc,a_{n},u)$ is dus opnieuw eindig en bijgevolg is de uitbreiding van $F$ tot $F(a_{0},a_{1},\dotsc,a_{n},u)$ ook eindig.
    \item Vanwege de rekenregel van uitbreidingen\stref{st:rekenregel-uitbreidingen} is $F(u)$ ook eindig.
      \[ F(a_{0},a_{1},\dotsc,a_{n},u) = F(u)(a_{0},a_{1},\dotsc,a_{n}) \]
    \end{itemize}
  \end{proof}
\end{st}

\subsection{Ontbindingsvelden}
\label{sec:ontbindingsvelden}

\begin{de}
  Zij $K,+,\cdot$ een veld en $f$ een niet-constante veelterm in $K[X]$.
  Een \term{ontbindingsveld} van $f$ over $K,+,\cdot$ is een uitbreidingsveld $E,+,\cdot$ van $K,+,\cdot$ met de volgende eigenschappen.
  \begin{itemize}
  \item $\exists c, a_{i} \in E:\ f = c(X-a_{1})(X-a_{2})\dotsc(X-a_{n})$
  \item Er is geen enkel veld $L,+,\cdot$, strikt tussen $K,+,\cdot$ en $E,+,\cdot$.
  \end{itemize}
  Met andere woorden is het ontbindingsveld van een veelterm in een veld het kleinste veld waarin $f$ ontbonden kan worden in eerstegraadsveeltermen.
\end{de}

\begin{st}
  Zij $K,+,\cdot$ een veld en $f$ een niet-constante veelterm in $K[X]$.
  De volgende eigenschappen vormen voor een uitbreidingsveld $E,+,\cdot$ van $K,+,\cdot$ een equivalente definite van een ontbindingsveld.
  \begin{itemize}
  \item $\exists c, a_{i} \in E:\ f = c(X-a_{1})(X-a_{2})\dotsc(X-a_{n})$
  \item $E = K(a_{1},a_{2},\dotsc,a_{n})$
  \end{itemize}
  We kunnen het ontbindingsveld van een veelterm $f$ over een veld $K,+,\cdot$ dus ook zien als ``de toevoeging van de wortels van $f$ aan $K,+,\cdot$''.

  \begin{proof}
    Bewijs van een equivalentie
    \begin{itemize}
    \item $\Rightarrow$\\
      Noem $E,+,\cdot$ het ontbindindingsveld van $f$ in $K[X]$, dan is $E$ zeker een deel van $ K(a_{1},a_{2},\dotsc,a_{n})$.\waarom
      Ook is $K(a_{1},a_{2},\dotsc,a_{n})$ een deel van $E$, want alle $a_{i}$ en $k$ zitten in $E$.
      \waarom
    \item $\Leftarrow$\\
    \end{itemize}
  \end{proof}
\end{st}

\begin{st}
  Zij $K,+,\cdot$ een veld en $f$ een niet-constante veelterm in $K[X]$ van graad $n$.
  Er bestaat een ontbindingsveld $E,+,\cdot$ van $f$ over $K,+,\cdot$ zodat het volgende geldt:
  \[ [E:K] \le n! \]

  \begin{proof}
    We bewijzen dit per inductie op de graad $n$ van $f$.
    \begin{itemize}
    \item De bewering geldt voor $n=1$.
      Elke veelterm van graad $1$ is zichzelf na ontbinding in eerstegraadsveeltermen.
      Het ontbindingsveld is dus $K,+,\cdot$ zelf.
    \item Als we ervan uitgaan dat de bewering geldt voor een bepaalde $n=k$ ...
    \item ... dan bewijzen we nu dat daaruit volgt dat de bewering geldt voor $n=k+1$.
      Als alle irreducibele factoren van $f$ in $K[X]$ graad $1$ hebben, is het ontbindingsveld opnieuw $K,+,\cdot$ zelf.
      Stel daarom dat $f$ deelbaar is door minstens \'e\'en irreducibele veelterm $p\in K[X]$ van graad groter dan $1$. 
      Er bestaat een uitbreiding van $K,+,\cdot$ waarin $p$ een wortel $u$ heeft.\stref{st:fundamentele-stelling-van-veldentheorie}
      Omdat $p$ een minimale veelterm is van $u$ over $K,+,\cdot$ is geldt het volgende.\stref{st:uitbreiding-van-algebraische-u-aan-veld-graad-minimale-veelterm}
      \[ [K(u):K] = gr(p) \le n \]
      We werken nu verder over $K(u)$.
      Aangezien $u$ een wortel is van $p$ is $X-u$ een deler van $p$ in $K(u)[X]$.
      Hieruit volgt dan dat ook $f$ deelbaar is door $X-u$ in $K(u)[X]$.
      Er bestaat dus een $h\in K(u)[X]$ van graad $k$ als volgt:
      \[ f = (X-u)h \]
      De inductiehypothese zegt ons dan dat er een ontbindingsveld $E,+,\cdot$ van $h$ over $K(u)$ bestaat als volgt:
      \[ [E:K(u)] \le (n-1)! \]
      $E,+,\cdot$ is nu een ontbindingsveld van $f$ over $K,+,\cdot$.
      Alle wortel van $h$: $a_{1},\dotsc a_{k}$ zitten immers in $E$, alsook $u$.
      Omdat $E,+,\cdot$ het kleinste deelveld is dat $K(u)$ en de $a_{i}$ omvat, moet $E$ als volgt geschreven kunnen worden:
      \[ E = K(u,a_{1},\dotsc,a_{k}) \]
      Uit de productformule volgt dan de bewering voor $n=k+1$.      
    \end{itemize}
  \end{proof}
\end{st}

\begin{st}
  Zij $K,+,\cdot$ een veld en $f$ een niet-constante veelterm in $K[X]$.
  Het ontbindingsveld van $f$ over $K,+,\cdot$ is uniek op isomorfisme na.
  \zb
\end{st}

\subsection{Cyclotome uitbreidingen en Cyclotome veeltermen}
\label{sec:cycl-uitbr-en}

\begin{de}
  Definieer $\omega=\omega_{n}$ als de $n$-de eenheidswortel van $X^{n}-1$ in $\mathbb{C},+,\cdot$.
  \[ \omega = e^{\frac{2\pi i}{n}} = \cos\frac{2\pi}{n} + i\sin\frac{2\pi}{n} \]
\end{de}

\begin{st}
  Het ontbindingsveld van $X^{n}-1$ over $\mathbb{Q}$ is de enkelvoudige uitbreiding $\mathbb{Q}(\omega)$.

  \begin{proof}
    De $n$-de machten van $\omega$ zijn de nulpunten van $X^{n}-1$, dus $\mathbb{Q}(1,\omega,\omega^{2},\dotsc,\omega^{n-1}) = \mathbb{Q}(\omega)$. is het ontbindingsveld over $\mathbb{Q}$.
  \end{proof}
\end{st}

\begin{de}
  Voor $n\in \mathbb{N}_{0}$ noemt men $\mathbb{Q}(\omega_{n})$ het $n$-de \term{cyclotome veld}.
\end{de}

\begin{de}
  Voor elk deelveld $F,+,\cdot$ van $\mathbb{C},+,\cdot$ noemt men $F(\omega_{n})$ de $n$-de \term{cyclotome uitbreiding} van $F,+,\cdot$.
\end{de}

\begin{st}
  Een regelmatige $n$-hoek is construeerbaar met enkel passer in lineaal als en slechts als $n$ van de volgende vorm is.
  \[ n = 2^{k}p_{1}p_{2}\dotsc p_{r} \text{ met } k\ge 0 \]
  In bovenstaande formule zijn de $p_{i}$ onderling verschillende priemgetallen van de vorm $p^{2^{m}} +1$.
  \zb
\end{st}

\begin{de}
  De $n$ machten van de $n$-de eenheidswortel, uitgerust met de vermenigvuldiging, vormen een cyclische deelgroep van $\mathbb{C}^{\times}$.
  De generatoren van deze groep noemen we de priemitieve $n$-de eenheidswortels.
\end{de}

\begin{ei}
  De primitieve $n$-de eenheidswortels zijn de $\omega^{k}$ waarbij $k$ en $n$ relatief priem zijn.
\extra{bewijs}
\end{ei}

\begin{de}
  Zij $n\in \mathbb{N}_{0}$ een getal.
  De $n$-de \term{cyclotome veelterm} (over $\mathbb{Q}$) is $\Phi_{n}$.
  \[ \Phi_{n} = \prod_{0\le i \le n-1,\ ggd(i,n)= 1}(X-\omega_{n}^{i})\]
  Merk op dat $\Phi_{n}$ monisch is in graad $\phi(n)$.
\end{de}

\begin{ei}
  \label{ei:formule-n-degraads-eenheidsveelterm}
  Zij $n\in \mathbb{N}_{0}$ een getal.
  \[ X^{n}-1 = \prod_{1\le d\le n,\ d|n}\Phi_{d} \]
  \begin{proof}
    Het rechterlid heet precies alle $n$-de eenheidswortels (zonder herhalingen).\waarom
    Aangezien beide veeltermen monisch zijn met juist dezelfde wortels moeten ze gelijk zijn.\stref{st:veeltermen-n-gelijke-beelden-gelijk}
  \end{proof}
\end{ei}

\begin{pr}
  Voor elke $n\in \mathbb{N}_{0}$ geldt $\Phi_{n} \in \mathbb{Z}[X]$.

  \begin{proof}
    Bewijs door inductie op $n$.
    \begin{itemize}
    \item De bewering geldt voor $n=1$: $\Phi_{1} = X-1 \in \mathbb{Z}[X]$. 
    \item Als de bewering geldt voor een bepaalde $n=k$ ...
    \item ... dan bewijzen we nu dat daaruit volgt dat de bewering geldt voor $n=k+1$.\\
      De inductiehypothese zegt dat alle $\Phi_{d}$ voor $d<n$ gehele co\"efficienten hebben.
      Beschouw nu $g$:
      \[ X^{k}-1 = g = \prod_{1\le d\le k,\ d|k}\Phi_{d}\]
      De vorige eigenschap\eiref{ei:formule-n-degraads-eenheidsveelterm} zegt ons nu het volgende:
      \[ X^{k+1}-1 = \Phi_{k+1}\cdot g\]
      Hieruit volgt meteen dat $\Phi_{k+1}$ een gehele veelterm is.
    \end{itemize}
  \end{proof}
\end{pr}

\begin{st}
  De cyclotome veelterm $\Phi_{n}$ is irreducibel in $\mathbb{Z}[X]$ voor elke $n\in \mathbb{N}_{0}$.
  \zb
\end{st}

\begin{gev}
  De minimale veelterm van $\omega_{n}$ over $\mathbb{Q}$ is de cyclotome veelterm $\Phi_{n}$.
  \zb
\end{gev}

\begin{gev} $\mathbb{Q}(\omega_{n}) \cong \mathbb{Q}[X]/(\Phi_{n})$ en $[\mathbb{Q}(\omega_{n}):\mathbb{Q}] = \phi_{n}$ \zb
\end{gev}

\subsection{Meervoudige wortels}
\label{sec:meervoudige-wortels}

\begin{de}
  Zij $K,+,\cdot$ een veld en $f\in K[X]$ een veelterm over $K,+,\cdot$.
  De veelterm $f$ heeft een \term{meervoudige wortel} $a$ in een uitbreiding $L,+,\cdot$ van $K,+,\cdot$ als er een veelterm $g\in L[X]$ over $L,+,\cdot$ bestaat zodat het volgende geldt binnen $L[X]$:
  \[ f = (X-a)^{2}g\]
\end{de}

\begin{de}
  Zij $K,+,\cdot$ een veld en $f\in K[X]$ een veelterm over $K,+,\cdot$.
  De \term{afgeleide} van $f$ is $f'\in K[X]$:
  \[ f= \sum_{i=0}^{n}a_{i}X^{i} \longrightarrow f' = \sum_{i=1}^{n}ia_{i}X^{i-1} \]
\end{de}

\begin{st}
  Rekenregels voor afgeleiden\\
  Zij $K,+,\cdot$ een veld, $c$ een element van $F$(???) en $f,g$ veeltermen over $K,+,\cdot$.
  \begin{itemize}
  \item $(cf)'= cf'$
  \item $(f+g)' = f'+g'$
  \item $(fg)' = f'g+ fg'$
  \end{itemize}
\extra{bewijs}
\end{st}

\begin{st}
  \label{st:criterium-meervoudige-wortel}
  Het \term{criterium voor een meervoudige wortel}\\
  Zij $K,+,\cdot$ een veld en $f\in K[X]$ een veelterm over $K,+,\cdot$.
  $f$ heeft een meervoudige wortel (in een uitbreiding van $K,+,\cdot$) als en slechts als $f$ en $f'$ een gemeenschappelijke niet-constante deler hebben in $K[X]$.

  \begin{proof}
    Bewijs van een equivalentie.
    \begin{itemize}
    \item $\Rightarrow$\\
      Stel dat $f$ een meervoudige wortel $a$ heeft in een uitbreiding $E,+,\cdot$ van $K,+,\cdot$. (Merk op dat het niet uitmaakt welke uitbreiding.)
      Er bestaat dan een veelterm $g\in E[X]$ zodat $f$ als volgt gescreven kan worden:
      \[ f = (X-a)^{2}g \]
      Vanwege de rekenregels voor afgelijden kunnen we $f'$ schrijven als volgt:
      \[ f' = (X-a)^{2}g' + 2(X-a)g \]
      $X-a$ is dus een deler van zowel $f$ als $f'$ in $E[X]$.
      Stel nu dat $f$ en $f'$ geen gemeenschappelijke niet-constante deler zouden hebben in het HID $K[X]$.\stref{st:veeltermen-over-veld-hid}
      De grootste gemeenschappelijke deler van $f$ en $f'$ in $K[X]$ is dan $1$.
      Dit betekent dat er veeltermen $h_{1}$ en $h_{2}\in K[X]$ bestaan als volgt.\stref{st:stelling-van-bezout}
      \[ f\cdot h_{1}+ f'\cdot h_{2} = 1 \]
      Wanneer we deze gelijkheid beschouwen in $E[X]$, maar dat zou betekenen dat $X-a$ een deler is van $1$ in $E[X]$. 
      \[ (X-a)^{2}gh_{1} + (X-a)^{2}g' + 2(X-a)gh_{2} = 1 \]
      Contradictie.
    \item $\Leftarrow$\\
      Stel dat $f$ en $f'$ een niet-constante gemeenschappelijke deler hebben in $K[X]$.
      Neem dan een wertel $a$ van deze deler in een uitbreiding $E,+,\cdot$ van $K,+,\cdot$ (die bestaat immers altijd\stref{st:fundamentele-stelling-van-veldentheorie}).
      Er bestaat dan een veelterm $g\in E[X]$ zodat $f$ als volgt geschreven kan worden:
      \[ f=(X-a)\cdot g\]
    \end{itemize}
  \end{proof}
\end{st}

\begin{st}
  Zij $K,+,\cdot$ een veld en $f$ een irriducibele veelterm in $K[X]$.
  \begin{itemize}
  \item Als $char(K)$ nul is, dan heeft $f$ geen meervoudige wortel.
  \item Als $char(K)$ niet nul is, dan heeft $f$ een meervoudige wortel als en slechts als er een $g\in K[X]$ bestaat zodat $f=g(X^{p})$ geldt.
  \end{itemize}

  \begin{proof}
    Het criterium voor meervoudige delers zegt ons dat $f$ een meervoudige wortel heeft als en slechts als $f$ en $f'$ een niet-constante gemeenschappelijke deler hebben in $K[X]$.\stref{st:criterium-meervoudige-wortel}
    Omdat $f$ irredicibel is, is echter enkel $f$ een niet-constante deler van $f$ (op eenheid na).\deref{de:irreducibel}
    $f$ heeft dus een meervoudige wortel als en slechts als $f'$ de nulveelterm is (want dan alleen is $f'$ deelbaar door $f$).
    Noteer verder $f$ en $f'$ als volgt:
    \[ f = \sum_{i=0}^{n}a_{i}X^{i} \text{ en } f' = \sum_{i=1}^{n}ia_{i}X^{i-1} \]
    \begin{itemize}
    \item Als de karakteristiek van $K,+,\cdot$ nul is, dan kan dit alleen als alle $a_{i}$ met $i\neq 0$ nul zijn, maar dat kan niet want dan dan zou $f=a_{0}$ een constante zijn.
    \item Als de karakteristiek van $K,+,\cdot$ (een priemgetal\gevref{gev:karakteristiek-van-domein-is-priem}) $p$ is, dan is dit equivalent mat $a_{i}=0$ wanneer $p$ $i$ niet deelt (anders is $ia_{i}$ al zeker nul).
      \[ f = \sum_{i=0,\ p\mid i}^{n}a_{i}X^{i} \]
      \clarify{hoe komen we aan deze $f$ en daarna aan de $g$?}
      Dit is precies wanneer $x$ van de vorm $f=g(X^{p})$ is met $g\in K[X]$.
    \end{itemize}
  \end{proof}
\end{st}

\begin{st}
  Een irreducibele veelterm over een eindig veld van karakteristiek niet nul heeft geen meervoudige wortel.
  \zb
\end{st}

\section{Algebra\"isch gesloten velden}
\label{sec:algebr-gesl-veld}

\begin{de}
  Een veld $F,+,\cdot$ is \term{algebra\"isch gesloten} als en slechts als elke niet-constante veelterm in $F[X]$ een wortel heeft in $F,+,\cdot$.
\end{de}

\begin{st}
  Alsternatieve definitie 1\\
  Een veld $F,+,\cdot$ is algebra\"isch gesloten als en slechts als elke niet-constante veelterm in $F[X]$ een product is van lineaire veeltermen in $F[X]$.

  \begin{proof}
    Bewijs van een equivalentie.
    \begin{itemize}
    \item $\Rightarrow$\\
      Kies een niet-constante veelterm $f$ in $F[X]$.
      We bewijzen dat $f$ een product is van lineaire veeltermen in $F[X]$.
      Omdat $F,+,\cdot$ algebra\"isch gesloten is, bestaat er een wortel $u$ van $f$ in $F$.
      $X-u$ is dan een deler van $f$\stref{st:domein-nulpunten-delen-veelterm} en bijgevolg bestaat er een $g$ als volgt:
      \[ f= (X-u)g \]
      Als $g$ constant is de bewering bewezen.
      Anders kunnen we opnieuw een wortel van $g$ nemen in $F$.
      We zetten deze redenering verder tot de bewering bewezen is.
    \item $\Leftarrow$: Omgekeerd zorgen de delers $(X-u_{i})$ voor elke niet-constante veelterm $f$ in $F[X]$ voor nulpunten $u_{i}$.
    \end{itemize}
  \end{proof}
\end{st}

\begin{st}
  Alsternatieve definitie 2\\
  Een veld $F,+,\cdot$ is algebra\"isch gesloten als en slechts als alle irreducibele veeltermen in $F[X]$ graad $1$ hebben.

  \begin{proof}
    Bewijs van een equivalentie.
    \begin{itemize}
    \item $\Rightarrow$\\
      Kies een veelterm $f$ uit $F[X]$ van graad groter dan $1$, anders is $f$ per definitie irreducibel.
      $f$ kan dan geschreven worden als een product van lineare veeltermen, en is bijgevolg reducibel.
    \item $\Leftarrow$\\
      Elke niet-constante veelterm $f$ uit $F[X]$ kan geschreven worden als een product van eerstegraads (irreducibele) veeltermen.
      Elk van die veeltermen is van de vorm $aX+b$ en $-frac{b}{a}$ is er dan een wortel van.
    \end{itemize}
  \end{proof}
\end{st}

\begin{ei}
  Een veld $F,+,\cdot$ is algebra\"isch gesloten als en slechts als voor elke algebra\"ische uitbreiding $E,+,\cdot$ van $F,+,\cdot$ $E$ gelijk is aan $F$.

  \begin{proof}
    Bewijs van een equivalentie.
    \begin{itemize}
    \item $\Rightarrow$\\
      Elk element $u\in E$ zit dan ook in $F$.
      Zij immers $f$ een minimale veelterm van $u$ over $F,+,\cdot$, dan moet $f$ graad $1$ hebben omdat $F,+,\cdot$ algebra\"isch gesloten is.
    \item $\Leftarrow$\\
    Zij $E,+,\cdot$ een algebra\"ische uitbreiding van $F,+,\cdot$.
      Kies een willekeurige irreducibele veelterm $f\in F[X]$.
      $f$ moet dan graad $1$ hebben. Bijgevolg is $F,+,\cdot$ algebra\"isch gesloten.
      Kies immers een uitbreiding $E,+,\cdot$ van $F,+,\cdot$ met een wortel $u$ van $f$ in $E$(die bestaat immers altijd\stref{st:fundamentele-stelling-van-veldentheorie}).
      $F(u)$ is dan een algebra\"ische uitbreiding van $F,+,\cdot$ met uitbreidingsgraad $gr(f)$.\stref{st:uitbreiding-van-algebraische-u-aan-veld-graad-minimale-veelterm}
      Omdat $F,+,\cdot$ algebra\"isch gesloten is, is $F(u)$ gelijk aan $F$ en is $gr(f)$ bijgevolg $1$. 
    \end{itemize}
  \end{proof}
\end{ei}

\begin{st}
  De \term{hoofdstelling van de (klassieke) algebra}\\
  $\mathbb{C}$ is algebra\"isch gesloten.
\zb
\end{st}


\begin{pr}
  Het veld der algebra\"ische getallen in $\mathbb{C}$ over $\mathbb{Q}$ is algebra\"isch gesloten.
  \[ \overline{\mathbb{Q}} = \{ a \in \mathbb{C} \ |\ a \text{ is algebra\"isch over } \mathbb{Q} \} \]

  \begin{proof}
    Merk eerst op dat $\overline{\mathbb{Q}}$ per definitie algebra\"isch is over $\mathbb{Q}$.
    Zij $f$ een niet-constante veelterm over $\overline{\mathbb{Q}}$, dan heeft $f$ zeker een wortel $u$ in $\mathbb{C}$ want $\mathbb{C}$ is algebra\"isch gesloten.
    Omdat $\overline{\mathbb{Q}}$, algebra\"isch is over $\overline{Q}$ en $\overline{Q}$ algebra\"isch is over $\mathbb{Q}$, is $\overline{\mathbb{Q}}(u)$ ook algebra\"isch over $\mathbb{Q}$.\stref{st:transitiviteit-algebraisch-zijn}
    In het bijzonder is dan $u$ algebra\"isch over $\mathbb{Q}$ en dus een element van $\overline{\mathbb{Q}}$.
  \end{proof}
\end{pr}

\begin{de}
  Zij $F,+,\cdot$ een veld.
  Een \term{algebra\"ische sluiting} van $F,+,\cdot$ is een algebra\"isch gesloten uitbreiding $\overline{F},+,\cdot$ van $F,+,\cdot$ die algebra\"isch is over $F,+,\cdot$.
\end{de}

\begin{lem}
  \label{lem:bestaan-algebraische-sluiting}
  Zij $K,+,\cdot$ een veld, dan bestaat er een algebra\"ische sluiting $K_{1},+,\cdot$ zodat elke niet-constante veelterm in $K[X]$ een wortel heeft in $K_{1}$.

  \begin{proof}
    Beschouw eerst $I$:
    \[ I = \{ f\in K[X] \mid f \text{ is irreducibel } \} \]
    Beschouw vervolgens voor elke veellterm $f$ in $I$ een veranderlijke $X_{f}$.
    We zullen verder de veeltermenring $R$ beschouwen:
    \[ R = K[X_{f}] \]
    Zij $J$ het ideaal van $R$ voortgebracht door de elementen $f(X_{f})$, met andere woorden de irreducibele veeltermen in hun eigen veranderlijke.
    We beweren nu dat $J$ verschillend is van $R$.
    \begin{itemize}
    \item 
      Stel immers dat $J$ gelijk is aan $R$, dan zit het eenheidselement $i$ van $K,+,\cdot$ in $J$.
      Dit betekent dat er (een eindig aantal\waarom) voortbrengers $f_{i}(X_{f_{i}})$ en elementen $r_{i}\in R$ bestaan als volgt:
      \[ i = \sum_{i}r_{i}f_{i}(X_{f_{i}}) \] Merk hierin op dat de
      veeltermen $r_{i}$ andere veranderlijken kunnen bevatten dat de
      $X_{f_{i}}$.  Neem nu een algebra\"ische uitbreiding $K'$ van $K$
      waarin elke $f_{i}$ een wortel $u_{i}$ in $K'$ heeft. (dit kan
      altijd \stref{st:fundamentele-stelling-van-veldentheorie}, maar
      enkel omdat we maar een eindig aantal $f_{i}$ beschouwen\waarom.)
      We substitueren nu in de gelijkheid hierboven de veranderlijken
      $X_{f_{i}}$ door $u_{i}$ en de andere optredende veranderlijken
      door een element naar keuze.  We verkrijgen dan dat het
      eenheidselement gelijk is aan het nulelement.  Contradictie.
    \end{itemize}
    Omdat $J$ niet gelijk is aan $R$, bestaat er een maximaal ideaal $M,+,\cdot$ van $R,+,\cdot$ als volgt \waarom:
    \[ J \subseteq M \subsetneq R \]
    Het veld $K_{1} = \nicefrac{R}{M}$ zal voldoen aan de gewenste voorwaarden.
    We beschouwen hiervoor het morfisme $j$ dat de samenstelling is van de inbedding van $K$ in $R$ en de natuurlijke afbeelding $R\rightarrow R/M$:
    \[ j: K \hookrightarrow R \rightarrow K_{1}: a \rightarrow \bar{a} \]
    \begin{itemize}
    \item $j$ is injectief.\stref{st:veldmorfisme-is-injectief}
      \begin{itemize}
      \item 
      \end{itemize}
      We kunnen daarom $K$ identificeren met zijn beeld $j(K)$ en dus als deelveld van $K_{1}$ beschouwen.\opmref{opm:veldmorfisme-is-injectief}
    \item Elke irreducibele veelterm $f\in K[X]$ heeft nu en worte in $K_{1}$, namelijk $\bar{X_{f}}$.\clarify{verifieer!}
      Bijgevolg heeft ook elke niet-constante veelterm in $K[X]$ een wortel in $K_{1}$.\waarom
    \item Alle $\bar{X_{f}}$ zijn algebra\"isch over $K,+,\cdot$.
      Nu is elk element van $K_{1}$ een veelterm-uitdrukkingin een eindig aantal van deze $\bar{X}_{f}$ met co\"efficienten in $K$ en dus ook algebra\"isch over $K,+,\cdot$.
    \end{itemize}
  \end{proof}
\TODO{bekijk dit opnieuw, en goed! Schrijf er ook de gebruikte strategie bij.}
\end{lem}

\begin{st}
  Elk veld heeft een algebra\"ische sluiting.
  \begin{proof}
    Kies een algebra\"ische uitbreiding $K_{1}$ van $K_{0}=K$ die voldoet aan de voorwaarden uit het lemma.\lemref{lem:bestaan-algebraische-sluiting}
    Neem dan analoog een dergelijke uitbreiding $K_{2}$ van $K_{1}$.
    We construeren op die manier een keten algebra\"ische uitbreidingen $K_{i}$ zodat elke niet-constante veelterm in $K_{i}[X]$ een wortel heeft in $K_{i+1}$.
    \[ K_{i} \subseteq K_{i+1} \]
    Definieer nu $E$:
    \[ E = \cup_{i}K_{i} \]
    \begin{itemize}
    \item $E$ is een veld. (want een algebra\"ische uitbreiding is opnieuw een veld.)
    \item $E,+,\cdot$ is algebra\"isch over $K,+,\cdot$.
      Elk element van $E$ behoort immers tot een $K_{i}$ en is dus algebra\"isch over $K,+,\cdot$.
    \item $E,+,\cdot$ is algebra\"isch gesloten.
      Elke niet-constante veelterm $g\in E[X]$ heeft immers een wortel in $E,+,\cdot$:
      Er bestaat een $i$ zodat alle co\"efficienten van $g$ tot $K_{i}$ behoren (want er zijn er maar eindig veel.)
      $g$ is dan ook een element van $K_{i}[X]$.
      Per constructie van $K_{i+1}$ heeft $g$ nu een wortel in $K_{i+1} \subseteq E$.
    \end{itemize}
  \end{proof}
  \TODO{bekijk dit opnieuw, en goed!}
\end{st}

\section{Eindige velden}
\label{sec:eindige-velden}

\begin{st}
  \label{st:multiplicatieve-groep-van-veld-is-cyclisch}
  De multiplicatieve groep $F^{\times},\cdot$ van een eindig veld $F,+,\cdot$ is cyclisch.
  
  \begin{proof}
    Merk eerst op dat de orde $q$ van $F$ een macht van een priemgetal is.\stref{st:eindig-veld-orde-macht-van-priemgetal}
    De orde van $F^{\times}$ is dan $q-1$ omdat $F$, naast het nulelement enkel eenheden bevat.\deref{de:veld}
    Voor $q=2$ is de multiplicatieve groep van orde $1$ en dus zeker cyclisch.
    Stel dus dat $q$ groter is dan $2$.
    \begin{itemize}
    \item We ontbinden $q-1 = |F^{\times}|$ in priemfactoren:
      \[ q-1 = \prod_{i=1}^{k}p_{i}^{r_{i}}\]
      De $p_{i}$ zijn hier verschillende priemgetallen.
    \item Voor elke $i\in \{1,\dotsc,k\}$ geldt nu het volgende:
      Aangezien $\frac{q-1}{p_{i}}$ kleiner is dan $q-1$, bestaat er een element $a_{i}\in F^{\times}$ dat geen wortel is van de veelterm $X^{\frac{q-1}{p_{i}}}-1$. \waarom
      Beschouw nu het element $b_{i}$:
      \[ b_{i} = a_{i}^{\left(\frac{q-1}{p_{i}^{r_{i}}}\right)}\]
      \begin{itemize}
      \item De orde van $b_{i}$ in $F^{\times},\cdot$ is nu $p_{i}^{r_{i}}$.
        \[ b^{p_{i}^{r_{i}}} = a^{q-1} = 1 \quad\text{ en }\quad b^{p_{i}^{r_{i}-1}} = a_{i}^{\frac{q-1}{p_{i}}} \neq 1 \]
      \item De orde van $b$ is $q-1$.
        \[ b = \prod_{i=1}^{k}b_{i}\]
        Stel immers dat de orde van $b$ een echte deler is van $q-1$, dan is deze orde een deler van \'e\'en van de $\frac{q-1}{p_{i}}$.
        Noem deze $\frac{q-1}{p_{a}}$.
        \[ 1 = b^{\frac{q-1}{p_{a}}} = b_{a}^{\frac{q-1}{p_{1}}} \cdot \prod_{i=1, i\neq a}^{k}b_{i}^{\frac{q-1}{p_{1}}} \]
        Nu geldt voor elke $i\neq a$ dat $p_{i}^{r_{i}}$ een deler is van $\frac{q-1}{p_{1}}$.
        Vanwege het vorig puntje is $b_{i}^{\frac{q-1}{p_{a}}}$ dan gelijk aan $1$, zodat de orde van $b_{1}$ een deler moet zijn van $\frac{q-1}{p_{a}}$.
        Dit is echter in tegenspraak met het vorig puntje.
      \end{itemize}
    \item $F^{\times},\cdot$ is cyclisch want $b$ is er een generator voor.
    \end{itemize}
  \end{proof}
\TODO{wutwut, wut uun bewuis, opnieuw doen!}
\extra{nog een bewijs van deze stelling}
\end{st}

\begin{opm}
  Bovenstaand bewijs is niet constructief.
  (We construeren geen generator)
  Er is hiervan immers geen constructief bewijs gekend.
\end{opm}

\begin{gev}
  Zij $F,+,\cdot$ een veld van orde $q$, dan is de multiplicatieve groep $F^{\times},\cdot$ ervan isomorf met $\mathbb{Z}_{q-1},+$.

  \begin{proof}
    $F^{\times},\cdot$ is cyclisch en van orde $q-1$, dus isomorf met $\mathbb{Z}_{q-1},+$. \needed
  \end{proof}
\end{gev}

\begin{de}
  Een generator van de multiplicatieve groep $F^{\times},\cdot$ van een eindig veld $F,+,\cdot$ noemt men soms een \term{primitief element} van $F,+,\cdot$.
\end{de}

\subsection{Bestaan en uniciteit}
\label{sec:bestaan-en-uniciteit}

\begin{st}
  \label{st:priemveld-bestaat}
  Zij $p$ een priemgetal waarvoor $q=p^{r}$ geldt met $r\in \mathbb{N}_{0}$.
  Er bestaat een veld met $q$ elementen, namelijk een ontbindingsveld van $X^{q}-X$ over $\mathbb{Z}_{p}$.

  \begin{proof}
    Zij $K,+,\cdot$ een ontbindingsveld van $X^{q}-X$ over $\mathbb{Z}_{p}$.
    We zullen aantonen dat $K$ $q$ elementen bevat.
    \begin{itemize}
    \item De afgeleide van $X^{q}-X$ is $qX^{q-1}-1 = -1\in \mathbb{Z}_{p}[X]$.
      Hieruit volgt meteen dat $X^{q}-X$ geen meervoudige wortels bevat.
      Noem de verzameling van wortels van $X^{q}-X$ nu $W$:
      \[ W = \{ x\in K \mid x^{q} = x\} \]
      $W$ bevat nu $q$ elementen omdat er geen herhalingen in voorkomen.
    \item $W$ is nu een deelveld van $K$ dat $\mathbb{Z}_{p}$ omvat. \waarom
      Per definitie van het ontbindingsveld is $K$ dan gelijk aan $W$.
    \end{itemize}
  \end{proof}
\end{st}

\begin{st}
  \label{st:priemveld-is-ontbindingsveld}
  Zij $p$ een priemgetal waarvoor $q=p^{r}$ geldt met $r\in \mathbb{N}_{0}$.
  Elk veld $F,+,\cdot$ met $q$ elementen is (isomorf met) een ontbindingsveld van $X^{q}-X$ over $\mathbb{Z}_p$.
  \[ X^{q}-X = \prod_{a\in F}(X-a) \]

  \begin{proof}
    Zij $F,+,\cdot$ een veld met $q$ elementen.
    We identificeren het priemdeelveld van $F$ met $\mathbb{Z}_{p}$.\gevref{gev:karakteristiek-van-domein-is-priem} \stref{st:char-priem-priemdeelveld-zp}.
    Omdat $F^{\times}$ een cyclische\stref{st:multiplicatieve-groep-van-veld-is-cyclisch} groep is\eiref{ei:eenhedengroep-is-groep} van orde $q-1$\deref{de:groep} geldt voor elke $x\in F^{\times}$ dat $x^{q-1}$ gelijk is aan $1$ en dus ook het volgende:
    \[ x^{q} = x \]
    Dit laatste geldt natuurlijk ook voor het nulelementen en bijgevolg voor alle $x\in F$.
    Anders gezegd heeft de veelterm $X^{q}-X$ van graad $q$ $q$ verschillende wortels in $F$.
    Omdat $F$ $q$ elementen heeft is $F$ dan precies de verzameling wortels van $X^{q}-X$ en dus een ontbindingsveld van $X^{q}-X$ over $\mathbb{Z}_{p}$.
  \end{proof}
\clarify{wat heeft dit met het priemdeelveld te maken??}
\end{st}


\begin{opm}
  De uniciteit van het veld met $p^{r}$ elementen volgt onmiddelijk als we de uniciteit van het ontbindingsveld gebruiken.
\end{opm}
\clarify{waarom dan nog het volgend bewijs?}

\begin{st}
  \label{st:priemveld-is-uniek}
  Zij $p$ een priemgetal waarvoor $q=p^{r}$ geldt met $r\in \mathbb{N}_{0}$.
  Twee verschillende velden met $q$ elementen zijn isomorf.

  \begin{proof}
    Zij $K,+,\cdot$ en $K',\star,*$ twee velden met $q$ elementen.
    We identificeren beide velden hun priemdeelvelden met $\mathbb{Z}_{p}$.\gevref{gev:karakteristiek-van-domein-is-priem} \stref{st:char-priem-priemdeelveld-zp}.
    Zij $u$ een generator\stref{st:multiplicatieve-groep-van-veld-is-cyclisch} van $K^{\times}$ met minimale veelterm $f$ over $\mathbb{Z}_{p}$.
    In het bijzonder is $f$ irreducibel in $\mathbb{Z}_{p}[X]$. \waarom
    $K$ is dan gelijk aan $\mathbb{Z}_{p}(u)$, precies omdat $u$ een generator is van de element in $K\setminus 0$.
    We weten nu dat $u$ een wortel is van $X^{q}-X$ over $\mathbb{Z}_{p}$.\stref{st:priemveld-is-ontbindingsveld}
    Nu volgt ook dat $X^{q}-X$ een product is van lineaire veeltermen in $K'[X]$.\waarom
    $f$ heet dus ook een wortel $u'$ in $K'$.
    Omdat $f$ bovendien irreducibel is over $\mathbb{Z}_{p}$ is $f$ ook de minimale veelterm van $u'$ over $\mathbb{Z}_{p}$.
    De structuurstelling voor enkelvoudige algebra\"ische uitbreidingen zegt nu het volgende: \waarom
    \[ K \cong \mathbb{Z}_{p}(u) \cong \nicefrac{\mathbb{Z}_{p}}{(f)} \cong \mathbb{Z}_{p}(u') \subseteq K' \]
    Omdat $K$ en $K'$ evenveel elementen hebben moet $K$ dan isomorf zijn met $K'$.
  \end{proof}
\end{st}


\begin{de}
  Zij $p$ een priemgetal waarvoor $q=p^{r}$ geldt met $r\in \mathbb{N}_{0}$.
  Het veld met $q$ elementen noteert men als $\mathbb{F}_{q}$ of $GF(p)$.
\end{de}

\begin{st}
  Zij $p$ een priemgetal waarvoor $q=p^{r}$ geldt met $r\in \mathbb{N}_{0}$.
  Er bestaat een irreducibele veelterm van graad $r$ in $\mathbb{F}_{p}[X]$.

  \begin{proof}
    Kies de minimale veelterm van een generator $u$ van $\mathbb{F}_{p}^{\times}$ over $\mathbb{F}_{p}$.
    Deze is dan inderdaad irreducibel met graad $[\mathbb{F}_{p}=\mathbb{F}_{p}(u) \mathbb{F}_{p}] = r$.
    \clarify{meer uitleg?}
  \end{proof}
\end{st}

\begin{st}
  Zij $p$ een priemgetal waarvoor $q=p^{r}$ geldt met $r\in \mathbb{N}_{0}$.
  Elke irreducibele veelterm van graad $r$ in $\mathbb{F}_{p}[X]$ is een deler van $X^{p}-X$ in $\mathbb{F}_{p}[X]$.

  \begin{proof}
    Zij $f$ een irreducibele veelterm van graad $r$ in $\mathbb{F}_{p}[X]$.
    Kies een velduitbreiding van $\mathbb{F}_{p}$ met daaring een wortel $a$ van $f$, dan geldt het volgende:
    \[ [\mathbb{F}_{p}(a):\mathbb{F}_{p}] = r \quad\text{ en }\quad |\mathbb{F}_{p}(a)| = p^{r} = q \]
    Uit de vorige stelling \needed volgt dat $a$ ook een wortel is van $X^{q}-X$ en dus is $X^{q}-X$ deelbaar door $f$.
  \end{proof}
\end{st}

\begin{lem}
  Zij $k$ een deler van $r$ in $\mathbb{N}_{0}$ en $p$ een priemgetal.
  Definieer $q$ en $q'$ als $q=p^{r}$ en $q'=p^{k}$.
  \[ X^{q'}-X | X^{q}-X \]

  \begin{proof}
    We gebruiken tweemaal de volgende identiteit.
    \[ y^{d} - 1= (y-1)(y^{d-1} + \dotsb + y+1) \text{ met } d\in \mathbb{N}_{0} \]
    Kies $y=q'$ en $d=\frac{r}{k}$. hieruit volgt dat $q-1$ deelbaar is door $q'-1$.
    Neem nu $y=X^{q'-1}$ en $d=\frac{q-1}{q'-1}$.
    Dan verkrijgen we het volgende:
    \[ X^{q'-1}-1 \mid X^{q-1}-1 \]
    Dit impliceert onmiddelijk de bewering.
  \end{proof}
\end{lem}

\begin{st}
  Zij $p$ een priemgetal waarvoor $q=p^{r}$ geldt met $r\in \mathbb{N}_{0}$.
  De irreducibele factoren van $X^{q}-X$ in $\mathbb{F}_{p}[X]$ zijn precies de irreducibele veeltermen in $\mathbb{F}_{p}[X]$.

  \begin{proof}
    Zij $f'$ een irreducibele veelterm in $\mathbb{F}_{p}[X]$ van graad $k$ waarbij $k$ een deler is van $r$.
    Noteer nu $q'=p^{k}$.
    Analoog als in de vorige stelling vinden we dat $f'$ een deler is van $X^{q'}-X$.
    Vanwege het lemma is dan $X^{q}-X$ deelbaar door $X^{q'}-X$ en dus door $f$.
    Zij anderzijds $g$ een irreducibele factor van $X^{q}-X$ in $\mathbb{F}_{p}[X]$ en noteer $l=gr(g)$.
    Neem een wortel $w$ van $g$ in het ontbindingsveld $\mathbb{F}_{q}$ van $X^{q}-X$ over $\mathbb{F}_{p}$.
    Omdat $g$ irreducibel is, is $g$ dan een minimale veelterm van $w$ over $\mathbb{F}_{p}$ en geldt $[\mathbb{F}_{p}(w):\mathbb{F}_{p}]=l$.
    Door tenslotte de productformule \needed toe te passen op de volgende uitbreidingen verkrijgen we dat $r$ deelbaar is door $l$.
    \[ \mathbb{F}_{p} \subseteq \mathbb{F}_{p}(w) \subseteq \mathbb{F}_{p} \]
    De alternatieve formulering volgt dadelijk uit het feit dat $X^{q}-X$ geen meervoudige wortels heeft en elke irreducibele factor dus maar \'e\'en keer voorkomt.
  \end{proof}
\end{st}
\TODO{deze sectie opnieuw!}

\subsection{Onderlinge inclusies}
\label{sec:onderlinge-inclusies}

\begin{st}
  Zij $p$ een priemgetal.
  Een veld met $p^{r}$ elementen heeft een uniek deelveld met $p^{k}$ elementen als en slechs als $k|r$ geldt.

  \begin{proof}
    Bewijs van een equivalentie
    \begin{itemize}
    \item $\Rightarrow$\\
      Als $\mathbb{F}_{p^{k}}$ een deelveld is van $\mathbb{F}_{p^{r}}$, dan volgt uit de productformule voor de volgende keten van velduitbreidingen dat $r$ deelbaar is door $k$.
      \[ \mathbb{F}_{P} \subseteq \mathbb{F}_{p^{r}} \subseteq \mathbb{F}_{p^{r}} \]
    \item $\Leftarrow$\\
      Als anderzijds $r$ deelbaar is door $k$, dan zegt het vorige lemma dat $X^{p^{k}}-X$ een deler is van $X^{p^{r}}-X$.
      Alle wortel van de veelterm $X^{p^{k}}-X$ behoren dus tot een ontbindingsveld $\mathbb{F}_{p}$ van $X^{p^{r}}-X$.
      Deze wortels vormen een deelveld met $p^{k}$ elementen.
      Stel nu dat $\mathbb{F}_{p^{r}}$ twee verschillende deelvelden heeft met $p^{k}$ elementen, dan zou de veelterm $X^{p^{k}}$ meer dan $p^{k}$ nulpunten hebben in het veld $\mathbb{F}_{p^{r}}$.
    \end{itemize}
  \end{proof}
\end{st}

\begin{pr}
  Zij $p$ een priemgetal.
  Beschouw voor elke $k$ en $r$ in $\mathbb{N}$ waarbij $k|r$ geldt het veld $\mathbb{F}_{p^{k}}$ als deelveld van $\mathbb{F}_{p^{r}}$.
  $\overline{\mathbb{F}_{p}}$ is dan een algebra\"ische sluiting van $\mathbb{F}_{p}$.
  \[ \overline{\mathbb{F}_{p}} = \cup_{i\in \mathbb{N}_{0}}\mathbb{F}_{p^{i}} \]

  \begin{proof}
    \begin{itemize}
    \item $\bar{\mathbb{F}_{p}}$ is een veld.
      Neem twee elementen $x$ en $y$ uit $\bar{\mathbb{F}_{p}}$, dan bestaan er getallen $i$ en $j$ in $\mathbb{N}_{0}$ zodat $x$ een element is van $\mathbb{F}_{p^{i}}$ en $y$ een element van $\mathbb{F}_{p^{j}}$.
      Via de de gegeven inbeddingen behoren $x$ en $y$ dan tot (bijvoorbeeld) $\mathbb{F}_{p^{ij}}$.
      We kunnen $x+y$ en $x\cdot y$ nu beschouwen in $\mathbb{F}_{p^{ij}} \subseteq \bar{\mathbb{F}}_{p}$.
      \extra{bewijs verder p 104}
    \item $\bar{\mathbb{F}_{p}}$ is algebra\"isch over $\mathbb{F}_{p}$.
      Neem immers een $u\in \bar{\mathbb{F}}_{p}$, dan bestaat er een $i\in \mathbb{N}_{0}$ zodat $u$ een element is van $\mathbb{F}_{p^{i}}$ en $u$ is dus algbera\"isch over $\mathbb{F}_{p}$.
    \item $\bar{\mathbb{F}_{p}}$ is algebra\"isch gesloten.
    \end{itemize}
  \end{proof}
\end{pr}

\TODO{deze sectie opnieuw!}

\end{document}

%%% Local Variables:
%%% mode: latex
%%% TeX-master: t
%%% End:
