\documentclass[main.tex]{subfiles}
\begin{document}
\chapter{Velden}
\label{cha:velden}

\TODO{morfismen van velden, zie toledo}
\section{Karakteristiek van een ring}
\label{sec:karakt-van-een}

\begin{de}
  Zij $R,+,\cdot$ een ring met nulelement $e$.
  De \term{karakteristiek} $char(R)$ van een ring $R,+,\cdot$ is ...
  \begin{itemize}
  \item ... de kleinste $n\in \mathbb{N}_{0}$ zodat $nx=e$ geldt voor alle elementen $x\in R$.
  \item ... $0$ indien er niet zo'n $n$ bestaat.
  \end{itemize}
\end{de} 

\begin{ei}
  Zij $R,+,\cdot$ een ring  en karakteristiek $char(R) = n \neq 0$, dan is de (additieve) orde van elk element in $R,+$ een deler van $n$.

  \begin{proof}
    Zij $s$ orde van een willekeurig element $x$ uit $R$.
    Omdat $nx$ het nulelement is, is $n$ een kandidaat voor de orde, maar groter dan $s$, dus is $s$ een deler van $n$.\eiref{ei:groep-eindige-orde-deelbaarheid}
  \end{proof}
\end{ei}

\begin{st}
  \label{st:eenheidselement-orde-karakteristiek}
  Zij $R,+,\cdot$ een ring met eenheidselement $i$.
  \[ char(R) = n \neq 0 \Leftrightarrow i \text{ heeft orde } n \]

  \begin{proof}
    Bewijs van een equivalentie.
    \begin{itemize}
    \item $\Rightarrow$: Dit volgt rechtstreeks uit de definitie van de karakteristiek.
    \item $\Leftarrow$\\
      Voor een gegeven $k\in \mathbb{N}_{0}$ geldt het volgende:
      \[ \forall x \in R: kx = k(i\cdot x) = (k \cdot i) \cdot x \]
      Dit betekent precies dat $kx$ het nulelement is als en slechts als $k$ de orde is van $i$ in $R,+,\cdot$.
      $k$ is dus ook de charakteristiek van $R,+,\cdot$.
    \end{itemize}
  \end{proof}
\end{st}

\begin{gev}
  Zij $R,+,\cdot$ een ring met eenheidselement $i$.
  \[ char(R) = 0 \Leftrightarrow i \text{ heeft orde } \infty \]

  \begin{proof}
    Dit volgt rechtstreeks.
    Als er immers geen enkele $k\in \mathbb{N}_{0}$ bestaat zodat $ki$ het nulelement is, kan er ook geen $k$ bestaan zodat voor elke $x\in R$ $kx$ het nuleleement is.
  \end{proof}
\end{gev}

\begin{st}
  Zij $R,+,\cdot$ een niet-triviale ring zonder nuldelers met nulelement $e$, dan hebben alle elementen uit $R_{e}$ hebben dezelfde orde in $R,+$

  \begin{proof}
    Als alle elementen in $R_{e}$ als orde $\infty$ hebben is de stelling evident.
    Stel daarom dat er een element $a\in R_{e}$ bestaat met eindige orde $n$.
    We moeten dan aantonen dat elk ander element $x\in R_{e}$ ook orde $n$ heeft.
    Beschouw nu $a \cdot (nx)$:
    \[ a \cdot (nx) = (na) \cdot x = e \cdot x = e \]
    Omdat $R$ geen nulderes heeft volgt hieruit dat $nx$ het nulelement is
    \clarify{waarom is $n$ minimaal?}
  \end{proof}
\extra{bewijs}
\end{st}

\begin{st}
 Zij $R,+,\cdot$ een niet-triviale ring zonder nuldelers, dan is de karakteristiek van $R,+,\cdot$ nul of een priemgetal.
 
 \begin{proof}
   Bewijs uit het ongerijmde.
   Stel dat de karakteristiek $n\neq 0$ van $R,+,\cdot$ valt te schrijven als $r\cdot s$ met $r, s \in \mathbb{N}_{0}$.
   Voor elk niet-nul-element $a$ uit $R$ geldt nu het volgende:
   \[ ra \cdot sa = (rs)a^{2} = na^{2} = 0 \]
   Dit betekent dat ofwel $ra$, ofwel $sa$ het nulelement is.
   $r$ of $s$ moet dan gelijk zijn aan $n$ (en de andere aan $1$).
   Dit houdt precies in dat $n$ een priemgetal is.
 \end{proof}
\end{st}

\begin{gev}
  De karakteristiek van een integriteitsdomein (of veld) is nul of een priemgetal.
\end{gev}

\begin{st}
  \label{st:rekenregel-in-integriteitsdomein}
  Een \term{rekenregel in integriteitsdomein}\\
  Zij $R,+,\cdot$ een integriteitsdomein van karakteristiek $p$.
  \[ \forall a,b \in R:\ (a+b)^{p} = a^{p}+b^{p} \]

  \begin{proof}
    De (normale) binomiaalontwikkeling van $(a+b)^{p}$ ziet er als volgt uit:
    \[ (a+b)^{p} = \sum_{i=0}^{p}\frac{p!}{i!(p-i)!} a^{i}b^{p-i} \]
    Merk nu op dat $p$ een deler is van $binom{p}{i}$ voor alle $i$ van $1$ tot en met $p-1$.
    Voor elk van die $i$'s wordt de term dan het nulelement.
    Er blijft dan enkel nog $a^{p} + b^{p}$ over in het rechterlid.
  \end{proof}
\end{st}

\begin{st}
  Zij $R,+,\cdot$ een integriteitsdomein van karakteristiek $p$.
  \[ \forall a,b \in R:\ (a-b)^{p} = a^{p}-b^{p} \]

  \begin{proof}
    \[ (a-b)^{p} = (a+(-b))^{p} = a^{p} + (-b)^{p} = a^{p} - b^{p} \]
    De derde gelijkheid geldt omwille van de rekenregel in integriteitsdomeinen.\stref{st:rekenregel-in-integriteitsdomein}
    De laatste gelijkheid geldt enkel omdat $p$ een priemgetal is.
\clarify{waarom mag dit ook voor $p=2$??}
  \end{proof}
\end{st}

\begin{st}
  Zij $R,+,\cdot$ een ring met priemkarakteristiek $p$.
  \[ \forall a,b \in R, \forall k \in \mathbb{N}:\ (a+b)^{p^{k}} = a^{p^{k}}+b^{p^{k}} \]
\extra{bewijs}
\end{st}

\begin{st}
  Zij $R,+,\cdot$ een ring met priemkarakteristiek $p$.
  \[ \forall a,b \in R, \forall k \in \mathbb{N}:\ (a-b)^{p^{k}} = a^{p^{k}}-b^{p^{k}} \]
\extra{bewijs}
\end{st}

\begin{st}
  Het \term{Frobeniusmorfisme}\\
  \begin{itemize}
  \item Zij $R,+,\cdot$ een integriteitsdomein van karakteristiek $p\neq 0$.
    De afbeelding $\phi$ is een injectief ringmorfisme.
    We noemen het het \term{Frobeniusmorfisme}.
    \[ \phi:\ R \rightarrow R:\ x \mapsto x^{p} \]
  \item Als $|R|$ eindig is (en dus een veld\stref{st:eindig-integriteitsdomein-is-veld}), dan is $\phi$ een isomorfisme.
  \end{itemize}

  \begin{proof}
    \begin{itemize}
    \item Zij $e$ het nulelement van $R,+,\cdot$.  $\phi$ is een
      ringmorfisme vanwege de rekenregel in
      integriteitsdomeinen\stref{st:rekenregel-in-integriteitsdomein}.
      Uit de implicatie $x^{p} = e \Rightarrow x = 0$ volgt dat $\phi$
      injectief is.  \clarify{vanwaar komt die implicatie?}
    \item Een injectieve afbeelding van een verzameling naar zichzelf is surjectief en bijgevolg ook een bijectie. \clarify{referentie naar bewijs?}
    \end{itemize}
  \end{proof}
\end{st}


\subsection{Deelvelden}
\label{sec:deelvelden}

\begin{de}
  Zij $E,+,\cdot$ een veld en $F$ een niet-lege deelverzameling van $E$.
  We noemen $F,+,\cdot$ een \term{deelveld} van $E,+,\cdot$ als $F,+,\cdot$ zelf een veld is (voor dezelfde bewerkingen).
\end{de}

\begin{st}
  Het \term{criterium voor deelvelden}\\
  Zij $F,+,\cdot$ een veld met eenheidselement $i$ en $E$ een deelverzameling van $F$, dan is $E,+,\cdot$ een deelveld van $f$ als en slechts als aan volgende voorwaarden voldaan is.
  \begin{itemize}
  \item $E$ is een deelring van $F$ met hetzelfde eenheidselement.
  \item $\forall x\in E:\ \exists y\in E:\ xy =i = yx$ (Elk element heeft een interne inverse.)
  \item $\forall x,y\in E:\ x\cdot y \in E$ (De multiplicatieve bewerking is intern.)
  \end{itemize}

  \begin{proof}
    \begin{itemize}
    \item $\Rightarrow$: Dit volgt rechtstreeks uit de definitie.
\clarify{waarom heeft $E$ noodzakelijk hetzelfde eenheidselement?}
    \item $\Leftarrow$\\
      Omdat $E$ een deelring is die $i$ bevat, volgt uit de twede voorwaarde dat $E$ een lichaam is.
      $E,+,\cdot$ is bovendien commutatief want het is een deelring van een commutatieve ring. \needed
      $E,+,\cdot$ is een commutatief lichaam en $E$ een deelverzameling van $F$, dus een deelveld.
    \end{itemize}
  \end{proof}
\end{st}

\subsection{Priemdeelring en Priemdeelveld}
\label{sec:priemd-en-priemd}

\begin{de}
  Zij $R,+,\cdot$ een ring met eenheidselement $i$.
  De doorsnede van alle deelringen van $R,+,\cdot$ die $i$ bevatten is zelf een deelring van $R,+,\cdot$ (die $i$ bevat).
  Deze `kleinste deelring die $i$ bevat' de \term{priemdeelring} van $R,+,\cdot$.
\end{de}

\begin{ei}
  \label{ei:priemdeelring-voortgebracht-door-eenheidselement}
  De priemdeelring van een ring $R,+,\cdot$ met eenheidselement $i$ is $<1>$:
  \[ <1> = \{ zi \ |\ z \in \mathbb{Z} \} \]

  \begin{proof}
    Kies $I$ zodat $I,+,\cdot$ een deelring is van $R,+,\cdot$.
    $I$ moet dan $<i>$ omvatten.
\question{en nu?}
  \end{proof}
\end{ei}

\begin{st}
  \label{st:karakteristiek-isomorfisme-met-zn}
  Zij $R,+,\cdot$ een ring met eenheidselement $i$.
  \begin{itemize}
  \item Als $char(R) = n \neq 0$ geldt, dan is de priemdeelring van $R$ isomorf met $\mathbb{Z}_{n}$.
  \item Als $char(R) = 0$ geldt, dan is de priemdeelring van $R$ isomorf met $\mathbb{Z}$.
  \end{itemize}

  \begin{proof}
    Beschouw de afbeelding $\phi$:
    \[ \phi:\ \mathbb{Z} \rightarrow \mathbb{R}: z \mapsto z\cdot i \]
    $\phi$ is nu een ringmorfisme:
    \[ \forall x,y\in \mathbb{Z}: \phi(x+y) = (x+y) \cdot i = (x \cdot i) + (y \cdot i) = \phi(x) + \phi(y) \]
    \[ \forall x,y\in \mathbb{Z}: \phi(x\cdot y) = (x\cdot y) \cdot i = (x \cdot i) \cdot (y \cdot i) = \phi(x) \cdot \phi(y) \]
    Merk op dat $\phi(\mathbb{Z}),+,\cdot$ de priemdeelring is van $R,+,\cdot$.\eiref{ei:priemdeelring-voortgebracht-door-eenheidselement}
    \begin{itemize}
    \item Als de karakteristiek van $R$ $n\neq 0$ is, dan wordt de kern van $\phi$ voortgebracht door $n$ in $\mathbb{Z}$ want $n$ is de orde van $i$ in $R,+$.
      \[ Ker(\phi) = (n) \]
      De priemdeelring is dan isomorf met $\nicefrac{\mathbb{Z}}{(n)} \cong \mathbb{Z}_{n}$.\needed
    \item Als de karakteristiek van $R,+,\cdot$ nul is,dan is $\phi$ injectief \waarom en is de priemdeelring van $R,+,\cdot$ dus isomorf met $\mathbb{Z}$.
    \end{itemize}
  \end{proof}
\end{st}


\begin{de}
  Zij $F,+,\cdot$ een veld.
  De doorsnede van alle deelvelden van $F$ is zelf een deelveld van $F$.
  Di kleinste deelveld heet het \term{priemdeelveld} van $F,+,\cdot$.
\end{de}
\begin{st}
  Zij $F,+,\cdot$ een veld.
  Als $char(F)$ een priemgetal is, dan is het priemdeelveld van $F,+,\cdot$ isomorf met $\mathbb{Z}_{p}$.
\end{st}

\begin{st}
  Zij $F,+,\cdot$ een veld.
  Als $char(F)$ nul is, dan is het priemdeelveld van $F,+,\cdot$ isomorf met $\mathbb{Q}$.

  \begin{proof}
    \begin{itemize}
    \item Als $char(F)$ een priemgetal $p$ is, weten we al dat de priemdeelring van $F,+,\cdot$ isomorf is met het veld $\mathbb{Z}_{p}$.
      Die deelring is dan ook een deelveld.\stref{st:karakteristiek-isomorfisme-met-zn}
    \item Als $char(F)$ nul is, dan is $\phi$ een injectief ringmorfisme:
      \[ \phi:\ \mathbb{Z} \rightarrow \mathbb{R}: z \mapsto z\cdot i \]
      Beschouw nu de afbeelding $\phi'$:
      \[ \phi':\ \mathbb{Q} \rightarrow F:\ \frac{a}{b} \mapsto \phi(a) \cdot (\phi(b))^{-1} = (ai) \cdot (bi)^{-1} \]
      Merk nu op dat $\phi'$ en ringmorfisme is:
      \[
      \forall \left(\frac{a}{b}\right),\left(\frac{c}{d}\right) \in
      \mathbb{Q}:\
      \begin{array}{rll}
        \phi'(\nicefrac{a}{b} + \nicefrac{c}{d}) &=  \phi'\left(\frac{ad+bc}{bd}\right) &\\
                                                 &= \phi(ad+bc) \cdot (\phi(bd))^{-1} &\\
                                                 &= ((ad+bc)i) \cdot ((bd)i)^{-1} &\\
                                                 &= ((ad)i) + ((bc)i)) \cdot ((bd)i)^{-1} &\\
                                                 &= ((ad)i)\cdot ((bd)i)^{-1}) +  ((bc)i) \cdot ((bd)\cdot i)^{-1}) &\\
                                                 &= \phi'(\nicefrac{ad}{bd}) + \phi'(\nicefrac{bc}{bd}) &= \phi'(\nicefrac{a}{b}) + \phi'(\nicefrac{c}{d}) \\
      \end{array}
      \]
      \[
      \forall \left(\frac{a}{b}\right),\left(\frac{c}{d}\right) \in
      \mathbb{Q}:\
      \begin{array}{rll}
        \phi'(\nicefrac{a}{b} \cdot \nicefrac{c}{d}) &= \phi'(\nicefrac{ac}{bd}) &\\
                                                     &= \phi(ac) \cdot (\phi(bd))^{-1} &\\
                                                     &= ((ac)i) \cdot ((bd) \cdot i)^{-1} &\\
                                                     &= ((ai) \cdot (ci)) \cdot ((bi) \cdot (di))^{-1} &\\
                                                     &= (ai) \cdot (bi)^{-1} \cdot (ci) \cdot (di)^{-1} &\\
                                                     &= \phi(a) \cdot (\phi(b))^{-1} \cdot \phi(c) \cdot (phi(d))^{-1} &= \phi'(\nicefrac{a}{b}) \cdot \phi'(\nicefrac{c}{d})
      \end{array}
      \]
      $\phi'$ is bovendien injectief.  Kies immers twee verschillende
      elementen $\frac{a}{b}$ en $\frac{c}{d}$ uit $\mathbb{Q}$ die op
      hetzelfde element van $F$ afgebeeld worden onder $\phi'$, dan
      geldt het volgende:
      \[ (ai) \cdot (bi)^{-1} = (ci) \cdot (di)^{-1} \]
      \[ \phi(a) \cdot \phi(d) = \phi(b) \cdot \phi(c) \]
      Omdat $\phi$ injectief is geldt $ad = bc$ en bijgevolg
      $\frac{a}{b} = \frac{c}{d}$.
      Het priemdeelveld van $F$ is nu precies het beeld van $\mathbb{Q}$ onder $\phi'$. \waarom
      De kern van $\phi$ is bovendien $\{0\}$, dus het priemdeelveld is isomorf met $\nicefrac{\mathbb{Q}}{\{0\}} \cong \mathbb{Q}$.
    \end{itemize}
  \end{proof}
  \begin{proof}
    Zij $F,+,\cdot$ een veld van karakteristiek $0$ en zij $P,+,\cdot$ het priemdeelveld van $F,+,\cdot$.
    $P$ omvat dan de priemdeelring van $F,+,\cdot$.
    We weten al dat die priemdeelring isomorf is met $\mathbb{Z}$ volgens $\phi$.\stref{st:karakteristiek-isomorfisme-met-zn}
    \[ \phi::\ \mathbb{Z} \Rightarrow F:\ z \mapsto (zi)\]
    Vermits het beeld tot $P$ behoort, kunnen we $\phi$ ook als volgt bekijken:
    \[ \mathbb{Z} \rightarrow P \]
    Er bestaat dan ook een ringmorfisme $\phi'$ zodat $\phi$ factoriseert als $\phi' \circ i$ waarbij $i$ de inclusie is van $\mathbb{Z}$ in $\mathbb{Q}$. \eiref{ei:factorisatie-breukenveld}
    \[ \phi': \mathbb{Q} \rightarrow P \]
    \begin{figure}[H]
      \centering
      \begin{tikzpicture}
        \matrix (m) [matrix of math nodes,row sep=3em,column sep=4em,minimum width=2em]
        {
          \mathbb{Z} & \mathbb{Q} \\
          P & \\};
        \path[-stealth]
        (m-1-1) edge node [left] {$\phi$} (m-2-1)
        edge node [above] {$i$} (m-1-2)
        (m-1-2) edge node [right] {$\phi'$} (m-2-1);
      \end{tikzpicture}
    \end{figure}
    Nu beeldt $\phi'$ $1$ af op het neutraal element van $P$.
    $\phi'$ is dus een veldmorfisme.
    OMdat $\phi'$ ook injectief is, is het beeld van $\mathbb{Q}$ onder $\phi'$ een deelveld van $P,+,\cdot$ en van $F,+,\cdot$.
    $P$ is echter het priemdeelveld van $F$, dus $\phi'$ moet ook surjectief zijn en bijgevolg een isomorfisme.
  \end{proof}
\end{st}

\begin{st}
  Zij $K,+,\cdot$ een deelveld van een veld $L,+,\cdot$.
  \[ char(K) = char(L) \]

  \begin{proof}
    Noem de karakteristiek van $L,+,\cdot$ $n$.
    De orde van het eenheidselement in $L$ is dan ook $n$.\stref{st:eenheidselement-orde-karakteristiek}
    Omdat $K,+,\cdot$ een deelveld is van $L,+,\cdot$ is $n$ ook de orde van het eenheidselement in $K,+,\cdot$.
  \end{proof}
\end{st}

\section{Veldmorfismen}
\label{sec:veldmorfismen}

\begin{de}
  Zij $F,+,\cdot$ en $E,\star,*$ velden, dan is een afbeelding $\phi: F \rightarrow E$ een \term{veld(homo)morfisme} als $\phi$ een ringmorfisme is dat het eenheidselement van $F$ op het eenheidselement van $E$ afbeeldt.
\end{de}

\begin{de}
  Een bijectief veldmorfisme noemen we een \term{veldisomorfisme}.
\end{de}

\begin{st}
  \label{st:veldmorfisme-is-injectief}
  Een veldmorfisme is steeds injectief.
\extra{bewijs}
\end{st}


\begin{opm}
  We zullen daarom een veld vaak identificeren met zijn beeld onder een veldmorfisme.
\end{opm}

\section{Velduitbreidingen: basisbegrippen}
\label{sec:veld-basisb}

\begin{de}
  Zij $F,+,\cdot$ een deelveld van een veld $E,+,\cdot$.
  Men noemt $E,+,\cdot$ een \term{velduitbreiding} of \term{uitbreiding} van $F$.
\end{de}

\begin{st}
  Zij $F,+,\cdot$ een deelveld van een veld $E,+,\cdot$.
  $E$ is een vectorruimte over $F$.
  
\extra{bewijs}
\end{st}

\begin{de}
  Zij $F,+,\cdot$ een deelveld van een veld $E,+,\cdot$.
  De \term{uitbreidingsgraad} of \term{graad} $[E:F]$ van $E$ over $F$ is de dimensie van $E$ als $F$-vectorruimte.
\end{de}

\begin{de}
  Een uitbreiding $E,+,\cdot$ van een veld $F,+,\cdot$ met een eindige uitbreidingsgraad noemen we een \term{eindige uitbreiding}.
\end{de}

\begin{de}
  Binnen een veld $F,+,\cdot$ noteren we $a\cdot b^{-1}$ vaak als $\frac{a}{b}$.
\end{de}

\begin{st}
  Het aantal elementen van een eindig veld is een macht van een priemgetal.

  \begin{proof}
    Zij $F,+,\cdot$ eeneindig veld, dan moet het priemdeelveld $P,+,\cdot$ van $F,+,\cdot$ isomorf zijn met een $\mathbb{Z}_{p}$ waarbij $p$ een priemgetal is.
    $F$ is een noodzakelijk eindigdimensionale vectorruimte over $P$.
    Noem de uitbreidingsgraad van $F,+,\cdot$ over $P,+,\cdot$ $n$.
    \[ |F| = |P|^{n} = p^{n} \]
\clarify{hoe komen we aan dit laatste??}
  \end{proof}
\end{st}

\begin{st}
  \examen
  De \term{fundamentele stelling van de veldentheorie}\\
  Zij $K,+,\cdot$ een veld en $f$ een niet-constante veelterm in $K[X]$, dan bestaat er een uitbreiding $E$ van $K$ zodat $f$ een wortel heeft in $E$.

  \begin{proof}
    Kies een irreducibele factor $p$ van $f$ in het UFD\stref{gev:veeltermen-over-veld-ufd} $K[X]$.
    Beschouw nu de quoti\"entring $\nicefrac{K[X]}{(p)}$.
    We beweren dat $p$ een wortel zal hebben $\nicefrac{K[X]}{(p)}$.
    \begin{itemize}
    \item $\nicefrac{K[X]}{(p)}$ is een veld omdat $p$ irreducibel is\gevref{gev:veld-door-hoofdideaal-van-irreducibele-veelterm-veld} en $(p)$ daarom een maximaal ideaal\stref{st:hoofdidiaal-van-irreducibele-veelterm-maximaal} in het HID\stref{st:veeltermen-over-veld-hid} $K[X]$.
    \item De afbeelding $f$ is een veldmorfisme en bijgevolg injectief.\stref{st:veldmorfisme-is-injectief}
      \[ f:\ K \rightarrow \nicefrac{K[X]}{(p)}:\ a \mapsto \bar{a} \]
      We identificeren daarom $K$ met het deelveld $\{ \bar{a} \mid a\in K \}$ van $\nicefrac{K[X]}{(p)}$.
      We beschouwen $\nicefrac{K[X]}{(p)}$ met andere woorden als een uitbreiding van $K$.
    \item $\bar{X}$ is nu een wortel van $p$ in $\nicefrac{K[X]}{(p)}$.
      \[ p(\bar{X}) = \sum_{i}\overline{a_{i}}(\overline{X})^{i}= \overline{\sum_{i}a_{i}X^{i}} = \bar{p} = \bar{e_{K}} \]
      Merk op dat we de identificatie gebruiken.
      We beschouwen $p\in K[X]$ als $p = \sum_{i}\bar{a_{i}}X^{i}$.
    \end{itemize}
\clarify{meer uitleg nodig!}
  \end{proof}
\end{st}

\begin{opm}
  Deze stelling geldt ook voor integriteitsdomeinen
\extra{bewijs: elk domein kan worden uitgebreid tot een breukenveld}
  maar niet voor algemene commutatieve ringen met eenheidselement.
\extra{voorbeeld p 79}
\end{opm}

\begin{st}
  De \term{productformule}\\
  Zij $F,+,\cdot$, $L,+,\cdot$ en $E,+,\cdot$ velden waarbij $L,+,\cdot$ een uitbreiding is van $F,+,\cdot$ en $E,+,\cdot$ een uitbreiding is van $L,+,\cdot$.
  Als zowel $[L:F]$ als $[E:L]$ eindig zijn, geldt het volgende:
  \[ [E:F] = [E:L] \cdot [L:F]\]

  \begin{proof}
    Kies een basis $\alpha = \{ u_{1},\dotsc,u_{n}\}$ van $E$ als vectorruimte over $L$ en een basis $\beta = \{ v_{1},\dotsc,v_{m}\}$.
    We zullen aantonen dat $\gamma = \{ u_{i}\cdot v_{i} \mid i\in \{1,\dotsc,n\}, j \in \{1,dotsc,m\} \}$ een basis vormt voor $E$ als vectorruimte over $F$.
  \begin{itemize}
  \item $\gamma$ is vrij.\\
    Stel dat $\gamma$ niet vrij zou zijn, dan bestaan er co\"efficienten $a_{ij}\in L$ als volgt:
    \[ \sum_{i=1}^{m}\sum_{j=1}^{n}a_{ij}u_{i}\cdot v_{j} = e_{F} \]
    Omdat $E,+,\cdot$ een veld is kunnen we dit herschrijven als volgt:
    \[ \sum_{i=1}^{m}\left(\sum_{j=1}^{n}(a_{ij}v_{i})\right)\cdot u_{j} = e_{F} \]
    Omdat $\alpha$ linear onafhankelijk is moeten de co\"efficienten $\sum_{j=1}^{m}a_{ij}v_{i}$ elk het nulelement zijn.
    Omdat $\beta$ lineair afankelijk is moeten ook nog eens alle $a_{ij}$ dan het nulelement zijn.
  \item $\gamma$ is voortbrengend.\\
\extra{bewijs}
  \end{itemize}
\end{proof}
\end{st}

\subsection{Algebraische en transcendente elementen}
\label{sec:algebr-en-transc}

\begin{de}
  Zij $F,+,\cdot$ een veld en $E,+,\cdot$ een uitbreiding van $F,+,\cdot$.
  Een element $u\in E$ is \term{algebra\"isch} over $F,+,\cdot$ als er een veelterm $f\in F[X], f\neq 0$ bestaat zodat $u$ een wortel is van $f$.
  Anders noemen we $u$ \term{transcendent}.
\end{de}

\begin{st}
  Elk element van een veld $F,+,\cdot$ is transcendent over $F,+,\cdot$.
  \extra{bewijs}
\end{st}

\begin{st}
  Elk element van de vorm $\sqrt[n]{q}$ met $n\in \mathbb{N}_{0}$ en $q\in \mathbb{Q}^{+}$ is algebra\"isch over $\mathbb{Q}$.
  \extra{bewijs}
\end{st}

\subsection{Enkelvoudige uitbreidingen}
\label{sec:enkelv-uitbr}

\begin{de}
  Zij $F,+,\cdot$ een veld en $E,+,\cdot$ een uitbreiding van $F,+,\cdot$.
  Zij $u\in E$ een element van $E$.
  Het veld bekomen door \term{adjunctie} of \term{toevoeging} van $u$ aan $F$: $F(u)$ is de doorsnede van alle velden $L$ die tussen $F$ en $E$ liggen en $u$ bevatten.
  Dit is dus het ``kleinste deelveld van $E$ dat $F$ en $u$ omvat''.
\end{de}

\begin{opm}
  $F(X)$ kunnen we dan zien als de verzameling van rationale veeltermen over $F,+,\cdot$.
\end{opm}

\begin{ei}
  Zij $F,+,\cdot$ een veld en $E,+,\cdot$ een uitbreiding van $F,+,\cdot$.
  Zij $u\in E$ een element van $E$.
  \[
  F(u) = 
  \left\{
      \frac
        {a_{r}u^{r} + \dotsb + a_{1} u + a_{0}}
        {b_{s}u^{s} + \dotsb + b_{1} u + b_{0}}
  \mid r,s \in \mathbb{N},\ a_{i},b_{i} \in F
  \right\}
  \]
  In bovenstaande formule mogen natuurijk niet alle $b_{i}$ het nulelement zijn.
\extra{bewijs}
\end{ei}

\begin{de}
  Zij $F,+,\cdot$ een veld en $E,+,\cdot$ een uitbreiding van $F,+,\cdot$.
  Zij $u$ een element van $E$, algebra\"isch over $F,+,\cdot$.
  Een \term{minimale veelterm} van $u$ over $F$ is een veelterm $f\in F[X]$ van minimale graad waarvoor $f\neq e$ en $f(u) = 0$ gelden.
\end{de}

\begin{pr}
  \label{pr:minimale-veelterm-uniek}
  Zi $F,+,\cdot$ een $E$-vectorruimte en $u$ algebra\"isch over $F$.
  De minimale veelterm van $u$ over $v$ is uniek bepaald op een constante factor (verschillend van het nulelement) na.

  \begin{proof}
    Beschouw $J$:
    \[ J = \{ g\in F[X] \mid g(u) = 0 \} \]
    $J$ is een ideaal van $F[X],+,\cdot$.
    Omdat $u$ algebra\"isch is over $F,+,\cdot$ bevat $g$ meer dan enkel het nulelement.
    De voortbrengers van een niet-triviaal-ideaal in een veeltermenring zijn precies de minimale veeltermen \lemref{lem:minimale-veelterm-brengt-ideaal-voort} en deze zijn allemaal gelijk op een constante factor na.\stref{st:minimale-veeltermen-bijna-uniek}
  \end{proof}
\end{pr}

\begin{ei}
  \label{ei:minimale-veelterm-uniek-en-irreducibel}
  Zij $F,+,\cdot$ een veld en $E,+,\cdot$ een uitbreiding van $F,+,\cdot$.
  Zij $u$ een element van $E$, algebra\"isch over $F,+,\cdot$.
  Zij $f$ een minimale veelterm van $u$ over $F,+,\cdot$.
  \begin{itemize}
  \item Als $g \in F[X]$ ook een veelterm is over $F,+,\cdot$ met waarde nul in $u$, dan is $f$ een deler van $g$.
  \item $f$ is irreducibel in $F[X]$.
  \end{itemize}

  \begin{proof}
    \begin{itemize}
    \item Inderdaad\prref{pr:minimale-veelterm-uniek}
    \item Stel dat $f$ reducibel zou zijn in $F[X]$, dan bestaan er veeltermen $g$ en $h$ over $F,+,\cdot$ van lagere graad zodat $f$ als volgt geschreven kan worden:
      \[ f = gh \]
      Omdat $u$ een nulpunt is van $f$ moet $u$ een nulpunt zijjn van $g$ of $h$, maar dat is in tegenspraak met de minimaliteit van $gr(f)$.
    \end{itemize}
  \end{proof}
\end{ei}

\begin{ei}
  Zij $F,+,\cdot$ een veld en $E,+,\cdot$ een uitbreiding van $F,+,\cdot$.
  Zij $u\in E$ een element van $E$.
  Zij $f\in F[X]$ een veelterm over $F[X]$ is met waarde nul in $u$.
  \[ f \text{ is irreducibel } \Rightarrow f \text{ is de minimale veelterm van } u \text{ over } F,+,\cdot \]

  \begin{proof}
    Bewijs via contrapositie:
    Stel dat $f$ niet de minimale veelterm is van $u$, maar $p$, is $p$ deelbaar door $f$\eiref{ei:minimale-veelterm-uniek-en-irreducibel}. En bestaat dan dus een veelterm $q$ zodat $f=gq$ geldt.
    Bijgevolg is $f$ reducibel.
  \end{proof}
\end{ei}

\begin{st}
  Zij $F,+,\cdot$ een veld en $E,+,\cdot$ een uitbreiding van $F,+,\cdot$.
  Zij $u\in E$ een element van $E$, transcendent over $F,+,\cdot$.
  \begin{itemize}
  \item $F(u) \cong F(X)$
  \item $[F(u):F]$ is niet eindig.
  \end{itemize}

  \begin{proof}
    \begin{itemize}
    \item Beschouw het substitutie-ringmorfisme $\phi$:
      \[ \phi:\ F[X] \rightarrow F(u):\ g \mapsto g(u) \]
      $u$ is transcendent als en slechts als $\phi$ injectief
      is.\waarom Omdat $\phi$ injectief is, kunnen we $\phi$
      uitbreiden tot een, nog steeds injectied, ringmorfisme $\phi'$.
      \[ \phi':\ F(X) \rightarrow F(u):\ \frac{g}{h} \mapsto
      \frac{g(u)}{h(u)} \]
      $\phi'(F(X))$ omvat nu duidelijk $F$ en $u$. \waarom Bovendien
      is veld omdat $\phi'$ injectief is. \waarom We besluiten dat
      $\phi'(F(X))$ een deelveld is van $F(u)$ dat $F$ en $u$ omvat
      \clarify{reken uit} $\phi'$ is nu ook surjectief en dus een
      isomorfisme \waarom \clarify{veel meer uitleg nodig!}
    \item De dimensie van $F[X]$ is niet eindig, dus de dimensie van $F(X)$ zeker niet.
      Er bestaat bovendien een bijectie van $F[X]$ naar $F(X)$.
    \end{itemize}
  \end{proof}
\end{st}

\begin{st}
  Zij $F,+,\cdot$ een veld met eenheidselement $i$ en $E,+,\cdot$ een uitbreiding van $F,+,\cdot$.
  Zij $u\in E$ een element van $E$, transcendent over $F,+,\cdot$ en $f$ een minimale veelterm van $u$ over $F,+,\cdot$.
  \begin{itemize}
  \item $F(u) \cong F(X)/(f)$
  \item $[F(u):F] = deg(f)$
  \item De volgende verzameling vormt een basis van $F(u)$ over $F$.
  \[ \{ i,u,u^{2},\dotsc u^{deg(f)-1} \}\]
  \end{itemize}

  \begin{proof}
    \begin{itemize}
    \item Beschouw opnieuw het substitutie-ringmorfisme $\phi$:
      \[ \phi:\ F[X] \rightarrow F(u):\ g \mapsto g(u) \]
      De kern van $\phi$ is nu het (hoofd\stref{st:veeltermen-over-veld-hid})ideaal van $F[X]$ voortgebracht door $f$.\prref{pr:minimale-veelterm-uniek}
      Dit betekent precies dat $\phi(F[X])$ isomorf is met $\nicefrac{F[X]}{(f)}$.\gevref{gev:eerste-ringmorfismestelling}
      Omdat $f$ irreducibel is, is $(f)$ ene makimaal in $F[X]$ en $\frac{F[X]}{(f)}$ bijgevolg een veld.
      $\phi(F[X])$ is dan een deelveld van $F(u)$ dat $F$ en $u$ omvat en bijgevolg is $\phi(F[X])$ gelijk aan $F(u)$.
    \item Vanwege het volgende isomofirme is het voldoende om te bewijzen dat de verzameling $\{ i,u,u^{2},\dotsc u^{deg(f)-1} \}$ een basis vormt van $\nicefrac{F[X]}{(f)}$ als vectorruimte over $F$.
      \[ \nicefrac{F[X]}{(f)} \rightarrow F(u): \bar{g} \mapsto g(u) \]
      \begin{itemize}
      \item Voortbrengend\\
        Kies een willekeurig element $\bar{g}$ uit $\nicefrac{F[X]}{(f)}$.
        $g$ zit dus in $F[X]$.
        Vanwege het delingsalgoritme in $F[X]$ kunnen we $g$ schrijven als $g=qf+r$ met $q$ en $r$ $\in F[X]$ en $gr(r) < gr(f)$.
        We noteren verder $d=gr(f)$ en $r$ als volgt:
        \[ r = \sum_{i=0}^{d-1}a_{i}X^{i} \]
        Er geld dan het volgende:\waarom
        \[ \bar{g} = \bar{q}\bar{f} + \bar{r} = \bar{r} = \overline{\sum_{i=0}^{d-1}a_{i}X^{i}} = \sum_{i=0}^{d-1}\bar{a}_{i}\bar{X}^{i} \]
        $\{ i,u,u^{2},\dotsc u^{deg(f)-1} \}$ is dus inderdaad een voortbrengende verzameling over $F,+,\cdot$.
        Merk op dat we hier de identificatie van $F$ met $\{ \bar{a}\mid a\in F\}$ gebruiken.
      \item Vrij\\
\extra{bewijs}
      \end{itemize}
      
    \end{itemize}
  \end{proof}
\end{st}

\begin{gev}
  Zij $F,+,\cdot$ een veld en $E,+,\cdot$ een uitbreiding van $F,+,\cdot$.
  \[ u \text{ is algebra\"isch over } F \Leftrightarrow [F(u):F] \text{ is eindig } \]
  \extra{bewijs}
\end{gev}
\extra{vb p 84}

\section{Algebra\"ische uitbreidingen en ontbindingsvelden}
\label{sec:algebr-uitbr-en}

\begin{de}
  Zij $F,+,\cdot$ en $S,+,\cdot$ velden en $E,+,\cdot$ een uitbreiding van zowal $F,+,\cdot$ als $S,+,\cdot$.
  Het veld verkregen door toevoeging van alle elementen van $S$ aan $F$: $F(S)$ is de doorsnede van alle velden $L$ die tussen $F,+,\cdot$ en $E,+,\cdot$ zitten en $S,+,\cdot$ uitbreiden.
  Dit is dus het ``kleinste deelveld van $E,+,\cdot$ dat $F,+,\cdot$ en $S,+,\cdot$ omvat''.
  \[ S = \{ u_{1},\dotsc,u_{n} \}:\ F(u_{1},\dotsc,u_{n}) = F(S) \]
\end{de}

\extra{formulering p 85 algebra}

\begin{st}
  Zij $F,+,\cdot$ en $S,+,\cdot$ velden en $E,+,\cdot$ een uitbreiding van zowal $F,+,\cdot$ als $S,+,\cdot$.
  \[ S = S_{1} \cup S_{2} \Rightarrow F(S) = F(S_{1})(S_{2}) \]
\extra{bewijs}
\end{st}

\begin{st}
  Zij $F,+,\cdot$ een veld en $E,+,\cdot$ een uitbreiding van $F,+,\cdot$.
  Zij $u_{1},u_{2}$ elementen van $E$.
  \[ F(u_{1},u_{2}) = F(u_{2})(u_{1})\]
\extra{bewijs}
\end{st}

\begin{st}
  Zij $F,+,\cdot$ een veld en $E,+,\cdot$ een uitbreiding van $F,+,\cdot$.
  Zij $u_{1},\dotsc,u_{n}$ elementen van $E$.
  \[ F(u_{1},\dotsc,u_{n}) = F(u_{1},\dotsc,u_{n-1})(u_{n})\]
\extra{bewijs}
\end{st}

\begin{st}
  De \term{stelling van het primitieve element}\\
  Zij $F,+,\cdot$ een veld van karakteristiek $0$.
  Zij $a$ en $b$ elementen in een uitbreiding van $F$ die algebra\"isch zijn over $F$,
  dan bestaat er een element $c\in F(a,b)$ zodat het volgende geldt:
  \[ F(a,b) = F(c) \]
  \zb
\end{st}

\begin{opm}
  Elke eindige uitbreiding is dus in feite een enkelvoudige uitbreiding.
\end{opm}

\begin{de}
  Het element dat van een eindige uitbreiding een enkelvoudige uitbreiding maakt noemt men een \term{primitief element}.
\end{de}

\subsection{Algebra\"ische uitbreidingen}
\label{sec:algebr-uitbr}
\begin{de}
  Zij $F,+,\cdot$ een veld en $E,+,\cdot$ een uitbreiding van $F,+,\cdot$.
  $E$ is \term{algebra\"isch} over $F$ als elk element $u\in E$ algebra\"isch is over $F$.
  Anders noemen we $E,+,\cdot$ \term{transcendent}.
\end{de}

\begin{pr}
  Zij $F,+,\cdot$ een veld en $E,+,\cdot$ een eindige uitbreiding van $F,+,\cdot$, dan is $E,+,\cdot$ algebra\"isch over $F,+,\cdot$.
\TODO{bewijs p 86}
\end{pr}

\begin{opm}
  Het omgekeerde van deze eigenschap geldt niet.
\extra{voorbeeld}
\end{opm}

\begin{st}
  Zij $F,+,\cdot$ een veld en $E,+,\cdot$ een uitbreiding van $F,+,\cdot$.
  Zij $u_{1},\dotsc,u_{n}$ elementen van $E$.
  \[ F(u) \text{ is algebra\"isch } over F \Leftrightarrow u \text{ is algebra\"isch over } F \]
\TODO{bewijs p 86}
\end{st}

\begin{st}
  Zij $F,+,\cdot$ een veld met nulelement $e$ en $E,+,\cdot$ een uitbreiding van $F,+,\cdot$.
  Zij $a$ en $b$ elementen van $E$.
  Als $a$ en $b$ algebra\"isch zijn over $F$, dan zijn ook de volgende elementen algebra\"isch over $F$:
  \begin{itemize}
  \item $a+b$
  \item $a-b$
  \item $a\cdot b$
  \item $\frac{a}{b}$ als $b \neq e$
  \end{itemize}
\TODO{bewijs p 87}
\end{st}

\begin{gev}
  Zij $F,+,\cdot$ een veld en $E,+,\cdot$ een uitbreiding van $F,+,\cdot$.
  De elementen van $E$ die algebra\"isch zijn over $F,+,\cdot$ vormen een deelveld van $E,+,\cdot$ en een uitbreiding van $F,+,\cdot$.
\extra{bewijs}
\end{gev}

\begin{de}
  Zij $F,+,\cdot$ een veld en $E,+,\cdot$ een uitbreiding van $F,+,\cdot$.
  Het deelveld van de elementen van $E$ die algebra\"isch zijn over $F,+,\cdot$ noemen we de algebra\"ische sluiting van $F,+,\cdot$ in $E,+,\cdot$.
\end{de}

\begin{st}
  De transitiviteit van het algebra\"isch zijn.\\
  Zij $F,+,\cdot$ een veld, $L,+,\cdot$ een uitbreiding van $F,+,\cdot$ en $E,+,\cdot$ een uitbreiding van $L,+,\cdot$.
  Als $E,+,\cdot$ algebra\"isch is over $L,+,\cdot$ en $L,+,\cdot$ algebra\"isch over $F,+,\cdot$, dan is $E,+,\cdot$ algebra\"isch over $F,+,\cdot$.
 \TODO{bewijs p 88}
\end{st}

\subsection{Ontbindingsvelden}
\label{sec:ontbindingsvelden}

\begin{de}
  Zij $K,+,\cdot$ een veld en $f$ een niet-constante veelterm in $K[X]$.
  Een \term{ontbindingsveld} van $f$ over $K,+,\cdot$ is een uitbreidingsveld $E,+,\cdot$ van $K,+,\cdot$ met de volgende eigenschappen.
  \begin{itemize}
  \item $\exists c, a_{i} \in E:\ f = c(X-a_{1})(X-a_{2})\dotsc(X-a_{n})$
  \item Er is geen enkel veld $L,+,\cdot$, strikt tussen $K,+,\cdot$ en $E,+,\cdot$.
  \end{itemize}
\end{de}

\begin{st}
  Zij $K,+,\cdot$ een veld en $f$ een niet-constante veelterm in $K[X]$.
  De volgende eigenschappen vormen voor een uitbreidingsveld $E,+,\cdot$ van $K,+,\cdot$ een equivalente definite van een ontbindingsveld.
  \begin{itemize}
  \item $\exists c, a_{i} \in E:\ f = c(X-a_{1})(X-a_{2})\dotsc(X-a_{n})$
  \item $E = K(a_{1},a_{2},\dotsc,a_{n})$
  \end{itemize} 
\extra{bewijs}
\end{st}

\begin{st}
  Zij $K,+,\cdot$ een veld en $f$ een niet-constante veelterm in $K[X]$ van graad $n$.
  Er bestaat een ontbindingsveld $E,+,\cdot$ van $f$ over $K,+,\cdot$ zodat het volgende geldt:
  \[ [E:K] \le n! \]
\TODO{bewijs p 89}
\end{st}

\begin{st}
  Zij $K,+,\cdot$ een veld en $f$ een niet-constante veelterm in $K[X]$.
  Het ontbindingsveld van $f$ over $K,+,\cdot$ is uniek op isomorfisme na.
  \zb
\end{st}

\subsection{Cyclotome uitbreidingen en Cyclotome veeltermen}
\label{sec:cycl-uitbr-en}

\begin{de}
  Definieer $\omega=\omega_{n}$ als de eenheidswortel van $X^{n}-1$ in $\mathbb{C},+,\cdot$.
  \[ \omega = e^{\frac{2\pi i}{n}} = \cos\frac{2\pi}{n} + i\sin\frac{2\pi}{n} \]
\end{de}

\begin{st}
  Het ontbindingsveld van $X^{n}-1$ over $\mathbb{Q}$ is de enkelvoudige uitbreiding $\mathbb{Q}(\omega)$.
\extra{bewijs}
\end{st}

\begin{de}
  Voor $n\in \mathbb{N}_{0}$ noemt men $\mathbb{Q}(\omega_{n})$ het $n$-de \term{cyclotome veld}.
\end{de}

\begin{de}
  Voor elk deelveld $F,+,\cdot$ van $\mathbb{C},+,\cdot$ noemt men $F(\omega_{n})$ de $n$-de \term{cyclotome uitbreiding} van $F,+,\cdot$.
\end{de}

\begin{st}
  Een regelmatige $n$-hoek is construeerbaar met enkel passer in lineaal als en slechts als $n$ van de volgende vorm is.
  \[ n = 2^{k}p_{1}p_{2}\dotsc p_{r} \text{ met } k\ge 0 \]
  In bovenstaande formule zijn de $p_{i}$ onderling verschillende priemgetallen van de vorm $p^{2^{m}} +1$.
  \zb
\end{st}

\begin{de}
  Zij $n\in \mathbb{N}_{0}$ een getal.
  De $n$-de \term{cyclotome veelterm} (over $\mathbb{Q}$) is $\Phi_{n}$.
  \[ \Phi_{n} = \prod_{0\le i \le n-1,\ ggd(i,n)= 1}(X-\omega_{n}^{i})\]
  Merk op dat $\Phi_{n}$ monisch is in graad $\phi(n)$.
\end{de}

\begin{ei}
  Zij $n\in \mathbb{N}_{0}$ een getal.
  \[ X^{n}-1 = \prod_{1\le d\le n,\ d|n}\Phi_{d} \]
\TODO{bewijs p 91}
\end{ei}

\begin{pr}
  Voor elke $n\in \mathbb{N}_{0}$ geldt $\Phi_{n} \in \mathbb{Z}[X]$.
\TODO{bewijs p 91}
\end{pr}

\begin{st}
  De cyclotome veelterm $\Phi_{n}$ is irreducibel in $\mathbb{Z}[X]$ voor elke $n\in \mathbb{N}_{0}$.
  \zb
\end{st}

\begin{gev}
  De minimalee veelterm van $\omega_{n}$ over $\mathbb{Q}$ is de cyclotome veelterm $\Phi_{n}$.
\extra{bewijs}
\end{gev}

\begin{gev}
  \[ \mathbb{Q}(\omega_{n}) \cong \mathbb{Q}[X]/(\Phi_{n}) \quad\text{ en }\quad [\mathbb{Q}(\omega_{n}):\mathbb{Q}] = \phi_{n} \]
\extra{bewijs}
\end{gev}

\subsection{Meervoudige wortels}
\label{sec:meervoudige-wortels}

\begin{de}
  Zij $K,+,\cdot$ een veld en $f\in K[X]$ een veelterm over $K,+,\cdot$.
  De veelterm $f$ heeft een \term{meervoudige wortel} $a$ in een uitbreiding $L,+,\cdot$ van $K,+,\cdot$ als er een veelterm $g\in L[X]$ over $L,+,\cdot$ bestaat zodat het volgende geldt binnen $L[X]$:
  \[ f = (X-a)^{2}g\]
\end{de}

\begin{de}
  Zij $K,+,\cdot$ een veld en $f\in K[X]$ een veelterm over $K,+,\cdot$.
  De \term{afgeleide} van $f$ is $f'\in K[X]$:
  \[ f= \sum_{i=0}^{n}a_{i}X^{i} \longrightarrow f' = \sum_{i=1}^{n}ia_{i}X^{i-1} \]
\end{de}

\begin{st}
  Rekenregels voor afgeleiden\\
  Zij $K,+,\cdot$ een veld, $c$ een element van $F$(???) en $f,g$ veeltermen over $K,+,\cdot$.
  \begin{itemize}
  \item $(cf)'= cf'$
  \item $(f+g)' = f'+g'$
  \item $(fg)' = f'g+ fg'$
  \end{itemize}
\extra{bewijs}
\end{st}

\begin{st}
  Het \term{criterium voor een meervoudige wortel}\\
  Zij $K,+,\cdot$ een veld en $f\in K[X]$ een veelterm over $K,+,\cdot$.
  $f$ heeft een meervoudige wortel (in een uitbreiding van $K,+,\cdot$) als en slechts als $f$ en $f'$ een gemeenschappelijke niet-constante deler hebben in $K[X]$.
\TODO{bewijs p 93}
\end{st}

\begin{st}
  Zij $K,+,\cdot$ een veld en $f$ een irriducibele veelterm in $K[X]$.
  \begin{itemize}
  \item Als $char(K)$ nul is, dan heeft $f$ geen meervoudige wortel.
  \item Als $char(K)$ niet nul is, dan eeft $f$ een meervoudige wortel als en slechts als er een $g\in K[X]$ bestaat zodat $f=g(X^{p})$ geldt.
\TODO{bewijs p 93}
  \end{itemize}
\end{st}

\begin{st}
  Een irreducibele veelterm over een eindig veld van karakteristiek niet nul heeft geen meervoudige wortel.
  \zb
\end{st}

\section{Algebra\"isch gesloten velden}
\label{sec:algebr-gesl-veld}

\begin{de}
  Een veld $F,+,\cdot$ is \term{algebra\"isch gesloten} als en slechts als elke niet-constante veelterm in $F[X]$ een wortel heeft in $F,+,\cdot$.
\end{de}

\begin{st}
  Alsternatieve definitie 1\\
  Een veld $F,+,\cdot$ is \term{algebra\"isch gesloten} als en slechts als elke niet-constante veelterm in $F[X]$ een product is van lineaire veeltermen in $F[X]$.
  \extra{bewijs}
\end{st}

\begin{st}
  Alsternatieve definitie 1\\
  Een veld $F,+,\cdot$ is \term{algebra\"isch gesloten} als en slechts als alle irreducibele veeltermen in $F[X]$ graad $1$ hebben.
  \extra{bewijs}
\end{st}

\begin{ei}
  Een veld $F,+,\cdot$ is algebra\"isch gesloten als en slechts als voor elke algebra\"ische uitbreiding $E,+,\cdot$ van $F,+,\cdot$ $E$ gelijk is aan $F$.
\TODO{bewijs p 95}
\end{ei}

\begin{st}
  De \term{hoofdstelling van de algebra}\\
  $\mathbb{C}$ is algebra\"isch gesloten.
\zb 
\extra{referentie naar bewijs?}
\end{st}


\begin{pr}
  Het veld der algebra\"ische getallen is algebra\"isch gesloten.
  \[ \mathbb{Q} = \{ a \in \mathbb{C} \ |\ a \text{ is algebra\"isch over } \mathbb{Q} \} \]
\TODO{bewijs p 96}
\end{pr}

\begin{de}
  Zij $F,+,\cdot$ een veld.
  Een \term{algebra\"ische sluiting} van $F,+,\cdot$ is een algebra\"isch gesloten uitbreiding $\overline{F},+,\cdot$ van $F,+,\cdot$ die algebra\"isch is over $F,+,\cdot$.
\end{de}

\begin{lem}
  Zij $K,+,\cdot$ een veld, dan bestaat er een algebra\"ische sluiting $K_{1},+,\cdot$ zodat elke niet-constante veelterm in $K[X]$ een wortel heeft in $K_{1}$.
\TODO{bewijs p 96}
\end{lem}

\begin{st}
  Elk veld heeft een algebra\"ische sluiting.
\TODO{bewijs p 97}
\end{st}

\section{Eindige velden}
\label{sec:eindige-velden}

\begin{st}
  De multiplicatieve groep $F^{\times},\cdot$ van een eindig veld $F,+,\cdot$ is cyclisch.
\TODO{bewijs p 99}
\TODO{bewijs op toledo}
\end{st}

\begin{opm}
  Bovenstaand bewijs is niet constructief.
  Er is hiervan geen constructief bewijs gekend.
\end{opm}

\begin{gev}
  Zij $F,+,\cdot$ een veld van orde $q$, dan is de multiplicatieve groep $F^{\times},\cdot$ ervan isomorf met $\mathbb{Z}_{q-1},+$.
\extra{bewijs}
\end{gev}

\begin{de}
  Een generator van de multiplicatieve groep $F^{\times},\cdot$ van een eindig veld $F,+,\cdot$ noemt men soms een \term{primitief element} van $F,+,\cdot$.
\end{de}

\subsection{Bestaan en uniciteit}
\label{sec:bestaan-en-uniciteit}

\begin{st}
  Zij $p$ een priemgetal waarvoor $q=p^{r}$ geldt met $r\in \mathbb{N}_{0}$.
  Er bestaat een veld met $q$ elementen, namelijk een ontbindingsveld van $X^{q}-X$ over $\mathbb{Z}_{p}$.
\TODO{bewijs p 100}
\end{st}

\begin{st}
  Zij $p$ een priemgetal waarvoor $q=p^{r}$ geldt met $r\in \mathbb{N}_{0}$.
  Elk veld $F,+,\cdot$ met $q$ elementen is een ontbindingsveld van $X^{q}-X$ over $\mathbb{Z}_p$
\TODO{bewijs p 100}
\end{st}

\begin{st}
  Zij $p$ een priemgetal waarvoor $q=p^{r}$ geldt met $r\in \mathbb{N}_{0}$.
  Twee verschillende velden met $q$ elementen zijn isomorf.
\TODO{bewijs p 100}
\end{st}

\begin{de}
  Zij $p$ een priemgetal waarvoor $q=p^{r}$ geldt met $r\in \mathbb{N}_{0}$.
  Het veld met $q$ elementen noteert men als $\mathbb{F}_{q}$ of $GF(p)$.
\end{de}

\begin{st}
  Zij $p$ een priemgetal waarvoor $q=p^{r}$ geldt met $r\in \mathbb{N}_{0}$.
  Er bestaat een irreducibele veelterm van graad $r$ in $\mathbb{F}_{p}[X]$.
\TODO{bewijs p 101}
\end{st}

\begin{st}
  Zij $p$ een priemgetal waarvoor $q=p^{r}$ geldt met $r\in \mathbb{N}_{0}$.
  Elke irreducibele veelterm van graad $r$ in $\mathbb{F}_{p}[X]$ is een deler van $X^{p}-X$ in $\mathbb{F}_{p}[X]$.
\TODO{bewijs p 101}
\end{st}

\begin{st}
  Zij $p$ een priemgetal waarvoor $q=p^{r}$ geldt met $r\in \mathbb{N}_{0}$.
  De irreducibele factoren van $X^{q}-X$ in $\mathbb{F}_{p}[X]$ zijn precies de irreducibele veeltermen in $\mathbb{F}_{p}[X]$.
\TODO{bewijs p 101}
\end{st}

\begin{lem}
  Zij $k$ een deler van $r$ in $\mathbb{N}_{0}$ en $p$ een priemgetal.
  Definieer $q$ en $q'$ als $q=p^{r}$ en $q'=p^{k}$.
  \[ X^{q'}-X | X^{q}-X \]
  \TODO{bewijs p 102}
\end{lem}


\subsection{Onderlinge inclusies}
\label{sec:onderlinge-inclusies}

\begin{st}
  Zij $p$ een priemgetal.
  Een veld met $p^{r}$ elementen heeft een uniek deelveld met $p^{k}$ elementen als en slechs als $k|r$ geldt.
\TODO{bewijs p 103}
\end{st}

\begin{pr}
  Zij $p$ een priemgetal.
  Beschouw voor elke $k$ en $r$ in $\mathbb{N}$ waarbij $k|r$ geldt het veld $\mathbb{F}_{p^{k}}$ als deelveld van $\mathbb{F}_{p^{r}}$.
  $\overline{\mathbb{F}_{p}}$ is dan een algebra\"ische sluiting van $\mathbb{F}_{p}$.
  \[ \overline{\mathbb{F}_{p}} = \cup_{i\in \mathbb{N}_{0}}\mathbb{F}_{p^{i}} \]
\TODO{bewijs p 104}
\end{pr}

\end{document}

%%% Local Variables:
%%% mode: latex
%%% TeX-master: t
%%% End:
