\documentclass[main.tex]{subfiles}
\begin{document}

\chapter{Ringen}
\label{cha:ringen}

\section{Abstracte ringen}
\label{sec:abstracte-ringen}

\subsection{Ring}
\label{sec:ring}

\begin{de}
  Een \term{ring} $R,+,\cdot$ is een (niet lege) verzameling $R$ waarop twee inwendige bewerkingen $+$ en $\cdot$ gedefinieerd zijn met de volgende eigenschappen:
  \begin{itemize}
  \item $R,+$ is een commutatieve groep.
  \item $\cdot$ is associatief.
    \[ \forall x,y,z \in R:\ (x\cdot y) \cdot z = x \cdot (y \cdot z) \]
  \item $\cdot$ is distributief ten opzichte van $+$.
    \[ \forall x,y,z \in R:\ x\cdot (y + z) = (x \cdot y) + (x \cdot z) = (y \cdot x) + (z \cdot x) \]
  \end{itemize}
\end{de}

\begin{de}
  In de context van een ring $R,+,\cdot$ noemen we het neutraal element van $R,+$ het \term{nulelement} van $R,+,\cdot$.
\end{de}

\begin{ei}
  \label{ei:nulelement-opslorpend}
  Zij $R,+,\cdot$ een ring met nulelement $e$.
  \[ \forall a \in R:\ a \cdot e = e = e \cdot a \]

  \begin{proof}
    \[
    \begin{array}{rll}
      a \cdot e &= a \cdot (-e+e) &\\
                &= a \cdot (e+e) &\\
                &= (a \cdot e) + (a \cdot e)\\
      \Rightarrow & a \cdot e = e\\
    \end{array}
    \]
    \[
    \begin{array}{rll}
      e \cdot a &= (-e+e) \cdot a &\\
                &= (e+e) \cdot a &\\
                &= (e \cdot a) + (e \cdot a) &= e\\
      \Rightarrow & e \cdot a = e\\
    \end{array}
    \]
    We gebruiken in de tweede gelijkheid dat $e$ zijn eigen inverse is en in de derde gelijkheid dat $\cdot$ distributief is ten opzichte van $+$.
  \end{proof}
\end{ei}

\begin{ei}
  \label{ei:min-door-maal}
  Zij $R,+,\cdot$ een ring met nulelement $e$. 
  \[ \forall a,b \in R:\ a \cdot (-b) = -(a \cdot b) = (-a) \cdot b \]
 
  \begin{proof}
    We bewijzen dat $a \cdot (-b)$ en $a \cdot b$ elkaars inversen zijn in $R,+$ alsook $-a \cdot b$ en $a \cdot b$.
    Dit betekent dat $a \cdot (-b)$ en $-a \cdot b$ gelijk zijn aan $-(a \cdot b)$ want de inverse is uniek.\stref{st:groep-uniek-invers-element}
    \[
    \begin{array}{rll}
      (a \cdot -b) + (a \cdot b) &= a \cdot (-b + b) &\\
                                 &= a \cdot e &= e 
    \end{array}
    \]  
    \[
    \begin{array}{rll}
      (-a \cdot b) + (a \cdot b) &= (-a + a) \cdot b &\\
                                 &= e \cdot b &= e 
    \end{array}
    \]      
  \end{proof}
\end{ei}

\begin{ei}
  Zij $R,+,\cdot$ een ring met nulelement.
  \[ \forall a,b \in R:\ (-a) \cdot (-b) = a \cdot b \]

  \begin{proof}
    We bewijzen dat $(-a) \cdot (-b)$ en $-(a \cdot b)$ elkaars inversen zijn in $R,+$
    Dit betekent dat $(-a) \cdot (-b)$ gelijk is aan $a \cdot b)$ want de inverse is uniek.\stref{st:groep-uniek-invers-element}
    \[
    \begin{array}{rll}
      (-a) \cdot (-b) + (- (a \cdot b)) &= (-a) \cdot (-b) + (-a) \cdot b &\\
                                        &= (-a) \cdot (-b + b) &\\
                                        &= (-a) \cdot e &= e
    \end{array}
    \]
  \end{proof}
\end{ei}

\begin{de}
  Wanneer in de context van een ring $R,+,\cdot$ de multiplicatieve notatie wordt gebruikt voor de tweede bewerking korten we ``$x \cdot y$'' vaak af als ``$xy$''.
\end{de}

\begin{de}
  Zij $R,+,\cdot$ een ring, dan korten we de verzameling $R\setminus \{e_{R,+}\}$ vaak af met $R_{0}$ of $R_{e}$.
\end{de}

\begin{de}
  Zij $R,+,\cdot$ een ring en $x$ en $y$ twee elementen van $R_{e}$, dan noemen we $x$...
  \begin{itemize}
  \item ... een \term{linker nuldeler} van $y$ als $x\cdot y$ het neutraal element is van $R,+$.
  \item ... een \term{rechter nuldeler} van $y$ als $y\cdot x$ het neutraal element is vn $R,+$.
  \end{itemize}
\end{de}

\begin{de}
  Zij $R,+,\cdot$ een ring, dan noemen we $x\in R$ een \term{nuldeler} van $y\in R$ als $x$ zowel een linker als rechter nuldeler is van $y$.
\end{de}

\begin{ei}
  \label{ei:nuldeler-asa-niet-schrapbaar}
  Zij $R,+,\cdot$ een niet-triviale ring en $z$ een element van $R$, dan is $z$ een nuldeler als en slechts als $z$ niet schrapbaar is voor $\cdot$.
  Noem het nulelement van $R,+,\cdot$ $e$.
  \[ z \text{ is een nuldeler } \Leftrightarrow z \text{ is niet schrapbaar voor } \cdot \]
  \begin{proof}
    Bewijs van een equivalentie.
    \begin{itemize}
    \item $\Rightarrow$\\
      Als $z$ een nuldeler is, dan bestaat er dus een $x \in R_{e}$ zodat $x\cdot z= e$ of $z \cdot x = e$ geldt.
      $z$ kan dan niet schrapbaar zijn voor $\cdot$ want $z\cdot e = e = e \cdot z$ geldt\eiref{ei:nulelement-opslorpend}:
      \[ z \cdot x = z \cdot e \wedge x \neq e \quad\text{ of }\quad x \cdot z = e \cdot z \wedge x \neq e \]
    \item $\Leftarrow$\\
      Stel dat er twee verschillende elementen $x$ en $y$ in $R$ bestaan zodat $z \cdot x$ gelijk is aan $z \cdot y$.
      $z$ is dan dus niet schrapbaar.
      \[ z \cdot x = z \cdot y \wedge x \neq y \]
      $z$ moet dan een nuldeler zijn:
      \[
      \begin{array}{rll}
        z \cdot (x + (-y)) &= z \cdot x + z \cdot (-y) &\\
                           &= z \cdot y + (-(z \cdot y)) &= e
      \end{array}
      \]
      Merk op dat dat de tweede gelijkheid enkel geldt omdat $z \cdot (-y) = -(z \cdot y)$ geldt voor elke twee elementen van een ring.\eiref{ei:min-door-maal}
    \end{itemize}
  \end{proof}
\end{ei}

\begin{de}
  \label{de:nulring}
  De \term{nulring} of \term{niet-triviale} ring is de ring met enkel een nulelement $e$.
  \[ \{ e \} \]
\end{de}

\subsection{Ring met eenheidselement}
\label{sec:ring-met-eenheidselement}

\begin{de}
  Zij $R,+,\cdot$ een ring.
  We noemen $R,+,\cdot$ een \term{ring met eenheidselement} als er een \term{eenheidselement} $i\in R$ bestaat als volgt:
  \[ \forall x\in R:\ i\cdot x = x = x \cdot i \]
\end{de}

\begin{ei}
  Zij $R,+,\cdot$ een ring met eenheidselement $i$ en nulelement $e$.
  \[ \forall a \in R:\ (-i)\cdot a = -a = a \cdot (-i) \]

  \begin{proof}
    We bewijzen dat $(-i) \cdot a$ en $a$ alsook $a \cdot (-i)$ en $a$ elkaars inversen zijn in $R,+$.
    Dit betekent dat $(-i) \cdot a$ gelijk is aan $-a$ alsook $a\cdot (-i)$ want de inverse is uniek.\stref{st:groep-uniek-invers-element}
    \[
    \begin{array}{rll}
      (-i) \cdot a + a &= (-i) \cdot a + i \cdot a &\\
                       &= (-i + i) \cdot a &\\
                       &= e \cdot a &= e
    \end{array}
    \]
    \[
    \begin{array}{rll}
      a \cdot (-i) + a &= a \cdot (-i) + a \cdot i &\\
                       &= a \cdot (-i+i) &\\
                       &= a \cdot e &= e
    \end{array}
    \]
  \end{proof}
\end{ei}

\begin{ei}
  Zij $R,+,\cdot$ een ring met eenheidselement $i$.
  \[ (-i) \cdot (-i) = i\]
  \begin{proof}
    We bewijzen dat $(-i) \cdot (-i)$ en $-i$ elkaars inversen zijn in $R,+$.
    Dit betekent dat $(-i) \cdot (-i)$ gelijk is aan $i$ want de inverse is uniek.\stref{st:groep-uniek-invers-element}
    \[
    \begin{array}{rll}
      (-i) \cdot (-i) + (-i) &= (-i) \cdot (-i) + (-i) \cdot i &\\
                             &= (-i) \cdot (-i + i) &\\
                             &= (-i) \cdot e &= e
    \end{array}
    \]
  \end{proof}
\end{ei}

\begin{de}
  Zij $R,+,\cdot$ een ring met eenheidselement $i$.
  We noemen een element $u$ van $R$ een \term{eenheid} als er een element $v$ bestaat van $R$ zodat het volgende geldt:
  \[ u \cdot v = i = v \cdot u \]
\end{de}

\begin{de}
  Zij $R,+,\cdot$ een ring met eenheidselement $i$.
  Zij $u$ een eenheid van $R,+,\cdot$ met $v$, dan noemen we $v$ de \term{inverse} van $u$.
  \[ u \cdot v = i = v \cdot u \]
\end{de}

\begin{st}
  Zij $R,+,\cdot$ een ring met eenheidselement $i$.
  Zij $u$ een eenheid van $R,+,\cdot$, dan is de inverse van $u$ uniek.

  \begin{proof}
    Zij $u$ een eenheid van een ring $R,+,\cdot$.
    Stel dat er twee elementen $v$ en $v'$ invers zijn van $u$.
    \[ u \cdot v = i = v \cdot u \quad\text{ en }\quad u \cdot v' = i = v' \cdot u \]
    \[
    \begin{array}{rll}
      v &= i \cdot v &\\
        &= (v' \cdot u) \cdot v &\\
        &= v' \cdot (u \cdot v) &\\
        &= v' \cdot i &= v'
    \end{array}
    \]
  \end{proof}
\end{st}

\begin{de}
  De \term{eenhedengroep} $R^{\times},\cdot$ van een ring $R,+,\cdot$ met eenheidselement $i$ is de verzameling van eenheden van $R$, uitgerust met de multiplicatieve bewerking van $R$.
  \[ R^{\times} = U(R) = \{ u \in R \ |\ \exists v \in R:\ u \cdot v = i = v \cdot u \} \]
\end{de}

\begin{st}
  Zij $R,+,\cdot$ en $S,\star,*$ ringen met eenheidselement.
  \[ (R \times S)^{\times} = R^{\times}\times S^{\times} \]

  \begin{proof}
    Noteer het eenheidselement van $R,+,\cdot$ als $i_{R}$, het eenheidselement van $S,\star,*$ als $i_{S}$ en het eenheidselement van $(R \times S),(+,\star),(\cdot,*)$ als $i_{R \times S}$.
    \[
    \begin{array}{rll}
      (R \times S)^{\times} &= \{ (r,s) \ |\ r \in R, s \in S \}^{\times} &\\
                          &= \{ (r,s) \in R\times S \ |\ \exists (r',s') \in R\times S:\ (r,s) \cdot (r',s') = i_{R \times S} = (r',s') \cdot (r,s) \}
    \end{array}
    \]
    Merk op dat $i_{R\times S}$ gelijk is aan $(i_{R},i_{S})$.
    \[
    \begin{array}{rll}
      R^{\times} \times S^{\times} &= \{ r \in R \ |\ \exists r' \in R:\ r \cdot r' = i_{R} = r' \cdot r \} \times \{ s \in S \ |\ \exists s' \in S:\ s * s' = i_{S} = s' * s \} &\\
                               &= \{ (r,s) \in R \times S \ |\ (\exists r' \in R:\ r \cdot r' = i_{R} = r' \cdot r) \wedge (\exists s' \in S:\ s * s' = i_{S} = s' * s) \} &\\
    \end{array}
    \]
    Er rest ons nu dus nog op te merken dat de volgende bewering wel degelijk geldt.
    \[
    \begin{array}{c}
      \exists (r',s') \in R\times S:\ (r,s) \cdot (r',s') = i_{R \times S} = (r',s') \cdot (r,s)\\
      \Leftrightarrow \\
      (\exists r' \in R:\ r \cdot r' = i_{R} = r' \cdot r) \wedge (\exists s' \in S:\ s * s' = i_{S} = s' * s)
    \end{array}
    \]
  \end{proof}
\end{st}

\begin{st}
  \label{st:nulring-nul-is-een}
  Een ring $R,+,\cdot$ met eenheidselement $i$ en nulelement $e$ waarbij $i$ gelijk is aan $e$ is de nulring.
  \extra{bewijs}
\end{st}

\subsection{Commutatieve ring}
\label{sec:commutatieve-rin}

\begin{de}
  Zij $R,+,\cdot$ een ring waarbij $\cdot$ commutatief is, dan noemen we $R,+,\cdot$ een \term{commutatieve ring}.
  \[ \forall x,y \in R:\ x\cdot y = y \cdot x \]
\end{de}

\begin{de}
  Zij $R,+,\cdot$ een commutatieve ring en $a$ en $b$ elementen van $R$.
  We noemen $a$ een \term{deler} van $b$ (in $R,+,\cdot$) als er een $q$ in $R$ bestaat zodat het volgende geldt:
  \[ b = q \cdot a \]
\end{de}

\subsection{Integriteitsdomeinen}
\label{sec:integriteitsdomeinen}

\begin{de}
  \label{de:integriteitsdomein}
  Een \term{(integriteits)domein} $R,+,\cdot$ is een (niet-triviale) commutatieve ring $R,+,\cdot$ met eenheidselement, zonder nuldelers.
  Zij $e$ het nulelement van $R,+,\cdot$:
  \[ \forall a, b \in R:\ (a \cdot x = e) \Rightarrow (x = e \vee y = e) \]
\end{de}

\begin{ei}
  Zij $R,+,\cdot$ een intergiteitsdomein met nulelement $e$
  \[ \forall a,b,c \in R:\ (a\cdot b = a\cdot c \wedge a \neq e) \Rightarrow b = c \]

  \begin{proof}
    Er zijn geen nuldelers in $R,+,\cdot$, want het is een integriteitsdomein, dus elk element is schrapbaar.\eiref{ei:nuldeler-asa-niet-schrapbaar}
  \end{proof}
\end{ei}

\begin{opm}
  In een willekeurige ring zijn niet alle elementen schrapbaar.
  In $\mathbb{Z}_{10}$ geldt $5\cdot 3 = 5 \cdot 7$ maar $3 \neq 7$.
\end{opm}

\subsection{Lichaam}
\label{sec:lichaam}

\begin{de}
  Zij $R,+,\cdot$ een (niet-triviale) ring met eenheidselement $i$.
  We noemen $R,+,\cdot$ een \term{lichaam} als $R_{e},\cdot$ een groep is:
  \[ \forall x \in R_{e}, \exists y \in R:\ x\cdot y = y \cdot x = i \]
  Elk element in $R_{e}$ is dus een eenheid.
\end{de}

\begin{st}
  \label{st:lichaam-geen-nuldelers}
  Een lichaam heeft geen nuldelers

  \begin{proof}
    Zij $R,+,\cdot$ een lichaam met eenheidselement $i$ en nulelement $e$.
    Stel nu dat er element $x$ en $y$ in $R$ bestaan zodat $x\cdot y = e$ of $y \cdot x = e$ geldt.
    \[
    \begin{array}{rrl}
      x \cdot y &= e\\
      \Rightarrow & x^{-1} \cdot (x \cdot y) \cdot y^{-1} &= x^{-1} \cdot e \cdot y^{-1}\\
      \Rightarrow & (x^{-1} \cdot x) \cdot (y \cdot y^{-1}) &= e\\
      \Rightarrow & i \cdot i &= e\\
      \Rightarrow & i &= e
    \end{array}
    \]
    De eerste equivalentie geldt omdat elk element een eenheid is.
    De tweede equivalentie geldt omdat het nulelement zich opslorpend gedraagt voor de multiplicatieve bewerking in een ring.\eiref{ei:nulelement-opslorpend}
    De laatste gelijkheid impliceert dat de ring $R$ triviaal is.
    Contradictie.\stref{st:nulring-nul-is-een}
  \end{proof}
\end{st}

\begin{st}
  \label{st:stelling-van-wedderburn}
  De \term{stelling van Wedderburn}\\
  Een eindig lichaam is commutatief en dus een veld.

  \begin{proof}
    Geen bewijs.
  \end{proof}
\end{st}

\subsection{Velden}
\label{sec:velden}

\begin{de}
  Een \term{veld} is een commutatief lichaam.
\end{de}

\begin{st}
  Een eindig integriteitsdomein is een veld.

  \begin{proof}
    Zij $D$ een eindig integriteitsdomein met nulelement $e$ en eenheidselement $i$.
    We moeten aantonen dat elk element $a$, verschillend van $e$, in $d$ een eenheid is.
    Kies een element $a$ van $D$ verschillend van $i$, (voor $a=i$ is $a$ namelijk zeker een eenheid).
    Beschouw nu de elementen $a^{n}$ voor elke $n\in \mathbb{N}$.
    Omdat $D$ eindig is bestaan er getallen $i$ en $j$ in $\mathbb{N}$ zodat $a^{i}$ gelijk is aan $a^{j}$ met $i > j$.
    \[ a^{i} = a^{j} \]
    Omdat er geen nuldelers zijn in $D$\deref{de:integriteitsdomein}, mogen we dit vereenvoudigen:
    \[ a^{i-j} = i \]
    Dit betekent precies dat $a^{i-j-1}$ de invers is van $a$.
  \end{proof}
\end{st}

\begin{gev}
  Zij $p$ een priemgetal, dan is $\mathbb{Z}_{p},+,\cdot$ een veld.
\extra{bewijs}
\end{gev}

\begin{st}
  In een veld hebben alle van het nulelement verschillende elementen dezelfde additieve orde.
  
  \begin{proof}
    Zij $F,+,\cdot$ een veld met eenheidselement $i$ en nulelement $e$.
    Kies nu twee elementen $a$ en $b$ uit $F_{e}$.
    Zij $r$ de additieve orde van $a$ zodat $ra = e$, dan bewijzen we nu dat $rb$ ook gelijk is aan het nulelement.
    \[
    \begin{array}{rll}
      rb &= r(ib) &\\
         &= r(aa^{-1}b) &\\
         &= (ra)a^{-1}b &\\
         &= e(a^{-1}b) &= e
    \end{array}
    \]
  \end{proof}
\end{st}

\begin{de}
  De additieve orde van de van $0$ verschillende elementen van een veld noemt met de \term{karakteristiek} van het veld.
\end{de}


\subsection{Direct product}
\label{sec:direct-product}


\begin{de}
  Zij $R_{i},+_{i},\cdot_{i}$ $n$ ringen, dan definieren we het direct product van deze $R_{i}$ als de productverzameling met dezelfde bewerkingen, maar dan op paarsgewijze elementen.
  \[ (R_{1} \times \dotsb \times R_{n}),+,\cdot \]
\end{de}

\begin{st}
  Het direct product van $n$ ringen $R_{i},+_{i},\cdot_{i}$ heeft een eenheidselement als en slechts als elk van de ringen $R_{i},+_{i},\cdot_{i}$ een eenheidselement heeft is.
  \extra{bewijs}
\end{st}

\begin{st}
  Het direct product van $n$ ringen $R_{i},+_{i},\cdot_{i}$ is commutatief als en slechts als elk van de ringen $R_{i},+_{i},\cdot_{i}$ commutatief is.
  \extra{bewijs}
\end{st}

\begin{st}
  Het direct product van $n$ ringen $R_{i},+_{i},\cdot_{i}$ kan geen veld zijn als elk van de ringen $R_{i},+_{i},\cdot_{i}$ verschillend is van de nulring.
  \extra{bewijs}
\end{st}

\subsection{Deelringen}
\label{sec:deelringen}

\begin{de}
  Zij $R,+,\cdot$ een ring en $S$ een niet-lege deelverzameling van $R$.
  We noemen $S$ een \term{deelring} van $R$ als $S$ een ring is voor dezelfde bewerkingen.
\end{de}

\begin{st}
  \label{st:deelring-criteria}
  Criteria voor een deelring.
  Zij $R,+,\cdot$ een ring en $S$ een niet-lege deelverzameling van $R$, dan is $S$ een deelring van $R$ als aan \'e\'en van de volgende criteria voldaan is.
  \begin{itemize}
  \item $S,+$ is een deelgroep van $R,+$ en $\cdot$ is intern in $S$.
  \item $S$ is niet leeg, $\cdot$ is intern in $S$ en het volgende geldt:
    \[ \forall a,b \in S: a - b \in S \]
  \end{itemize}
\extra{bewijs}
\end{st}

\subsection{Ringmorfismen}
\label{sec:ringmorfismen}

\begin{de}
  Zij $R,+,\cdot$ en $S,\star,*$ ringen.
  Een \term{(ring)(homo)morfisme} van $R$ naar $S$ is een afbeelding $f: R\rightarrow S$ die voldoet aan twee voorwaarden:
  \begin{enumerate}
  \item $\forall x,y \in R:\ f(x + y) = f(x) \star f(y)$
  \item $\forall x,y \in R:\ f(x \cdot y) = f(x) * f(y)$
  \end{enumerate}
\end{de}

\begin{opm}
  Een ringisomorfisme $f$ van een ring $R,+,\cdot$ naar $S,\star,*$ is een groepsisomorfisme van de groep $R,+$ naar $S,\star$
\end{opm}

\begin{de}
  Een bijectief ringhomomorfisme is een \term{(ring)isomorfisme}.
\end{de}

\begin{de}
  Een morfisme van een ring naar zichzelf heet een \term{endomorfisme}.
\end{de}

\begin{de}
  Een isomorfisme van een ring naar zichzelf heet een \term{automorfisme}.
\end{de}

\begin{de}
  Zij $R,+,\cdot$ en $S,\star,*$ ringen en $f:R \rightarrow S$ een ringmorfisme.
  Zij $e$ het nulelement van $S$.
  De \term{kern} $Ker(f)$ van $f$ is de verzameling van elementen van $R$ die onder $f$ op $e\in S$ worden afgebeeldt.
  \[ Ker(f) = \{ x \in R \ |\ f(x) = 0 \} \]
\end{de}

\begin{ei}
  Zij $R,+,\cdot$ en $S,\star,*$ ringen met nulelementen $e_{R}$ en $e_{S}$, en $f:R \rightarrow S$ een morfisme.
  \[ f(e_{R}) = e_{S} \]

  \begin{proof}
    $f$ moet de groep $R,+$ afbeelden op $S,\star$, dus het neutraal element van $R,+$ moet op het neutraal element van $S,\star$ afgebeeld worden.\stref{st:groepsmorfisme-behoudt-neutraal-element}
  \end{proof}
\end{ei}
 
\begin{ei}
  Zij $R,+,\cdot$ en $S,\star,*$ ringen en $f:R \rightarrow S$ een morfisme.
  \[ \forall a \in R: f(-a) = -f(a) \]
  \begin{proof}
    $f$ moet de groep $R,+$ afbeelden op $S,\star$, dus $f$ moet de inverse behouden.\stref{st:groepsmorfisme-behoudt-inverse}
  \end{proof}
\end{ei}

\begin{ei}
  Zij $R,+,\cdot$ en $S,\star,*$ ringen en $f:R \rightarrow S$ een morfisme.
  Zij $A$ een deelring van $R$, dan is $f(A)$ een deelring van $S$.
  \extra{bewijs}
\end{ei}

\begin{ei}
  Zij $R,+,\cdot$ en $S,\star,*$ ringen en $f:R \rightarrow S$ een morfisme.
  Zij $B$ een deelring van $S$, dan is $f^{-1}(B)$ een deelring van $R$.
  \extra{bewijs}
\end{ei}

\begin{ei}
  Zij $R,+,\cdot$ een ring met eenheidselement $i$, $S,\star,*$ een ring en $f:R \rightarrow S$ een morfisme.
  De deelring $f(R)$ heeft $f(i)$ als eenheidselement.
  \extra{bewijs}
\end{ei}

\begin{opm}
  Het eenheidselement van een deelring van een ring hoeft niet gelijk zijn aan het eenheidselement van de ring.
  \clarify{hoeft de ring er \'e\'en te hebben?}
  \extra{verwijzen naar vb op p 47 A}
\end{opm}

\begin{ei}
  Zij $R,+,\cdot$ en $S,\star,*$ ringen en $f:R \rightarrow S$ een morfisme.
  Zij $e$ het nulelement van $R$.
  $f$ is injectief als en slechts als $Ker(f) = \{e\}$.
  \extra{bewijs}
\end{ei}

\begin{ei}
  Zij $R,+,\cdot$ en $S,\star,*$ ringen en $f$ een ringisomorfisme van $R$ naar $S$, dan is ook $f^{-1}: S \rightarrow R$ een ringisomorfisme.
  \extra{bewijs}
\end{ei}

\begin{ei}
  \label{ei:ringmorfisme-behoudt-multiplicatieve-inverse}
  Zij $R,+,\cdot$ en $S,\star,*$ ringen met eenheidselement $i_{R}$ en $i_{S}$ en zij $f: R\rightarrow S$ een isomorfisme dat het eenheidselement $i_{R}$ van $R,+,\cdot$ afstuurt op het eenheidselement $i_{S}$ van $S,+,\cdot$.
  Als $u$ een eenheid is in $R$, dan is $f(u)$ een eenheid in $S$.
  Bovendien geldt het volgende:
  \[ (f(u))^{-1} = f(u^{-1}) \]
  \TODO{bewijs}
\end{ei}


\begin{st}
  Zij $R,+,\cdot$ en $S,\star,*$ ringen met eenheidselement $i_{R}$ en $i_{S}$ en zij $f: R\rightarrow S$ een isomorfisme dat het eenheidselement $i_{R}$ van $R,+,\cdot$ afstuurt op het eenheidselement $i_{S}$ van $S,+,\cdot$, dan zijn de eenheidsgroepen $R^{\times},\cdot$ en $S^{\times},\cdot$ isomorf.

  \begin{proof}
    Het is voldoende om aan te tonen dat $f(R^{\times})$ gelijk is aan $S^{\times}$ want $f$ is een isomorfisme.
    Noem $i_{R}$ het eenheidselement van $R^{\times}$ en $i_{S}$ het eenheidselement van $S^{\times}$
    \begin{itemize}
    \item $\subseteq$\\
      Kies een $r\in R^{\times}$, dan bestaat er een inverse $r'$ van $r$ in $R^{\times}$.
      \[ r\cdot r' = i_{R} = r' \cdot r \]
      We beelden nu $i_{R}$ af onder $f$:
      \[ f(r\cdot r') = f(i_{R}) = f(r' \cdot r) \]
      \[ f(r)\cdot f(r') = i_{S} = f(r') \cdot f(r)\]
      Voor element $f(r)$ van $S^{\times}$ bestaat er dus een element $f(r')$ in $S^{\times}$.
      \[ f(R^{\times}) \subseteq S^{\times} \]
    \item $\supseteq$\\
      Omdat $f$ bijectief is, is $f^{-1}$ ook bijectief en bovendien zinvol om te beschouwen.\stref{st:afb-inverse-asa-bijectief}
      Kies nu een element $f(r)$ met inverse $f(r)^{-1}$ in $S,\star,*$.
      \[ f(r) \cdot f(r)^{-1} = i_{S} = f(r)^{-1} \cdot f(r) \]
     We beelden nu $i_{S}$ af onder $f^{-1}$:
     \[ f^{-1}(f(r) \cdot f(r)^{-1}) = f^{-1}(i_{S}) = f^{-1}(f(r)^{-1} \cdot f(r))\]
     \[ f^{-1}(f(r)) \cdot f^{-1}(f(r^{-1})) = f^{-1}(i_{S}) = f^{-1}(f(r^{-1})) \cdot f^{-1}(f(r))\]
     \[ r \cdot r^{-1} = f^{-1} = i_{R} = r^{-1} \cdot r \]
      Voor element $r$ van $R^{\times}$ bestaat er dus een element $r^{-1}$ in $R^{\times}$.
      \[ f^{-1}(S^{\times}) \subseteq R^{\times} \]
    \end{itemize}
  \end{proof}
\end{st}

\begin{pr}
  Zij $R,+,\cdot$ een ring met eenheidselement $i_{R}$ en nulelement $e_{R}$ en zij $S,\star,*$ een integriteitsdomein met eenheidselement $i_{S}$ en nulelement $e_{S}$.
  Zij $f: R \rightarrow S$ een ringmorfisme.
  \[ f(R) = \{e_{S}\} \quad\vee\quad f(i_{R}) = i_{S} \]
\TODO{bewijs p 47 A}
\end{pr}

\begin{pr}
  Zij $R,+,\cdot$ en $S,\star,*$ ringen en $f: R \rightarrow S$ een morfisme.
  $f$ is een isomorfisme als en slechts als en een morfisme $g: S \rightarrow R$ bestaat zodat het volgende geldt:
  \[ g \circ f = I_{R} \quad\wedge\quad f \circ g = I_{S} \]
\TODO{bewijs}
\end{pr}

\subsection{Breukenveld van een integriteitsdomein}
\label{sec:breukenveld-van-een-integriteitsdomein}

\begin{de}
  Zij $R$ een integriteitsdomein, $F$ een veld en $i: R \rightarrow F$ een ringmorfisme als volgt:
  \begin{enumerate}
  \item $i$ is injectief
  \item $\forall q \in F,\ \exists a \in R, b\in R_{e}:\ q = i(a)\cdot i(b)^{-1}$
  \end{enumerate}
  $F$ is dan het \term{breukenveld} van $R$.
\end{de}

\begin{st}
  We kunnen een integriteitsdomein identificeren met zijn beeld onder de inbedding $i$ en dus als deelring van $F$ beschouwen.
  \TODO{bewijs p 48 A}
\end{st}

\begin{ei}
  Het breukenveld van een integriteitsdomein is uniek.
  \TODO{bewijs p 49 A}
\end{ei}

\begin{ei}
  Zij $F$ het breukenveld van een integriteitsdomein $R$ met inbedding $i: R \rightarrow F$.
  Er bestaat voor elk injectief ringmorfisme $f: R \rightarrow K$ met $K$ een veld een uniek ringmorfisme $f': F \rightarrow K$ zodat $f = f' \circ i$ geldt.
  \TODO{bewijs p 49 A}
\end{ei}

\subsection{Idealen}
\label{sec:idealen}

\TODO{linker, en rechterideaal}

\begin{de}
  Zij $R,+,\cdot$ een ring en $I$ een deelverzameling van $R$.
  $I$ is een \term{ideaal} van $R$ als het valgende geldt:
  \begin{itemize}
  \item $I,+$ is een deelgroep van $R,+$.
  \item $forall a \in R, \forall b \in I:\ a\cdot b \subseteq I \wedge b\cdot a \subseteq I$. 
  \end{itemize}
\end{de}

\begin{st}
  Elke ring heeft twee triviale idealen.
  \begin{enumerate}
  \item De ring met enkel het nulelement.
  \item Heel de ring.
  \end{enumerate}
\extra{bewijs}
\end{st}

\begin{ei}
  Een ideaal $I$ van een ring $R,+,\cdot$ is een normaaldeler van de groep $R,+$.
\end{ei}

\begin{de}
  De nevenklassen van een ideaal, beschouwd als normaaldeler, noemen we \term{restklassen}.
\end{de}

\begin{de}
  De \term{quotientring} $R/I$ van een ring $R,+,\cdot$ ten opzichte van een ideaal is de partitie van $R$ is restklassen van $I$, uitgerust met de quotientwetten:
  \begin{itemize}
  \item $\scalebox{1.5}{$+$}:\ R/I \rightarrow R/I:\ (x+I)\ \scalebox{1.5}{$+$}\ (y+I) = (x+y)+I$
  \item $\scalebox{1.5}{$\cdot$}:\ R/I \rightarrow R/I:\ (x+I)\ \scalebox{1.5}{$\cdot$}\ (y+I) = (x\cdot y) + I$
  \end{itemize}
\end{de}




\extra{eenhedengroepen van resklassenringen pagina 35 tot 49 structuren}


\section{deelbaarheid}
\label{sec:deelbaarheid}

\begin{de}
  Zij $a$, $b$ en $q$ elementen van een integriteitsdomein $D,+,\cdot$ zodat $a= qb$ met $b$ niet het nulelement $e$, dan noemen we $b$ een \term{deler} van $a$.
  \[ a\ |\ b \Leftrightarrow \exists q \in D: b = q\cdot a \]
\end{de}

\begin{st}
  Zij $a$, $b$ en $c$ elementen van een integriteitsdomein $D,+,\cdot$.
  \[ (a\ |\ b \wedge a\ |\ c) \Rightarrow a\ |\ (b+c) \]

  \begin{proof}
    $a$ deelt $b$, dus er bestaat een $q \in D$ zodat $b=q\cdot a$ geldt.
    $a$ deelt bovendien $c$, dus er bestaat ook een $q'\in D$ zodat $c = q'\cdot a$ geldt. 
    \[
      b + c = q\cdot a + q'\cdot a = (q+q')\cdot a 
    \]
    Dit betekent dat $a$ ook $b+c$ deelt.
  \end{proof}
\end{st}

\begin{st}
  Zij $a$ en $b$ elementen van een integriteitsdomein $D,+,\cdot$.
  \[ \forall r \in D:\ a\ |\ b \Rightarrow a\ |\ br  \]

  \begin{proof}
    Stel dat $a$ een deler is van $b$ in $D,+,\cdot$, dan bestaat er dus een $q\in D$ zodat $b = qa$ geldt.
    Beschouw nu $br$ voor een willekeurige $r\in D$.
    \[ br = qar = qra \]
    De tweede gelijkheid geldt omdat $\cdot$ commutatief is in $D$.\deref{de:integriteitsdomein}
    $br =qra$ betekent precies dat $a$ een deler is van $br$.
  \end{proof}
\end{st}

\begin{st}
  \label{st:deler-transitief}
  Zij $a$, $b$ en $c$ elementen van een integriteitsdomein $D,+,\cdot$.
  \[ (a\ |\ b \wedge b\ |\ c) \Rightarrow a\ |\ c \]

  \begin{proof}
    $a$ deelt $b$, dus er bestaat een $q \in D$ zodat $b=q\cdot a$ geldt.
    $b$ deelt bovendien $c$, dus er bestaat ook een $q'\in D$ zodat $c = q'\cdot b$ geldt. 
    \[ c = q'b = q'qa \]
    Dit betekent precies dat $a$ $c$ deelt.
  \end{proof}
\end{st}

\begin{de}
  We noemen twee (niet-nul)elementen $a$ en $b$ uit een integriteitsdomein $D,+,\cdot$, met eenheidselement $i$, \term{geassocieerd} als er een eenheid $u$ bestaat in $D$ zodat $a=ub$ geldt:
  \[ a \sim_{D} b \Leftrightarrow \exists (u,u'\in D:\ uu'= i = u'u) \wedge a = ub \]
\end{de}

\begin{st}
  Zij $a$ en $b$ twee elementen uit een integriteitsdomein $D,+,\cdot$.
  \[ a \sim_{D} b \Leftrightarrow (a\ |\ b \wedge b\ |\ a) \]

  \begin{proof}
    Bewijs van een equivalentie.\\
    Noem het eenheidselement van $D$ $i$.
    \begin{itemize}
    \item $\Rightarrow$\\
      Als $a\sim_{D} b$ geldt, dan bestaat er een inverteerbaar element $u$ zodat $a= ub$, dus $b|a$ geldt, maar omdat $u$ inverteerbaar is, geldt ook $u^{-1}a=b$.
      Bijgevolg is $a$ ook een deler van $b$: $a|b$.
    \item $\Leftarrow$\\
      Omdat $a$ en $b$ onderling elkaar delen, bestaan er elementen $x$ en $y$ in $D$ zodat zowel $a=xb$ als $b=ya$ gelden.
      \[ a = xb = xya \Rightarrow x\cdot y = i\]
      $x$ en $y$ zijn elkaars inversen, dus $x$ is inverteerbaar, zodat $a\sim_{D} b$ geldt.
    \end{itemize}
  \end{proof}
\end{st}

\begin{ei}
  \label{ei:associatie-is-equivalentierelatie}
  De associatie $\sim_{D}$ is een equivalientierelatie.
  \begin{proof}
    Inderdaad, de associatie is reflexief, transitief\stref{st:deler-transitief} en symmetrisch.
  \end{proof}
\end{ei}

\begin{st}
  Zij $a$ en $b$ twee elementen uit een integriteitsdomein $D,+,\cdot$.
  \[ (a \sim_{D} b \wedge c \sim_{D} d) \Rightarrow (a\ |\ c \Rightarrow b\ |\ d) \]

  \begin{proof}
    Als $a\sim_{D} b$ en $c\sim_{D} d$ gelden, bestaan er dus inverteerbare elementen $u$ en $v$ in $D$ zodat $a=ub$ en $c=vd$ gelden.
    Stel nu dat $a$ een deler is van $c$, dan bestaat er ook nog een $q\in D$ zodat $c=qa$ geldt.
    \[ d = v^{-1}c = v^{-1}qa = v^{-1}qub \]
    $b$ is dus een deler van $d$.
  \end{proof}
\end{st}

\begin{de}
  We noemen een deler $a$ van $b$ een \term{echte deler} van $b$ als $a$ niet-inverteerbaar is en niet geassocieerd is met $b$.
\end{de}

\begin{de}
  \label{de:ggd}
  Zij $a$ en $b$ twee elementen uit een integriteitsdomein $D,+,\cdot$.
  We noemen $g\in D$ een \term{grootste gemene deler} van $a$ en $b$ als $g$ zowel een deler is van $a$ als van $b$ en elke andere deler van $a$ en $b$ een deler is van $g$.
  \[ g = ggd(a,b) \Leftrightarrow (g\ |\ a \wedge g\ |\ b \wedge (\forall c\in D: (c\ |\ a \wedge c\ |\ b) \Rightarrow c\ |\ g)) \]
\end{de}

\begin{de}
  \label{de:kgv}
  Zij $a$ en $b$ twee elementen uit een integriteitsdomein $D,+,\cdot$.
  We noemmen $k\in D$ een \term{kleinst gemeen veelvoud} van $a$ en $b$ als $k$ zowel een veelvoud is van $a$ als van $b$ en elk ander gemeen veelvoud van $a$ en $b$ een veelvoud is van $k$.
  \[ g = kgv(a,b) \Leftrightarrow (a\ |\ k \wedge b\ |\ k \wedge (\forall c\in D: (a\ |\ c \wedge b\ |\ c) \Rightarrow k\ |\ c)) \]
\end{de}

\begin{st}
  \label{st:ggd-uniek}
  Een grootst gemene deler is uniek op een inverteerbaar element na.\\
  Zij $D,+,\cdot$ een integriteitsdomein.
  \[ \forall a,b,g_{1},g_{2} \in D:\ (g_{1} = ggd(a,b) \wedge g_{2} = ggd(a,b)) \Rightarrow g_{1} \sim_{D} g_{2} \]
  \begin{proof}
    Stel dat $g_{1}$ en $g_{2}$ beide een $ggd$ zijn van $a$ en $b$, dan geldt $g_{1}|g_{2}$ alsook $g_{2}|g_{1}$.\deref{ggd}
    $g_{1}$ is dus geaccosieerd met $g_{2}$: $g_{1} \sim_{D} g_{2}$.
  \end{proof}
\end{st}

\begin{st}
  Zij $D,+,\cdot$ een integriteitsdomein.
  \[ \forall a,b,g_{1},g_{2} \in D:\ (g_{1} \sim_{D} g_{2} \wedge g_{1} = ggd(a,b)) \Rightarrow g_{2} = ggd(a,b) \]

  \begin{proof}
    Stel dat $g_{1}$ geassocieerd is met een $ggd$ $g_{2}$ van $a$ en $b$ in $D,+,\cdot$, dan gelden volgende beweringen:
    \[ a | g_{1} \wedge b | g_{1} \wedge g_{1} | g_{2} \wedge g_{2} | g_{1} \]
    Door de transitiviteit van ``is een deler van''\stref{st:deler-transitief} zijn $a$ en $b$ dan ook delers van $g_{2}$.
    Bovendien is elke andere $ggd$ van $a$ en $b$ geassocieerd met $g_{2}$, dus ook deelbaar door $g_{2}$.\stref{st:ggd-uniek}
  \end{proof}
\end{st}

\begin{de}
  Zij $D,+,\cdot$ een integriteitsdomein en $p$ een niet-nulelement van $D$, dan noemen we $p$ een \term{priemelement} als uit $p=ab$ volgt dat $a$ of $b$ inverteerbaar is in $D$.
\end{de}

\begin{opm}
  Een priemgetal heeft dus geen echte delers.
\end{opm}

\subsection{Euclidische Ring}
\label{sec:euclidische-ring}

\begin{de}
  We noemen een integriteitsdomein $D,+,\cdot$ een \term{euclidische ring} als er een afbeelding $d:\ D\setminus \{e_{D,+}\} \rightarrow \mathbb{R}^{+}$ bestaat die elk niet-nulelement $a$ afbeeldt op een niet-negatief geheel getal $d(a)$ zodat het volgende geldt:
\begin{itemize}
\item $\forall a,b \in D\setminus \{e_{D,+}\}:\ d(a) \le d(ab)$
\item $\forall a \in D,\ \forall b\in D\setminus \{e_{D,+}\}:\ \exists q,r \in D:\ (a = qb+r \wedge (r=0 \vee d(r) < d(b)))$
\end{itemize}
\end{de}

\begin{st}
  In een euclidische ring $D,+,\cdot$ met afbeelding $d$ geldt het volgende:
  \[ \forall a,b \in D:\ d(a) = d(ab) \Leftrightarrow b \text{ is inverteerbaar} \]
\TODO{bewijs p 125}
\end{st}

\begin{st}
  In een euclidische ring $D,+,\cdot$ met afbeelding $d$ geldt het volgende:
  \[ \forall a,b \in D:\ d(a) < d(ab) \Leftrightarrow b \text{ is niet inverteerbaar} \]
\TODO{bewijs p 125}
\end{st}

\begin{st}
  In een euclidische ring $D,+,\cdot$ met afbeelding $d$ geldt het volgende:
  \[ \forall a,b \in D:\ b \text{ is een echte deler van } a \Rightarrow d(b) < d(a) \]
\TODO{bewijs p 125}
\end{st}

\begin{st}
  Stelling van B\'ezout\\
  In een Euclidische ring $D,+,\cdot$ bestaat er een grootste gemene deler van elk paar elementen $(a,b)$ die niet allebei het nulelement zijn.
  Bovendien kan die grootste gemene deler $g$ beschreven worden als volgt:
  \[ ggd(a,b) = g = \alpha \cdot a + \beta \cdot b \text{ met }\alpha,\beta \in D \]
\TODO{bewijs p 126 TAI}
\end{st}

\TODO{ vanaf algoritme van Euclides om grootste gemene deler te vinden
  p 126 TAI tot sectie 7 p 129}

\begin{al}
  \label{al:algoritme-van-euclides}
  Het \term{Algoritme van Euclides}\\
  Zij $a$ en $b$ twee elementen van een euclidische ring $D,+,\cdot,d$, dan kunnen we de grootste gemene deler $ggd(a,b)$ van $a$ en $b$ als volgt vinden.
  \[
  \begin{array}{rcll}
    a &=& bq_{1} + r_{1} &\text{ met } d(r_{1}) < d(b) \\
    b &=& r_{1}q_{1} + r_{2} &\text{ met } d(r_{2}) < d(r_{1}) \\
    r_{1} &=& r_{2}q_{2} + r_{3} &\text{ met } d(r_{3}) < d(r_{2}) \\
    \vdots && \vdots \\
    r_{k-2} &=& r_{k-1}q_{k} + r_{k} &\text{ met } d(r_{k}) < d(r_{k-1}) \\
    r_{k-1} &=& r_{k}q_{k+1} + 0
  \end{array}
  \]
  $r_{k}$ is dan de grootste gemene deler $ggd(a,b)$ van $a$ en $b$.
  \begin{proof}
    Het algoritme van Euclides is eindig en correct.
    \begin{itemize}
    \item Eindigheid\\
      De rij $r_{1},\dotsc$ is een dalende rij in $\mathbb{N}$, dus voor een zekere $k$ moet $r_{k+1}$ nul worden.
    \item Correctheid\\
      Allereerst bewijzen we dat de grootst gemene deler $ggd(a,b)$ van $a$ en $b$ gelijk is aan de grootst gemene deler $ggd(b,r_{1})$ van $b$ en $r_{1}$.
      \[ ggd(a,b) = ggd(b,r_{1}) \]
      Noem $g$ de grootst gemene deler van $a$ en $b$, dan deelt $g$ zeker $a$ en $b$.
      Nu geldt bovendien $r_{1} = a - bq_{1}$, dus $g$ deelt ook $r_{1}$.
      Elke andere deler van zowel $b$ als $r_{1}$ zal ook een deler zijn van $a$, en dus ook een deler van $g$.
      \[ ggd(r_{i-1},r_{i}) = ggd(r_{i},r_{i+1}) \]
      We kunnen deze redenering verder zetten voor $r_{i-1}$, $r_{i}$ en $r_{i+1}$ op volledig analoge wijze.
      Het resultaat van deze redenering is dat $ggd(a,b)$ gelijk is aan $ggd(r_{k-1},r_{k})$.
      $r_{k}$ is echter een deler van $r_{k-1}$, dus ook de grootst gemene deler van $r_{k}$ en $r_{k+1}$ en bijgevolg de grootst gemene deler $ggd(a,b)$ van $a$ en $b$.
    \end{itemize}
  \end{proof}
\end{al}

\begin{de}
  Als de grootst gemene deler $ggd(a,b)$ van twee elementen $a$ en $b$ van een euclidische ring $D,+,\cdot,d$ niet inverteerbaar is, dan noemen we $a$ en $b$ \term{onderling ondeelbaar} of \term{copriem}.
\end{de}

\begin{ei}
  In een euclidische ring $D,+,\cdot,d$ met eenheidselement $i$ zal $i$ steeds een grootst gemene deler zijn van twee onderling ondeelbare elementen $a$ en $b$.
\extra{bewijs}
\end{ei}

\begin{st}
  Zij $D,+,\cdot,d$ een euclidische ring en $a$ en $b$ elementen van $D$.
  \[ a \text{ en } b \text{ zijn onderling ondeelbaar } \Rightarrow (\forall c \in D)(a | bc \Rightarrow a|c) \]
  \TODO{bewijs TAI p 128}
\end{st}

\begin{st}
  Zij $D,+,\cdot,d$ een euclidische ring en $a$ en $b$ elementen van $D$.
  \[ a \text{ en } b \text{ zijn onderling ondeelbaar } \Rightarrow (\forall c \in D)(a | c \wedge b | c \Rightarrow ab|c) \]
  \TODO{bewijs TAI p 128}
\end{st}

\begin{st}
  Zij $D,+,\cdot,d$ een euclidische ring en $a$ en $b$ elementen van $D$.
  \[ p \text{ priem } \Rightarrow (p|ab \Rightarrow p|a \vee p|b) \]
  \TODO{bewijs TAI p 128}
\end{st}

\begin{st}
  In een euclidische ring $D,+,\cdot,d$ kan elk niet-nulelement ontbonden worden in een product van priemelementen en een inverteerbaar element.
  Deze ontbinding is bovendien uniek op de volgorde van de factoren na en op een inverteerbaar element na.
  \TODO{bewijs TAI p 128}
\end{st}

\section{Veeltermen over ringen}
\label{sec:veelt-over-ring}

\begin{de}
  Een \term{veelterm over een ring} $R,+,\cdot$ is een uitdrukking $a(x)$ van de volgende vorm.
  \[ a(x) = \sum_{i=0}^{n}a_{i}x^{i} \text{ met } n \ge 0 \]
  In $a(x)$ zijn alle $a_{i}$ elementen van $R$ en $a_{n}$ niet het nulelement. 
  De $a_{i}$ noemt met de \term{co\"efficienten} en $x$ de \term{onbepaalde}.
  $n$ noemt men de \term{graad} van $a(x)$.
  $a_{n}$ noemt met de \term{leidende co\"efficient} en $a_{n}x^{n}$ de \term{leidende term}.
\end{de}

\begin{de}
  De \term{graad} van een veelterm $a(x)$ noteren we als $gr(a(x))$.
\end{de}

\begin{de}
  Een veelterm van graad $0$ noemt men een \term{constante (veel)term}.
\end{de}

\begin{de}
  De veelterm van graad $0$ met het nulelement $e$ als co\"efficient noemt met de \term{nul(veel)term}
\end{de}

\begin{de}
  Een veelterm waarbij de $n$-de co\"efficient $a_{n}$ gelijk is aan het eenheidselement $i$ noemt men een \term{monische veelterm}.
\end{de}

\begin{de}
  We definieren de som van twee veeltermen $a(x)$ en $b(x)$ als volgt:
  \[ a(x) = \sum_{i=0}^{m}a_{i}x^{i} \]
  \[ b(x) = \sum_{i=0}^{n}b_{i}x^{i} \]
  \[  a(x) + b(x) = \sum_{i=0}^{\max(m,n)}(a_{i} + b_{i})x^{i} \]
  Uiteraard zijn $a_{i}$ en $b_{i}$ voor $i$ groter dan $m$, respectievelijk $n$ nulelementen.
\end{de}

\begin{de}
  We definieren het product van twee veeltermen $a(x)$ en $b(x)$ als volgt:
  \[ a(x) = \sum_{i=0}^{m}a_{i}x^{i} \]
  \[ b(x) = \sum_{i=0}^{n}b_{i}x^{i} \]
  \[  a(x)\cdot b(x) = \sum_{i=0}^{n+m}(a_{i} \cdot b_{i})x^{i} \]
  Uiteraard zijn $a_{i}$ en $b_{i}$ voor $i$ groter dan $m$, respectievelijk $n$ nulelementen.
\end{de}

\begin{de}
  Men noteert de verzameling van veeltermen in $x$ over een ring $R,+,\cdot$ als $R[x]$.
\end{de}

\begin{ei}
  De verzameling van veeltermen $R[x]$ in een variabele $x$ over een ring $R,+,\cdot$, uitgerust met de veelterm-som en het veelterm-product, vormt een ring $R[x],+,\cdot$.
\extra{bewijs}
\end{ei}

\begin{st}
  Als de ring $R,+,\cdot$ commutatief is, dan is $R[x],+,\cdot$ ook commutatief.
\extra{bewijs}
\end{st}

\begin{st}
  Het nulelement $e$ van $R,+,\cdot$ is ook het nulelement van $R[x],+,\cdot$.
\extra{bewijs}
\end{st}

\begin{st}
  Het eenheidselement $i$ van $R,+,\cdot$(, als het bestaat,) is ook het eenheidselement van $R[x],+,\cdot$.
\extra{bewijs}
\end{st}

\begin{st}
  $R[x],+,\cdot$ heeft geen nuldelers als en slechts als $R,+,\cdot$ geen nuldelers heeft.
\extra{bewijs}
\end{st}

\begin{gev}
  $R[x],+,\cdot$ is een integriteitsdomein als en slechts als $R,+,\cdot$ een integriteitsdomein is.
\extra{bewijs}
\end{gev}


\section{Veeltermen over velden}
\label{sec:veelt-over-veld}

\begin{opm}
  De inverse van een veelterm over een ring $R,+,\cdot$ is in $R,+,\cdot$ is over het algemeen geen veelterm.
\end{opm}

\begin{opm}
  Zij $F,+,\cdot$ een veld, dan is $F[x]$ niet noodzakelijk een veld.
\end{opm}

\begin{st}
  \label{st:deling-veelterm-over-veld}
  Zij $a(x)$ en $b(x)$ veeltermen over een veld $F[x],+,\cdot$ met $b(x)$ niet het nulelement, dan bestaat er precies \'e\'en veelterm $q(x) \in F[x]$ en een $r(x) \in F[x]$ zodat het volgende geldt:
  \[ a(x) = q(x)b(x) + r(x) \quad\text{ met }\quad gr(r(x)) < gr(b(x)) \]
\TODO{bewijs p 131 TAI}
\end{st}

\begin{de}
  Zij $a(x)$ en $b(x)$ veeltermen over een veld $F[x],+,\cdot$ met $b(x)$ niet het nulelement.
  Als $q(x)$ en $r(x)$ veeltermen zijn zoals in stelling \ref{st:deling-veelterm-over-veld} op pagina \pageref{st:deling-veelterm-over-veld}, dan noemen we $q(x)$ het \term{quotient} en $r(x)$ de \term{rest} van $a(x)$ bij deling door $b(x)$.
\end{de}

\begin{st}
  De veeltermen in een variabele $x$ over een veld $F,+,\cdot$, uitgerust met $gr$ als afbeelding $d$, vormen een euclidische ring.
\extra{bewijs}
\end{st}

\begin{de}
  De productverzameling $(\mathbb{Z}_{2})^{m}$ bevat $m$-tallen van bits.
  We noteren $(\mathbb{Z}_{2})^{m}$ vaak als $F_{2^{m}}$.
\end{de}

\begin{de}
  We associeren met elk element $(a_{m-1},\dotsc,a_{0})$ van $F_{2^{m}}$ een veelterm $a(x)$:
  \[ a = (a_{m-1},\dotsc,a_{0}) \longleftrightarrow a(x) = \sum_{i=0}^{n}a_{i}x^{i} \]
\end{de}

\begin{de}
  We definieren een optelling in $F_{2^{m}}$ als de optelling van veeltermen met co\"efficienten in de $m$-tallen.
  \[ (a_{m-1},\dotsc,a_{0}) + (b_{m-1},\dotsc,b_{0}) =  (c_{m-1},\dotsc,c_{0}) \Leftrightarrow a(x) + b(x) = c(x) \]
\end{de}

\begin{de}
  We definieren een vermenigvuldiging in $F_{2^{m}}$ (modulo een $m$-tal $g$) als de optelling van veeltermen met co\"efficienten in de $m$-tallen.
  \[ (a_{m-1},\dotsc,a_{0}) \cdot (b_{m-1},\dotsc,b_{0}) =  (c_{m-1},\dotsc,c_{0}) \Leftrightarrow a(x) \cdot b(x) = c(x) \mod g(x) \]
\end{de}

\begin{ei}
  De verzameling veeltermen van graad $m-1$ of lager, uitgerust met de optelling en vermenigvuldiging (modulo $g$) vormt een veld.
\TODO{bewijs p 134}
\end{ei}

\section{Veeltermfuncties}
\label{sec:veeltermfuncties}

\begin{de}
  Zij $R[x],+,\cdot$ de ring der veeltermen over een deelring $R,+,\cdot$ van $S,+,\cdot$.
  We defini\"eren de \term{waarde} van een veelterm $a(x)$ voor een element $s$ van $s$ als het element $a(s)$ van $S$.
\end{de}

\begin{de}
  Zij $R[x],+,\cdot$ de ring der veeltermen over een deelring $R,+,\cdot$ van $S,+,\cdot$.
  De waarde van een veelterm $a(x)$ voor een element $s\in S$ bepaalt een functie $a$:
  \[ a:\ S \rightarrow S:\ s \mapsto a(s) \]
  We noemen $a$ de \term{veeltermfunctie} van $a$.
\end{de}

\begin{de}
  Zij $R[x],+,\cdot$ de ring der veeltermen over een deelring $R,+,\cdot$ van $S,+,\cdot$.
  Zij $s$ een element van $S$ en $a(x)$ een veelterm van $R[x]$, dan noemen we $s$ een \term{nulpunt} of een \term{wortel} van $a$ als en slechts als $a(s)$ het nulelement is van $s$.
\end{de}

\begin{st}
  De rest bij deling van een veelterm $p(x)$ door $(x-a)$ is $p(a)$.
\extra{bewijs}
\end{st}

\begin{st}
  $p(x)$ is deelbaar door $(x-a)$ als en slechts als $a$ een nulpunt is van $p$.
\extra{bewijs}
\end{st}

\begin{st}
  Een veelterm $p$ van graad $2$ over een veld $F,+,\cdot$ is irreduceerbaar als en slechts $p$ geen wortel heeft in $F$.
\extra{bewijs}
\end{st}

\begin{st}
  \label{st:fundamentele-stelling-van-de-algebra}
  De \term{fundamentele stelling van de algebra}.\\
  Een niet-constante veelterm over het veld $\mathbb{C},+,\cdot$ der complexe getallen heeft minstens $1$ wortel in $\mathbb{C}$.
  \extra{verwijzing naar een bewijs.}
\end{st}
  
\begin{gev}
  Elke veelterm van graad $n$ over het veld $\mathbb{C},+,\cdot$ der complexe getallen heeft precies $n$ wortels in $\mathbb{C}$.
\extra{bewijs}
\end{gev}

\begin{gev}
  Elke veelterm van graad $n$ over het veld $\mathbb{C},+,\cdot$ der complexe getallen is op een unieke manier te ontbinden in priemelementen van de vorm $(x-a)$ op een constante (en dus inverteerbare) veelterm na.
\extra{bewijs}
\end{gev}

\begin{st}
  Als $a\in \mathbb{C}$ een complexe wortel is van een re\"ele veelterm $p(x)\in \mathbb{R}[x]$, dan zal ook het complex toegevoegde $\bar{a}$ van $a$ een wortel zijn van $p(x)$.
\TODO{bewijs p 136}
\end{st}

\section{Codetheorie}
\label{sec:codetheorie}

\begin{de}
  Een \term{boodschap} van $n$ bits is een opeenvolging van $n$ elementen van een eindig veld (meestal $\mathbb{Z}_{2}$).
\end{de}

\begin{de}
  \label{de:code}
  Een $(n,k)$-\term{code} is een afbeelding $c$ die een boodschap van $k$ bits afbeeldt op een boodschap van $n$ bits.
  \[ c:\ \mathbb{Z}_{2}^{k} \rightarrow \mathbb{Z}_{2}^{n}:\ b \mapsto c(b) \]

  \begin{figure}[H]
    \centering
    \begin{tikzpicture}
      \tikzset{
        elps/.style 2 args={draw,ellipse,minimum width=#1,minimum height=#2},
        node distance=3cm,
        font=\footnotesize,
        >=latex,
      }
      \node(x)[elps={1.3cm}{1cm},label={below left:$\mathbb{Z}_{2}^{k}$}]{};
      \node(y)[elps={2cm}{1.2cm},right=of x,label={below left:$\mathbb{Z}_{2}^{n}$}]{};
      \fill[gray!50]($(y.center)-(5pt,5pt)$)circle[x radius=.7cm,y radius=.3cm]coordinate(im);
      \node at (im){$\mathrm{im}(c)$};
      \draw[->](x)to[bend left=20]node[above]{$c$}(y);
    \end{tikzpicture}
    \caption{Een code, schematisch voorgesteld}
    \label{fig:code}
  \end{figure}
\end{de}

\begin{opm}
  Een code is nooit surjectief. (Dit is de bedoeling natuurlijk.)
\end{opm}

\begin{de}
  Een \term{codewoord} is een boodschap $c(b)$ in het beeld van een code $c$.
\end{de}

\begin{de}
  Een \term{code gegenereerd door een veelterm} $p(x)\in \mathbb{Z}_{2}[x]$ van graad $n-k$ is een $(n,k)$-code $c$:
  \[ c:\ \mathbb{Z}_{2}^{k} \rightarrow \mathbb{Z}_{2}^{n}:\ b(x) \mapsto c(b) = r(x) + x^{n-k}x(x) \quad\text{ met }\quad x^{n-k}b(x) = p(x)q(x) + r(x) \]
\end{de}

\begin{opm}
  Vermits de graad $gr(r(x))$ van de rest $r(x)$ een kleiner is dan $n-k$, passen de extra $n-k$ bits netjes na de boodschap-bits.
\end{opm}

\TODO{zeker een voorbeeld maken, wss examenvraag!}


\section{Het veld der breuken}
\label{sec:het-veld-der}

\begin{st}
  Zij $D,+,\cdot$ een integriteitsdomein, dan is het mogelijk om een veld $F,\star,*$ te construeren zodat ...
  \begin{itemize}
  \item ... $R,+,\cdot$ isomorf is met een deelring $R'$ van $F$.
  \item ... ieder element in $F$ te schrijven valt als $p\star q^{-1}$ met $p,q \in R'$.
  \end{itemize}
\TODO{bewijs p 139}
\end{st}

\begin{de}
  Het veld $F$ uit de vorige stelling noemt met het \term{veld der breuken} of het \term{veld der quoti\"enten} van $R$.
\end{de}

\begin{opm}
  Men kan $F$ zien als een uitbreiding van $R$ (isomorf met $R'$) zodat in $F$ elk niet-nulelement van $R$ een inverse heeft gekregen.
\end{opm}

\begin{st}
  Het breukenveld van een integriteitsdomein is op isomorfisme na uniek.
\extra{zeker? bewijs!}
\end{st}

\end{document}






