\documentclass[main.tex]{subfiles}
\begin{document}

\chapter{Lineaire Algebra}
\label{cha:lineaire-algebra}

\section{Herhaling}
\label{sec:herhaling}

\subsection{Conventies}
\label{sec:conventies}


In heel deze sectie zullen we spreken over een veld $K,+,\cdot$ en een $K$-vectorruimte $V,+,\cdot$ van eindige dimensie $n$.

\begin{de}
  Een $K$-lineaire afbeelding van $V$ naar $V$ noemen we ook een \term{lineaire transformatie}, \term{(lineaire) operator} of \term{endomorfisme} van $V$.
\end{de}

\begin{st}
  Zij $\varepsilon = \{e_{1},\dotsc,e_{n}\}$ en $\varepsilon'= \{ e_{1}',\dotsc,e_{n}'\}$ twee basissen van $V$.
  De matrix $P \in M^{n\times n}(K)$ van basisverandering van $\varepsilon$ naar $\varepsilon'$ heeft in de $j$-de kolomn de co\"ordinaten van $e_{j}'$ ten opzichte van $\varepsilon$.
  \[ e_{j}' = \sum_{i=1}^{n}c_{ij}e_{i} \]
  \[
  P = 
  \begin{pmatrix}
    c_{11} & c_{12} & \hdots & c_{1n}\\
    c_{21} & c_{22} & \hdots & c_{2n}\\
    \vdots & \vdots & \ddots & \vdots\\
    c_{n1} & c_{n2} & \hdots & c_{nn}\\
  \end{pmatrix}
  \]
\extra{bewijs}
\end{st}

\begin{st}
  Zij $A$ een lineaire transformatie van $V$.
  De matrix $M\in M^{n\times n}(K)$ van $A$ ten opzichte van een basis $\varepsilon = \{e_{1},\dotsc,e_{n}\}$ van $V$ heeft in de $j$-de kolom de co\"ordinaten van $A(e_{j})$ ten opzichte van $\varepsilon$.
  \[ A(e_{j}) = \sum_{i=1}^{n}a_{ij}e_{j} \]
  \[ 
  M = 
  \begin{pmatrix}
    a_{11} & a_{12} & \hdots & a_{1n}\\
    a_{21} & a_{22} & \hdots & a_{2n}\\
    \vdots & \vdots & \ddots & \vdots\\
    a_{n1} & a_{n2} & \hdots & a_{nn}\\
  \end{pmatrix}
  \]
\extra{bewijs}
\end{st}

\begin{st}
  Zij $\varepsilon = \{e_{1},\dotsc,e_{n}\}$ en $\varepsilon'= \{ e_{1}',\dotsc,e_{n}'\}$ twee basissen van $V$ en $P$ de matrix van basisverandering van $\varepsilon$ naar $\varepsilon'$.
  \[ 
  \begin{pmatrix}
    e'_{1} & e'_{2} & \hdots & e'_{n}\\
  \end{pmatrix}
  =
  \begin{pmatrix}
    e_{1} & e_{2} & \hdots & e_{n}\\
  \end{pmatrix}
  P
  \]
\extra{bewijs}
\end{st}
 
\begin{st}
  Zij $x_{1},\dotsc,x_{n}\in K$ de co\"ordinaten van $v\in V$ ten opzicht van $\varepsilon$ en $x'_{1},\dotsc,x'_{n}\in K$ de co\"ordinaten van diezelfde $v$ ten opzicht van $\varepsilon'$.
  \[ v = \sum_{i=1}^{n}x_{i}e_{i} = \sum_{i=1}^{n}x'_{i}e'_{i}\]
  \[ 
  \begin{pmatrix}
    x_{1} \\ x_{2} \\ \vdots \\ x_{n}\\
  \end{pmatrix}
  =
  P
  \begin{pmatrix}
    x'_{1} \\ x'_{2} \\ \vdots \\ x'_{n}\\
  \end{pmatrix}
  \]
\extra{bewijs}
\end{st}

\begin{st}
  Zi $A$ een lineaire transformatie van $V$ en $M$ de matrix van $A$ ten opzichte van $\varepsilon$.
  Zij $x_{1},\dotsc,x_{n}\in K$ de co\"ordinaten van $v\in V$ ten opzichte van $\varepsilon$ en $y{1},\dotsc,y_{n}\in K$ de co\"ordinaten van $A(v)$ ten opzichte van $\varepsilon$.
  \[ 
  \begin{pmatrix}
    y_{1} \\ y_{2} \\ \vdots \\ y_{n}\\
  \end{pmatrix}
  =
  M
  \begin{pmatrix}
    x_{1} \\ x_{2} \\ \vdots \\ x_{n}\\
  \end{pmatrix}
  \]
\extra{bewijs}
\end{st}

\begin{st}
  Zij ook nog $M'$ de matrix van $A$ ten opzichte van $\varepsilon'$.
  \[ M' = P^{-1}MP \]
  $M$ en $M'$ zijn dus gelijkvormig.
\extra{bewijs}
\end{st}

\subsection{Herhaling}
\label{sec:herhaling-1}

\begin{de}
  Zij $K,+,\cdot$ een veld en $V,+,\cdot$ een eindigdimensionale $K$-vectorruimte.
  Een lineaire transformatie $A$ van $V,+,\cdot$ noemen we \term{diagonaliseerbaar} over $K,+,\cdot$ als er een basis van $V,+,\cdot$ bestaat zodat de matrix van $A$ ten opzichte van die basis een diagonaalmatrix is over $K$.
\end{de}

\begin{de}
  Een matrix $M\in M^{n\times n}(K)$ is diagonaliseerbaar over $K$ als er een matrix $P\in GL_{n}(K)$ bestaat zodat $P^{-1}MP$ een diagonaalmatrix is.
\end{de}

\begin{ei}
  Zij $K,+,\cdot$ een veld en $V$ een eindigdimensionale $K$-vectorruimte met basis $\varepsilon$.
  Een lineaire transformatie $A$ van $V$ is diagonaliseerbaar over $K$ als en slechts als de matrix van $A$ ten opzichte van $\varepsilon$ diagonaliseerbaar is over $K$.
\extra{bewijs}
\end{ei}

\begin{de}
  Zij $K,+,\cdot$ een veld en $A$ een lineaire tranformatie van een $K$-vectorruimte $V,+,\cdot$.
  De vector $v\in V$ is een \term{eigenvector} van $A$ met \term{eigenwaarde} $\lambda\in K$ als het volgende geldt
  \[ A(v) = \lambda v \text{ en } v\neq \vec{0} \]
\end{de}

\begin{ei}
  Zij $K,+,\cdot$ een veld en $V$ een eindigdimensiolane $K$-vectorruimte.
  Een lineaire transformatie $A$ van $V$ is diagonaliseerbaar over $K$ als en slechts als $V$ een basis heeft van eigenvectoren (met eigenwaarden in $K$).
\extra{bewijs}
\end{ei}

\begin{de}
  Zij $K,+,\cdot$ een veld en $M$ een matrix in $M^{n\times n}(K)$.
  De \term{karakteristieke veelterm} van $M$ is $f_{M}$.
  \[ f_{M} = \det(X\mathbb{I}_{n} - M) \in K[X] \]
\end{de}

\begin{de}
  Zij $A$ een lineaire transformatie van een eindigdimensionale $K$-vectorruimte $V$.
  De \term{karakteristieke veelterm} van $A$ is de karakteristieke veelterm van de matrix van $A$ ten opzichte van een willekeurige basis van $V$.
\end{de}

\begin{st}
  De karakteristieke veelterm van een lineaire transformatie is onafhankelijk van de gekozen veelterm.
\extra{bewijs}
\end{st}

\begin{pr}
  Zij $K,+,\cdot$ een veld en $A$ een lineaire tranfformatie van een eindigdimensionale $K$-vectorruimte $V$, dan is $\lambda\in K$ een eigenwaarde van $A$ als en slechts als $\lamda$ een wortel is van de karakteristieke veelterm van $A$.
\extra{bewijs}
\end{pr}

\begin{de}
  Naar aanleiding van bovenstaande propositie spreken we ook over de eigenwaarden van een matrix.
\end{de}

\begin{gev}
  Zij $K,+,\cdot$ een veld en $V$ een eindigdimensionale $K$-vectorruimte van dimensie $n$ en $A$ een lineaire transformatie van $V$.
  Als de karakteristieke veelterm $f_{A}$ $n$ verschillende wortel heeft in $K$, dan is $A$ diagonaliseerbaar over $K$.
\extra{bewijs}
\end{gev}

\subsection{Triagulatie}
\label{sec:triagulatie}

\begin{st}
  De \term{triagulatiestelling}\\
  Zij $K,+,\cdot$ een veld en $V$ een $n$-dimensionale $K$-vectorruimte.
  Zij $A$ een lineaire transformatie van $V$ waarvoor $f_{A}$ volledig splitst in lineaire factoren over $K$, dan bestaat er een basis $\mathcal{B} = \{ v_{1},\dotsc,v_{n} \}$ van $V$ zodat de matrix van $A$ tev opzichte van $\mathcal{B}$ een bovendriehoeksmatrix is met de eigenwaarden van $A$ op de hoofddiagonaal.
\extra{bewijs?}
\end{st}

\begin{gev}
  Zij $K,+,\cdot$ een veld en $M\in M^{n\times n}(K)$ een matrix over $K$ zodat $f_{M}$ volledig splitst in lineaire factoren over $K$, dan bestaat er een matrix $P\in GL_{n}(K)$ zodat $P^{-1}AP$ een bovendriehoeksmatrix is met de eigenwaarden van $M$ op de hoofddiagonaal.
\extra{bewijs?}
\end{gev}

\begin{opm}
  Als $K,+,\cdot$ een algebra\"isch gesloten veld is zijn alle matrices en lineaire transformaties van eindigdimensionale vectorruimtes over $K$ trianguleerbaar.
\end{opm}

\end{document}