\documentclass[main.tex]{subfiles}
\begin{document}

\chapter{Lineaire Algebra}
\label{cha:lineaire-algebra}

\section{Herhaling}
\label{sec:herhaling}

\subsection{Conventies}
\label{sec:conventies}


In heel deze sectie zullen we spreken over een veld $K,+,\cdot$ en een $K$-vectorruimte $V,+,\cdot$ van eindige dimensie $n$.

\begin{de}
  Een $K$-lineaire afbeelding $L$ is een afbeelding die lineair is in $K$:
  \[ \forall \lambda_{1},\lambda_{2}\in K,\ \forall v_{1},v_{2}\in V:\ L(\lambda_{1}\cdot v_{1}+\lambda_{2}\cdot v_{2}) = \lambda_{1} \cdot  L(v_{1}) + \lambda_{2} \cdot L(v_{2}) \]
\end{de}
\begin{de}
  Een $K$-lineaire afbeelding van $V$ naar zichzelf noemen we ook een \term{lineaire transformatie}, \term{(lineaire) operator} of \term{endomorfisme} van $V$.
\end{de}

\begin{st}
  Zij $\varepsilon = \{e_{1},\dotsc,e_{n}\}$ en $\varepsilon'= \{ e_{1}',\dotsc,e_{n}'\}$ twee basissen van $V$.
  De matrix $P \in M^{n\times n}(K)$ van basisverandering van $\varepsilon$ naar $\varepsilon'$ heeft in de $j$-de kolomn de co\"ordinaten van $e_{j}'$ ten opzichte van $\varepsilon$.
  \[ e_{j}' = \sum_{i=1}^{n}c_{ij}e_{i} \]
  \[
  P = 
  \begin{pmatrix}
    c_{11} & c_{12} & \hdots & c_{1n}\\
    c_{21} & c_{22} & \hdots & c_{2n}\\
    \vdots & \vdots & \ddots & \vdots\\
    c_{n1} & c_{n2} & \hdots & c_{nn}\\
  \end{pmatrix}
  \]
\extra{bewijs}
\end{st}

\begin{st}
  Zij $A$ een lineaire transformatie van $V$.
  De matrix $M\in M^{n\times n}(K)$ van $A$ ten opzichte van een basis $\varepsilon = \{e_{1},\dotsc,e_{n}\}$ van $V$ heeft in de $j$-de kolom de co\"ordinaten van $A(e_{j})$ ten opzichte van $\varepsilon$.
  \[ A(e_{j}) = \sum_{i=1}^{n}a_{ij}e_{j} \]
  \[ 
  M = 
  \begin{pmatrix}
    a_{11} & a_{12} & \hdots & a_{1n}\\
    a_{21} & a_{22} & \hdots & a_{2n}\\
    \vdots & \vdots & \ddots & \vdots\\
    a_{n1} & a_{n2} & \hdots & a_{nn}\\
  \end{pmatrix}
  \]
\extra{bewijs}
\end{st}

\begin{st}
  Zij $\varepsilon = \{e_{1},\dotsc,e_{n}\}$ en $\varepsilon'= \{ e_{1}',\dotsc,e_{n}'\}$ twee basissen van $V$ en $P$ de matrix van basisverandering van $\varepsilon$ naar $\varepsilon'$.
  \[ 
  \begin{pmatrix}
    e'_{1} & e'_{2} & \hdots & e'_{n}\\
  \end{pmatrix}
  =
  \begin{pmatrix}
    e_{1} & e_{2} & \hdots & e_{n}\\
  \end{pmatrix}
  P
  \]
\extra{bewijs}
\end{st}
 
\begin{st}
  Zij $x_{1},\dotsc,x_{n}\in K$ de co\"ordinaten van $v\in V$ ten opzicht van $\varepsilon$ en $x'_{1},\dotsc,x'_{n}\in K$ de co\"ordinaten van diezelfde $v$ ten opzicht van $\varepsilon'$.
  \[ v = \sum_{i=1}^{n}x_{i}e_{i} = \sum_{i=1}^{n}x'_{i}e'_{i}\]
  \[ 
  \begin{pmatrix}
    x_{1} \\ x_{2} \\ \vdots \\ x_{n}\\
  \end{pmatrix}
  =
  P
  \begin{pmatrix}
    x'_{1} \\ x'_{2} \\ \vdots \\ x'_{n}\\
  \end{pmatrix}
  \]
\extra{bewijs}
\end{st}

\begin{st}
  Zi $A$ een lineaire transformatie van $V$ en $M$ de matrix van $A$ ten opzichte van $\varepsilon$.
  Zij $x_{1},\dotsc,x_{n}\in K$ de co\"ordinaten van $v\in V$ ten opzichte van $\varepsilon$ en $y{1},\dotsc,y_{n}\in K$ de co\"ordinaten van $A(v)$ ten opzichte van $\varepsilon$.
  \[ 
  \begin{pmatrix}
    y_{1} \\ y_{2} \\ \vdots \\ y_{n}\\
  \end{pmatrix}
  =
  M
  \begin{pmatrix}
    x_{1} \\ x_{2} \\ \vdots \\ x_{n}\\
  \end{pmatrix}
  \]
\extra{bewijs}
\end{st}

\begin{st}
  Zij ook nog $M'$ de matrix van $A$ ten opzichte van $\varepsilon'$.
  \[ M' = P^{-1}MP \]
  $M$ en $M'$ zijn dus gelijkvormig.
\extra{bewijs}
\end{st}

\subsection{Herhaling}
\label{sec:herhaling-1}

\begin{de}
  Zij $K,+,\cdot$ een veld en $V,+,\cdot$ een eindigdimensionale $K$-vectorruimte.
  Een lineaire transformatie $A$ van $V,+,\cdot$ noemen we \term{diagonaliseerbaar} over $K,+,\cdot$ als er een basis van $V,+,\cdot$ bestaat zodat de matrix van $A$ ten opzichte van die basis een diagonaalmatrix is over $K$.
\end{de}

\begin{de}
  Een matrix $M\in M^{n\times n}(K)$ is diagonaliseerbaar over $K$ als er een inverteerbare matrix $P\in GL_{n}(K)$ bestaat zodat $P^{-1}MP$ een diagonaalmatrix is.
\end{de}

\begin{ei}
  Zij $K,+,\cdot$ een veld en $V$ een eindigdimensionale $K$-vectorruimte met basis $\varepsilon$.
  Een lineaire transformatie $A$ van $V$ is diagonaliseerbaar over $K$ als en slechts als de matrix van $A$ ten opzichte van $\varepsilon$ diagonaliseerbaar is over $K$.
\extra{bewijs}
\end{ei}

\begin{de}
  Zij $K,+,\cdot$ een veld en $A$ een lineaire tranformatie van een $K$-vectorruimte $V,+,\cdot$.
  De vector $v\in V$ is een \term{eigenvector} van $A$ met \term{eigenwaarde} $\lambda\in K$ als het volgende geldt
  \[ A(v) = \lambda v \text{ en } v\neq \vec{0} \]
\end{de}

\begin{ei}
  Zij $K,+,\cdot$ een veld en $V$ een eindigdimensiolane $K$-vectorruimte.
  Een lineaire transformatie $A$ van $V$ is diagonaliseerbaar over $K$ als en slechts als $V$ een basis heeft van eigenvectoren (met eigenwaarden in $K$).
\extra{bewijs}
\end{ei}

\begin{de}
  Zij $K,+,\cdot$ een veld en $M$ een matrix in $M^{n\times n}(K)$.
  De \term{karakteristieke veelterm} van $M$ is $f_{M}$.
  \[ f_{M} = \det(X\mathbb{I}_{n} - M) \in K[X] \]
\end{de}

\begin{de}
  Zij $A$ een lineaire transformatie van een eindigdimensionale $K$-vectorruimte $V$.
  De \term{karakteristieke veelterm} van $A$ is de karakteristieke veelterm van de matrix van $A$ ten opzichte van een willekeurige basis van $V$.
\end{de}

\begin{st}
  De karakteristieke veelterm van een lineaire transformatie is onafhankelijk van de gekozen basis.
\extra{bewijs}
\end{st}

\begin{pr}
  Zij $K,+,\cdot$ een veld en $A$ een lineaire transformatie van een eindigdimensionale $K$-vectorruimte $V$, dan is $\lambda\in K$ een eigenwaarde van $A$ als en slechts als $\lambda$ een wortel is van de karakteristieke veelterm van $A$.
\extra{bewijs}
\end{pr}

\begin{de}
  Naar aanleiding van bovenstaande propositie spreken we ook over de eigenwaarden van een matrix.
\end{de}

\begin{gev}
  Zij $K,+,\cdot$ een veld en $V$ een eindigdimensionale $K$-vectorruimte van dimensie $n$ en $A$ een lineaire transformatie van $V$.
  Als de karakteristieke veelterm $f_{A}$ $n$ verschillende wortel heeft in $K$, dan is $A$ diagonaliseerbaar over $K$.
\extra{bewijs}
\end{gev}

\begin{gev}
  Als $K,+,\cdot$ een algebra\"isch gesloten veld is, dan zijn alle matrices en lineaire transformmaties van eindigdimensionale vectorruimtes over $K$ trianguleerbaar.
\end{gev}

\subsection{Triagulatie}
\label{sec:triagulatie}

\begin{st}
  De \term{triagulatiestelling}\\
  Zij $K,+,\cdot$ een veld en $V$ een $n$-dimensionale $K$-vectorruimte.
  Zij $A$ een lineaire transformatie van $V$ waarvoor $f_{A}$ volledig splitst in lineaire factoren over $K$, dan bestaat er een basis $\mathcal{B} = \{ v_{1},\dotsc,v_{n} \}$ van $V$ zodat de matrix van $A$ tev opzichte van $\mathcal{B}$ een bovendriehoeksmatrix is met de eigenwaarden van $A$ op de hoofddiagonaal.
\extra{bewijs?}
\end{st}

\begin{gev}
  Zij $K,+,\cdot$ een veld en $M\in M^{n\times n}(K)$ een matrix over $K$ zodat $f_{M}$ volledig splitst in lineaire factoren over $K$, dan bestaat er een matrix $P\in GL_{n}(K)$ zodat $P^{-1}AP$ een bovendriehoeksmatrix is met de eigenwaarden van $M$ op de hoofddiagonaal.
\extra{bewijs?}
\end{gev}

\begin{opm}
  Als $K,+,\cdot$ een algebra\"isch gesloten veld is zijn alle matrices en lineaire transformaties van eindigdimensionale vectorruimtes over $K$ trianguleerbaar.
\end{opm}

\section{Cayley-Hamilton}
\label{sec:cayley-hamilton}

\subsection{$K$-algebra's}
\label{sec:k-algebras}

\begin{de}
  ZIj $K,+,\cdot$ een veld.
  Een $K$-\term{algebra} $A,+,\cdot,\bowtie$ is een verzameling $A$ voorzien van twee inwendige bewerkingen...
  \[ (+):\ A \times A \rightarrow A \quad\text{ en }\quad (\cdot): A \times A \rightarrow A \]
  ... en een scalaire vermenigvuldiging (een uitwendige bewerking)...
  \[ \bowtie: K \times A \rightarrow A: (k,a) \mapsto k \bowtie a \]
  ... met de volgende eigenschappen:
  \begin{itemize}
  \item $A,+,\cdot$ is een ring.
  \item $K,A,+$ is een vectorruimte met scalaire vermenigvuldiging $\bowtie$.
  \item $\forall a,b \in A: \forall k \in K: k \bowtie (a\cdot b) = (k \bowtie a) \cdot b = a \cdot (k \bowtie b)$
  \end{itemize}
\end{de}

\begin{de}
  Een $K$-algebra $A,+,\cdot,\bowtie$ is \term{commutatief} als de ring $A,+,\cdot$ commutatief is.
\end{de}

\begin{de}
  Een $K$-algebra $A,+,\cdot,\bowtie$ heeft een eenheidselement als de ring $A,+,\cdot$ een eenheidselement heeft.
\end{de}

\begin{de}
  Zij $M \in M^{n\times n}(K)$ een matrix over $K$.
  We kunnen $M$ substitueren voor $X$ in een veelterm $g\in K[X]$ over $K$.
  \[ g = \sum_{i=0}^{r}k_{i}\cdot X^{i} \]
  \[ g(M) = \sum_{i=0}k_{i}\bowtie M^{i} \in M^{n\times n}(K)\text{ met } M^{0} = \mathbb{I}_{n} \]
\end{de}

\begin{de}
  Zij $V$ een $K$-vectorruimte van willekeurige dimensie.
  Een lineare transformatie $A \in Hom(V,V)$ kunnen we substitueren voor $X$ in een veelterm $g\in K[X]$ over $K$.
  \[ g = \sum_{i=0}^{r}k_{i}\cdot X^{i} \]
  \[ g(A) = \sum_{i=0}k_{i}\bowtie M^{i} \text{ met } A^{0} = Id \text{ en } A^{n} = A\circ A^{n-1}\]
\end{de}

\begin{de}
  Zij $K,+,\cdot$ een veld en $A,+,\cdot,\bowtie$ en $B,+,\cdot,\heartsuit$ $K$-algebra's.
  Een $K$-\term{algebramorfisme} is een afbeelding $f$ met de volgende eigenschappen.
  \[ f:\ A \rightarrow B:\ a \mapsto f(a) \]
  \[ \forall k\in K: a\in A: f(k\bowtie a) = k \heartsuit f(b) \]
\question{klopt dit? was oefening}
\end{de}

\begin{st}
  Zij $M\in M^{n\times n}(K)$ een matrix over $K$ De afbeelding $\phi$ is een $K$-algebramorfisme:
  \[ \phi:\ K[X] \rightarrow M^{n\times n}(K): g \mapsto g(M) \]
\extra{bewijs}
\end{st}

\begin{st}
  De volgende verzameling is een commutatieve deel-$K$-algebra van $M^{n\times n}(K)$.
  \[ \{ g(M) \ |\ g \in K[X] \} \]
\extra{bewijs}
\end{st}

\subsection{Cayley-Hamilton}
\label{sec:cayley-hamilton-1}

\begin{ei}
  Zij $K,+,\cdot$ een veld en $A\in M^{n\times n}(K)$ een matrix, dan bestaat er een veelterm $g\in K[X]$ met $g$ verschillend van de nulveelterm zodat $A$ een wortel is van $g$.

  \begin{proof}
    De dimensie van $M^{n \times n}(K)$ als vectorruimte over $K$ is $n^{2}$.\waarom
    De $n^{2}+1$ matrices $A^{n^{2}},A^{n^{2}-1},\dotsc,A^{2},A,I$ lineair afhankelijk.\waarom
    Er bestaan dus $c_{i}\in K$ als volgt:
    \[ \sum_{i=0}^{n^{2}}c_{i}A^{i} = 0 \]
    Kies dan $g= \sum_{i=0}^{n^{2}}c_{i}X^{i}$ als veelterm die $A$ als wortel heeft.
  \end{proof}
\end{ei}

\begin{st}
  De \term{stelling van Cayley-Hamilton}.\\
  Zij $K,+,\cdot$ een veld.
  Zij $A$ een lineaire transformatie van een eindigdimensionale $K$-vectorruimte (of een $n\times n$ matrix over $K$). Noteer met $f_{A}$ de karakteristieke veelterm van $A$.
  \[ f_{A}(A) = 0 \]

  \begin{proof}
    Zij $V$ een $n$-dimensionale $K$-vectorruimte en $A$ een lineaire transformatie van $V$ met matrix $M$ ten opzichte van een basis $\{v_{1},\dotsc,v_{n}\}$ van $V$.
    \[ \forall i \in \{1,\dotsc,n\}:\ A(v_{i}) = \sum_{j=1}^{n}m_{ij}v_{j} \]
    We kunnen deze vergelijkingen als volgt opschrijven:
    \[
    \left\{
      \begin{array}{cccccc}
        A(v_{1}) &          &       && =\sum_{j=1}^{n}m_{1j}v_{j}\\
        & A(v_{2}) &       && =\sum_{j=1}^{n}m_{2j}v_{j}\\
        &         & \ddots && \vdots\\
        &         &        & A(v_{n}) &=\sum_{j=1}^{n}m_{nj}v_{j}\\
      \end{array}
    \right.
    \]
    We kunnen de vergelijkingen opnieuw herschrijven als volgt:\clarify{wut?} (vraag nog niet waarom.)
    \[
    \left\{
      \begin{array}{ccccccc}
        (A-m_{11}I)(v_{1}) & -m_{12}I(v_{2}) & \hdots & -m_{1n}I(v_{n}) &=& 0\\
        m_{21}I(v_{1}) & (A-m_{12}I)(v_{2}) & \hdots & -m_{1n}I(v_{n}) &=& 0\\
        \vdots & \vdots &\ddots & \vdots & &\vdots\\
        m_{n1}I(v_{1}) & m_{n2}I(v_{2}) & \hdots & (A-m_{nn}I)(v_{n}) &=& 0\\
      \end{array}
    \right.
    \]
    We definieren nu de $(n\times n)$-matrix $\Delta$ als volgt:
    \[
    \Delta = \delta_{ij}A - m_{ij}I = 
    \begin{pmatrix}
      A-m_{11}I & -m_{12}I & \hdots & -m_{1n}I\\
      m_{21}I & A-m_{12}I & \hdots & -m_{1n}I\\
      \vdots & \vdots &\ddots & \vdots & &\vdots\\
      m_{n1}I & m_{n2}I & \hdots & A-m_{nn}I\\
    \end{pmatrix}
    \]
    We kunnen de vergelijkingen dan nog eens herschrijven als volgt:
    \[ \Delta
    \begin{pmatrix}
      v_{1}\\ v_{2}\\ \vdots \\ v_{n}
    \end{pmatrix}
    =
    \begin{pmatrix}
      0\\ 0\\ \vdots \\ 0
    \end{pmatrix}
    \]
    Merk op dat $\Delta$ elementen heeft in de commutatieve deelring $R_{A}$ van $Hom(V,V)$:
    \[ R_{A} = \{g(A) \mid g \in K[X]\} \]
    Beschouw nu de adjunctmatrix $adj(\Delta)$ van $\Delta$.
    Deze bestaat ook uit elementen in $R_{A}$.
    \[
    adj(\Delta) \cdot \Delta = det(\Delta)I
    \]
    Hieruit halen we nu tenslotte het volgende:
    \[ det(\Delta)I 
    \begin{pmatrix}
      v_{1}\\ v_{2}\\ \vdots \\ v_{n}
    \end{pmatrix}
    = adj(\Delta)\Delta 
    \begin{pmatrix}
      v_{1}\\ v_{2}\\ \vdots \\ v_{n}
    \end{pmatrix}
    = adj(\Delta)
    \begin{pmatrix}
      0\\ 0\\ \vdots \\ 0
    \end{pmatrix}
    =
    \begin{pmatrix}
      0\\ 0\\ \vdots \\ 0
    \end{pmatrix}
    \]
    Dit betekent dat de lineaire transformatie $det(\Delta)I$ elke basisvector $v_{i}$ op $0$ afbeeldt.
    Die determinant moet dus nul zijn, en die determinant is precies $f_{A}(A)$.
  \end{proof}
\clarify{holy shit!}
\end{st}

\section{Minimale veeltermen}
\label{sec:minimale-veeltermen}

\begin{de}
  Zij $K,+,\cdot$ en $A$ een lineaire transformatie van een eindigdimensionale $K$-vectorruimte (of een $n\times n$ matrix over $K$).
  De \term{minimale veelterm} van $A$ is de (unieke) monische veelterm $g\in K[X]$ van minimale graad zodat $g(A) = 0$ geldt.
  We noteren de minimale veelterm van $A$ als $\phi_{A}$.
\end{de}

\begin{st}
  De minimale veelterm van een lineaire transformatie is een deler van de karakteristieke veelterm.
\extra{bewijs, volgt uit Cayley-Hamilton}
\end{st}

\begin{ei}
  Zij $A$ een lineaire transformatie van een eindigdimensionale $K$-vectorruimte $V$ en $M$ de matrix van $A$ ten opzichte van een basis van $V$.
  \[ \phi_{A} = \phi_{M} \]
\extra{bewijs}
\end{ei}

\begin{ei}
  Zij $K'$ een velduidbreiding van een veld $K,+,\cdot$ en $M\in M^{n\times n}(K)$ een matrix over $K$ met minimale veelterm $\phi_{M} \in K[X]$.
  We kunnen $M$ ook beschouwen als element van $M^{n\times n}(K')$ met minimale veelterm $\phi'_{M} \in K'[X]$
  \[ \phi_{M} = \phi'_{M} \]
  \begin{proof}
    \begin{itemize}
    \item $\phi'_{M}$ is zeker een deler van $\phi_{M}$, mant we kunnen $\phi_{M}$ ook beschouwen als veelterm in $K'[X]$.
    \item Neem een basis $\beta$ van $K'$ als $K$-vectorruimte.
      \[ \phi'_{M}= \sum_{j}\lambda_{j}X^{j} \in K'[X] \]
      We kunnen $\phi'_{M}$ dan herschrijven als volgt:
      \[ \phi'_{M} = \sum_{j}\left(\sum_{i}a_{ji}b_{i}\right)X^{j} = \sum_{j}\left(\sum_{i}a_{ji}X\right)b_{i} \]
      Hierein zijn de $a_{ji}\in K$ de co\"ordinaten van $\lambda_{j}$ ten opzichte van de basiselementen $b_{i}\in \beta$.
      Noem nu $f_{i}=\sum_{j}a_{ji}X^{j}\in K[X]$:
      \[ \sum_{j}f_{i}b_{i} \text{ met } f_{i}=\sum_{j}a_{ji}X^{j}\in K[X] \]
      Wanneer we deze veelterm evalueren in $M'$ krijgen we de volgende gelijkheid:
      \[0 = \phi'_{M}(M) = \sum_{i}f_{i}(M)b_{i} \]
      Omdat de $b_{i}$ lineair onafhankelijk zijn over $K$ moeten alle $f_{i}(M)$ het nulelemente zijn in $M^{n\times n}(K)$.

    \item Per definitie van $\phi_{M}$ is $\phi_{M}$ dan een deler van alle $f_{i}$ in $K[X]$.
      $\phi_{M}$ is dus ook een deler van alle $f_{i}$ in $K'[X]$ en daarom ook van $\phi'_{M}$.      
    \end{itemize}
  \end{proof}
\end{ei}

\subsection{Decompositie via minimale veelterm}
\label{sec:decomp-via-minim}

\begin{pr}
  Zij $K,+,\cdot$ een veld en $V$ een eindigdimensionale $K$-vectorruimte.
  Zij $A$ een lineaire transformatie van $V$ met minimale veelterm $\phi_{A}$.
  Zij $\phi_{A} = \phi_{1} \cdot \phi_{2}$ met $\phi_{1}$ en $\phi_{2}$ monische veeltermen over $K$ en relatief priem.
  We beschouwen de lineaire transformaties $\phi_{1}(A)$ en $\phi_{2}(A)$ van $V$ en definieren $V_{1}$ en $V_{2}$ als volgt:
  \[ V_{1} = Ker(\phi_{1}(A)) \quad\text{ en }\quad V_{2} = Ker(\phi_{2}(A)) \]
  \begin{itemize}
  \item $V_{1} = Im(\phi_{2}(A))$ en $V_{2} = Im(\phi_{1}(A))$
  \item $V = V_{1} \oplus V_{2}$, $A(V_{1}) \subseteq V_{1}$ en $A(V_{2}) \subseteq V_{2}$.
  \item De minimale veelterm van $A|_{V_{1}}$ is $\phi_{1}$ en de minimale veelterm van $A|_{V_{2}}$ is $\phi_{2}$.
  \end{itemize}
  
  \begin{proof}
    We merken eerst het volgende op...
    \[ \phi_{1}(A) \circ \phi_{2}(A) = \phi_{2}(A) \circ \phi_{1}(A) = 0 \]
    ... en dat er veeltermen $g$ en $h$ in $K[X]$ bestaan als volgt: ($\phi_{1}$ en $\phi_{2}$ zijn immers relatief priem.$ggd(\phi_{1},\phi_{2}) =1$)\needed
    \[ g\phi_{1}+h\phi_{2} =1\]
    \[ g(A) \circ \phi_{1}(A) + h(A) \circ \phi_{2}(A) = I \]
    \begin{itemize}
    \item We bewijzen dat $Im(\phi_{1}(A))$ gelijk is aan $Ker(\phi_{2}(A))$.
      De andere gelijkheid geldt dan ook.
      \begin{itemize}
      \item $\subseteq$\\
        Uit de eerste opmerking volgt dat $Im(\phi_{1}(A))$ een deel is van $Ker(\phi_{2}(A))$.
      \item $\supseteq$\\
        Kies nu een element $a$ uit $Ker(\phi_{2}(A))$, dan zit $a$ ook in $Im(\phi_{1}(A)$:
        \[ a = \left(\phi_{1}(A) \circ g(A)\right)(a) + \left(h(A) \circ \phi_{2}(A)\right)(a) = \phi_{1}(A)(g(A)(a)) = 0\]
      \end{itemize}
    \item
      \begin{itemize}
      \item $V=V_{1}+V_{2}$\\
        Kies een $v$ uit $V$:
        \[
        \begin{array}{rll}
          v&= \left( \phi_{1}(A) \circ g(A) + h(A) \circ \phi_{2}(A)\right)(v) &\\
          &= \phi_{1}(A)(g(A)(a))+\phi_{2}(A)(h(A)(a)) & \in Im(\phi_{1}(A))+Im(\phi_{2}(A))=V_{1}+V_{2} \\
        \end{array}\]
      \item $V_{1}\cap V_{2} = \{0\}$\\
        Kies een $v$ uit $V_{1}\cap V_{2}$, dan geldt het volgende vanuit de tweede opmerking.
        \[ \phi_{1}(A)(g(A)(v)) + \phi_{2}(A) (h(A)(v)) = v \]
        $v$ moet dan het nulelement zijn.
      \end{itemize}
   We tonen nog aan dat $A(V_{1})$ een deel is van $V_{1}$.
      Kies hiervoor een $v\in V_{1}$; dus met $(\phi_{1}(A))(v) = 0$.
      \[
      \phi_{1}(A)(A(v)) = (\phi_{1}(A) \circ A)(v) = (A \circ \phi_{1}(A))(v) = A(0) = 0
      \]
      Bijgevolg is $A(V_{1})$ inderdaad een deel van $Kep(\phi_{1}(A)) = V$.
    \item Noteer voorlopig de minimale veeltermen van $A|_{V_{1}}$ en $A|_{V_{2}}$ als $\gamma_{1}$ en $\gamma_{2}$.
      Uit de definitie van $V_{1}$ en $V_{2}$ volgt dan het volgende:
      \[ \gamma_{1}|\phi_{1} \text{ en } \gamma_{2}|\phi_{2} \]
      Anderzijds beweren we dat $\phi_{1}\phi_{2}|\gamma_{1}\gamma_{2}$ geldt.
      Hieruit volgt dat $\gamma_{1}$ en $\gamma_{2}$ respectievelijk gelijk zijn aan $\phi_{1}$ en $\phi_{2}$.
      Om tenslotte onze bewering te bewijzen zullen we nagaan dat $(\gamma_{1}\gamma_{2})(A) = 0$ geldt.
      Neem hiervoor een willekeurige $v\in V$.
      We kunnen $v$ dan ontbinden als $v=v_{1}+v_{2}$ met $v_{1}\in V_{1}$ en $v_{2}\in V_{2}$.
      \[ ((\gamma_{1}\gamma_{2})(A))(v) = \gamma_{2}(A)(\gamma_{1}(A)(v_{1})) + \gamma_{1}(A)(\gamma_{2}(A)(v_{2})) = 0+ 0= 0 \]
    \end{itemize}
  \end{proof}
\end{pr} 

\begin{gev}
  Zij $K,+,\cdot$ een veld en $V$ een eindigdimensionale $K$-vectorruimte.
  Zij $A$ een lineaire transformatie van $V$ met minimale veelterm $\phi_{A}$.
  Zij $\phi_{A} = \phi_{1}^{p_{1}}\cdot phi_{2}^{p_{2}}\cdot \dotsb phi_{k}^{p_{k}}$ met $\phi_{1},\dotsc,\phi_{k}\in K[X]$ onderling verschillende monische irreducibele veeltermen.
  Noteer $V_{i} = Ker(\phi_{i}(A)^{p_{i}}$.
  \[ V = V_{1} \oplus V_{2} \oplus \dotsb \oplus V_{k} \]
  \[ \forall i: A(V_{i}) \subseteq V_{i} \]
  De minimale veelterm van $A|_{V_{i}}$ is $\phi_{i}^{p_{i}}$.
\extra{bewijs}
\end{gev}

\begin{st}
  Zij $K,+,\cdot$ een algebra\"isch gesloten vold en $V$ een eindigdimensionale $K$-vectorruimte.
  Zij $A$ een lineaire transformatie van $V$ met minimale veelterm $\phi_{A}$.
  \[ \phi_{A} = \prod_{i=1}^{k}(X-c_{i})^{p_{i}} \]
  In bovenstaande gelijkheid zijn alle $c_{i} \in K$ verschillend.
  Noteer $V_{i}=Ker((A-c_{i}I)^{p_{i}})$.
  \begin{itemize}
  \item $V = \bigoplus_{i=1}^{k}V_{i}$, $A(V_{i}) \subseteq V_{i}$ en de minimale veelterm van $A_{i} = A|_{V_{i}}$ is $(X-c_{i})^{p_{i}}$.
  \item De eigenwaarden van $A$ zijn de $c_{i}$
  \item Noteer $m_{i}$ voor de dimensie van $V_{i}$.
    De karakteristieke veeltermen $f_{A}$ en $f_{A_{i}}$ van $A$ en $A_{i}$ zijn gegeven als volgt:
    \[ f_{A_{i}} = (X-c_{i})^{m_{i}} \quad\text{ en }\quad f_{A} = \prod_{i=1}^{k}f_{A_{i}} \]
  \end{itemize}
\TODO{bewijs p 118}
\end{st}

\begin{de}
  De deelruimte $V_{i}$ noemt men de \term{veralgemeende eigenruimte} of \term{karakteristieke deelruimte} van de eigenwaarde $c_{i}$.
\end{de}

\begin{ei}
  Zij $K,+,\cdot$ een veld en $A$ een lineaire transformatie van een eindigdimensionale $K$-vectorruimte, dan hebben $\phi_{A}$ en $f_{A}$ dezelfde irreducibele factoren in $K[X]$.
\end{ei}

\subsection{Diagonaliseren}
\label{sec:diagonaliseren}

\begin{st}
  \examen
  Zij $K,+,\cdot$ een willekeurig veld, $V$ een eindigdimensionale $K$-vectorruimte en $A$ een lineaire transformatie van $V$.
  $A$ is diagonaliseerbaar (over $K,+,\cdot$) als en slechts als de minimale veelterm $\phi_{A}\in K[X]$ van $A$ het product is van onderling verschillende lineaire veeltermen in $K[X]$.
  \[ \phi_{A} = \prod_{i=1}^{k}(X-c_{i}) \text{ met } \forall i,j:\ c_{i} \neq c_{j} \]
\TODO{bewijs p 119}
\end{st}

\begin{ei}
  Zij $K,+,\cdot$ een veld en $M\in M^{n\times n}(K)$ een matrix, dan bestaat er een matrix $P \in GL_{n}(K)$ zodat $P^{-1}MP$ een diagonaalmatrix is als en slechts als de minimale veelterm $\phi_{M}\in K[X]$ van $M$ het product is van onderling verschillende lineaire veeltermen in $K[X]$.
\extra{bewijs} 
\end{ei}

\section{Nilpotente transformaties en Jordanvorm}
\label{sec:nilp-transf-en}

\subsection{Nilpotente transformaties}
\label{sec:nilp-transf}

\begin{de}
  Zijk $K,+,\cdot$ een willekeurig veld.
  Een lineaire transformatie $A$ van een (niet-triviale) $K$-vectorruimte $V$(, of een matrix $A\in M^{n\times n}(K)$) is \term{nilpotent} van index $k\in \mathbb{N}_{0}$ als het volgende geldt:
  \[ A^{k} = 0 \text{ en } A^{k-1} \neq 0 \]
\end{de}

\begin{ei}
  Een lineaire transformatie $A$ van een eindigdimensionale $K$-vectorruimte $V$(, of een matrix $A\in M^{n\times n}(K)$) is nilpotent van index $k$ als en slechts als de minimale veelterm $\phi_{A}$ gelijk is aan $X^{k}$.
\extra{bewijs}
\end{ei}

\begin{ei}
  \examen
  Zij $K,+,\cdot$ een veld en $V$ een eindigdimensionale $K$-vectorruimte.
  Zij $A$ een nilpotente transformatie van $V$ van index $k$.
  Neem $v\in V$ een vector zodat $A^{k-1}(v) \neq \vec{0}$.
  $v, A(v), A^{2}(V), \dotsc, A^{k-1}(v)$ zijn lineair onafhankelijk in $V$.
\TODO{bewijs p 122}
\end{ei}

\begin{st}
  Zij $K,+,\cdot$ een veld en $V$ een $K$-vectorruimte van eindige dimensie $n$.
  Zij $A$ een nilpotente transformatie van $V$ van index $k$, dan bestaan er...
  \begin{itemize}
  \item ... strikt positieve natuurlijke getallen $p$ en $k_{1}, \dotsc, k_{p}$ met $k_{i} \le k_{i+1}$ als volgt:
    \[ \sum_{i=1}^{p}k_{i} = n \]
  \item ...  en $p$ vectoren $v_{i}\in V$ met $A^{k_{i}}(v_{i}) = 0$ ...
  \end{itemize}
  ... zodat de volgende verzameling een basis vormt van $V$.
  \[ \bigcup_{i=1}^{p} \{ v_{i}, A(v_{i}), \dotsc, A^{k_{i}-1}(v_{i}) \} \]
  Bovendien zijn $p$ en de $k_{i}$ hierin uniek bepaald.
\extra{bewijs}
\end{st}

\begin{de}
  De $k_{1}, \dotsc, k_{p}$ uit vorige stelling noemen we het \term{invariant systeem} van de nilpotente verzameling $A$.
\end{de}

\begin{gev}
  \examen
  Zij $K,+,\cdot$ een veld en $V$ een eindigdimensionale $K$-vectorruimte.
  Zij $A$ een nilpotente transformatie van $V$ van index $k$ en met invariant systeem $\{k =k_{1},\dotsc,k_{p}\}$
  Er bestaat dan een basis van $V$ zodat de matrix van $A$ ten opzicte van die basis de volgende blokvorm heeft:
  \[ 
  \begin{pmatrix}
    A_{1} & 0 & \hdots & 0\\
    0 & A_{2} & \hdots & 0\\
    \vdots & \vdots & \ddots & \vdots\\
    0 & 0 & \hdots & A_{p}\\
  \end{pmatrix}
  \]
  Hierin is elke $A_{i}$ een $k_{i}\times k_{i}$-matrix met de volgende vorm:
  \[ 
  \begin{pmatrix}
    0 & 0 & 0 & \hdots & 0 & 0 & 0\\
    1 & 0 & 0 & \hdots & 0 & 0 & 0\\
    0 & 1 & 0 & \hdots & 0 & 0 & 0\\
    \vdots & \vdots & \vdots & \ddots & \vdots & \vdots & \vdots\\
    0 & 0 & 0 & \hdots & 0 & 0 & 0\\
    0 & 0 & 0 & \hdots & 1 & 0 & 0\\
    0 & 0 & 0 & \hdots & 0 & 1 & 0\\
  \end{pmatrix}
  \]
\extra{bewijs}
\end{gev}

\begin{pr}
  Zij $K,+,\cdot$ een veld en zij gegeven strikt positieve natuurlijke getallen $n$, $p$ en $k_{1}, \dotsc, k_{p}$ zodat $n$ de som is van de $k_{i}$.
  \begin{itemize}
  \item Er bestaat voor elke $K$-vectorruimte $V$ van dimensie $n$ een nilpotente transformatie van $V$ met invariant syteem $k_{1}, \dotsc, k_{p}$.
  \item Zij $V$ en $V'$ precies zo'n $K$-vectorruimten en $A$ en $A'$ nilpotente transformaties met dat invariant systeem, dan bestaat er een isomorfisme (van $K$-vectorruimte) $f: V\rightarrow V'$ zodat het volgende diagram commuteert:
    \begin{figure}[H]
      \centering
      \begin{tikzpicture}
        \matrix (m) [matrix of math nodes,row sep=3em,column sep=4em,minimum width=2em]
        {
          V & V \\
          V' & V' \\};
        \path[-stealth]
        (m-1-1) edge node [left] {$f$} (m-2-1)
        edge node [above] {$A$} (m-1-2)
        (m-2-1.east|-m-2-2) edge node [below] {$A'$} (m-2-2)
        (m-1-2) edge node [right] {$f$} (m-2-2);
      \end{tikzpicture}
    \end{figure}
  \end{itemize}
  \extra{bewijs}
\end{pr}


\subsection{Jordanvorm}
\label{sec:jordanvorm}

\begin{de}
  Zij $K,+,\cdot$ een veld en $c\in K$ een element van $K$.
  We definieren voor $k\in \mathbb{N}_{0}$ een matrix $J(c;k)$ als volgt:
  \[
  J(c;k) = 
  \begin{pmatrix}
    c & 0 & 0 & \hdots & 0 & 0 & 0\\
    1 & c & 0 & \hdots & 0 & 0 & 0\\
    0 & 1 & c & \hdots & 0 & 0 & 0\\
    \vdots & \vdots & \vdots & \ddots & \vdots & \vdots & \vdots\\
    0 & 0 & 0 & \hdots & c & 0 & 0\\
    0 & 0 & 0 & \hdots & 1 & c & 0\\
    0 & 0 & 0 & \hdots & 0 & 1 & c\\
  \end{pmatrix}
  \in M^{k\times k}(K)
  \]
  We definieren $J(c;k_{1},\dotsc,k_{p})$ verder als de volgende blokmatrix:
  \[
  J(c;k_{1},\dotsc,k_{p}) = 
  \begin{pmatrix}
    J(c;k_{1}) & 0 & \hdots & 0\\
    0 & J(c;k_{2}) & \hdots & 0\\
    \vdots & \vdots & \ddots & \vdots\\
    0 & 0 & \hdots & J(c;k_{p})\\
  \end{pmatrix}
  \in M^{n\times n}(K)
  \]
\end{de}

\begin{de}
  Een \term{Jordanmatrik} over een veld $K,+,\cdot$ is een blokmatrix van de volgende vorm, waarin de $c_{i}$ onderling verschillende elementen zijn van $K$.
  \[
  \begin{pmatrix}
    J(c;k_{1}^{(1)},\dotsc,k_{p}^{(2)}) & 0 & \hdots & 0\\
    0 & J(c;k_{1}^{(2)},\dotsc,k_{p}^{(2)}) & \hdots & 0\\
    \vdots & \vdots & \ddots & \vdots\\
    0 & 0 & \hdots & J(c;k_{1}^{(m)},\dotsc,k_{p}^{(m)})\\
  \end{pmatrix}
  \]
\end{de}

\begin{st}
  Zij $K,+,\cdot$ een algebra\"isch gesloten veld en $A$ een lineaire transformatie van een eindigdimensionale $K$-vectorruimte, dan bestaat er een basis van $V$ zodat de matrix $A$ ten opzichte van die basis een Jordanmatrix is.
  Deze matrix is bovendien uniek op de volgorde van de blokken na.
  \TODO{bewijs p 126}
\end{st}

\begin{ei}
  Zij $K,+,\cdot$ een algebra\"isch gesloten veld.
  Zij $V$, respectievelijk $V'$ een $K$-vectorruimte van dimensie $n$ en $A$, respectievelijk $A'$ een lineaire transformatie hierop.
  $A$ en $A'$ hebben dezelfde Jordanvorm als en slechs als er een isomorfisme van $K$-vectorruimten $f$ bestaat zodat het volgende diagram commuteert
  \begin{figure}[H]
    \centering
    \begin{tikzpicture}
      \matrix (m) [matrix of math nodes,row sep=3em,column sep=4em,minimum width=2em]
      {
        V & V \\
        V' & V' \\};
      \path[-stealth]
      (m-1-1) edge node [left] {$f$} (m-2-1)
      edge node [above] {$A$} (m-1-2)
      (m-2-1.east|-m-2-2) edge node [below] {$A'$} (m-2-2)
      (m-1-2) edge node [right] {$f$} (m-2-2);
    \end{tikzpicture}
  \end{figure}
  \extra{bewijs}
\end{ei}

\begin{de}
  Zij $K,+,\cdot$ een algebra\"isch gesloten veld.
  Zij $M\in M^{n\times n}(K)$ een matrix over $K$.
  De Jordanvorm van $M$ is de unieke Jordanmatrix $J\in M^{n\times n}(K)$ zodat er een $P\in GL_{n}(K)$ bestaat opdat $J = P^{-1}MP$ geldt.
\end{de}

\begin{pr}
  Twee matrices $M$, $M'$ over $K$ hebben dezelfde Jordanvorm als en slechts als er een $P\in GL_{n}(K)$ bestaat met $M' = P^{-1}MP$.
  Met andere woorden: ``Gelijkvormige matrices hebben dezelfde Jordanvorm.''
\extra{bewijs}
\end{pr}

\section{Normale transformaties}
\label{sec:norm-transf}

\subsection{Inproducten}
\label{sec:inproducten}

\begin{de}
  Zij $V$ een $\mathbb{R}$-vectorruimte.
  Een \term{inproduct} is een afbeelding $\langle \cdot, \cdot\rangle$ ...
  \[ \langle \cdot , \cdot \rangle :\ V \times V \rightarrow \mathbb{R}:\ (v,w) \mapsto \langle v, w \rangle \]
  ... met de volgende eigenschappen: 
  \begin{itemize}
  \item $\forall v, w \in V:\ \langle v,w \rangle = \langle w, v \rangle$
  \item $\forall v,v',w \in V,\ \forall \alpha,\beta \in \mathbb{R}:\ \langle \alpha v + \beta v', w \rangle = \alpha \langle v,w \rangle + \beta \langle v', w \rangle $ (lineariteit)
  \item $\forall v\in V:\ \langle v,v \rangle \in \mathbb{R}^{+}$
  \item $\forall v\in V:\ \langle v,v \rangle = 0 \Leftrightarrow v = \vec{0}$
  \end{itemize}
\end{de}

\begin{de}
  Zij $V$ een $\mathbb{C}$-vectorruimte.
  Een \term{hermetisch product} is een afbeelding $\langle \cdot, \cdot\rangle$ ...
  \[ \langle \cdot , \cdot \rangle :\ V \times V \rightarrow \mathbb{C}:\ (v,w) \mapsto \langle v, w \rangle \]
  ... met de volgende eigenschappen:
  \begin{itemize}
  \item $\forall v, w \in V:\ \langle v,w \rangle = \overline{\langle w, v \rangle}$
  \item $\forall v,v',w \in V,\ \forall \alpha,\beta \in \mathbb{C}:\ \langle \alpha v + \beta v', w \rangle = \alpha \langle v,w \rangle + \beta \langle v', w \rangle $
  \item $\forall v\in V:\ \langle v,v \rangle \in \mathbb{R}^{+}$
  \item $\forall v\in V:\ \langle v,v \rangle = 0 \Leftrightarrow v = \vec{0}$
  \end{itemize}
\end{de}

\begin{de}
  Een re\"ele of complexe vectorruimte, uitgerust met een inproduct of hermitisch product, noemen we een \term{inproductruimte}.
\end{de}

\begin{de}
  Zij $V$ een re\"ele of complexe inproductruimte.
  Voor elke vector $v\in V$ defini\"eren we de \term{norm} $|v|$ van $v$ als volgt:
  \[ |v| = \sqrt{\langle v,v\rangle} \]
\end{de}

\begin{de}
  Zij $V$ een re\"ele of complexe inproductruimte.
  Twee vectoren $v$ en $w$ uit $V$ noemen we \term{orthogonaal} als hun inproduct nul is.
\end{de}

\begin{de}
  Zij $V$ een re\"ele of complexe inproductruimte.
  Twee vectoren $v$ en $w$ uit $V$ noemen we \term{orthonormaal} als zo orthogonaal zijn en norm $1$ hebben.
\end{de}

\begin{st}
  De \term{ongelijkheid van Cauchy-Schwarz}\\
  Zij $V$ een re\"ele of complexe inproductruimte.
  \[ \forall v,w \in V:\ |\langle v,w \rangle| \le |v| \cdot |w| \]
\TODO{bewijs p 3}
\end{st}

\begin{de}
  Zij $V,\langle\cdot,\cdot\rangle$ en $W,[\cdot,\cdot]$ inproductruimten over hetzelfde veld.
  Een lineaire afbeelding $A:V \rightarrow W$ heet een \term{isometrie} als het volgende geldt:
  \[ \forall v,w \in V: \langle v,w \rangle =[A(v),A(w)] \]
\end{de}

\begin{ei}
  Een isometrie is bijectief.
\extra{bewijs}
\end{ei}

\begin{ei}
  Zij $A$ een lineaire afbeelding van de inproductruimte $V,\langle\cdot,\cdot\rangle$ naar de inproductruimte $W,[\cdot,\cdot]$.
  $A$ is een isometrie als en slechts als $A$ de norm van vectoren bewaart.
  \[ \forall v \in V:\ |v| = |A(v)| \]
\TODO{bewijs p 3}
\end{ei}

\subsection{De adjunct van een lineaire transformatien}
\label{sec:de-adjunct-van}

\begin{st}
  \examen
  Zij $V,\langle\cdot,\cdot\rangle$ een eindigdimensionale re\"ele of complexe inproductruimte en zij $A\in Hom(V,V)$ een lineaire transformatie van $V$.
  Er bestaat een unieke lineaire transformatie $A^{*} \in Hom(V,V)$ zodat het volgende geldt:
  \[ \langle A(v),w \rangle = \langle v, A^{*}(w) \rangle \]
\TODO{bewijs p 4}
\end{st}

\begin{de}
  De lineaire transformatie $A^{*}$ uit de vorige stelling noemen we de \term{adjuncte lineaire transformatie} van $A$.
\end{de}

\begin{de}
  De adjunct $M^{*}$ van een matrix $M$ is de matrix $\overline{M}^{T}$.
\end{de}

\begin{ei}
  Zij $V,\langle\cdot,\cdot\rangle$ een eindigdimensionale re\"ele of complexe inproductruimte. 
  Zij $\varepsilon$ een orthonormale basis van $V$ en $A \in Hom{V,V}$ een lineaire tranformatie van $V$.
  Als $M$ de matrix is van $A$ ten opzichte van $\varepsilon$, dan is $M^{*}$ de matrix van $A^{*}$ ten opzichte van $\varepsilon$.
\TODO{bewijs p 6}
\end{ei}

\subsection{Normale transformaties}
\label{sec:norm-transf-1}

\begin{de}
  Zij $V,\langle\cdot,\cdot\rangle$ een inproductruimte en $A\in H(V,V)$ een lineaire transformatie van $V$ waarvoor de adjunct $A^{*}$ bestaat.
  \begin{itemize}
  \item We noemen $A$ \term{normaal} als $A\circ A^{*} = A^{\circ} \circ A$ geldt.
  \item We noemen $A$ \term{hermetisch} als $A=A^{*}$ geldt.
  \item We noemen $A$ \term{unitair} als $A^{-1} = A^{*}$ geldt.
  \end{itemize}
\end{de}
\TODO{eigenschappen p 7}

\begin{lem}
  \examen
  Zij $V,\langle\cdot,\cdot\rangle$ een inproductruimte en $A,B \in H(V,V)$ lineaire transformaties van $V$.
  \begin{itemize}
  \item $(A+B)^{*} = A^{*} + B^{*}$
  \item $(A^{k})^{*} = (A^{*})^{k}$
  \item $(\lambda A)^{*} = \overline{\lambda}A^{*}$ voor $\lambda \in F$.
  \item $(A^{*})^{*}$
  \end{itemize}
\TODO{bewijs p 8}
\end{lem}

\begin{lem}
  Zij $A$ een normale transformatie van een inproductruimte $V,\langle\cdot,\cdot\rangle$.
  \begin{itemize}
  \item Voor alle $f\in F[X]$ is $f(A)$ een normale transformatie van $V$.
  \item $\forall v\in V: |A(v)| = |A^{*}(v)|$
  \item Als $v$ een eigenvector is van $A$ bij eigenwaarde $\lambda \in F$, dan is $v$ een eigenvector van $A^{*}$ bij eigenwaarde $\overline{\lambda}$. 
  \item Eigenvectoren van $A$ bij verschillende eigenwaarden zijn orthogonaal.
  \item Zij $v$ een vector in $V$.
    \[ A^{k}(v) = 0 \Rightarrow A(v) = 0 \]
  \end{itemize}
\TODO{bewijs p 8}
\end{lem}

\begin{lem}
  Zij $V,\langle\cdot,\cdot\rangle$ een eindigdimensionale complexe improductruimte.
  Zij $A\in Hom(V,V)$ een normale transformatie.
  De minimale veelterm $\varphi_{A}$ een product van verschillende lineaire factoren.
\TODO{bewijs p 9}
\end{lem}

\begin{st}
  \examen
  Zij $V,\langle\cdot,\cdot\rangle$ een eindigdimensionale complexe improductruimte.
  $A$ is normaal als en slechts als $V$ een orthonormale basis van eigenvectoren voor $A$ heeft.
  \TODO{bewijs p 9}
\end{st}

\begin{gev}
  Zij $M$ een normale matrix, dan bestaat er een unitaire matrix $U$ zodat $U^{-1}MU$ een diagonaalmatrix is.
  \extra{bewijs}
\end{gev}

\begin{de}
  Zij $O$ een re\"ele matrix, dan heet $O$ \term{orthogonaal} als $O^{T}= O^{-1}$ geldt.
\end{de}

\end{document}