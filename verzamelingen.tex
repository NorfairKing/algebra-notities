\documentclass[main.tex]{subfiles}
\begin{document}

\chapter{Verzamelingen}
\label{cha:verzamelingen}

\begin{de}
  Een \emph{verzameling} is een geheel van onderling verschillende, ongeordende objecten. Deze objecten noemt men de elementen van de verzameling.
de 
\end{de}

\begin{de}
  Een \emph{formele beschrijving} van een verzameling met behulp van een predikaat $p$ ziet er als volgt uit.
  \[ \{x\ |\ p(x)\} \]
  Dit is de verzameling van all elementen die aan het predikaat $p$ voldoen.
\end{de}

\begin{de}
  Twee verzamelingen $A$ en $B$ zijn \emph{gelijk} als en slechts als ze dezelfde elementen bevatten. 
  \[ A = B \Leftrightarrow \forall x:\ x \in A \Leftrightarrow x \in B \]
\end{de}

\begin{st}
  \emph{Transitiviteit van '$=$':} Gegeven drie willekeurige verzamelingen $A$, $B$ en $C$.
  \[ A = B \wedge B = C \Rightarrow A = C \]
  \begin{proof}
    \[
    \begin{array}{cll}
      A = B \wedge B = C &\Leftrightarrow (\forall x:\ x \in A \Leftrightarrow x \in B) \wedge (\forall x:\ x \in B \Leftrightarrow x \in C) &\\
      &\Rightarrow \forall x:\ ((x \in A \Leftrightarrow x \in B) \wedge (x \in B \Leftrightarrow x \in C)) &\\
      &\Leftrightarrow \forall x:\ (x \in A \Leftrightarrow x \in C) &\Leftrightarrow  A = C\\
    \end{array}
    \]
  \end{proof}
\end{st}

\begin{de}
  Een verzameling $A$ is \emph{een deelverzameling} van een verzameling $B$ als en slechts als $B$ alle elementen van $A$ bevat.
  \[ A \subseteq B \Leftrightarrow \forall x:\ x \in A \Rightarrow x \in B\]
\end{de}

\begin{st}
  \emph{Anti-symmetrie van '$\subseteq$':} Gegeven twee willekeurige verzamelingen $A$ en $B$.
  \[ A \subseteq B \wedge B \subseteq A  \Leftrightarrow A = B \]
  \begin{proof}
    \[
    \begin{array}{cll}
      A \subseteq B \wedge B \subseteq A &\Leftrightarrow (\forall x:\ x \in A \Rightarrow x \in B) \wedge (\forall x:\ x \in B \Rightarrow x \in A) &\\
      & \Leftrightarrow \forall x:\ ((x \in A \Rightarrow x \in B) \wedge (x \in B \Rightarrow x \in A)) &\\
      & \Leftrightarrow \forall x:\ x \in A \Leftrightarrow x \in B &\Leftrightarrow A = B  \\
    \end{array}
    \]
  \end{proof}
\end{st}

\begin{st}
  \emph{Transitiviteit van '$\subseteq$':} Gegeven drie willekeurige verzamelingen $A$, $B$ en $C$.
  \[ A \subseteq B \wedge B \subseteq C  \Leftrightarrow A \subseteq C \]
  \begin{proof}
    \[
    \begin{array}{cll}
      A \subseteq B \wedge B \subseteq C &\Leftrightarrow (\forall x:\ x \in A \Rightarrow x \in B) \wedge (\forall x:\ x \in B \Rightarrow x \in C) &\\
      & \Rightarrow \forall x:\ ((x \in A \Rightarrow x \in B) \wedge (x \in B \Rightarrow x \in C)) &\\
      & \Leftrightarrow \forall x:\ x \in A \Rightarrow x \in C &\Leftrightarrow A \subseteq C  \\
    \end{array}
    \]
  \end{proof}
\end{st}

\begin{de}
  Een verzameling $A$ is een \emph{strikte deelverzameling} van een verzameling $B$ als en slechts als $A$ een deelverzameling is van $B$ en niet gelijk is aan $B$.
  \[ A \subset B \Leftrightarrow A \subseteq B \wedge a \neq B \]
\end{de}



\end{document}
