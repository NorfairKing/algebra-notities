% Voor subfiles
\usepackage{subfiles}

% Voor algoritmes
\usepackage{algorithm2e}

% Voor todo's
\usepackage{todonotes}

% Voor wiskunde
\usepackage{amsmath}
\usepackage{amsfonts}
\usepackage{amssymb}
\usepackage{amsthm}


% Voor mooiere breuken
\usepackage{nicefrac}

% Voor urls
\usepackage{hyperref}

% Voor svg illustraties
\usepackage[clean,pdf]{svg}
\setsvg{svgpath = illustraties/}
\setsvg{inkscape = inkscape -z -D}

% Een mooier bestand met wiskunde ondersteuning
\usepackage{libertine}
\usepackage[libertine]{newtxmath}

% Om het totaal aantal pagina's te tellen
\usepackage{lastpage}
\usepackage{afterpage}

% Voor tekeningen
\usepackage{tikz}
\usetikzlibrary{decorations}
\usetikzlibrary{calc}
\usetikzlibrary{arrows.meta}
\usetikzlibrary{matrix}
\usetikzlibrary{fit}
\usetikzlibrary{shapes}

% Nog tekeningen
\usepackage{pgfplots}

% Om figuren op de juiste plaats te krijgen
\usepackage{float}

% Voor frames
\usepackage{mdframed}

% Om de marges aan te passen
\usepackage[left=2cm,right=2cm,top=2cm,bottom=2cm]{geometry}

% Voor headers en footers
\usepackage{fancyhdr}

% fancy verbatim
\usepackage{fancyvrb}

% program listings
\usepackage{listings}

% Voor mooie links
\usepackage{hyperref}
\hypersetup{
colorlinks,
citecolor=black,
filecolor=black,
linkcolor=black,
urlcolor=black
}

%indices
\usepackage{makeidx}
\makeindex