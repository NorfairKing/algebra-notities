\documentclass[main.tex]{subfiles}
\begin{document}


\chapter{Examenvragen: Toepassingen van Algebra}
\label{cha:examen-tai}

\section{Benoem deze algebra}
\subsection{Abstract}

\subsubsection{Vraag}
\begin{center}
  Benoem een gegeven algebra.
\end{center}

\subsubsection{Antwoord}
De algebra zal $2$, $3$ of $4$ bewerkingen bevatten.
Afhankelijk hiervan vallen er al een aantal opties weg.
\begin{itemize}
\item $1$ bewerking: Zeker geen ring of booleaanse algebra
\item $2$ bewerking: Zeker geen booleaanse algebra
\end{itemize}
(Hoogstwaarschijnlijk zal de structuur een speciaal geval zijn van een ring maar geen veld.)
Beantwoord zeker de volgende vragen:
\begin{itemize}
\item $1$ bewerkingen: $G,\heartsuit$
  \begin{itemize}
  \item Is $G,\heartsuit$ een groep?
  \item Is $G,\heartsuit$ commutatief?
  \end{itemize}
\item $2$ bewerkingen: $R, \Box, \spadesuit$
  \begin{itemize}
  \item Is $R, \Box$ een groep?
  \item Is $R, \Box$ commutatief?
  \item Is $R, \Box, \spadesuit$ een ring?
  \item Is $R, \Box, \spadesuit$ commutatief?
  \item Heeft $R, \Box, \spadesuit$ nuldelers?
  \item Heeft $R, \Box, \spadesuit$ een eenheidselement?
    \begin{itemize}
    \item Bevat $R, \Box, \spadesuit$ enkel eenheden?
    \end{itemize}
  \item Bekijk dan figuur \ref{fig:ringen} op pagina \pageref{fig:ringen} en beslis verder.
  \end{itemize}
\item $3$ bewerkingen: $B,\star,\oslash, \triangleright$
  \begin{itemize}
  \item Is $B,\star,\oslash, \triangleright$ een booleaanse algebra?
  \end{itemize}
\end{itemize}

\subsection{Voorbeeld}
\subsubsection{Vraag}
\begin{center}
  Benoem de algebra $\mathbb{R}^{2\times 2},+,\cdot$.
\end{center}

\subsubsection{Antwoord}
$\mathbb{R}^{2\times 2}$, heeft twee bewerkingen.

\begin{itemize}
\item $\mathbb{R}^{2\times 2},+$ is een groep.
\item $\mathbb{R}^{2\times 2},+$ is commutatief.
  Het heeft dus zin om verder te kijken of het $\mathbb{R}^{2 \times 2}$ een ring is.
\item $\mathbb{R}^{2\times 2},+,\cdot$ is een ring.
\item $\mathbb{R}^{2\times 2},+,\cdot$ is niet commutatief.
  Qua specifieke structuren kan het dus enkel een lichaam zijn, maar dan mag het geen nuldelers hebben.
\item $\mathbb{R}^{2\times 2},+,\cdot$ bevat nuldelers:
  \[
  \begin{pmatrix}
    1 & 0\\
    0 & 0
  \end{pmatrix}
  \cdot
  \begin{pmatrix}
    0 & 0\\
    0 & 1
  \end{pmatrix}
  =
  \begin{pmatrix}
    0 & 0\\
    0 & 0
  \end{pmatrix}
  \]
\end{itemize}





\newpage
\section{Ontbind in priemfatoren}

\section{BM algoritme}

\section{Convolutionele codes}


\end{document}
