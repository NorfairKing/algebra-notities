\documentclass[main.tex]{subfiles}
\begin{document}

\chapter{Groepen}
\label{cha:groepen}

\section{Basisbegrippen}
\label{sec:basisbegrippen}

\subsection{De groep}
\label{sec:de-groep}

\begin{de}
  Een \term{halfgroep} of \term{mono\"ide} $G,*$ is een algebra\footnote{Zie definitie \ref{de:algebra}.} die bestaat uit een (niet-lege) verzameling $G$ en een overal bepaalde, interne afbeelding $*$ (De bewerking).
  \[ *: G \times G \rightarrow: (x,y) \mapsto x * y \]
  De bewerking $*$ voldoet aan de volgende voorwaarden.
  \begin{itemize}
  \item De bewerking $*$ is \term{associatief}:
    \[ \forall x, y, z \in G: (x*y)*z = x*(y*z) \] 
  \item Er bestaat een \term{neutraal element} $e \in G$ voor de bewerking $*$.
    \[ \forall x \in G: x*e = e = e*x \]
  \end{itemize}
\end{de}

\begin{de}
  \label{de:groep}
  Een \term{groep} $G,*$ is een mono\"ide waarvoor nog een extra voorwaarde geldt:\\
  Elk element $x \in G$ heeft een \term{inverse} met betrekking tot de bewerking $*$: $x'$
  \[ \forall x \in G, \exists x' \in G:\ x * x' = e = x' * x \]
\end{de}

\begin{st}
  \label{st:groep-uniek-invers-element}
  De inverse $x^{-1}$ van een element $x$ van een groep $G$ is uniek.
  
  \begin{proof}
    Bewijs uit het ongerijmde.\\
    Stel dat er twee verschillende inversen $y$ en $z$ zijn van $x$ in $G$.
    \[  
    \begin{array}{rl}
      y &= y * e_{G}\\
        &= y * (x * z)\\
        &= (y * x) * z\\
        &= e_{G} * z\\
        &= z
    \end{array}
    \]
    De derde gelijkheid geldt omdat de bewerkin $*$ associatief is.\footnote{Zie de definitie van een groep (Definitie \ref{de:groep}).} De vierde gelijkheid geldt omdat het neutraal element van een groep uniek is.\footnote{Zie stelling \ref{st:neutraal-element-uniek}.}
  \end{proof}
\end{st}

\begin{de}
  Een \term{commutatieve groep} of \term{abelse groep} $G,*$ is een groep waarbij de bewerking $*$ commutatief is.
  \[ \forall x,y \in G: x * y = y * x\]
\end{de}

\begin{de}
  Zij $G,*$ een groep en $H$ een (niet-lege) deelverzameling van $G$. We noemen $H$ een \term{deelgroep} van $G$ als $H$ zelf ook een groep is met dezelfde bewerking $*$.
\end{de}

\begin{st}
  \label{st:deelgroep-zelfde-neutraal-element}
  Zij $H$ een deelgroep van $G,*$, dan is $e_{G}$ ook het neutraal element van $H$.

  \begin{proof}
    Noem $e_{G}$ het neutraal element van $G,*$ en $e_{h}$ dat van $H,*$. Noem het invers van een element $x$ in $G$ $x^{-1}$ en het invers van $x$ in $H$ $\bar{x}$.
    \[
    \begin{array}{rrl}
                  & e_{H} * e_{H} &= e_{H} * e_{G}\\
      \Rightarrow & e^{-1}_{H} * (e_{H} * e_{H}) &= e^{-1}_{H} * (e_{H} * e_{G})\\
      \Rightarrow & (e^{-1}_{H} * e_{H}) * e_{H} &= (e^{-1}_{H} * e_{H}) * e_{G}\\
      \Rightarrow & e_{G} * e_{H} &= e_{G} * e_{G}\\
      \Rightarrow & e_{H} &= e_{G}\\
    \end{array}
    \]
  \end{proof}
\end{st}

\begin{st}
  \label{st:deelgroep-houdt-invers-ook-in}
  Zij $H$ een deelgroep van $G,*$, dan is elk invers element $x^{-1}$ van een element $x$ in $H$ ook het invers element van $x$ in $G$.

  \begin{proof}
    Noem $e_{G}$ het neutraal element van $G,*$ en $e_{h}$ dat van $H,*$.
    Noem het invers van een element $x$ in $G$ $x^{-1}$ en het invers van $x$ in $H$ $\bar{x}$.
    \[
    \begin{array}{rll}
      x * \bar x &= e_{H} &\\
                 &= e_{G} &\\
                 &= x * x^{-1} &\\
    \end{array}
    \Rightarrow \bar x = x^{-1}
    \]
  \end{proof}
\end{st}

\begin{st}
  Het \term{criterium van een deelgroep}.\\
  Zij $G,*$ een groep, en $H$ een deelverzameling van $G$.
  $H$ is een deelgroep van $G$ als en slechts als aan de volgende voorwaarden voldaan is.
  \begin{enumerate}
  \item $e_{G} \in H$
  \item $\forall x,y \in H: x * y \in H$
  \item $\forall x \in H: x^{-1} \in H$
  \end{enumerate}

  \begin{proof}
    Bewijs van een equivalentie.
    \begin{itemize}
    \item $\Rightarrow$\\
      Als $H$ een deelgroep is van $G$, dan gelden de voorwaarden al omdat $H$ zelf een groep is.\footnote{Zie bovendien stelling \ref{st:deelgroep-zelfde-neutraal-element}.}
    \item $\Leftarrow$\\
      Stel dat de voorwaarden voldaan zijn. Vanwege voorwaarde twee is de beperking van $*$ tot $H$ alvast een interne bewerking in $H$.
      \[ *: H \times H \rightarrow H: (x,y) \mapsto x*y \]
      \begin{itemize}
      \item associativiteit\\
      Deze bewerking is associatief in $G$, dus ook in $H$.
      \item Neutraal element\\
      Vanwege de eerste voorwaarde is $e_{G}$ ook een neutraal element van $H$.
      \item Inverse\\
      Elk element $x$ in $H$ heeft bovendien ook een invers in $H$ volgens de derde voorwaarde.
      \end{itemize}
    \end{itemize}
  \end{proof}
\end{st}

\begin{st}
  \term{alternatieve criteria}.\\
  We kunnen in het vorige criterium de volgende aanpassingen maken.
  \begin{itemize}
  \item Vervang de eerste voorwaarde door voorwaarde $1'$:
    \[ H \neq \emptyset \]
  \item Vervang de tweede en derde voorwaarde samen door voorwaarde $4$:
      \[ \forall x,y \in H: x * y^{-1} \in H \]
  \end{itemize}

  \begin{proof}
    We bewijzen dat de voorwaarden die we vervangen equivalent zijn.
    \begin{itemize}
    \item $e_{G} \in H \Leftrightarrow H \neq \emptyset$.
      Als $e_{G}$ een element is van $H$, is $H$ natuurlijk niet leeg.
      Als $H$ niet leeg is, bestaat er een element $x$ in $H$.
      Vanwege de derde voorwaarde zit de inverse van dat element ook in $H$.
      Vanwege de tweede voorwaarde zit $x * x^{-1} = e_{G}$ ook in $H$.
    \item 
      \[ (\forall x,y \in H: x * y \in H) \wedge (\forall x \in H: x^{-1} \in H)  \Leftrightarrow \forall x,y \in H: x * y^{-1} \in H \]
      Als voorwaarde 2 en 3 gelden is het duidelijk dat voorwaarde 4 geldt.
      Als voorwaarde 4 geldt, kies dan $e_{G}$ voor $x$ in voorwaarde 4 om voorwaarde 3 te bekomen.
      \[ \forall y \in H: e_{G} * y^{-1} = y^{-1} \in H \]
      Kies nu de inverse $z^{-1}$ van een willekeurig element $x$ in $H$ voor $y$ in voorwaarde 4 om voorwaarde 2 te bekomen.
      \[ \forall x, z \in H: x * (z^{-1})^{-1} = x * z \in H \]
    \end{itemize}
  \end{proof}
\end{st}

\begin{st}
  Zij $G$ een verzamelinge met een bewerking $*$ die voldoet aan de volgende voorwaarden.
  \begin{itemize}
  \item $*$ is associatief
  \item er bestaat een $e$ in $G$ waarvoor geldt $\forall x \in G: x * e = x$
  \item voor elk element $e$ dat voldoet aan de vorige voorwaarde:
    \[ \forall x \in G, \exists y \in G: x * y = e \]
  \end{itemize}
  $G,*$ is dan een groep.

  \begin{proof}
    Om te bewijzen dat $G,*$ een groep is, moeten we nog bewijzen dat er een neutraal element bestaat in $G$ en dat elk element een inverse heeft in $G$.
\TODO{ voor doorbijters: Bewijs }
  \end{proof}
\end{st}

\subsection{Morfismen}
\label{sec:morfismen}

\begin{de}
  \label{de:groepsmorfisme}
  Zij $G,*$ en $H,\Box$ groepen.
  Een (groeps)(homo)\term{morfisme} $f$ is een morfisme\footnote{Zie definitie \ref{de:morfisme}.} tussen twee groepen $G,*$ en $H,\Box$.
  \[ \forall x,y \in G: f(x*y) = f(x) \Box f(y) \]
\end{de}

\begin{de}
  Zij $f: G \rightarrow H$ een groepsmorfisme. De \term{kern} $Ker f$ wordt gedefinieerd als volgt.
  \[ Ker f = \{ x \in G \ |\ f(x) = e_{H} \} \]
\end{de}

\begin{de}
  Zij $f: G \rightarrow H$ een groepsmorfisme. Het \term{beeld} $Im f$ wordt gedefinieerd als volgt.
  \[ Im f = f(G) = \{ f(u) \ |\ u \in G \} \]
\end{de}


\begin{st}
  \label{st:groepsmorfisme-behoudt-neutraal-element}
  Zij $G,*$ en $H,\Box$ groepen met een morfisme $f: G \rightarrow H$.
  \[ e_{H} = f(e_{G}) \]

  \begin{proof}
    Beschouw de neutrale elementen $e_{G}$ en $e_{H}$ in de groepen.
    Begin bij de definite van een groepsmorfisme.\footnote{Zie definitie \ref{de:groepsmorfisme}.}
    \[ f(e_{G}*e_{G}) = f(e_{G})*f(e_{G}) \]
    $e_{G}$ is het neutraal element in $G$. $e_{G}*e_{G}$ is dus opnieuw $G$.
    \[ f(e_{G}) = f(e_{G})*f(e_{G}) \]
    Voeg links $e_{H}$ toe. Dit mag omdat $e_{H}$ het neutraal element is in $H$.
    \[ f(e_{G})*e_{H} = f(e_{G})*f(e_{G}) \]
    Schrap tenslotte $f(e_{G})$ aan beide kanten.
    \[ e_{H} = f(e_{G}) \]
  \end{proof}
\end{st}

\begin{st}
  \label{st:groepsmorfisme-behoudt-inverse}
  Zij $G,*$ en $H,\Box$ groepen met een morfisme $f: G \rightarrow H$.
  \[ \forall x \in G: f(x^{-1}) = (f(x))^{-1} \]

  \begin{proof}
    Kies een willekeurig element $x$ in $G$.
    Nu geldt het volgende:
    \[
    \begin{array}{rl}
    f(x) * f(x^{-1}) &= f(x*x^{-1})\\
                    &= f(e_{G})\\
                    &= e_{H}
    \end{array}
    \]
    De eerste gelijkheid is precies de definitie van een groepsmorfisme.\footnote{Zie definitie \ref{de:groepsmorfisme}.}
    De tweede gelijkheid volgt uit de definitie van de inverse van een element van een groep.\footnote{Zie definitie \ref{de:groep} puntje 3.}
    De laatste gelijkheid geldt omdat een groepsmorfisme het neutraal element behoudt.\footnote{Zie stelling \ref{st:groepsmorfisme-behoudt-neutraal-element}.}
    Wat we bekomen is de definitie van het neutraal element $f(x^{-1})$ van $f(x)$.
  \end{proof}
\end{st}

\begin{st}
  \label{st:beeld-is-deelgroep}
  Zij $G,*$ en $H,\Box$ groepen met een morfisme $f: G \rightarrow H$.
  \[ Im(f) \text{ is een deelgroep van } H \]

  \begin{proof}
    We bewijzen elke voorwaarde uit het criterium voor deelgroepen.
    \begin{enumerate}
    \item $e_{H} \in Im(f)$\\
      Inderdaad!\footnote{Zie stelling \ref{st:groepsmorfisme-behoudt-neutraal-element}.}
    \item $\forall x,y \in Im(f): x \Box y \in Im(f)$\\
      Kies twee elementen $f(x)$ en $f(y)$ in $Im(f)$, nu bestaan er dus twee elementen $x$ en $y$ in $G$.
      In $G$ is de bewerking $*$ intern.\footnote{Zie definitie \ref{de:groep}}.
      Kijk nu naar de definitie van een groepsmorfisme.\footnote{Zie definitie \ref{de:groepsmorfisme}.}
      \[ f(x*y) = f(x) \Box f(y) \]
      $f(x) \Box f(y)$ is dus een element van $Im(f)$.
    \item $\forall x \in Im(f): x^{-1} \in Im(f)$\\
      Kies een element $f(x)$ in $Im(f)$, er bestaat er dus een element $x$ in $G$.
      Nu is de inverse van $f(x)$ precies $f(x^{-1})$.\footnote{Zie stelling \ref{st:groepsmorfisme-behoudt-inverse}.}
    \end{enumerate}
  \end{proof}
\end{st}

\begin{st}
  \label{st:kern-is-deelgroep}
  Zij $G,*$ en $H,\Box$ groepen met een morfisme $f: G \rightarrow H$.
  \[ Ker(f) \text{ is een deelgroep van } G \]
  \begin{proof}
    We bewijzen elke voorwaarde uit het criterium voor deelgroepen.
    \begin{enumerate}
    \item $e_{H} \in Ker(f)$\\
      Inderdaad!\footnote{Zie stelling \ref{st:groepsmorfisme-behoudt-neutraal-element}.}
    \item $\forall x,y \in Ker(f): x * y \in Ker (f)$\\
      Kies twee willekeurige elementen $x$ en $y$ in de kern $Ker(f)$ van $f$.
      Nu geldt het volgende.
      \[
      \begin{array}{rl}
      f(x * y) &= f(x) \Box f(y)\\
               &= e_{H} \Box e_{H}\\
               &= e_{H}
      \end{array}
      \]
      $x * y$ zit dus in $Ker (f)$ voor elke $x$ en $y$.
    \item $\forall x \in Ker(f): x^{-1} \in Ker (f)$\\
      Kies een willekeurig element $x$ in de kern $Ker(f)$ van $f$.
      Nu geld het volgende.
      \[
      \begin{array}{rl}
      f(x^{-1}) &= (f(x))^{-1}\\
               &= e_{H}^{-1}\\
               &= e_{H}
      \end{array}
      \]
      $x^{-1}$ zit dus in $Ker (f)$ voor elke $x$.
    \end{enumerate}
  \end{proof}
\end{st}

\begin{st}
  Zij $G,*$ en $H,\Box$ groepen met een morfisme $f: G \rightarrow H$.
  \[ Ker(f) = \{e_{G}\}\Leftrightarrow f \text{ is injectief} \]

  \begin{proof}
    Bewijs van een equivalentie.
  \[ \forall x,y \in G: f(x*y) = f(x) \Box f(y) \]
    \begin{itemize}
    \item $\Rightarrow$\\
      Bewijs uit het ongerijmde: Stel dat er twee verschillende elementen $x$ en $y$ in $G$ zitten die door $x$ op hetzelfde element $f(x) = f(y) \in H$ afgebeeldt worden.
      \[ f(x*y) = f(x) \Box f(y) = f(x) \Box f(x) \]
      \[ f(y) = f(x) \]
      Contradictie.
    \item $\Leftarrow$\\
      Bewijs door contrapositie: Als de kern van $f$ niet triviaal is, dan bestaan er minstens twee verschillende elementen in $G$ die door $f$ op $e_{H}$ afgebeeldt worden en is $f$ dus niet injectief.
    \end{itemize}
  \end{proof}
\end{st}

\begin{st}
  Zij $G,*$ en $H,\Box$ groepen met een morfisme $f: G \rightarrow H$.
  \[ f \text{ is een isomorfisme} \Rightarrow f^{-1} \text{ is een isomorfisme} \]
  Merk op dat de afbeelding $f^{-1}$ slechts bestaat als $f$ een injectie is.
  \begin{proof}
    $f^{-1}$ is een morfisme:
    \[ 
    \begin{array}{rll}
      f^{-1}(y_{1} \Box y_{2}) &= f^{-1}(f(x_{1}) \Box f(x_{2})) &\\
                             &= f^{-1}(f(x_{1} * x_{2})) &\\
                             &= x_{1} * x_{2} &= f^{-1}(y_{1}) \Box f^{-1}(y_{2})
    \end{array}
    \]
    $f^{-1}$ is bovendien bijectief, want $f$ is bijectief.
    \TODO{bewijs in het hoofdstuk over afbeeldingen.}
  \end{proof}
\end{st}

\begin{st}
  Zij $G,*$ en $H,\Box$ groepen met een morfisme $f: G \rightarrow H$.
  Als een verzameling $A$ een deelgroep is van $G$, dan is $f(A)$ een deelgroep van $H$.

  \begin{proof}
    We gaan elke voorwaarde in het criterium van een deelgroep af.
    \begin{itemize}
    \item $e_{f(A)} \in H$.\\
      $A$ is een deelgroep van $G$, dus geldt $e_{A}\in G$.
      Bovendien wordt $e_{A} = e_{G}$ afgebeeldt op $e_{H} = e_{f(A)}$.\footnote{Zie stelling \ref{st:groepsmorfisme-behoudt-neutraal-element}.}
      $e_{f(A)}$ zit dus wel degelijk in $H$.
    \item $\forall x,y \in f(A): x \Box y \in f(A)$\\
      Kies twee elementen $f(x)$ en $f(y)$ in $f(A)$, nu bestaan er dus twee elementen $x$ en $y$ in $A$.
      In $A$ is de bewerking $*$ intern.\footnote{Zie definitie \ref{de:groep}.}
      Kijk nu naar de definitie van een groepsmorfisme.\footnote{Zie definitie \ref{de:groepsmorfisme}.}
      \[ f(x * y) = f(x) \Box f(y) \]
      $f(x) \Box f(y)$ is dus een element van $f(A)$.
    \item $\forall x \in f(A): x^{-1} \in f(A)$\\
      Kies een element $f(x)$ in $f(A)$, er bestaat er dus een element $x$ in $A$.
      Nu is de inverse van $f(x)$ precies $f(x^{-1})$.\footnote{Zie stelling \ref{st:groepsmorfisme-behoudt-inverse}.}
    \end{itemize}
  \end{proof}
  Merk op dat deze stelling een algemener geval is van stelling \ref{st:beeld-is-deelgroep}.
\end{st}

\begin{st}
  Zij $G,*$ en $H,\Box$ groepen met een morfisme $f: G \rightarrow H$.
  Als een verzameling $B$ een deelgroep is van $H$, dan is $f^{-1}(B)$ een deelgroep van $G$.

  \begin{proof}
    We gaan elke voorwaarde in het criterium van een deelgroep af.
    \begin{itemize}
    \item $e_{f^{-1}(B)} \in G$.\\
      Omdat $e_{G} = e_{B}$ geldt\footnote{Zie stelling \ref{st:deelgroep-zelfde-neutraal-element}.}, geldt ook $f(e_{B}) = e_{H}$.
      Bijgevolg geldt ook $e_{f^{-1}(B)} = e_{B} \in G$. 
    \item $\forall x,y \in f^{-1}(B): x \Box y \in f^{-1}(B) $\\
      Kies twee willekeurige elementen $a$ en $b$ uit $f^{-1}(B)$.
      Dit houdt in dat er twee elementen $f(a)$ en $f(b)$ in $B$ bestaan.
      $f(a) \Box f(b) \in B$ geldt omdat $B$ een deelgroep is van $B$. \footnote{Zie definitie \ref{de:groep}.}
      Dit is bovendien gelijk aan $f(a * b) \in f^{-1}(B)$, dus zitten $a$ en $b$ beide in $f^{-1}(B)$.
    \item $\forall x \in f^{-1}(B): x^{-1} \in f^{-1}(B)$\\
      Kies een willekeurig element $a$ uit $f^{-1}(B)$.
      Dit houdt in dat er een element $f(a)$ in $B$ bestaat.
      Het inverse element van $f(a)$ is $(f(a)^{-1})$ en zit ook in B.\footnote{Zie stelling \ref{st:deelgroep-houdt-invers-ook-in}.}
      Er bestaat dus ook een element $a^{-1}$ dat bovendien in $f^{-1}(B)$ zit.\footnote{Zie stelling \ref{st:groepsmorfisme-behoudt-inverse}.}
    \end{itemize}
  \end{proof}
\end{st}


\begin{st}
  De verzamelingen van automorfismen $Aut G$, uitgeruist met de samenstellingsfunctie $\circ$ vormt een groep.

  \begin{proof}
    We bewijzen elk deel van de definitie appart.\footnote{Zie definitie \ref{de:groep}.}
    \begin{itemize}
    \item associativiteit
      \[ \forall x, y, z \in Aut G: (x \circ y) \circ z = x \circ (y \circ z) \] 
      De samenstelling van afbeeldingen is inderdaad associatief.\footnote{Zie stelling \ref{st:samenstelling-relaties-associatief}.}
    \item neutraal element
      \[ \forall x \in Aut G: x \circ e = e = e \circ x \]
      Er bestaat een neutraal element voor $Aut G$, namelijk $Id_{G}$.\footnote{Zie definitie \ref{identieke-transformatie}.}
    \item inverse
      \[ \forall x \in Aut G, \exists x' \in Aut G:\ x \circ x' = e = x' \circ x \]
      Kies een willekeurige $x$ in $G$. Er bestaat nu wel degelijk een inverse afbieelding, precies omdat $x$ een bijectie is.
    \end{itemize}
  \end{proof}
\end{st}

\begin{st}
  Zij $G,*$ en $H,\Box$ groepen en $\alpha: G \rightarrow H$ een morfisme.
  $\alpha$ is een isomorfisme als en slechts als er een morfisme $\beta: H \rightarrow G$ bestaat zodat $\beta \circ \alpha = Id_{G}$ en $\alpha \circ \beta = Id_{G}$ gelden.

  \begin{proof}
    Bewijs van een equivalentie.
    Zij $G,*$ en $H,\Box$ willekeurige groepen. 
    \begin{itemize}
    \item $\Rightarrow$\\
      Zij $\alpha: G \rightarrow H$ een groepsisomorfisme.
      $\alpha$ is een bijectie, dus $\alpha^{-1}$ is goed gedefinieerd.
      Noem $\alpha^{-1}$ nu $\beta$, dan gelden $\beta \circ \alpha = Id_{G}$ en $\alpha \circ \beta = Id_{G}$.\footnote{Zie stelling \ref{st:afb+inverse=identieke}.}
    \item $\Leftarrow$\\
      Zij $\alpha: G \rightarrow H$ $\beta: H \rightarrow G$ morfismes, zodat $\beta \circ \alpha = Id_{G}$ en $\alpha \circ \beta = Id_{G}$ gelden.
      Volgens $\beta$ is nu de inverse van $\alpha$.\footnote{Zie stelling \ref{st:afb+inverse=identieke}.}
      Omdat $\alpha$ een inverse heeft, is $\alpha$ bijectief en bijgevolg een isomorfisme.\footnote{Zie stelling \ref{st:afb-inverse-asa-bijectief}.}
    \end{itemize}
  \end{proof}
\end{st}


\subsection{Orde}
\label{sec:orde}

\begin{de}
  De \term{orde} $n$ van het element $x$ van een groep $G,*$ is de kleinste $n \in N_{0}$ waarvoor $x^{n} = e_{G}$ geldt, indien die bestaat en anders $\infty$.
\end{de}

\begin{de}
  De \term{orde} $|G|$ of $\#G$ van een groep $G$ is het aantal elementen van $G$.
\end{de}

\begin{ei}
  \label{ei:groep-eindige-orde-deelbaarheid}
  Zij $G,*$ een groep en $x \in G$ een element met een eindige orde $n$ in die groep.
  \[ \forall r,s \in \mathbb{Z}:\ (x^{s} = e \Leftrightarrow n | s) \wedge (x^{r} = x^{s} \Leftrightarrow n | r-s)\]
  \begin{proof}
    Bewijs van conjunctie.
    Kies willekeurige elementen $r$ en $s$ in $\mathbb{Z}$.
    \begin{itemize}
    \item $(x^{s} = e \Leftrightarrow n | s)$
      \begin{itemize}
      \item $\Rightarrow$\\
        Stel $x^{s} = e$ geldt.
        Deel nu $s$ euclidisch door $n$.\footnote{Zie stelling \ref{st:euclidische-deling}.}
        \[ s = nq + r \text{ met } 0 \le r < n \]
        Nu geldt het volgende:
        \[ x^{s} = x^{nq + r} = (x^{n})^{q} * x^{r} = e^{q} * x^{r} = x^{r} = e\]
        Omdat $n$ de kleinste waarde is waarvoor $x^{n}=e$ geldt en omdat $r$ tussen $0$ en $n$ ligt, besluiten we dat $r$ nul is.
        \[ s = nq \]
        $n$ is dus een deler van s.\footnote{Zie definitie \ref{de:deler}.}
      \item $\Leftarrow$\\
        Stel dat $n | s$ geldt, dan bestaat er een $q$ zodat volgende gelijkheid geldt.\footnote{Zie definitie \ref{de:deler}.}
        \[ s = nq \]
        We beschouwen nu $x^{s}$
        \[ x^{s} = x^{nq} = (x^{n})^{q} = e^{q} = e\]
      \end{itemize}
    \item $(x^{r} = x^{s} \Leftrightarrow n | r-s)$
      \begin{itemize}
      \item $\Rightarrow$\\
        Stel dat $x^{r} = x^{s}$ geldt.
        \[ x^{r} = x^{s} \Leftrightarrow x^{r-s} = e \]
        In het vorige deel van dit bewijs hebben we bewezin dat $n$ dan een deler is van $r-s$.
        \[ n|r-s \]
      \item $\Leftarrow$\\
        Stel dat $n | r-s$ geldt, dan geldt volgend deel \'e\'en van dit bewijs het volgende:
        \[ x^{r-s} = e \]
        Dit betekent precies dat $x^{r}$ en $x^{s}$ gelijk zijn.
        \[ \Rightarrow x^{r} = x^{s} \]
      \end{itemize}
    \end{itemize}
  \end{proof}
\end{ei}

\begin{de}
  Zij $x$ een element van een groep $G,*$, dan is $<x>$ de \term{groep} voortgebracht door $x$.
  \[ <x> = \{ x^{s}\ |\ s \in \mathbb{Z} \} \]
\end{de}

\begin{st}
  \label{st:voortbrenging-is-groep}
  Zij $x$ een element van een groep $G,*$.
  De `groep' $<x>$ voortgebracht door $x$ is wel degelijk een groep.

  \begin{proof}
    We bewijzen elke eigenschap van een groep voor $<x>$.\footnote{Zie definitie \ref{de:groep}.}
    \begin{itemize}
    \item Associativiteit\\
      Kies drie willekeurige elementen uit $<x>$, met andere woorden kies drie getallen $a$, $b$ en $c$ uit $mathbb{Z}$:
      \[
      \begin{array}{rll}
      (x^{a} * x^{b})* x^{c} &= x^{a+b} * x^{c} &\\
                           &= x^{a+b+c}       &\\
                           &= x^{a} * x^{b+c} &= x^{a} * (x^{b} * x^{c})
      \end{array}
      \]
    \item Neutraal element
      Kies willekeurig een element uit $<x>$, kies dus een $a\in \mathbb{Z}$.
      Het neutraal element voor $<x>$ is $x^{0}$:
      \[ x^{a} * x^{0} = x^{a+0} = x^{a} = x^{0+a} = x^{0} + x^{a} \]
    \item Invers element
      Kies een willekeurig element uit $<x>$, kies dus een $a \in \mathbb{Z}$.
      Het invers element van $x^{a}$ is nu $x^{-a}$.
      \[ x^{a}*x^{-a} = x^{a-a} = x^{0} = x^{-a+a} = x^{-a} * x^{a} \]
    \end{itemize}
  \end{proof}

\end{st}

\begin{st}
  \label{st:orde-van-generator-is-orde-van-groep}
  Zij $x$ een element in een groep $G,*$, dan is de orde $n$ van de groep voortgebracht door $x$ gelijk aan de orde $m$ van $x$ in $G$.
  \[ |x| = |<x>| \]

  \begin{proof}
    Bekijk de groep $<x>$.
    \[ <x> = \{ x^{n}, x,x^{2},\dotsc,x^{n-1}\} \]
    Er zitten precies $n$ elementen in $<x>$.
    Dat $m$ de orde is van $x$ in $G$ houdt het volgende in:
    \[ x^{m} = e \text{ en } \neg (\exists\ m')(m'<m \wedge x^{m'} = e) \]
    Alle machten van $x$ tot en met $x^{m}$ zijn dus verschillend.
    Die $m$ machten zijn precies de elementen van $<x>$.
  \end{proof}
\end{st}
 
\begin{de}
  \label{de:cyclische-groep}
  Een groep $G,*$ is een \term{cyclishe groep} als en slechts als er een element in $G$ bestaat dat $G$ voortbrengt.
  We noemen bovendien $x$ de \term{generator} van $G$.
\end{de}

\begin{ei}
  Elke oneindige cyclishe groep is isomorf met $\mathbb{Z},+$ en is dus aftelbaar.

  \begin{proof}
    We tonen dat er een bijectie $b$ bestaat tussen de willekeurige oneindige cyclische groep $G,*$ met generator $x$ en $\mathbb{Z},+$.
    \[ G = <x> = \{ x^{s}\ |\ s \in \mathbb{Z} \} \]
    Inderdaad, kies $b$ als volgt.
    \[ b:\ \mathbb{Z} \rightarrow G:\ s \mapsto x^{s} \]
  \end{proof}
\end{ei}

\begin{ei}
  Elke cyclishe groep van eindige orde $n \in \mathbb{N}$ is isomorf met $\mathbb{Z}_{n},+$.

  \begin{proof}
    We tonen dat er een bijectie $b$ bestaat tussen de willekeurige eindige cyclische groep $G,*$ met generator $x$ en orde $n$ en $\mathbb{Z},+$.
    \[ G = <x> = \{ x^{s}\ |\ s \in \mathbb{Z} \} \]
    Inderdaad, kies $b$ als volgt.
    \[ b:\ \mathbb{Z} \rightarrow G:\ [s]_{n} \mapsto x^{s} \]
  \end{proof}
\end{ei}

\begin{st}
  \label{st:deelgroep-van-cyclische-groep-is-cyclisch}
  Elke deelgroep $H$ van een cyclische groep $G,* = <x>$ is cyclish.
  Sterker nog: $H$ wordt voortgebracht door $x^{s}$ waarbij $s \in \mathbb{N}_{0 }$ het kleinste getal is waarvoor $x^{s} \in H$ geldt. 

  \begin{proof}
    Als $H = \{e\}$ geldt, dan is $H$ inderdaad cyclisch.
    Stel nu dat $H$ niet enkel het neutraal element bevat, dan bevat $H$ mistens nog de elementen $x^{m}$ en $x^{-m}$ met $m \in \mathbb{N}$.
    Het kan zijn dat er meerdere van die $m$'s bestaan, maar er bestaat er altijd een kleinste: $n$.
    \[ x^{n} \in H \]
    Voor elk element $x^{s}$ van $G$ geldt nu dat het in $H$ zit als $n$ een deler is van $s$.
    \[ x^{s} \in H \Leftrightarrow n|s \]
    Inderdaad, als $x^{s}$ in $H$ zit, dan kunnen we $s$ euclidisch delen door $n$\footnote{Zie stelling \ref{st:euclidische-deling}.}:
    \[ s = nq + r \text{ met } 0 \le r < n \]
    \[ x^{s} = x^{nq + r} \Rightarrow x^{r} = x^{s-nq} = x^{s}(x^{n})^{-q} \in H \]
    Omdat $n$ minimaal is, en $r$ tussen $0$ en $n$ zit, is $r$ gelijk aan nul, en $s$ dus deelbaar door $s$.
    Omgekeerd, wanneer $s$ deelbaar is door $n$ bestaat er een $q$ zodat volgende bewering geldt\footnote{Zie definitie \ref{de:deler}.}:
    \[ s = nq \]
    \[ x^{s} = x^{nq} = (x^{n})^{q} \in H \]
    Met andere woorden: elke macht van $x^{n} \in H$ zit opnieuw in $H$. Dit betekent precies dat $H$ cyclisch is.
  \end{proof}
\end{st}

\begin{st}
  \label{st:orde-element-cyclische-groep}
  Zij $G = <a>$ een cyclishe groep van eindige orde $n$ met $*$ als bewerking.
  De orde van $a^{k}$ is gelijk aan $\frac{n}{ggd(k,n)}$:
  \[ (a^{k})^{\frac{n}{ggd(k,n)}} = e\]
  
  \begin{proof}
    \[ (a^{k})^{\frac{n}{ggd(k,n)}} = (a^{n})^{\frac{k}{ggd(k,n)}} = e \]
    Nu moeten we nog bewijzen dat $\frac{n}{ggd(k,n)}$ minimaal is.
    Stel dus dat er en $0 < r < \frac{n}{ggd(k,n)}$ bestaat zodat $a^{kr}$ geldt.
    Bekijk nu de stelling van B\'ezout-Bachet.\footnote{Zie stelling \ref{st:bezout-bachet}.}
    Er bestaan een $\alpha$ en $beta$ zodat het volgende geldt:
    \[ ggd(k,n) = \alpha k + \beta n \]
    Nu bekijken we $a^{kr} = e$ opnieuw, en verheffen we beide kanten tot de macht $\alpha$ zodat er $\alpha k$ in de macht staat.
    \[ a^{kr\alpha} = a^{ggd(k,n)r - \beta nr} = a^{ggd(k,n)r}* a^{-\beta nr} = a^{ggd(k,n)r}\]
    Omdat $a^{ggd(k,n)r} = e$ geldt, is $n$ een deler van $ggd(k,n)r$.\footnote{Zie stelling \ref{ei:groep-eindige-orde-deelbaarheid}.}
    Dat houdt in dat $\frac{n}{d}$ een deler is van $r$, en dus kleiner.
    \TODO{bewijzen in hoofdstuk van deelbaarheid}
  \end{proof}

\end{st}

\begin{st}
  Zij $G = <a>$ een cyclishe groep van eindige orde $n$.
  $a^{k}$ is een generator van $G$ als en slechts als $ggd(k.n)$ gelijk is aan $1$.
  \[ <a^{k}> = G \Leftrightarrow ggd(k,n) = 1 \]

  \begin{proof}
     We weten dat de orde van $a^{k}$ in $G$ gelijk is aan $\frac{n}{ggd(k,n)}$.\footnote{Zie stelling \ref{st:orde-element-cyclische-groep}.}
     Als en slechts als $a^{k}$ een generator is voor $G$, dan is de orde van $a^{k}$ gelijk aan de orde van $G$.\footnote{Zie stelling \ref{st:orde-van-generator-is-orde-van-groep}.}
     \[ \frac{n}{ggd(k,n)} = n \Leftrightarrow ggd(k,n) = 1 \]
  \end{proof}
\end{st}

\begin{st}
  Zij $G,* = <a>$ een cyclishe groep van eindige orde $n$.
  Voor elke $positieve$ deler $m$ van $n$ geldt dat $G$ precies \'e\'en deelgroep heeft van orde $m$, namelijk $<a^{\frac{n}{m}}>$.
  \begin{proof}
    $<a^{\frac{n}{m}}>$ is een deelgroep\footnote{Zie stelling \ref{st:voortbrenging-is-groep}.} van $G$.
    Omdat $G$ cyclisch is, is $<a^{\frac{n}{m}}>$ ook cyclisch.\footnote{Zie stelling \ref{st:deelgroep-van-cyclische-groep-is-cyclisch}.} De orde van $<a^{\frac{n}{m}}>$ is bovendien gelijk aan $\frac{n}{ggd({\frac{n}{m},n)}}$.\footnote{Zie stelling \ref{st:orde-element-cyclische-groep}.}
    \[ \frac{n}{ggd({\frac{n}{m},n)}} = m \]
\TODO{Bewijs in hoofdstuk van deelbaarheid}
    We moeten nu dus nog bewijzen dat elke deelgroep $H$ van orde $m$ gelijk is aan $<a^{\frac{n}{m}}>$.
    Kies zo'n deelgroep $H$ van orde $m$. $H$ is nu zeker cyclisch.
    Noem de generator van $H$ $a^{k}$.
    \[ H = <a^{k}> \]
    We weten nu opnieuw dat $H$ orde $\frac{n}{ggd(k,n)}$ heeft.
    \[ m = \frac{n}{ggd(k,n)} = \frac{kgv(k,n)}{k} \]
\TODO{Bewijs in hoofdstuk van deelbaarheid}
    \[ k = \frac{kgv(k,n)}{m} \]
    $k$ is dus een veelvoud van $\frac{n}{m}$.
    Bijgevolg zit $a^{k}$ zeker in $<a^{k}>$.
    $H$ is nu dus een deel van $<a^{k}>$, dus zijn $H$ en $<a^{\frac{n}{m}}>$ gelijk vanwege hun gelijke orde.
  \end{proof}
\end{st}

\subsection{Nevenklassen}
\label{sec:nevenklassen}

\begin{de}
  \label{de:nevenklassen}
  Zij $G,*$ een groep en $H$ een deelgroep van $G$.
  \begin{itemize}
  \item De \term{linkse nevenklasse} $xH$ van $H$ in $G$ bepaald door $x$:
    \[ xH = \{ x * h\ |\ h \in H \} \]
  \item De \term{rechtse nevenklasse} $Hx$ van $H$ in $G$ bepaald door $x$:
    \[ Hx = \{ h * x\ |\ h \in H \} \]
  \end{itemize}
  De verzameling van linker nevenklassen van $H$ in $G$ noteren we als $G/H$.
  \[ G/H = \{ x*h\ |\ x \in G \} \]
\end{de}

\begin{ei}
  \label{ei:linker-nevenklasse-eig}
  Zij $G$ een groep en $H$ een deelgroep van $G$.
  \[ \forall a,b \in G:\ aH = bH \Leftrightarrow a \in bH \Leftrightarrow b^{-1}a \in H \]

  \begin{proof}
    Bewijs door circulaire implicaties.
    \begin{itemize}
    \item $\forall a,b \in G:\ aH = bH \Rightarrow a \in bH$\\
      $H$ is een groep, en bevat dus een neutraal element.\footnote{Zie definitie \ref{de:groep}.}
      $a$ is dus een element van $aH$, wat gelijk is aan $bH$, dus $a$ zit in $bH$.
    \item $\forall a,b \in G:\ a \in bH \Rightarrow b^{-1}a \in H$\\
      Ofwel zijn $b$ en $a$ gelijk, en dan is $b^{-1}a$ een element van $H$ omdat $H$ een groep is.
      Ofwel zijn $b$ en $a$ niet gelijk, en dan bestaat er dus een $c \in H$ waarvoor het volgende geldt:
      \[ b*c = a\]
      \[ b^{-1}*b*c = b^{-1}a\]
      \[ c = b^{-1}a \in H \]
    \item $\forall a,b \in G:\ b^{-1}a \in H\Rightarrow aH = bH$\\
      Noem $b^{-1}a$ $h_{0}$.
      \[ b^{-1}a = h_{0} \Rightarrow (a = bh_{0}) \text { en } b = a h_{0}^{-1} \]
      Nu geldt voor alle $h \in H$ het volgende:
      \[ a * h = bh_{0}*h \]
      $aH$ is dus al een deel van $bH$.
      Bovendien geldt, opnieuw voor alle $h \in H$ ook het volgende:
      \[ b * h = ah_{0}^{-1} \]
      $bH$ is dus ook een deel van $aH$.
      We besluiten dat $aH$ en $bH$ gelijk zijn.
   \end{itemize}
  \end{proof}
\end{ei}

\begin{ei}
  \label{ei:rechter-nevenklasse-eig}
  Zij $G$ een groep en $H$ een deelgroep van $G$.
  \[ \forall a,b \in G:\ Ha = Hb \Leftrightarrow a \in Hb \Leftrightarrow ab^{-1} \in H \]

\TODO{ bewijs analoog }
\end{ei}

\begin{ei}
  Zij $G,*$ een groep en $H$ een deelgroep van $G$.
  \[ \forall a \in G:\ aH = H \Leftrightarrow a \in H \Leftrightarrow Ha = H \]

  \begin{proof}
    Bewijs door circulaire implicaties.
    \begin{itemize}
    \item $\forall a \in G:\ aH = H \Rightarrow a \in H$\\
      $a$ is een element van $aH$. $aH$ en $H$ zijn gelijk, dus $a$ zit ook in $H$.
    \item $\forall a \in G:\ a \in H \Rightarrow Ha = H$\\
      $H$ is een deelgroep met bewerking $*$ die intern is in $H$.\footnote{Zie definitie \ref{de:groep}.}
      Wanneer we elk element in $H$ rechts bewerken met $a$, komen we telkens een element in $H$ uit.
      Omgekeerd kan elk element in $H$ geschreven worden als een ander element in $H$ rechts bewerkt met $a$.
    \item $\forall a \in G:\ Ha = H \Rightarrow aH = H$\\
      $a$ is een element van $Ha$. $Ha$ en $H$ zijn gelijk, dus $a$ zit ook in $H$.
      $H$ is een deelgroep met bewerking $*$ die intern is in $H$.\footnote{Zie definitie \ref{de:groep}.}
      Wanneer we elk element in $H$ links bewerken met $a$, komen we telkens een element in $H$ uit.
      Omgekeerd kan elk element in $H$ geschreven worden als een ander element in $H$ links bewerkt met $a$.
      $Ha$ en $aH$ zijn dus gelijk.
    \end{itemize}
  \end{proof}
\end{ei}

\begin{st}
  Zij $G$ een groep en $H$ een deelgroep van $G$, De linkse(/rechter) nevenklassen van $H$ vormen een partitie van $G$
  De relatie ``X en Y liggen in dezelfde linkernevenklasse van $H$ in $G$'' is een equivalentierelatie.

  \begin{proof}
    We bewijzen dat de linkse(/rechter) nevenklassen elke definierende eigenschap van een partitie heeft.
    \begin{itemize}
    \item Elke linker(/rechter) nevenklasse is niet leeg. Ze bevat altijd het element waarmee bewerkt wordt
    \item Elk element van $G$ behoort tot een linker(/rechter) nevenklasse.
      Elk element $x$ behoort al zeker tot de nevenklasse van $H$ bepaald door $x$ in $G$.
    \item Twee linker(/rechter) nevenklassen zijn ofwel gelijk ofwel disjunct.
      Ofwel zijn twee nevenklassen disjunct, ofwel bevatten hun doorsnede minstens \'e\'en element.
      We bewijzen dat als twee nevenklassen minstens \'e\'en element gemeenschapplijk hebben, dat ze dan gelijk zijn.
      Kies een willekeurige $x$ en $y$ uit $G$ zodat de doorsnede van de nevenklasse van $H$ bepaald door $x$ en door $y$ niet disjunct zijn.
      Er bestaat dan een element $z$ in $G$ dat in de doorsnede zit.
      Er bestaan dus elementen $h$ en $h'$ in $G$ die aan volgende bewering voldoen:
      \begin{itemize}
      \item Linker nevenklassen
        \[ z = x*h = y*h' \]
        Nu volgt dit:
        \[ y^{-1}*x = h'*h^{-1} \in H \]
        $xH$ en $yH$ zijn dus gelijk.\footnote{Zie eigenschap \ref{ei:linker-nevenklasse-eig}.}
      \item Rechter nevenklassen
        \[ z = h*x = h'*y \]
        Nu volgt dit:
        \[ y*x^{-1} = h'^{-1}h \in H \]
        $Hx$ en $Hy$ zijn dus gelijk.\footnote{Zie eigenschap \ref{ei:rechter-nevenklasse-eig}.}
      \end{itemize}
    \end{itemize}
    De relatie ``X en Y liggen in dezelfde linkernevenklasse van $H$ in $G$'' is dus een equivalentierelatie.\footnote{Zie stelling \ref{st:partitie-equivalentierelatie}.}
  \end{proof}
\end{st}

\begin{st}
  \label{st:nevenklassen-zelfde-orde}
  Zij $G,*$ een eindige groep en $H$ een deelgroep van $G,*$. Nu geldt voor elke $x\in G$ dat de orde van $H$ en zowel de linker als de rechter nevenklasse van $x$ in $G$ alledrie dezelfde orde.
  \[ |H| = |x*H| = |H*x| \]

  \begin{proof}
    We bewijzen dat er een bijectie $f: H \rightarrow xH$ bestaat tussen $H$ en $xH$.
    Beschouw de afbeelding $f$:
    \[ f:\ H\rightarrow xH:\ h \rightarrow x*h \]
    \begin{itemize}
    \item $f$ is injectief.\\
      Kies twee willekeurige elementen $h_{1}$ en $h_{2}$ uit $H$ waarvoor het beeld onder $f$ gelijk is.
      \[ 
      \begin{array}{rrl}
                  & f(h_{1})    &= f(h_{2})   \\ 
      \Rightarrow & xh_{1}      &= xh_{2}     \\
      \Rightarrow & x^{-1}xh_{1} &= x^{-1}xh_{2}\\
      \Rightarrow & h_{1}       &= h_{2}
      \end{array}
      \]
    \item $f$ is surjectief.\\
      Kies een willekeurig element $a$ uit $xH$, dan bestaat er dus een $h \in H$ zodat $a = x*h$ geldt.
    \end{itemize}
    Analoog kunnen we bewijven dat er een bijectie $f: H \rightarrow Hx$ bestaat.
  \end{proof}
\end{st}

\begin{st}
  \label{st:stelling-van-lagrange}
  \term{Stelling van Lagrange}:\\
  Zij $G,*$ een eindige groep en $H \subset G$ een deelgroep. Het aantal linkse (of rechtse) nevenklassen van $H$ in $G$ is dan gelijk aan:
  \[ [G:H] = \frac{|G|}{|H|} \]
  We noemen dit de index van $H$ in $G$.
  De orde van $H$ is bijgevolg een deler van de orde van $G$.

  \begin{proof}
    De linkernevenklassen van $H$ in $G$ vormen een partitie van $G$, dus we kunnen elementen $g_{1} = e,g_{2},\dotsc,g_{k} \in G$ kiezen zodat de nevenklassen die erdoor bepaald zijn samen $G$ vormen en onderling disjunct zijn.
    \[ G = g_{1}H \cup \dotsb \cup g_{k}H \]
    \[ \forall i,j:\ i \neq j \Rightarrow g_{i}H \cap g_{j}H = \emptyset \]
    Nu geldt dus dat de orde van $G$ $k$ keer de orde van $H$ is.\footnote{Zie stelling \ref{st:nevenklassen-zelfde-orde}.}
    \[ |G| = k \cdot |H|\]
    Hieruit volgt dat $|H|$ een deler is van $|G|$ en dat het aantal linker nevenklassen $k$ gelijk is aan $\frac{|G|}{|H|}$.
  \end{proof}
\end{st}

\begin{gev}
  \label{gev:orde-van-element-deelt-orde-van-groep}
  Zij $G,*$ een eindige groep en $x$ een element van $G$.
  De orde van $x$ in $G$ is een deler van $|G|$.
  \[ x^{|G|} = e_{G} \]

  \begin{proof}
    Beschouw de deelgroep\footnote{Zie stelling \ref{st:voortbrenging-is-groep}.} $<x>$ van $G$.
    Het aantal elementen van $<x>$ is precies de orde van $x$ in $G$.\footnote{Zie stelling \ref{st:orde-van-generator-is-orde-van-groep}.}
    De orde van een deelgroep is steeds een deler van de orde van de groep.\footnote{Zie stelling \ref{st:orde-van-generator-is-orde-van-groep}.}
  \end{proof}
\end{gev}

\begin{st}
  \label{st:priemgroep-is-cyclisch}
  Zij $G,*$ een groep waarvan de orde gelijk is aan een priemgetal $p$, dan is $G$ cyclisch.
  \begin{proof}
    Kies een willekeurig element $x$ uit $G$ en beschouw de deelgroep\footnote{Zie stelling \ref{st:voortbrenging-is-groep}.} $<x>$ voortgebracht door $x$ van $G$.
    De orde van $<x>$ is een deler van de orde van $G$.\footnote{Zie stelling \ref{gev:orde-van-element-deelt-orde-van-groep}.}
    $p$ heeft maar twee delers, namelijk $1$ en $p$, dus ofwel is $x$ gelijk aan het neutraal element, ofwel is $x$ een generator voon $G$ en is $G$ dus cyclisch.\footnote{Zie definitie \ref{de:cyclische-groep}.}
  \end{proof}
\end{st}

\begin{gev}
  Op isomorfisme na, bestaat er slechts \'e\'en groep voor elk priemgetal, waarbij dat getal de orde is van die groep.
  \begin{proof}
    Kies een priemgetal $p$ en twee groepen $G,*$ en $H,\Box$ met $p$ als orde.
    We weten al dat $G$ en $H$ cyclisch zijn.\footnote{Zie stelling \ref{st:priemgroep-is-cyclisch}.}
    Kies dus een generator $g$ voor $G$ en een generator $h$ voor $H$.
    Beschouw nu het volgende morfisme.
    \[ f:\ G \rightarrow H:\ g^{k} \mapsto h^{k} \]
    $f$ is een isomorfisme, dus $G$ en $H$ zijn isomorf.
  \end{proof}
\end{gev}

\begin{st}
  Zij $f: G \rightarrow H$ een morfisme tussen twee groepen $G,*$ en $H,\Box$. Voor alle $x,y$ in $G$ geldt het volgende:
  \[ f(x) = f(y) \Leftrightarrow x * Ker(f) = y * Ker(f) \]

  \begin{proof}
    \[ x * Ker(f) = y * Ker(f) \Leftrightarrow x^{-1} * y \in Ker(f) \]
    $Ker(f)$ is een deelgroep van $G$.\footnote{Zie stelling \ref{st:kern-is-deelgroep}.}
    Bovenstaande gelijkheid geldt voor elke linker neverklasse, dus ook voor $Ker(f)$.\footnote{Zie eigenschap \ref{ei:linker-nevenklasse-eig} van linker neverklassen.}
    Omdat $x^{-1} * y$ in de kern van $f$ zit is het beeld ervan $e_{H}$.
    \[ f(x^{-1} * y) = e_{H} \]
    $f$ is een morfisme\footnote{Zie definitie \ref{de:groepsmorfisme}.}:
    \[ f(x^{-1}) \Box f(y) = e_{H}\]
    Wat er nu over blijft is precies dat $f(x)$ en $f(y)$ gelijk zijn.
    We hebben in dit bewijs enkel equivalenties gebruikt. De omgekeerde redenering geldt dus ook.
  \end{proof}
\end{st}


\begin{st}
  Als $G$ een commutatieve groep is ,dan is de verzameling $G/H$ met als bewerking $\bar{*}$ vormt een groep.
  \[ \bar{*}: G/H \times G/H \rightarrow G/H: (x*H,y*H) \rightarrow (x*y)*H \]

\TODO{ bewijs str}
\TODO{voorwaarde afzwakken bij normaaldelers}
\end{st}


\subsection{Directe som}
\label{sec:directe-som}

\begin{de}
  Zij $G_{1},\dotsc,G_{n}$ groepen met bewerkingen $*_{1},\dotsc,*_{n}$, dan is de \term{directe som} van de groepen $G_{i}$, genoteerd als $G_{1} \oplus \dotsb \oplus G_{n}$ de verzameling $G$, voorzien van de bewerking $*$ waarbij het volgende geldt.
  \[
  G = G_{1} \times \dotsb \times G_{n}
  \]
  \[
  (g_{1},\dotsc,g_{n}) *(h_{1},\dotsc,h_{n}) = (g_{1} *_{1} h_{1}, \dotsc, g_{2} *_{2} h_{2})
  \]
\end{de}

\begin{st}
  De directe som van $n$ groepen $G_{1},\dotsc,G_{n}$ is een groep.

\TODO{ bewijs }
\end{st}

\begin{ei}
  Elke groep van orde $4$ is isomorf met $\mathbb{Z}_{4}$ of met de viergroep $V$.

\TODO{ bewijs }
\end{ei}

\begin{ei}
  Elke groep van orde $6$ is isomorf met $\mathbb{Z}_{6}$ of met $\mathcal{S}$. 

\TODO{ bewijs }
\end{ei}

\section{Permutatiegroepen}
\label{sec:permutatiegroepen}

\begin{de}
  Zij $V$ een verzameling.
  De verzameling van permutaties van $V$ noemen we $\mathcal{S}V$.
  De \term{symmetrische groep} van $V$ is de groep $\mathcal{S}V, \circ$ waarbij $\circ$ de samenstelling van afbeeldingen is.
\end{de}

\begin{de}
  Een \term{permutatiegroep} van $V$ is een deelgroep van $\mathcal{S}V,\circ$.
\end{de}

\begin{de}
  De \term{symmetrische groep} van graad $n$: $\mathcal{S}_{n}$ is de groep van permutaties van $\{1,\dotsc,n\}$,
\end{de}

\begin{st}
  $\mathcal{S}V,\circ$ is een groep.

\TODO{ bewijs }
\end{st}

\begin{st}
  De orde van $\mathcal{S}_{n}$ is $n!$.

\TODO{ bewijs }
\end{st}

\begin{st}
  \label{st:stelling-van-cayley}
  \term{Stelling van Cayley}:\\
  Elke groep is isomorf met een permutatiegroep.

\TODO{ bewijs }
\end{st}

\begin{gev}
  Elke eindige groep met $n$ elementen is isomorf met een deelgroep van $\mathcal{S}_{n}$.

\TODO{ bewijs }
\end{gev}

\end{document}
