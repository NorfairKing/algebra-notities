\documentclass[main.tex]{subfiles}
\begin{document}

\chapter{Groepen}
\label{cha:groepen}

\section{Basisbegrippen}
\label{sec:basisbegrippen}

\subsection{De groep}
\label{sec:de-groep}

\begin{de}
  Een \term{halfgroep} $G,*$ is een algebra\footnote{Zie definitie \ref{de:algebra}.} die bestaat uit een (niet-lege) verzameling $G$ en een afbeelding $*$ (De bewerking).
  \[ *: G \times G \rightarrow: (x,y) \mapsto x * y \]
  De bewerking $*$ is associatief.
\end{de}

\begin{de}
  Een \term{mono\"ide} is een halfgroep $G,*$ met een neutraal element $e_{G}$.
\end{de}

\TODO{ cyclische mono\"ide }

\begin{de}
  \label{de:groep}
  Een \term{groep} $G,*$ is een mono\"ide waarin elk element symmetriseerbaar is.
  \[ \forall x \in G, \exists x' \in G:\ x * x' = e = x' * x \]
\end{de}

\begin{st}
  \label{st:groep-uniek-invers-element}
  De inverse $x^{-1}$ van een element $x$ van een groep $G$ is uniek.
  
  \begin{proof}
    Bewijs uit het ongerijmde.\\
    Stel dat er twee verschillende inversen $y$ en $z$ zijn van $x$ in $G$.
    \[  
    \begin{array}{rl}
      y &= y * e_{G}\\
        &= y * (x * z)\\
        &= (y * x) * z\\
        &= e_{G} * z\\
        &= z
    \end{array}
    \]
    De derde gelijkheid geldt omdat de bewerking $*$ associatief is.\footnote{Zie de definitie van een groep (Definitie \ref{de:groep}).} De vierde gelijkheid geldt omdat het neutraal element van een groep uniek is.\footnote{Zie stelling \ref{st:neutraal-element-uniek}.}
  \end{proof}
\end{st}

\begin{st}
  Zij $G,*$ een groep.
  \[ (a*b)^{-1} = b^{-1} * a^{-1} \]
\extra{bewijs}
\end{st}

\begin{de}
  Een \term{commutatieve groep} of \term{abelse groep} $G,*$ is een groep waarbij de bewerking $*$ commutatief is.
  \[ \forall x,y \in G: x * y = y * x\]
\end{de}

\begin{de}
  Zij $G,*$ een groep en $H$ een (niet-lege) deelverzameling van $G$. We noemen $H$ een \term{deelgroep} van $G$ als $H$ zelf ook een groep is met dezelfde bewerking $*$.
  Met andere woorden: ``Een deelgroep is een deelalgebra die ook een groep is''.
\end{de}

\begin{st}
  \label{st:deelgroep-zelfde-neutraal-element}
  Zij $H$ een deelgroep van $G,*$, dan is $e_{G}$ ook het neutraal element van $H$.

  \begin{proof}
    Noem $e_{G}$ het neutraal element van $G,*$ en $e_{h}$ dat van $H,*$. Noem het invers van een element $x$ in $G$ $x^{-1}$ en het invers van $x$ in $H$ $\bar{x}$.
    \[
    \begin{array}{rrl}
                  & e_{H} * e_{H} &= e_{H} * e_{G}\\
      \Rightarrow & e^{-1}_{H} * (e_{H} * e_{H}) &= e^{-1}_{H} * (e_{H} * e_{G})\\
      \Rightarrow & (e^{-1}_{H} * e_{H}) * e_{H} &= (e^{-1}_{H} * e_{H}) * e_{G}\\
      \Rightarrow & e_{G} * e_{H} &= e_{G} * e_{G}\\
      \Rightarrow & e_{H} &= e_{G}\\
    \end{array}
    \]
  \end{proof}
\end{st}

\begin{st}
  \label{st:deelgroep-houdt-invers-ook-in}
  Zij $H$ een deelgroep van $G,*$, dan is elk invers element $x^{-1}$ van een element $x$ in $H$ ook het invers element van $x$ in $G$.

  \begin{proof}
    Noem $e_{G}$ het neutraal element van $G,*$ en $e_{h}$ dat van $H,*$.
    Noem het invers van een element $x$ in $G$ $x^{-1}$ en het invers van $x$ in $H$ $\bar{x}$.
    \[
    \begin{array}{rll}
      x * \bar x &= e_{H} &\\
                 &= e_{G} &\\
                 &= x * x^{-1} &\\
    \end{array}
    \Rightarrow \bar x = x^{-1}
    \]
  \end{proof}
\end{st}

\begin{de}
  Zij $G,*$ een groep met neutraal element $e$, dan noemen we $\{e\},*$ en $G,*$ \term{triviale deelgroepen} van $G,*$.
\end{de}

\begin{st}
  \label{st:criterium-deelgroep}
  Het \term{criterium van een deelgroep}.\\
  Zij $G,*$ een groep, en $H$ een deelverzameling van $G$.
  $H$ is een deelgroep van $G$ als en slechts als aan de volgende voorwaarden voldaan is.
  \begin{enumerate}
  \item $e_{G} \in H$
  \item $\forall x,y \in H: x * y \in H$
  \item $\forall x \in H: x^{-1} \in H$
  \end{enumerate}

  \begin{proof}
    Bewijs van een equivalentie.
    \begin{itemize}
    \item $\Rightarrow$\\
      Als $H$ een deelgroep is van $G$, dan gelden de voorwaarden al omdat $H$ zelf een groep is.\footnote{Zie bovendien stelling \ref{st:deelgroep-zelfde-neutraal-element}.}
    \item $\Leftarrow$\\
      Stel dat de voorwaarden voldaan zijn. Vanwege voorwaarde twee is de beperking van $*$ tot $H$ alvast een interne bewerking in $H$.
      \[ *: H \times H \rightarrow H: (x,y) \mapsto x*y \]
      \begin{itemize}
      \item associativiteit\\
      Deze bewerking is associatief in $G$, dus ook in $H$.
      \item Neutraal element\\
      Vanwege de eerste voorwaarde is $e_{G}$ ook een neutraal element van $H$.
      \item Inverse\\
      Elk element $x$ in $H$ heeft bovendien ook een invers in $H$ volgens de derde voorwaarde.
      \end{itemize}
    \end{itemize}
  \end{proof}
\end{st}
hopl

\begin{st}
  \term{alternatieve criteria}.\\
  We kunnen in het vorige criterium de volgende aanpassingen maken.
  \begin{itemize}
  \item Vervang de eerste voorwaarde door voorwaarde $1'$:
    \[ H \neq \emptyset \]
  \item Vervang de tweede en derde voorwaarde samen door voorwaarde $4$:
      \[ \forall x,y \in H: x * y^{-1} \in H \]
   \end{itemize}

  \begin{proof}
    We bewijzen dat de voorwaarden die we vervangen equivalent zijn.
    \begin{itemize}
    \item $e_{G} \in H \Leftrightarrow H \neq \emptyset$.
      Als $e_{G}$ een element is van $H$, is $H$ natuurlijk niet leeg.
      Als $H$ niet leeg is, bestaat er een element $x$ in $H$.
      Vanwege de derde voorwaarde zit de inverse van dat element ook in $H$.
      Vanwege de tweede voorwaarde zit $x * x^{-1} = e_{G}$ ook in $H$.
    \item 
      \[ (\forall x,y \in H: x * y \in H) \wedge (\forall x \in H: x^{-1} \in H)  \Leftrightarrow \forall x,y \in H: x * y^{-1} \in H \]
      Als voorwaarde 2 en 3 gelden is het duidelijk dat voorwaarde 4 geldt.
      Als voorwaarde 4 geldt, kies dan $e_{G}$ voor $x$ in voorwaarde 4 om voorwaarde 3 te bekomen.
      \[ \forall y \in H: e_{G} * y^{-1} = y^{-1} \in H \]
      Kies nu de inverse $z^{-1}$ van een willekeurig element $x$ in $H$ voor $y$ in voorwaarde 4 om voorwaarde 2 te bekomen.
      \[ \forall x, z \in H: x * (z^{-1})^{-1} = x * z \in H \]
    \end{itemize}
  \end{proof}
\end{st}

\begin{st}
  Zij $G$ een verzamelinge met een bewerking $*$ die voldoet aan de volgende voorwaarden.
  \begin{itemize}
  \item $*$ is associatief
  \item er bestaat een $e$ in $G$ waarvoor geldt $\forall x \in G: x * e = x$
  \item voor elk element $e$ dat voldoet aan de vorige voorwaarde:
    \[ \forall x \in G, \exists y \in G: x * y = e \]
  \end{itemize}
  $G,*$ is dan een groep.

  \begin{proof}
    Om te bewijzen dat $G,*$ een groep is, moeten we nog bewijzen dat er een neutraal element bestaat in $G$ en dat elk element een inverse heeft in $G$.
\TODO{ voor doorbijters: Bewijs }
  \end{proof}
\end{st}

\begin{st}
  Zij $G,*$ een groep waarop een equivalentierelatie $\sim$ is gedefinieerd, dan is de verzameling van alle elementen equivalent met het neutraal element als $\sim$ links-(of rechts-)verenigbaar is met $*$.
  
\TODO{bewijs p 106 tai}
\end{st}

\begin{st}
  \label{st:doorsnede-deelgroepen}
  De doorsneden van twee deelgroepen is opnieuw een deelgroep.
\extra{bewijs}
\end{st}

\subsection{Morfismen}
\label{sec:morfismen}

\begin{de}
  \label{de:groepsmorfisme}
  Zij $G,*$ en $H,\Box$ groepen.
  Een (groeps)(homo)\term{morfisme} $f$ is een morfisme\footnote{Zie definitie \ref{de:morfisme}.} tussen twee groepen $G,*$ en $H,\Box$.
  \[ \forall x,y \in G: f(x*y) = f(x) \Box f(y) \]
\end{de}

\begin{de}
  Zij $f: G \rightarrow H$ een groepsmorfisme. De \term{kern} $Ker f$ wordt gedefinieerd als volgt.
  \[ Ker f = \{ x \in G \ |\ f(x) = e_{H} \} \]
\end{de}

\begin{de}
  Zij $f: G \rightarrow H$ een groepsmorfisme. Het \term{beeld} $Im f$ wordt gedefinieerd als volgt.
  \[ Im f = f(G) = \{ f(u) \ |\ u \in G \} \]
\end{de}


\begin{st}
  \label{st:groepsmorfisme-behoudt-neutraal-element}
  Zij $G,*$ en $H,\Box$ groepen met een morfisme $f: G \rightarrow H$.
  \[ e_{H} = f(e_{G}) \]

  \begin{proof}
    Beschouw de neutrale elementen $e_{G}$ en $e_{H}$ in de groepen.
    Begin bij de definite van een groepsmorfisme.\footnote{Zie definitie \ref{de:groepsmorfisme}.}
    \[ f(e_{G}*e_{G}) = f(e_{G})*f(e_{G}) \]
    $e_{G}$ is het neutraal element in $G$. $e_{G}*e_{G}$ is dus opnieuw $G$.
    \[ f(e_{G}) = f(e_{G})*f(e_{G}) \]
    Voeg links $e_{H}$ toe. Dit mag omdat $e_{H}$ het neutraal element is in $H$.
    \[ f(e_{G})*e_{H} = f(e_{G})*f(e_{G}) \]
    Schrap tenslotte $f(e_{G})$ aan beide kanten.
    \[ e_{H} = f(e_{G}) \]
  \end{proof}
\end{st}

\begin{st}
  \label{st:groepsmorfisme-behoudt-inverse}
  Zij $G,*$ en $H,\Box$ groepen met een morfisme $f: G \rightarrow H$.
  \[ \forall x \in G: f(x^{-1}) = (f(x))^{-1} \]

  \begin{proof}
    Kies een willekeurig element $x$ in $G$.
    Nu geldt het volgende:
    \[
    \begin{array}{rl}
    f(x) * f(x^{-1}) &= f(x*x^{-1})\\
                    &= f(e_{G})\\
                    &= e_{H}
    \end{array}
    \]
    De eerste gelijkheid is precies de definitie van een groepsmorfisme.\footnote{Zie definitie \ref{de:groepsmorfisme}.}
    De tweede gelijkheid volgt uit de definitie van de inverse van een element van een groep.\footnote{Zie definitie \ref{de:groep} puntje 3.}
    De laatste gelijkheid geldt omdat een groepsmorfisme het neutraal element behoudt.\footnote{Zie stelling \ref{st:groepsmorfisme-behoudt-neutraal-element}.}
    Wat we bekomen is de definitie van het neutraal element $f(x^{-1})$ van $f(x)$.
  \end{proof}
\end{st}

\begin{st}
  \label{st:beeld-is-deelgroep}
  Zij $G,*$ en $H,\Box$ groepen met een morfisme $f: G \rightarrow H$.
  \[ Im(f) \text{ is een deelgroep van } H \]

  \begin{proof}
    We bewijzen elke voorwaarde uit het criterium voor deelgroepen.
    \begin{enumerate}
    \item $e_{H} \in Im(f)$\\
      Inderdaad!\footnote{Zie stelling \ref{st:groepsmorfisme-behoudt-neutraal-element}.}
    \item $\forall x,y \in Im(f): x \Box y \in Im(f)$\\
      Kies twee elementen $f(x)$ en $f(y)$ in $Im(f)$, nu bestaan er dus twee elementen $x$ en $y$ in $G$.
      In $G$ is de bewerking $*$ intern.\footnote{Zie definitie \ref{de:groep}}.
      Kijk nu naar de definitie van een groepsmorfisme.\footnote{Zie definitie \ref{de:groepsmorfisme}.}
      \[ f(x*y) = f(x) \Box f(y) \]
      $f(x) \Box f(y)$ is dus een element van $Im(f)$.
    \item $\forall x \in Im(f): x^{-1} \in Im(f)$\\
      Kies een element $f(x)$ in $Im(f)$, er bestaat er dus een element $x$ in $G$.
      Nu is de inverse van $f(x)$ precies $f(x^{-1})$.\footnote{Zie stelling \ref{st:groepsmorfisme-behoudt-inverse}.}
    \end{enumerate}
  \end{proof}
\end{st}

\begin{st}
  \label{st:kern-is-deelgroep}
  Zij $G,*$ en $H,\Box$ groepen met een morfisme $f: G \rightarrow H$.
  \[ Ker(f) \text{ is een deelgroep van } G \]
  \begin{proof}
    We bewijzen elke voorwaarde uit het criterium voor deelgroepen.
    \begin{enumerate}
    \item $e_{H} \in Ker(f)$\\
      Inderdaad!\footnote{Zie stelling \ref{st:groepsmorfisme-behoudt-neutraal-element}.}
    \item $\forall x,y \in Ker(f): x * y \in Ker (f)$\\
      Kies twee willekeurige elementen $x$ en $y$ in de kern $Ker(f)$ van $f$.
      Nu geldt het volgende.
      \[
      \begin{array}{rl}
      f(x * y) &= f(x) \Box f(y)\\
               &= e_{H} \Box e_{H}\\
               &= e_{H}
      \end{array}
      \]
      $x * y$ zit dus in $Ker (f)$ voor elke $x$ en $y$.
    \item $\forall x \in Ker(f): x^{-1} \in Ker (f)$\\
      Kies een willekeurig element $x$ in de kern $Ker(f)$ van $f$.
      Nu geld het volgende.
      \[
      \begin{array}{rl}
      f(x^{-1}) &= (f(x))^{-1}\\
               &= e_{H}^{-1}\\
               &= e_{H}
      \end{array}
      \]
      $x^{-1}$ zit dus in $Ker (f)$ voor elke $x$.
    \end{enumerate}
  \end{proof}
\end{st}

\begin{st}
  \label{st:kern-triviaal-asa-morfisme-injectief}
  Zij $G,*$ en $H,\Box$ groepen met een morfisme $f: G \rightarrow H$.
  \[ Ker(f) = \{e_{G}\}\Leftrightarrow f \text{ is injectief} \]

  \begin{proof}
    Bewijs van een equivalentie.
  \[ \forall x,y \in G: f(x*y) = f(x) \Box f(y) \]
    \begin{itemize}
    \item $\Rightarrow$\\
      Bewijs uit het ongerijmde: Stel dat er twee verschillende elementen $x$ en $y$ in $G$ zitten die door $x$ op hetzelfde element $f(x) = f(y) \in H$ afgebeeldt worden.
      \[ f(x*y) = f(x) \Box f(y) = f(x) \Box f(x) \]
      \[ f(y) = f(x) \]
      Contradictie.
    \item $\Leftarrow$\\
      Bewijs door contrapositie: Als de kern van $f$ niet triviaal is, dan bestaan er minstens twee verschillende elementen in $G$ die door $f$ op $e_{H}$ afgebeeldt worden en is $f$ dus niet injectief.
    \end{itemize}
  \end{proof}
\end{st}

\begin{st}
  Zij $G,*$ en $H,\Box$ groepen met een morfisme $f: G \rightarrow H$.
  \[ f \text{ is een isomorfisme} \Rightarrow f^{-1} \text{ is een isomorfisme} \]
  Merk op dat de afbeelding $f^{-1}$ slechts bestaat als $f$ een injectie is.
  \begin{proof}
    $f^{-1}$ is een morfisme:
    \[ 
    \begin{array}{rll}
      f^{-1}(y_{1} \Box y_{2}) &= f^{-1}(f(x_{1}) \Box f(x_{2})) &\\
                             &= f^{-1}(f(x_{1} * x_{2})) &\\
                             &= x_{1} * x_{2} &= f^{-1}(y_{1}) \Box f^{-1}(y_{2})
    \end{array}
    \]
    $f^{-1}$ is bovendien bijectief, want $f$ is bijectief.
    \TODO{bewijs in het hoofdstuk over afbeeldingen.}
  \end{proof}
\end{st}

\begin{st}
  \label{st:fa-deelgroep-h}
  Zij $G,*$ en $H,\Box$ groepen met een morfisme $f: G \rightarrow H$.
  Als een verzameling $A$ een deelgroep is van $G$, dan is $f(A)$ een deelgroep van $H$.

  \begin{proof}
    We gaan elke voorwaarde in het criterium van een deelgroep af.
    \begin{itemize}
    \item $e_{f(A)} \in H$.\\
      $A$ is een deelgroep van $G$, dus geldt $e_{A}\in G$.
      Bovendien wordt $e_{A} = e_{G}$ afgebeeldt op $e_{H} = e_{f(A)}$.\footnote{Zie stelling \ref{st:groepsmorfisme-behoudt-neutraal-element}.}
      $e_{f(A)}$ zit dus wel degelijk in $H$.
    \item $\forall x,y \in f(A): x \Box y \in f(A)$\\
      Kies twee elementen $f(x)$ en $f(y)$ in $f(A)$, nu bestaan er dus twee elementen $x$ en $y$ in $A$.
      In $A$ is de bewerking $*$ intern.\footnote{Zie definitie \ref{de:groep}.}
      Kijk nu naar de definitie van een groepsmorfisme.\footnote{Zie definitie \ref{de:groepsmorfisme}.}
      \[ f(x * y) = f(x) \Box f(y) \]
      $f(x) \Box f(y)$ is dus een element van $f(A)$.
    \item $\forall x \in f(A): x^{-1} \in f(A)$\\
      Kies een element $f(x)$ in $f(A)$, er bestaat er dus een element $x$ in $A$.
      Nu is de inverse van $f(x)$ precies $f(x^{-1})$.\footnote{Zie stelling \ref{st:groepsmorfisme-behoudt-inverse}.}
    \end{itemize}
  \end{proof}
  Merk op dat deze stelling een algemener geval is van stelling \ref{st:beeld-is-deelgroep}.
\end{st}

\begin{st}
  \label{st:fbm-deelgroep-g}
  Zij $G,*$ en $H,\Box$ groepen met een morfisme $f: G \rightarrow H$.
  Als een verzameling $B$ een deelgroep is van $H$, dan is $f^{-1}(B)$ een deelgroep van $G$.

  \begin{proof}
    We gaan elke voorwaarde in het criterium van een deelgroep af.
    \begin{itemize}
    \item $e_{f^{-1}(B)} \in G$.\\
      Omdat $e_{G} = e_{B}$ geldt\footnote{Zie stelling \ref{st:deelgroep-zelfde-neutraal-element}.}, geldt ook $f(e_{B}) = e_{H}$.
      Bijgevolg geldt ook $e_{f^{-1}(B)} = e_{B} \in G$. 
    \item $\forall x,y \in f^{-1}(B): x \Box y \in f^{-1}(B) $\\
      Kies twee willekeurige elementen $a$ en $b$ uit $f^{-1}(B)$.
      Dit houdt in dat er twee elementen $f(a)$ en $f(b)$ in $B$ bestaan.
      $f(a) \Box f(b) \in B$ geldt omdat $B$ een deelgroep is van $B$. \footnote{Zie definitie \ref{de:groep}.}
      Dit is bovendien gelijk aan $f(a * b) \in f^{-1}(B)$, dus zitten $a$ en $b$ beide in $f^{-1}(B)$.
    \item $\forall x \in f^{-1}(B): x^{-1} \in f^{-1}(B)$\\
      Kies een willekeurig element $a$ uit $f^{-1}(B)$.
      Dit houdt in dat er een element $f(a)$ in $B$ bestaat.
      Het inverse element van $f(a)$ is $(f(a)^{-1})$ en zit ook in B.\footnote{Zie stelling \ref{st:deelgroep-houdt-invers-ook-in}.}
      Er bestaat dus ook een element $a^{-1}$ dat bovendien in $f^{-1}(B)$ zit.\footnote{Zie stelling \ref{st:groepsmorfisme-behoudt-inverse}.}
    \end{itemize}
  \end{proof}
\end{st}


\begin{st}
  De verzamelingen van automorfismen $Aut G$, uitgeruist met de samenstellingsfunctie $\circ$ vormt een groep.

  \begin{proof}
    We bewijzen elk deel van de definitie appart.\footnote{Zie definitie \ref{de:groep}.}
    \begin{itemize}
    \item associativiteit
      \[ \forall x, y, z \in Aut G: (x \circ y) \circ z = x \circ (y \circ z) \] 
      De samenstelling van afbeeldingen is inderdaad associatief.\footnote{Zie stelling \ref{st:samenstelling-relaties-associatief}.}
    \item neutraal element
      \[ \forall x \in Aut G: x \circ e = e = e \circ x \]
      Er bestaat een neutraal element voor $Aut G$, namelijk $Id_{G}$.\footnote{Zie definitie \ref{identieke-transformatie}.}
    \item inverse
      \[ \forall x \in Aut G, \exists x' \in Aut G:\ x \circ x' = e = x' \circ x \]
      Kies een willekeurige $x$ in $G$. Er bestaat nu wel degelijk een inverse afbieelding, precies omdat $x$ een bijectie is.
    \end{itemize}
  \end{proof}
\end{st}

\begin{st}
  Zij $G,*$ en $H,\Box$ groepen en $\alpha: G \rightarrow H$ een morfisme.
  $\alpha$ is een isomorfisme als en slechts als er een morfisme $\beta: H \rightarrow G$ bestaat zodat $\beta \circ \alpha = Id_{G}$ en $\alpha \circ \beta = Id_{G}$ gelden.

  \begin{proof}
    Bewijs van een equivalentie.
    Zij $G,*$ en $H,\Box$ willekeurige groepen. 
    \begin{itemize}
    \item $\Rightarrow$\\
      Zij $\alpha: G \rightarrow H$ een groepsisomorfisme.
      $\alpha$ is een bijectie, dus $\alpha^{-1}$ is goed gedefinieerd.
      Noem $\alpha^{-1}$ nu $\beta$, dan gelden $\beta \circ \alpha = Id_{G}$ en $\alpha \circ \beta = Id_{G}$.\footnote{Zie stelling \ref{st:afb+inverse=identieke}.}
    \item $\Leftarrow$\\
      Zij $\alpha: G \rightarrow H$ $\beta: H \rightarrow G$ morfismes, zodat $\beta \circ \alpha = Id_{G}$ en $\alpha \circ \beta = Id_{G}$ gelden.
      Volgens $\beta$ is nu de inverse van $\alpha$.\footnote{Zie stelling \ref{st:afb+inverse=identieke}.}
      Omdat $\alpha$ een inverse heeft, is $\alpha$ bijectief en bijgevolg een isomorfisme.\footnote{Zie stelling \ref{st:afb-inverse-asa-bijectief}.}
    \end{itemize}
  \end{proof}
\end{st}


\subsection{Orde}
\label{sec:orde}

\begin{de}
  De \term{orde} $n$ van het element $x$ van een groep $G,*$ is de kleinste $n \in N_{0}$ waarvoor $x^{n} = e_{G}$ geldt, indien die bestaat en anders $\infty$.
\end{de}

\begin{de}
  De \term{orde} $|G|$ of $\#G$ van een groep $G$ is het aantal elementen van $G$.
\end{de}

\begin{ei}
  \label{ei:groep-eindige-orde-deelbaarheid}
  Zij $G,*$ een groep en $x \in G$ een element met een eindige orde $n$ in die groep.
  \[ \forall r,s \in \mathbb{Z}:\ (x^{s} = e \Leftrightarrow n | s) \wedge (x^{r} = x^{s} \Leftrightarrow n | r-s)\]
  \begin{proof}
    Bewijs van conjunctie.
    Kies willekeurige elementen $r$ en $s$ in $\mathbb{Z}$.
    \begin{itemize}
    \item $(x^{s} = e \Leftrightarrow n | s)$
      \begin{itemize}
      \item $\Rightarrow$\\
        Stel $x^{s} = e$ geldt.
        Deel nu $s$ euclidisch door $n$.\footnote{Zie stelling \ref{st:euclidische-deling}.}
        \[ s = nq + r \text{ met } 0 \le r < n \]
        Nu geldt het volgende:
        \[ x^{s} = x^{nq + r} = (x^{n})^{q} * x^{r} = e^{q} * x^{r} = x^{r} = e\]
        Omdat $n$ de kleinste waarde is waarvoor $x^{n}=e$ geldt en omdat $r$ tussen $0$ en $n$ ligt, besluiten we dat $r$ nul is.
        \[ s = nq \]
        $n$ is dus een deler van s.\footnote{Zie definitie \ref{de:deler}.}
      \item $\Leftarrow$\\
        Stel dat $n | s$ geldt, dan bestaat er een $q$ zodat volgende gelijkheid geldt.\footnote{Zie definitie \ref{de:deler}.}
        \[ s = nq \]
        We beschouwen nu $x^{s}$
        \[ x^{s} = x^{nq} = (x^{n})^{q} = e^{q} = e\]
      \end{itemize}
    \item $(x^{r} = x^{s} \Leftrightarrow n | r-s)$
      \begin{itemize}
      \item $\Rightarrow$\\
        Stel dat $x^{r} = x^{s}$ geldt.
        \[ x^{r} = x^{s} \Leftrightarrow x^{r-s} = e \]
        In het vorige deel van dit bewijs hebben we bewezin dat $n$ dan een deler is van $r-s$.
        \[ n|r-s \]
      \item $\Leftarrow$\\
        Stel dat $n | r-s$ geldt, dan geldt volgend deel \'e\'en van dit bewijs het volgende:
        \[ x^{r-s} = e \]
        Dit betekent precies dat $x^{r}$ en $x^{s}$ gelijk zijn.
        \[ \Rightarrow x^{r} = x^{s} \]
      \end{itemize}
    \end{itemize}
  \end{proof}
\end{ei}

\begin{de}
  Zij $x$ een element van een groep $G,*$, dan is $<x>$ de \term{groep} voortgebracht door $x$.
  \[ <x> = \{ x^{s}\ |\ s \in \mathbb{Z} \} \]
\end{de}

\begin{st}
  \label{st:voortbrenging-is-groep}
  Zij $x$ een element van een groep $G,*$.
  De `groep' $<x>$ voortgebracht door $x$ is wel degelijk een groep.

  \begin{proof}
    We bewijzen elke eigenschap van een groep voor $<x>$.\footnote{Zie definitie \ref{de:groep}.}
    \begin{itemize}
    \item Associativiteit\\
      Kies drie willekeurige elementen uit $<x>$, met andere woorden kies drie getallen $a$, $b$ en $c$ uit $mathbb{Z}$:
      \[
      \begin{array}{rll}
      (x^{a} * x^{b})* x^{c} &= x^{a+b} * x^{c} &\\
                           &= x^{a+b+c}       &\\
                           &= x^{a} * x^{b+c} &= x^{a} * (x^{b} * x^{c})
      \end{array}
      \]
    \item Neutraal element
      Kies willekeurig een element uit $<x>$, kies dus een $a\in \mathbb{Z}$.
      Het neutraal element voor $<x>$ is $x^{0}$:
      \[ x^{a} * x^{0} = x^{a+0} = x^{a} = x^{0+a} = x^{0} + x^{a} \]
    \item Invers element
      Kies een willekeurig element uit $<x>$, kies dus een $a \in \mathbb{Z}$.
      Het invers element van $x^{a}$ is nu $x^{-a}$.
      \[ x^{a}*x^{-a} = x^{a-a} = x^{0} = x^{-a+a} = x^{-a} * x^{a} \]
    \end{itemize}
  \end{proof}

\end{st}

\begin{st}
  \label{st:orde-van-generator-is-orde-van-groep}
  Zij $x$ een element in een groep $G,*$, dan is de orde $n$ van de groep voortgebracht door $x$ gelijk aan de orde $m$ van $x$ in $G$.
  \[ |x| = |<x>| \]

  \begin{proof}
    Bekijk de groep $<x>$.
    \[ <x> = \{ x^{n}, x,x^{2},\dotsc,x^{n-1}\} \]
    Er zitten precies $n$ elementen in $<x>$.
    Dat $m$ de orde is van $x$ in $G$ houdt het volgende in:
    \[ x^{m} = e \text{ en } \neg (\exists\ m')(m'<m \wedge x^{m'} = e) \]
    Alle machten van $x$ tot en met $x^{m}$ zijn dus verschillend.
    Die $m$ machten zijn precies de elementen van $<x>$.
  \end{proof}
\end{st}
 
\begin{de}
  \label{de:cyclische-groep}
  Een groep $G,*$ is een \term{cyclishe groep} als en slechts als er een element in $G$ bestaat dat $G$ voortbrengt.
  We noemen bovendien $x$ de \term{generator} van $G$.
\end{de}

\begin{st}
  \label{st:cyclishe-groep-is-commutatief}
  Elke cyclische groep is commutatief.
\extra{bewijs}
\end{st}

\begin{ei}
  Elke oneindige cyclishe groep is isomorf met $\mathbb{Z},+$ en is dus aftelbaar.

  \begin{proof}
    We tonen dat er een bijectie $b$ bestaat tussen de willekeurige oneindige cyclische groep $G,*$ met generator $x$ en $\mathbb{Z},+$.
    \[ G = <x> = \{ x^{s}\ |\ s \in \mathbb{Z} \} \]
    Inderdaad, kies $b$ als volgt.
    \[ b:\ \mathbb{Z} \rightarrow G:\ s \mapsto x^{s} \]
  \end{proof}
\end{ei}

\begin{ei}
  Elke cyclishe groep van eindige orde $n \in \mathbb{N}$ is isomorf met $\mathbb{Z}_{n},+$.

  \begin{proof}
    We tonen dat er een bijectie $b$ bestaat tussen de willekeurige eindige cyclische groep $G,*$ met generator $x$ en orde $n$ en $\mathbb{Z},+$.
    \[ G = <x> = \{ x^{s}\ |\ s \in \mathbb{Z} \} \]
    Inderdaad, kies $b$ als volgt.
    \[ b:\ \mathbb{Z} \rightarrow G:\ [s]_{n} \mapsto x^{s} \]
  \end{proof}
\end{ei}

\begin{st}
  \label{st:deelgroep-van-cyclische-groep-is-cyclisch}
  Elke deelgroep $H$ van een cyclische groep $G,* = <x>$ is cyclish.
  Sterker nog: $H$ wordt voortgebracht door $x^{s}$ waarbij $s \in \mathbb{N}_{0 }$ het kleinste getal is waarvoor $x^{s} \in H$ geldt. 

  \begin{proof}
    Als $H = \{e\}$ geldt, dan is $H$ inderdaad cyclisch.
    Stel nu dat $H$ niet enkel het neutraal element bevat, dan bevat $H$ mistens nog de elementen $x^{m}$ en $x^{-m}$ met $m \in \mathbb{N}$.
    Het kan zijn dat er meerdere van die $m$'s bestaan, maar er bestaat er altijd een kleinste: $n$.
    \[ x^{n} \in H \]
    Voor elk element $x^{s}$ van $G$ geldt nu dat het in $H$ zit als $n$ een deler is van $s$.
    \[ x^{s} \in H \Leftrightarrow n|s \]
    Inderdaad, als $x^{s}$ in $H$ zit, dan kunnen we $s$ euclidisch delen door $n$\footnote{Zie stelling \ref{st:euclidische-deling}.}:
    \[ s = nq + r \text{ met } 0 \le r < n \]
    \[ x^{s} = x^{nq + r} \Rightarrow x^{r} = x^{s-nq} = x^{s}(x^{n})^{-q} \in H \]
    Omdat $n$ minimaal is, en $r$ tussen $0$ en $n$ zit, is $r$ gelijk aan nul, en $s$ dus deelbaar door $s$.
    Omgekeerd, wanneer $s$ deelbaar is door $n$ bestaat er een $q$ zodat volgende bewering geldt\footnote{Zie definitie \ref{de:deler}.}:
    \[ s = nq \]
    \[ x^{s} = x^{nq} = (x^{n})^{q} \in H \]
    Met andere woorden: elke macht van $x^{n} \in H$ zit opnieuw in $H$. Dit betekent precies dat $H$ cyclisch is.
  \end{proof}
\end{st}

\begin{st}
  \label{st:orde-element-cyclische-groep}
  Zij $G = <a>$ een cyclishe groep van eindige orde $n$ met $*$ als bewerking.
  De orde van $a^{k}$ is gelijk aan $\frac{n}{ggd(k,n)}$:
  \[ (a^{k})^{\frac{n}{ggd(k,n)}} = e\]
  
  \begin{proof}
    \[ (a^{k})^{\frac{n}{ggd(k,n)}} = (a^{n})^{\frac{k}{ggd(k,n)}} = e \]
    Nu moeten we nog bewijzen dat $\frac{n}{ggd(k,n)}$ minimaal is.
    Stel dus dat er en $0 < r < \frac{n}{ggd(k,n)}$ bestaat zodat $a^{kr}$ geldt.
    Bekijk nu de stelling van B\'ezout-Bachet.\footnote{Zie stelling \ref{st:bezout-bachet}.}
    Er bestaan een $\alpha$ en $beta$ zodat het volgende geldt:
    \[ ggd(k,n) = \alpha k + \beta n \]
    Nu bekijken we $a^{kr} = e$ opnieuw, en verheffen we beide kanten tot de macht $\alpha$ zodat er $\alpha k$ in de macht staat.
    \[ a^{kr\alpha} = a^{ggd(k,n)r - \beta nr} = a^{ggd(k,n)r}* a^{-\beta nr} = a^{ggd(k,n)r}\]
    Omdat $a^{ggd(k,n)r} = e$ geldt, is $n$ een deler van $ggd(k,n)r$.\footnote{Zie stelling \ref{ei:groep-eindige-orde-deelbaarheid}.}
    Dat houdt in dat $\frac{n}{d}$ een deler is van $r$, en dus kleiner.
    \TODO{bewijzen in hoofdstuk van deelbaarheid}
  \end{proof}

\end{st}

\begin{st}
  Zij $G = <a>$ een cyclishe groep van eindige orde $n$.
  $a^{k}$ is een generator van $G$ als en slechts als $ggd(k.n)$ gelijk is aan $1$.
  \[ <a^{k}> = G \Leftrightarrow ggd(k,n) = 1 \]

  \begin{proof}
     We weten dat de orde van $a^{k}$ in $G$ gelijk is aan $\frac{n}{ggd(k,n)}$.\footnote{Zie stelling \ref{st:orde-element-cyclische-groep}.}
     Als en slechts als $a^{k}$ een generator is voor $G$, dan is de orde van $a^{k}$ gelijk aan de orde van $G$.\footnote{Zie stelling \ref{st:orde-van-generator-is-orde-van-groep}.}
     \[ \frac{n}{ggd(k,n)} = n \Leftrightarrow ggd(k,n) = 1 \]
  \end{proof}
\end{st}

\begin{st}
  Zij $G,* = <a>$ een cyclishe groep van eindige orde $n$.
  Voor elke $positieve$ deler $m$ van $n$ geldt dat $G$ precies \'e\'en deelgroep heeft van orde $m$, namelijk $<a^{\frac{n}{m}}>$.
  \begin{proof}
    $<a^{\frac{n}{m}}>$ is een deelgroep\footnote{Zie stelling \ref{st:voortbrenging-is-groep}.} van $G$.
    Omdat $G$ cyclisch is, is $<a^{\frac{n}{m}}>$ ook cyclisch.\footnote{Zie stelling \ref{st:deelgroep-van-cyclische-groep-is-cyclisch}.} De orde van $<a^{\frac{n}{m}}>$ is bovendien gelijk aan $\frac{n}{ggd({\frac{n}{m},n)}}$.\footnote{Zie stelling \ref{st:orde-element-cyclische-groep}.}
    \[ \frac{n}{ggd({\frac{n}{m},n)}} = m \]
\TODO{Bewijs in hoofdstuk van deelbaarheid}
    We moeten nu dus nog bewijzen dat elke deelgroep $H$ van orde $m$ gelijk is aan $<a^{\frac{n}{m}}>$.
    Kies zo'n deelgroep $H$ van orde $m$. $H$ is nu zeker cyclisch.
    Noem de generator van $H$ $a^{k}$.
    \[ H = <a^{k}> \]
    We weten nu opnieuw dat $H$ orde $\frac{n}{ggd(k,n)}$ heeft.
    \[ m = \frac{n}{ggd(k,n)} = \frac{kgv(k,n)}{k} \]
\TODO{Bewijs in hoofdstuk van deelbaarheid}
    \[ k = \frac{kgv(k,n)}{m} \]
    $k$ is dus een veelvoud van $\frac{n}{m}$.
    Bijgevolg zit $a^{k}$ zeker in $<a^{k}>$.
    $H$ is nu dus een deel van $<a^{k}>$, dus zijn $H$ en $<a^{\frac{n}{m}}>$ gelijk vanwege hun gelijke orde.
  \end{proof}
\end{st}

\begin{st}
  \label{st:stelling-van-ruffini}
  Stelling van Ruffini\\
  Zij $\alpha\in \mathcal{S}_{n}$ verschillend van $Id$. De orde van $\alpha$ in $\mathcal{S}_{n}$ is gelijk aan het kleinst gemeen veelvoud van de lengtes van de cykels in de disjuncte cykelnotatie van $\alpha$.
\TODO{bewijs p 15}
\end{st}

\subsection{Nevenklassen}
\label{sec:nevenklassen}

\begin{de}
  \label{de:nevenklassen}
  Zij $G,*$ een groep en $H$ een deelgroep van $G$.
  \begin{itemize}
  \item De \term{linkse nevenklasse} $xH$ van $H$ in $G$ bepaald door $x$:
    \[ xH = \{ x * h\ |\ h \in H \} \]
  \item De \term{rechtse nevenklasse} $Hx$ van $H$ in $G$ bepaald door $x$:
    \[ Hx = \{ h * x\ |\ h \in H \} \]
  \end{itemize}
  De verzameling van linker nevenklassen van $H$ in $G$ noteren we als $G/H$.
  \[ G/H = \{ xH\ |\ x \in G \} \]
\end{de}

\begin{ei}
  \label{ei:linker-nevenklasse-eig}
  Zij $G$ een groep en $H$ een deelgroep van $G$.
  \[ \forall a,b \in G:\ aH = bH \Leftrightarrow a \in bH \Leftrightarrow b^{-1}a \in H \]

  \begin{proof}
    Bewijs door circulaire implicaties.
    \begin{itemize}
    \item $\forall a,b \in G:\ aH = bH \Rightarrow a \in bH$\\
      $H$ is een groep, en bevat dus een neutraal element.\footnote{Zie definitie \ref{de:groep}.}
      $a$ is dus een element van $aH$, wat gelijk is aan $bH$, dus $a$ zit in $bH$.
    \item $\forall a,b \in G:\ a \in bH \Rightarrow b^{-1}a \in H$\\
      Ofwel zijn $b$ en $a$ gelijk, en dan is $b^{-1}a$ een element van $H$ omdat $H$ een groep is.
      Ofwel zijn $b$ en $a$ niet gelijk, en dan bestaat er dus een $c \in H$ waarvoor het volgende geldt:
      \[ b*c = a\]
      \[ b^{-1}*b*c = b^{-1}a\]
      \[ c = b^{-1}a \in H \]
    \item $\forall a,b \in G:\ b^{-1}a \in H\Rightarrow aH = bH$\\
      Noem $b^{-1}a$ $h_{0}$.
      \[ b^{-1}a = h_{0} \Rightarrow (a = bh_{0}) \text { en } b = a h_{0}^{-1} \]
      Nu geldt voor alle $h \in H$ het volgende:
      \[ a * h = bh_{0}*h \]
      $aH$ is dus al een deel van $bH$.
      Bovendien geldt, opnieuw voor alle $h \in H$ ook het volgende:
      \[ b * h = ah_{0}^{-1} \]
      $bH$ is dus ook een deel van $aH$.
      We besluiten dat $aH$ en $bH$ gelijk zijn.
   \end{itemize}
  \end{proof}
\end{ei}

\begin{ei}
  \label{ei:rechter-nevenklasse-eig}
  Zij $G$ een groep en $H$ een deelgroep van $G$.
  \[ \forall a,b \in G:\ Ha = Hb \Leftrightarrow a \in Hb \Leftrightarrow ab^{-1} \in H \]

\TODO{ bewijs analoog }
\end{ei}

\begin{ei}
  \label{ei:nevenklasse-eigen-element-gelijk}
  Zij $G,*$ een groep en $H$ een deelgroep van $G$.
  \[ \forall a \in G:\ aH = H \Leftrightarrow a \in H \Leftrightarrow Ha = H \]

  \begin{proof}
    Bewijs door circulaire implicaties.
    \begin{itemize}
    \item $\forall a \in G:\ aH = H \Rightarrow a \in H$\\
      $a$ is een element van $aH$. $aH$ en $H$ zijn gelijk, dus $a$ zit ook in $H$.
    \item $\forall a \in G:\ a \in H \Rightarrow Ha = H$\\
      $H$ is een deelgroep met bewerking $*$ die intern is in $H$.\footnote{Zie definitie \ref{de:groep}.}
      Wanneer we elk element in $H$ rechts bewerken met $a$, komen we telkens een element in $H$ uit.
      Omgekeerd kan elk element in $H$ geschreven worden als een ander element in $H$ rechts bewerkt met $a$.
    \item $\forall a \in G:\ Ha = H \Rightarrow aH = H$\\
      $a$ is een element van $Ha$. $Ha$ en $H$ zijn gelijk, dus $a$ zit ook in $H$.
      $H$ is een deelgroep met bewerking $*$ die intern is in $H$.\footnote{Zie definitie \ref{de:groep}.}
      Wanneer we elk element in $H$ links bewerken met $a$, komen we telkens een element in $H$ uit.
      Omgekeerd kan elk element in $H$ geschreven worden als een ander element in $H$ links bewerkt met $a$.
      $Ha$ en $aH$ zijn dus gelijk.
    \end{itemize}
  \end{proof}
\end{ei}

\begin{st}
  De verzamelingen der linker en rechter nevenklassen van een deelgroep $H,*$ van een groep $G,*$ zijn equipotent.
  \[ |\{ xH\ |\ x \in G \}| = |\{ Hx\ |\ x \in G \}| \]
  \TODO{bewijs p 101 tai}
\end{st}

\begin{st}
  Zij $G$ een groep en $H$ een deelgroep van $G$, De linkse(/rechter) nevenklassen van $H$ vormen een partitie van $G$
  De relatie ``X en Y liggen in dezelfde linkernevenklasse van $H$ in $G$'' is een equivalentierelatie.

  \begin{proof}
    We bewijzen dat de linkse(/rechter) nevenklassen elke definierende eigenschap van een partitie heeft.
    \begin{itemize}
    \item Elke linker(/rechter) nevenklasse is niet leeg. Ze bevat altijd het element waarmee bewerkt wordt
    \item Elk element van $G$ behoort tot een linker(/rechter) nevenklasse.
      Elk element $x$ behoort al zeker tot de nevenklasse van $H$ bepaald door $x$ in $G$.
    \item Twee linker(/rechter) nevenklassen zijn ofwel gelijk ofwel disjunct.
      Ofwel zijn twee nevenklassen disjunct, ofwel bevatten hun doorsnede minstens \'e\'en element.
      We bewijzen dat als twee nevenklassen minstens \'e\'en element gemeenschapplijk hebben, dat ze dan gelijk zijn.
      Kies een willekeurige $x$ en $y$ uit $G$ zodat de doorsnede van de nevenklasse van $H$ bepaald door $x$ en door $y$ niet disjunct zijn.
      Er bestaat dan een element $z$ in $G$ dat in de doorsnede zit.
      Er bestaan dus elementen $h$ en $h'$ in $G$ die aan volgende bewering voldoen:
      \begin{itemize}
      \item Linker nevenklassen
        \[ z = x*h = y*h' \]
        Nu volgt dit:
        \[ y^{-1}*x = h'*h^{-1} \in H \]
        $xH$ en $yH$ zijn dus gelijk.\footnote{Zie eigenschap \ref{ei:linker-nevenklasse-eig}.}
      \item Rechter nevenklassen
        \[ z = h*x = h'*y \]
        Nu volgt dit:
        \[ y*x^{-1} = h'^{-1}h \in H \]
        $Hx$ en $Hy$ zijn dus gelijk.\footnote{Zie eigenschap \ref{ei:rechter-nevenklasse-eig}.}
      \end{itemize}
    \end{itemize}
    De relatie ``X en Y liggen in dezelfde linkernevenklasse van $H$ in $G$'' is dus een equivalentierelatie.\footnote{Zie stelling \ref{st:partitie-equivalentierelatie}.}
  \end{proof}
\end{st}

\begin{st}
  \label{st:nevenklassen-zelfde-orde}
  Zij $G,*$ een eindige groep en $H$ een deelgroep van $G,*$. Nu geldt voor elke $x\in G$ dat de orde van $H$ en zowel de linker als de rechter nevenklasse van $x$ in $G$ alledrie dezelfde orde.
  \[ |H| = |x*H| = |H*x| \]

  \begin{proof}
    We bewijzen dat er een bijectie $f: H \rightarrow xH$ bestaat tussen $H$ en $xH$.
    Beschouw de afbeelding $f$:
    \[ f:\ H\rightarrow xH:\ h \rightarrow x*h \]
    \begin{itemize}
    \item $f$ is injectief.\\
      Kies twee willekeurige elementen $h_{1}$ en $h_{2}$ uit $H$ waarvoor het beeld onder $f$ gelijk is.
      \[ 
      \begin{array}{rrl}
                  & f(h_{1})    &= f(h_{2})   \\ 
      \Rightarrow & xh_{1}      &= xh_{2}     \\
      \Rightarrow & x^{-1}xh_{1} &= x^{-1}xh_{2}\\
      \Rightarrow & h_{1}       &= h_{2}
      \end{array}
      \]
    \item $f$ is surjectief.\\
      Kies een willekeurig element $a$ uit $xH$, dan bestaat er dus een $h \in H$ zodat $a = x*h$ geldt.
    \end{itemize}
    Analoog kunnen we bewijven dat er een bijectie $f: H \rightarrow Hx$ bestaat.
  \end{proof}
\end{st}

\begin{st}
  Een groep $G,*$ heeft evenveel linkse als rechtse nevenklassen.

\TODO{bewijs}
\end{st}

\begin{st}
  \label{st:stelling-van-lagrange}
  \term{Stelling van Lagrange}:\\
  Zij $G,*$ een eindige groep en $H \subset G$ een deelgroep. Het aantal linkse (of rechtse) nevenklassen van $H$ in $G$ is dan gelijk aan:
  \[ [G:H] = \frac{|G|}{|H|} \]
  We noemen dit de \term{index} van $H$ in $G$.
  De orde van $H$ is bijgevolg een deler van de orde van $G$.

  \begin{proof}
    De linkernevenklassen van $H$ in $G$ vormen een partitie van $G$, dus we kunnen elementen $g_{1} = e,g_{2},\dotsc,g_{k} \in G$ kiezen zodat de nevenklassen die erdoor bepaald zijn samen $G$ vormen en onderling disjunct zijn.
    \[ G = g_{1}H \cup \dotsb \cup g_{k}H \]
    \[ \forall i,j:\ i \neq j \Rightarrow g_{i}H \cap g_{j}H = \emptyset \]
    Nu geldt dus dat de orde van $G$ $k$ keer de orde van $H$ is.\footnote{Zie stelling \ref{st:nevenklassen-zelfde-orde}.}
    \[ |G| = k \cdot |H|\]
    Hieruit volgt dat $|H|$ een deler is van $|G|$ en dat het aantal linker nevenklassen $k$ gelijk is aan $\frac{|G|}{|H|}$.
  \end{proof}
\end{st}

\begin{gev}
  \label{gev:orde-van-element-deelt-orde-van-groep}
  Zij $G,*$ een eindige groep en $x$ een element van $G$.
  De orde van $x$ in $G$ is een deler van $|G|$.
  \[ x^{|G|} = e_{G} \]

  \begin{proof}
    Beschouw de deelgroep\footnote{Zie stelling \ref{st:voortbrenging-is-groep}.} $<x>$ van $G$.
    Het aantal elementen van $<x>$ is precies de orde van $x$ in $G$.\footnote{Zie stelling \ref{st:orde-van-generator-is-orde-van-groep}.}
    De orde van een deelgroep is steeds een deler van de orde van de groep.\footnote{Zie stelling \ref{st:orde-van-generator-is-orde-van-groep}.}
  \end{proof}
\end{gev}

\begin{st}
  \label{st:priemgroep-is-cyclisch}
  Zij $G,*$ een groep waarvan de orde gelijk is aan een priemgetal $p$, dan is $G$ cyclisch.
  \begin{proof}
    Kies een willekeurig element $x$ uit $G$ en beschouw de deelgroep\footnote{Zie stelling \ref{st:voortbrenging-is-groep}.} $<x>$ voortgebracht door $x$ van $G$.
    De orde van $<x>$ is een deler van de orde van $G$.\footnote{Zie stelling \ref{gev:orde-van-element-deelt-orde-van-groep}.}
    $p$ heeft maar twee delers, namelijk $1$ en $p$, dus ofwel is $x$ gelijk aan het neutraal element, ofwel is $x$ een generator voon $G$ en is $G$ dus cyclisch.\footnote{Zie definitie \ref{de:cyclische-groep}.}
  \end{proof}
\end{st}

\begin{gev}
  Op isomorfisme na, bestaat er slechts \'e\'en groep voor elk priemgetal, waarbij dat getal de orde is van die groep.
  \begin{proof}
    Kies een priemgetal $p$ en twee groepen $G,*$ en $H,\Box$ met $p$ als orde.
    We weten al dat $G$ en $H$ cyclisch zijn.\footnote{Zie stelling \ref{st:priemgroep-is-cyclisch}.}
    Kies dus een generator $g$ voor $G$ en een generator $h$ voor $H$.
    Beschouw nu het volgende morfisme.
    \[ f:\ G \rightarrow H:\ g^{k} \mapsto h^{k} \]
    $f$ is een isomorfisme, dus $G$ en $H$ zijn isomorf.
  \end{proof}
\end{gev}

\begin{st}
  Zij $f: G \rightarrow H$ een morfisme tussen twee groepen $G,*$ en $H,\Box$. Voor alle $x,y$ in $G$ geldt het volgende:
  \[ f(x) = f(y) \Leftrightarrow x * Ker(f) = y * Ker(f) \]

  \begin{proof}
    \[ x * Ker(f) = y * Ker(f) \Leftrightarrow x^{-1} * y \in Ker(f) \]
    $Ker(f)$ is een deelgroep van $G$.\footnote{Zie stelling \ref{st:kern-is-deelgroep}.}
    Bovenstaande gelijkheid geldt voor elke linker neverklasse, dus ook voor $Ker(f)$.\footnote{Zie eigenschap \ref{ei:linker-nevenklasse-eig} van linker neverklassen.}
    Omdat $x^{-1} * y$ in de kern van $f$ zit is het beeld ervan $e_{H}$.
    \[ f(x^{-1} * y) = e_{H} \]
    $f$ is een morfisme\footnote{Zie definitie \ref{de:groepsmorfisme}.}:
    \[ f(x^{-1}) \Box f(y) = e_{H}\]
    Wat er nu over blijft is precies dat $f(x)$ en $f(y)$ gelijk zijn.
    We hebben in dit bewijs enkel equivalenties gebruikt. De omgekeerde redenering geldt dus ook.
  \end{proof}
\end{st}


\begin{st}
  Als $G$ een commutatieve groep is ,dan is de verzameling $G/H$ met als bewerking $\bar{*}$ vormt een groep.
  \[ \bar{*}: G/H \times G/H \rightarrow G/H: (x*H,y*H) \rightarrow (x*y)*H \]

\TODO{ bewijs str}
\TODO{voorwaarde afzwakken bij normaaldelers}
\end{st}


\subsection{Directe som}
\label{sec:directe-som}

\begin{de}
  Zij $G_{1},\dotsc,G_{n}$ groepen met bewerkingen $*_{1},\dotsc,*_{n}$, dan is de \term{directe som} van de groepen $G_{i}$, genoteerd als $G_{1} \oplus \dotsb \oplus G_{n}$ de verzameling $G$, voorzien van de bewerking $*$ waarbij het volgende geldt.
  \[
  G = G_{1} \times \dotsb \times G_{n}
  \]
  \[
  (g_{1},\dotsc,g_{n}) *(h_{1},\dotsc,h_{n}) = (g_{1} *_{1} h_{1}, \dotsc, g_{2} *_{2} h_{2})
  \]
\end{de}

\begin{st}
  De directe som van $n$ groepen $G_{1},\dotsc,G_{n}$ is een groep.

\TODO{ bewijs }
\end{st}

\begin{ei}
  \label{ei:groep-orde-vier}
  Elke groep van orde $4$ is isomorf met $\mathbb{Z}_{4}$ of met de viergroep $V$.

\TODO{ bewijs }
\end{ei}

\begin{ei}
  \label{ei-groep-orde-zes}
  Elke groep van orde $6$ is isomorf met $\mathbb{Z}_{6}$ of met $\mathcal{S}$. 

\TODO{ bewijs }
\end{ei}

\begin{de}
  Zij $G,*$ een commutatieve groep en $G_{1},*$ en $G_{2},*$ deelgroepen van $G$, en elk element $g$ valt op een unieke manier te schrijven als $g=g_{1}g_{2}$ met $g_{1}\in G_{1}$ en $g_{2}\in G_{2}$, dan noemen we $G$ de inwendige directe som van $G_{1}$ en $G_{2}$.
\end{de}

\section{Permutatiegroepen}
\label{sec:permutatiegroepen}

\begin{de}
  Zij $V$ een verzameling.
  De verzameling van permutaties van $V$ noemen we $\mathcal{S}V$.
  De \term{symmetrische groep} van $V$ is de groep $\mathcal{S}V, \circ$ waarbij $\circ$ de samenstelling van afbeeldingen is.
\end{de}

\begin{de}
  Een \term{permutatiegroep} van $V$ is een deelgroep van $\mathcal{S}V,\circ$.
\end{de}

\begin{de}
  De \term{symmetrische groep} van graad $n$: $\mathcal{S}_{n}$ is de groep van permutaties van $\{1,\dotsc,n\}$,
\end{de}

\begin{st}
  $\mathcal{S}V,\circ$ is een groep.

\TODO{ bewijs }
\end{st}

\begin{st}
  De orde van $\mathcal{S}_{n}$ is $n!$.

\TODO{ bewijs }
\end{st}

\begin{st}
  De orde van $\mathcal{A}_{n}$ is $\frac{n!}{2}$.

\TODO{ bewijs }
\end{st}

\begin{st}
  In een permutatiegroep heeft een element dat als een cykel van lengte $n$ wordt voorgesteld orde $n$

\TODO{ bewijs }
\end{st}

\begin{st}
  \label{st:stelling-van-cayley}
  \term{Stelling van Cayley}:\\
  Elke groep is isomorf met een permutatiegroep (een deelgroep van $S(G),\circ$).

\TODO{ bewijs }
\end{st}

\begin{gev}
  Elke eindige groep met $n$ elementen is isomorf met een deelgroep van $\mathcal{S}_{n}$.

\TODO{ bewijs }
\end{gev}

\begin{pr}
  De afbeelding $sgn: \mathcal{S}_{n},\circ \rightarrow \{1,-1\},\cdot$ is een groepsmorfisme met als kern $A_{n}$.

  \begin{proof}
    Kies twee willekeurige permutaties $\sigma$ en $\tau$ uit $\mathcal{S}_{n}$.
    Nu geldt het volgende, bijgevolg is $sgn$ een morfisme.\footnote{Zie stelling \ref{ei:samenstelling-permutaties-teken}}
    \[ sgn(\sigma \circ \tau) = sgn(\sigma) \cdot sgn(\tau) \]
    De kern van dit morfisme zijn alle permutaties met teken $1$.
    Dit zijn precies de permutaties in $A_{n}$.\footnote{Zie definitie \ref{de:even-permutaties}.}
  \end{proof}
\end{pr}
 
\begin{pr}
  De groep van even permutaties $\mathcal{A}_{n}$ is een deelgroep van $\mathcal{S}_{n}$.
  \begin{proof}
    We gaan de voorwaarden in het criterium voor een deelgroep na.\footnote{Zie stelling \ref{st:criterium-deelgroep}}.
    \begin{itemize}
    \item Het neutraal element van $\mathcal{S}_{n}$ is een even permutatie.
    \item De samenstelling van twee even permutaties is opnieuw een even permutatie.
      Kies namelijk twee willekeurige even permutaties $\sigma$ en $\tau$:
      \[ sgn(\sigma \circ \tau) = sgn(\sigma) \cdot sgn(\tau) = 1 \cdot 1 = 1 \]
    \item Voor elke even permutatie bestaat er een inverse permutatie die ook even is.
      Kies een willekeurige even permutatie $\tau$:
      \[
      \begin{array}{rl}
        \tau \circ \tau^{-1} &= Id\\
        sgn(tau \circ \tau^{-1}) &= sgn(Id)\\
        sgn(tau) \cdot sgn(\tau^{-1}) &= 1\\
        1 \cdot sgn(\tau^{-1}) &= 1
      \end{array}
      \]
      Het teken van $\tau^{-1}$ moet dus $1$ zijn en $\tau^{-1}$ bijgevolg even.
    \end{itemize}
  \end{proof}
\end{pr}

\section{Conjugatie}
\label{sec:conjugatie}

\begin{de}
  Zij $G$ een groep en $x$ een element van $G$.
  Een element $axa^{-1}$ met $a\in G$ in $G$ noemen we een \term{geconjugeerde} of \term{toegevoegde} van $x$.
\end{de}

\begin{de}
  De \term{conjugatieklas} van $x$ in $G$ is de verzameling $Cl(x)$.
  \[ Cl(x) = \{ axa^{-1}\ |\ a \in G \} \]
\end{de}

\begin{st}
  In elke groep is de conjugatieklas van het neutraal element het singleton van het neutraal element.
\extra{bewijs}
\end{st}

\begin{ei}
  \label{ei:element-eigen-conjugatieklas}
  Een element behoort steeds tot zijn eigen conjugatieklas.

  \begin{proof}
    Kies een willekeurig element $x$ van een groep $G,*$.
    Omdat $e$ een element is van $G$ bevat de conjugatieklas het element $exe^{-1} = x$.
  \end{proof}
\end{ei}

\begin{ei}
  \label{ei:conjugatieklassen-partitie}
  De verzameling van alle conjugatieklassen van een groep $G,*$ vormt een partitie van $G$ en bepaalt bijgevolg een equivalentierelatie.
  \begin{proof}
    We gaan alle eigenschappen van een partitie na.\footnote{Zie definitie \ref{de:partitie}.}
    \begin{itemize}
    \item Een conjugatieklasse van een element $x$ is nooit leeg want ze bevat steeds minstens $x$.
    \item Conjugatieklassen zijn onderling disjunct.
      Stel immers dat twee verschillende conjugatieklassen $Cl(x)$ en $Cl(y)$ van elementen $x$ en $y$ een niet-lege doorsnede hebben, dan bestaan er elementen $a,b \in G$ en een element $z$ in de doorsnede zodat het volgende geldt:
      \[ axa^{-1} = z = byb^{-1} \]
      Nu zit $x$ in de conjugatieklas $Cl(y)$ van $y$ met als geconjugeerde $b^{-1}a$ en $y$ in de conjugatieklas $Cl(x)$ van $x$ met als geconjugeerde $a^{-1}b$:
      \[ (b^{-1}a)x(a^{-1}b) = y \]
      \[ x = (a^{-1}b) y(b^{-1}a) \]
\clarify{ Ik ben nog niet overtuigd... }
      Contradictie.
    \item Elk element van $G$ zit in een conjugatieklas.\footnote{Zie eigenschap \ref{ei:element-eigen-conjugatieklas}.}
    \end{itemize}
  \end{proof}
\end{ei}

\begin{st}
  Zij $G,*$ een groep en $a$ en $b$ twee elementen van $G$.
  \[ Cl(a) = Cl(b) \Leftrightarrow b \in Cl(a) \Leftrightarrow a \in Cl(b) \]

  \begin{proof}
    Twee verschillende conjugatieklassen zijn disjunct\footnote{Zie eigenschap \ref{ei:conjugatieklassen-partitie}.}, dus twee elementen zitten in dezelfde conjugatieklas als en slechts als een van de elementen in de conjugatieklas van het andere zit.
  \end{proof}
\end{st}

\begin{ei}
  Twee permutaties $\sigma$ en $\tau$ in $\mathcal{S}_{n}$ zijn geconjugeerd als en slechts als ze dezelfde cykelstructuur hebben.
  Met andere woorden als en slechts als $\sigma$ en $\tau$ evenveel cykels hebben van elke optredende lengte.
  
  \begin{proof}
    Bewijs van een equivalentie.
    \begin{itemize}
    \item $\Rightarrow$\\
      Zij $\sigma$ en $\tau$ twee geconjugeerde permutaties in $\mathcal{S}_{n}$.
      Er bestaat dan een geconjugeerd element $\alpha$:
      \[ \alpha \sigma \alpha^{-1} = \tau \]
      Schrijf nu $\sigma$ in de disjuncte cykelnotatie als volgt:
      \[ \sigma = (i_{1,1}\dotsc i_{1,j_{1}})(i_{2,1}\dotsc i_{2,j_{2}}) \dotsc (i_{k,1}\dotsc i_{k,j_{k}}) \]
      Het aantal elementen in de $m$-de cykel is $j_{m}$ en het totaal aantal cykels is $k$.
      Allereerst merken we het volgende op:
      \[ \alpha \circ (i_{m,1}\dotsc i_{m,j_{m}}) \circ \alpha^{-1} = (\alpha(i_{m,1})\dotsc\alpha(i_{m,j_{m}})) \]
\clarify{waarom? maak algoritme voor samenstelling van cykels en toon dan aan.}
      Dit is bovendien opnieuw een $j_{m}$-cykel.
      Bekijk nu $\tau$:
      \[ 
      \begin{array}{rl}
      \tau &= \alpha \circ \sigma \circ \alpha^{-1}\\
           &= \alpha \circ (i_{1,1}\dotsc i_{1,j_{1}})(i_{2,1}\dotsc i_{2,j_{2}}) \dotsc (i_{k,1}\dotsc i_{k,j_{k}}) \circ \alpha^{-1}\\
           &= (\alpha \circ (i_{1,1}\dotsc i_{1,j_{1}})\circ \alpha^{-1}) \circ (\alpha \circ(i_{2,1}\dotsc i_{2,j_{2}})\circ \alpha^{-1}) \circ \dotsc \circ (\alpha \circ(i_{k,1}\dotsc i_{k,j_{k}}) \circ \alpha^{-1})\\
           &= (\alpha(i_{2,1})\dotsc\alpha(i_{2,j_{2}})) \circ (\alpha(i_{2,1})\dotsc\alpha(i_{2,j_{2}})) \circ \dotsc \circ (\alpha(i_{k,1})\dotsc\alpha(i_{k,j_{k}}))
      \end{array}
      \]
      Deze schrijfwijze bestaat opnieuw uit disjuncte cykels.
      \clarify{waarom?}
      De cykelstructuur is bovendien gelijk aan die van $\sigma$.
    \item $\Leftarrow$\\
      Stel dat twee permutaties $\sigma$ en $\tau$ dezelfde cykelstructuur hebben.
      \[ \sigma = (i_{1,1}\dotsc i_{1,j_{1}})(i_{2,1}\dotsc i_{2,j_{2}}) \dotsc (i_{k,1}\dotsc i_{k,j_{k}}) \]
      \[ \sigma = (i'_{1,1}\dotsc i'_{1,j_{1}})(i'_{2,1}\dotsc i'_{2,j_{2}}) \dotsc (i'_{k,1}\dotsc i'_{k,j_{k}}) \]
      We construeren nu een permutatie $\alpha$ van $\{ 1,\dotsc,n \}$ zodat $\tau$ en $\sigma$ conjugeren als volgt:
      \[
        \forall p,q:\ \alpha(i_{p,q}) = i'_{p,q}\\
      \]
      De andere elementen beeldt $\alpha$ op zichzelf af.
      \clarify{Zeker?}
      Nu conjugeert $\alpha$ $\tau$ met $\sigma$.
      \clarify{Ja?}
    \end{itemize}
  \end{proof}
\end{ei}

\begin{de}
  \label{de:sigma-a}
  Zij $G,*$ een groep en $a$ een element van $G$.
  De \term{conjugatie met een element} $a$ of het \term{inwendig automorfisme bepaald door een element} $a$ is een afbeelding $\sigma_{a}$:
  \[ \sigma_{a}:\ G \rightarrow G:\ x \mapsto axa^{-1} \]
\end{de}

\begin{ei}
  Zij $G,*$ een groep en $a$ een element van $G$, dan is $\sigma_{a}$ een automorfisme.
  
  \begin{proof}
    $\sigma_{a}$ is een transformatie van $G$\footnote{Zie definitie \ref{de:sigma-a}.}, dus we moeten enkel nog bewijzen dat $\sigma_{a}$ voor elke $a$ een bijectie is.
    Kies daarvoor een willekeurige $a$ uit $G$.
    \begin{itemize}
    \item $\sigma_{a}$ is een surjuctie.
      Kies een element $b$ in het beeld van $\sigma_{a}$, dan bestaat er een $x$ zodat $x$ op $b$ afgebeeldt wordt:
      \[ \sigma_{a}(x) = b \]
      $b$ is namelijk gelijk aan $axa^{-1}$ dus $x$ is gelijk aan $a^{-1}ba$.
    \item $\sigma_{a}$ is een injectie.
      Kies twee elementen $x$ en $y$ met eenzelfde beeld $b$ onder $\sigma_{a}$.
      \[ axa^{-1} = aya^{-1} \]
      $x$ is dan gelijk aan $y$.
    \end{itemize}
  \end{proof}
\end{ei}

\begin{ei}
  Zij $G,*$ een groep, dan is de afbeelding $\sigma$ een groepsmorfisme:
  \[ \sigma:\ G,* \rightarrow Aut G,\circ:\ a \mapsto \sigma_{a} \]

  \begin{proof}
    Kies twee willekeurige elementen $a$ en $b$ uit $G$.
    \[ \sigma(a * b) = \sigma_{a * b}\]
    \[ \sigma(a) \circ \sigma(b) = \sigma_{a} \circ \sigma_{b} \]
    Kies nu een willekeurig element $x$ uit $G$:
    \[
    \begin{array}{rll}
      \sigma_{a * b}(x) &= (a*b)*x*(a*b)^{-1} &\\
                       &= a*b*x*b^{-1}a^{-1} &\\
                       &= a*\sigma_{b}(x)*a^{-1} &\\
                       &= \sigma_{a}(\sigma_{b}(x)) = (\sigma_{a} \circ \sigma_{b})(x)      
    \end{array}
    \]
  \end{proof}
\end{ei}

\begin{de}
  \label{de:inn-g}
  De deelgroep $InnG$ van $AutG$ is de \term{groep der inwendige automorfismen} van $G$.
  \[ InnG = Im(\sigma) = \{ \sigma_{a}\ |\ a \in G \} \]
\end{de}

\begin{st}
  De groep $InnG$ der inwendige automorfismen van een groep $G$ is wel degelijk een deelgroep van $AutG$.
\extra{bewijs}
\end{st}

\extra{bekijk expliciet wat er gebeurt in commutatieve groepen voor bovenstaande begrippen.}

\begin{de}
  Zij $G,*$ een groep en $g$ een element van $G$, dan is de \term{centralisator} van $g$ in $G,*$ de verzameling $C_{G,*}(g)$ van de elemneten in $G$ die commuteren met $g$.
  \[ C_{G,*}(g) = \{ x \in G\ |\ gx = xg \} \]
\end{de}

\begin{ei}
  \label{ei:centralisator-bevat-element}
  De centralisator $C_{G}(a)$ van $a$ in een groep $G,*$ bevat steeds $a$.
  
  \begin{proof}
    Inderdaad, $ga = ag$ geldt steeds voor $g=a$.
  \end{proof}
\end{ei}

\begin{ei}
  \label{ei:centralisator-is-deelgroep}
  Zij $G,*$ een groep, dan is elke centralisator $C_{G,*}(g)$ van $G,*$ een deelgroep van $G,*$.
  \begin{proof}
    We gaan het criterium voor een deelgroep na voor een willekeurig element $g$ in $G$
    \begin{itemize}
    \item Het neutraal element van $G,*$ is een element van $C_{G,*}(g)$.
      \[ \forall g\in G: ge_{G} = e_{G}g \]
    \item Kies twee elementen $x$ en $y$ die commuteren met $g$
      \[
      \begin{array}{rll}
        (x*y)*g &= x*(y*g)&\\
                &= x*(g*y) &\\
                &= (x*g)*y &\\
                &= (g*x)*y &= g*(x*y)
      \end{array}
      \]
    \item Kies een element $x$ dat commuteert met $g$:
      \[
      \begin{array}{rl}
        xx^{-1} &= e\\
        gxx^{-1} &= g\\
        xgx^{-1} &= g\\
        gx^{-1} &= x^{-1}g
      \end{array}
      \]
    \end{itemize}
  \end{proof}
\end{ei}

\begin{de}
  \label{de:centrum}
  Zij $G$ een groep, dan is het \term{centrum} van $G$ de deelverzameling van $G$ van elementen die met alle elementen van de groep commuteren.
  \[ Z(G) = \{ x \in G\ |\ \forall g \in G:\ gx = xg\} = \bigcap_{g \in G}C_{G}(g) \]
\end{de}

\begin{ei}
  \label{ei:centrum-is-deelgroep}
  Het centrum van een groep is een deelgroep van die groep.

  \begin{proof}
    We kunnen dit op drie manieren bewijzen:
    \begin{enumerate}
    \item Het criterium van een deelgroep afgaan.
    \item Het centrum van een groep is de doorsnede van deelgroepen. \footnote{Zie eigenschap \ref{ei:centralisator-is-deelgroep}.}
\footnote{Zie stelling \ref{st:doorsnede-deelgroepen}.}
    \item Het centrum van een een groep is de kern van $\sigma:\ G,* \rightarrow Aut G,\circ:\ a \mapsto \sigma_{a}$\footnote{Zie stelling \ref{st:kern-is-deelgroep}.}
    \end{enumerate}
  \end{proof}
\end{ei}

\begin{st}
  Voor $n$ groter dan $1$ geldt het volgende over het centrum van $D_{n}$
  \[ Z(D_{2n-1}) = \{e\} \]
  \[ Z(D_{2n}) = \{e,a^{n}\} \]
  
\extra{bewijs}
\end{st}

\begin{st}
  \label{st:orde-conjugatieklasse-centralisator}
  Zij $G$ een eindige groep en $a$ een element van $G$.
  \[ |G| = |C_{G}(a)|\cdot|Cl(a)| \]

  \begin{proof}
    We construeren een bijectie tussen de conjugatieklas $Cl(a)$ van $a$ en de verzameling van linkse nevenklassen van $C_{G}(a)$ in $G,*$.
    Er geldt dan het volgende, wat equivalent is met de te bewijzen stelling.\footnote{Zie stelling \ref{st:stelling-van-lagrange}.}
    \[ \frac{|G|}{|C_{G}(a)|} = |\{ gC_{G}(a) \ |\ a \in G \}| = Cl(a) \]
    Beschouw nu de afbeelding $\phi$:
    \[ \phi: \{ gC_{G}(a) \ |\ a \in G \} \rightarrow Cl(a): gC_{G}(a) \mapsto gag^{-1} \]
    We bewijzen eerst dat $\phi$ goed gedefinieerd is.
    Als we immers twee verschillende elementen $g$ en $h$ zouden kiezen van een linker nevenklasse $gC_{G}(a)$ dan zou $gag^{-1}$ gelijk moeten zijn aan $hah^{-1}$ opdat $\phi$ goed gedefinieerd zou zijn.
    \[ gC_{G}(a) = hC_{G}(a) \Leftrightarrow h^{-1}gC_{G}(a) = C_{g}(a) \]
    Omdat $h^{-1}g$ bijgevolge in $C_{g}(a)$ zit commuteert $h^{-1}g$ met $a$:
    \[ \Leftrightarrow h^{-1}g \in C_{g}(a) \Leftrightarrow h^{-1}ga = ah^{-1}g \Leftrightarrow gag^{-1} = hah^{-1} \]
    \begin{itemize}
    \item $\phi$ is een injectie:\\
      Stel dat $\phi(gC_{g}(a))$ gelijk is aan $\phi(hC_{g}(a))$, dus dat $gag^{-1}$ gelijk is aan $hah^{-1}$, dan geldt eenvoudigweg dat $gC_{G}(a)$ gelijk is aan $hC_{G}(a)$.
    \item $\phi$ is een surjectie:\\
      Inderdaad, voor elke mogelijke conjugatie (met $g$) van $a$ bestaat er een linker nevenklasse $gC_{G}(a)$.
    \end{itemize}
  \end{proof}
\end{st}

\begin{gev}
  \label{gev:orde-conjugatieklasse-deelt-orde-groep}
  Zij $G,*$ een eindige groep en $a$ een element van $G$, dan is $|Cl(a)|$ een deler van $|G|$.
  
  \begin{proof}
    Voor elk element $a$ van $G$ bestaat er een $q \in \mathbb{Z}$ zodat het volgende geldt:
    \[ |G| = q \cdot |Cl(a)| \]
    Sterker nog, die $q$ is gelijk aan $|C_{G}(a)|$.
  \end{proof}
\end{gev}

\begin{st}
  \label{st:klasvergelijking}
  De \term{klasvergelijking}\\
  Zij $G$ een eindige groep van orde $k$.
  De som van het aantal orde in het centrum van $G$ en het aantal elementen van de conjungatieklassen met meer dan 1 element is gelijk aan het aantal elementen in $G$.\\
  In symbolen: Noteer de conjungatieklassen van $G$ als volgt:
  \[ Cl(a_{1}),\dotsc, Cl(a_{m}), Cl(a_{m+1}), \dotsc, Cl(a_{k}) \]
  De elementen met index kleiner of gelijk aan $m$ hebben een conjungatieklasse van grootte $1$, en de anderen een grotere.
  \[ |Cl(a_{1})| = \dotsb = |Cl(a_{m})| = 1 \text{ en } |Cl(a_{m+i})| > 1\]
  \[ |G| = |Z(G)| + \sum_{i=m+1}^{k}|Cl(a_{i})| \]

  \begin{proof}
    De conjugatieklassen van $G,*$ vormen een partitie.\footnote{Zie stellin \ref{ei:conjugatieklassen-partitie}.}
    Bijgevolg geldt het volgende over $|G|$.
    \[ |G| = \sum_{i=1}^{k}|Cl(a_{i})| \]
    Merk nu op dat de orde van een conjugatieklasse $Cl(a_{i})$ $1$ is wanneer die conjugatieklasse $Cl(a_{i})$ enkel $a_{i}$ bevat.
    Dit is precies wanneer de centralisator $C_{G,*}(a)$ van $a$ heel $G$ is\footnote{Zie stelling \ref{st:orde-conjugatieklasse-centralisator}.}, dus wanneer $a$ in het centrum van $G,*$ zit.
    \[ \sum_{i=1}^{m}|Cl(a_{i})| = |Z(G)| \]
    Hieruit volgt meteen de stelling.
  \end{proof}
\end{st}

\begin{de}
  Een groep $P,*$ met een macht $p^{k}$ van een priemgetal $p$ als orde noemen we een $p$-groep.
\end{de}

\begin{pr}
  \label{pr:orde-centrum-pgroep-groter-dan-een}
  Zij $p$ een priemgetal en $P,*$ een $p$-groep, dan bevat het centrum $Z(P)$ van $P,*$ meer dan \'e\'en element.
  \[ |Z(P)| > 1 \]
  
  \begin{proof}
    Het aantal elementen in het centrum van een $p$-groep bevat meer dan \'e\'en element als en slechts als het aantal elementen van dat centrum groter is of gelijk aan $p$.
    Dit als en slechts als $p$ de orde van $Z(P)$ deelt.
    De orde van $Z(P)$ deelt immers de orde van $P$ want $Z(P)$ is een deelgroep van $G,*$.\footnote{Zie eigenschap \ref{ei:centrum-is-deelgroep}.} \stref{st:stelling-van-lagrange}
    Als $|P|$ niet $1$ is, deelt $p$ dus zeker de orde van $Z(P)$.
    \[ |Z(P)| > 1 \Leftrightarrow |Z(P)| \ge p \Leftrightarrow p \text{ deelt } |Z(p)| \]
    Zij de orde van $|P|$ de $r$-de macht van $p$.
    \[ |P| = p^{r} \]
    Zij $Cl(a_{1})$ tot en met $Cl(a_{k})$ de conjugatieklassen van $G,*$ zoals in de formulering van de klasvergelijking.
    \[ p^{r} = |P| = |Z(G)| + \sum_{i=m+1}^{k}|Cl(a_{i})| \]
    Omdat de orde van $Cl(a_{i})$ ook $|P|$ deelt\footnote{Zie gevolg \ref{gev:orde-conjugatieklasse-deelt-orde-groep}.}, moet $|Cl(a_{i})|$ met $i$ groter dan $m$ een strikt positieve macht zijn van $p$.
    Bijgevolg moet $p$ een deler zijn van $|Z(p)|$ \clarify{waarom?} en $|Z(p)|$ dus groter dan $1$.
  \end{proof}
\end{pr}


\begin{st}
  \label{st:priemgroep-kwadraat-abels}
  Zij $p$ een priemgetal, dan is elke groep met $p^{2}$ elementen commutatief.

  \begin{proof}
    Zij $P$ een groep met orde $p^{2}$.
    De orde van het centrum $Z(P)$ van $P$ is groter dan $1$\prref{pr:orde-centrum-pgroep-groter-dan-een}.
    De orde van het centrum is bovendien een deler van $p^{2}$\eiref{ei:centrum-is-deelgroep} \stref{st:stelling-van-lagrange}.
    Bijgevolg zijn er precies twee mogelijkheden:
    \[ |Z(P)| = p \text { of } |Z(P)| = p^{2} \]
    Als $|Z(p)| = p^{2}$ geldt, is de groep gelijk aan het centrum van de groep en bijgevolg commutatief.\deref{de:centrum}
    Als $|Z(p)|$ $p$ is, kies dan een element $a$ uit $P$ dat niet in het centrum zit. Dit kan omdat $p$ strikt kleiner is dan $p^{2}$.
    De centralisator $C_{P}(a)$ van $a$ zal minstens zichzelf bevatten\eiref{ei:centralisator-bevat-element}, alsook het centrum van $P$, minstens $p+1$ elementen dus.
    Die centralisator moet als orde een deler hebben van $|P|$,\eiref{ei:centralisator-is-deelgroep} \stref{st:stelling-van-lagrange} dus moet die centralisator heel $P$ omvatten.
    Dit betekent echter dat $a$ toch een element is van het centrum want deze redenering gaat op voor elke $a$ buiten $Z(P)$, contradictie. 
  \end{proof}
\end{st}

\begin{st}
  Elke groep van orde $9$ is isomorf met $\mathbb{Z}_{9}$ of met $\mathbb{Z}_{3} \oplus \mathbb{Z}_{3}$.
  
\extra{bewijs}
\end{st}

\section{Normaaldelers en Quotientgroepen}
\label{sec:normaaldelers-en-quotientgroepen}

\begin{de}
  Een \term{normaaldeler} $H,*$ is een deelgroep $H,*$ van $G,*$ met de volgende eigenschap:
  \[ N \triangleleft G \Leftrightarrow \forall g \in G: g*H = H*g \]
  In woorden: ``De linker en rechter nevenklasse van $N$ bepaald door elk element $g\in G$ zijn gelijk.''
\end{de}

\begin{opm}
  Merk op dat volgende bewering niet noodzakelijk geldt voor een normaaldeler $H,*$ van een groep $G,*$:
  \[ \forall h \in H, g \in G:\ g*h = h*g \]
  \clarify{is dit equivalent met: ``en normaaldeler is niet noodzakelijk het centrum van een groep.'' ?}
\end{opm}

\begin{de}
  Een \term{quotientgroep} $G/H,\bar{*}$ van een groep $G$ door een deelgroep $H$ van $G$ is de groep der linker nevenklassen van $H$ met een bewerking $\bar{*}$ die onafhankelijk is van de representantenkeuze.
  \[ (\bar{*}):\ G/H \rightarrow G/H:\ (x*H)\bar{*}(y*H) = (x' * y') * H \]
  In de bovenstaande definitie zijn $x'$ en $y'$ twee willekeurig gekozen representaten.
\end{de}

\begin{ei}
  \label{ei:normaaldelers-tussen-groep-en-deler}
  Als $N$ een normaaldeler is van $G$, dan is $N$ ook een normaaldeler van elke deelgroep tussen $N$ en $G$ en geldt bovendien, nog sterker, het volgende in $G,*$:
  \[ N \triangleleft G \Rightarrow (\forall N':\ (N \subseteq N' \wedge N' \subseteq G) \Rightarrow (\forall n\in N: nN' = N'n)) \]
\extra{bewijs}
\end{ei}

\begin{st}
  Van een commutatieve groep is elke deelgroep een normaaldeler.

  \begin{proof}
    Kies een deelgroep $D,*$ van een commutatieve groep $G,*$.
    \[ \forall x\in G: x*D = \{ x*d | d \in D \} = \{ d*x | d \in D \} = D*x \]
  \end{proof}
\end{st}

\begin{st}
  In elke groep $G,*$ zijn $\{e_{G}\},*$ en $G,*$ normaaldelers.
  
  \begin{proof}
    Aangezien de bewerking $*$ in $G$ intern is\deref{de:groep} geldt het volgende:
    \[ gG = Gg \]
    $G,*$ is dus een normaaldeler van $G,*$\\
    Precies omdat $e_{G}$ het neutraal element is in $G,*$, is $\{e_{G}\},*$ een normaaldeler van $G,*$.
    \[ e_G*\{e_{G}\} = \{e_{G} * e_{G}\} = \{e_{G}\}*e_{G} \]
  \end{proof}
\end{st}

\begin{de}
  Zij $G$ een groep, dan noemen we $\{e_{G}\}$ en $G$ de triviale normaaldelers van $G$.
\end{de}

\begin{st}
  \label{st:criteria-voor-normaaldeler}
  Criteria voor een normaaldeler\\
  Zij $G,*$ een groep en $H$ een deelgroep van $G$.
  \[
  \begin{array}{rcl}
    H \triangleleft G &\overset{(1)}{\Leftrightarrow} & \forall g \in G:\ g*H*g^{-1} = H\\
                      &\overset{(2)}{\Leftrightarrow} & \forall g \in G:\ g*H*g^{-1} \subset H\\
                      &\overset{(3)}{\Leftrightarrow} & \forall g \in G,\forall h \in H:\ g*h*g^{-1} \in H\\
  \end{array}
  \]

  \begin{proof}
    \begin{enumerate}
    \item Bewijs van een equivalentie.
      \begin{itemize}
      \item $\Rightarrow$\\
        \begin{itemize}
        \item $gNg^{-1}\subseteq N$\\
          Kies een willekeurig element $g*n*g^{-1}$ in $g*N*g^{-1}$.
          Omdat $g*n$ in de nevenklasse van $N$ zit, moet er een $n'\in N$ bestaan zodat $g*n =n'*g$ geldt.
          Laat dit nu precies $gng^{-1}$ zijn.
          \[ gng^{-1} = n'gg^{-1} = n' \]
          $gng^{-1}$ zit dus zeker in $N$.\\
        \item $N \subseteq gNg^{-1}$\\
          Kies een willekeurig element $n$ in $N$.
          Omdat $n*g$ voor elk element $g\in G$ in de nevenklasse van $N$ zit, bestaat er een $n'\in N$ zodat $n*g = g*n'$ geldt.
          \[ n = n * g * g^{-1} = g*n'*g^{-1}\]
          $g*n'*g^{-1}$ zit zeker in $g*N*g^{-1}$.
        \end{itemize}
      \item $\Leftarrow$\\
        Kies een willekeurig element $g$ van $G,*$. voor elk element $n$ van $N$ bestaat er nu een element $n'\in N$ als volgt:
        \[ \forall n\in N,\ \exists n'\in N:\ g*n*g^{-1} =n' \]
        Dit houdt precies het volgende in.
        \[ \forall n\in N,\ \exists n'\in N:\ g*n=n'*g^{-1} \]
        $N$ is dus een normaaldeler van $G$.
        \[ g*N = N*g \]
      \end{itemize}

    \item \extra{bewijs}
    \item Dit is precies hetzelfde als het tweede criterium, enkel geherformuleerd.
    \end{enumerate}
  \end{proof}
\end{st}

\begin{st}
  ZIj $G,*$ een groep, $H,*$ een deelgroep van $G,*$ en $\sim$ een equivalentierelatie op $G$, dan is $H$ een normaaldeler van $G$ als en slechts als $\sim$ zowel links als rechts verenigbaar is.

\TODO{bewijs p 106 tai}
\end{st}

\begin{pr}
  Zij $G,*$ een groep, dan is het centrum $Z(G)$ een normaaldeler van $G$.
  \begin{proof}
    \[ \forall g\in G:\ g*Z(G) = \{ g*z \ |\ z\in Z(G) \} = \{ z*g \ |\ z\in Z(G) \} = Z(G)*g \} \]
  \end{proof}
\end{pr}

\begin{pr}
  Zij $G,*$ een groep, dan is elke deelgroep van een groep $G,*$ met index $2$ een normaaldeler van $G,*$.
  \begin{proof}
    Zij $H,*$ een deelgroep van een groep $G,*$ met index $2$.
    \[ [G:H] = 2 \Leftrightarrow \frac{|G|}{|H|} = 2 \]
    Het aantal nevenklassen van $H$ in $G$ in dan precies $2$.\stref{st:stelling-van-lagrange}
    De nevenklassen zijn dan $H$ en $G\setminus H$.
\clarify{waarom?}
    We gaan nu na of $gH$ gelijk is aan $Hg$ voor elke $g\in G$.
    Als $g$ in $H$ zit, dan is $gH$ gelijk aan $H$ en $Hg$. $gH=Hg$ geldt dan.
    Als $g$ in $G\setminus H$ zit, dan zijn zowel $gH$ als $Hg$ niet gelijk aan $H$ geldt bijgevolg $gH = G\setminus H = Hg$.
    \clarify{waarom?}
  \end{proof}
\end{pr}

\begin{pr}
  Zij $G,*$ een groep en $N,*$ de enige deelgroep van $G,*$ met orde $|N|$, dan is $N,*$ een normaaldeler van $G,*$.

  \begin{proof}
    Beschouw voor elke $g$ in $G$ het inwendig automorfisme $\sigma_{g}$:
    \[ \sigma_{g}:\ G \rightarrow G:\ x \mapsto gxg^{-1} \]
    Dit morfisme beeldt de deelgroep $N$ bijectief af op de deelgroep $gNg^{-1}$.
    \clarify{waarom?}
    \[ |gNg^{-1}| = |N| \]
    Omdat $N$ de enige deelgroep is met $|N|$ elementen en $gNg^{-1}$ een deelgroep is \waarom van $G,*$ moeten $gNg^{-1}$ en $N$ gelijk zijn.
  \end{proof}
\end{pr}

\begin{pr}
  \label{pr:kern-is-normaaldeler}
  De kern $Ker(f)$ van een groepsmorfisme $f:\ G,* \rightarrow H,\Box$ is een normaaldeler van $G,*$.
  \[ Ker(f) \triangleleft G \]
  
  \begin{proof}
    Kies een willekeurig element $g$.
    We bewijzen nu dat er voor elk element $k$ in de kern $Ker(f)$ van $f$ een element $k'$ bestaat zodat $gk$ gelijk is aan $k'g$.
    Omdat zowel $k$ als $k'$ in de kern van $f$ zitten geldt het volgende:\eiref{ei:nevenklasse-eigen-element-gelijk}
    \[ kKer(f) = Ker(f) = Ker(f)k'\]
    Dit betekent precies dat de kern van $f$ een normaaldeler is van $G,*$.
  \end{proof}
\end{pr}

\begin{ei}
  Zij $f:\ G,* \rightarrow H,\Box$ een groepsmorfisme
  $f$ behoudt het normaaldelen.
  \[ N \triangleleft G \Rightarrow f(N) \triangleleft f(G) \]

  \begin{proof}
    Omdat $N$ een normaaldeler is van $G$, geldt voor elk element $g$ van $G$, dat er voor elk element $n$ van $N$ een element $n'$ in $N$ bestaat zodat het volgende geldt:
    \[ g*n =n'*g \]
    We beelden nu beiden kanten af onder $f$.
    \[
    \begin{array}{rl}
      f(g*n) &= f(n'*g)\\  
      f(g)\Box f(n) &= f(n') \Box f(g)\\
    \end{array}
    \]
    Dit betekent precies dat $f(N)$ een normaaldeler is van $G$.
  \end{proof}
\end{ei}

\begin{ei}
  Zij $f:\ G,* \rightarrow H,\Box$ een groepsmorfisme
  $f^{-1}$ behoudt het normaaldelen.
  \[ f(N) \triangleleft H \Rightarrow N \triangleleft G \]

  \begin{proof}
    Omdat $N$ een normaaldeler is van $H$, geldt voor elk element $f(g)$ van $H$, dat er voor elk element $f(n)$ van $N$ een element $f(n')$ in $N$ bestaat zodat het volgende geldt:
    \[ f(g) \Box f(n) = f(n') \Box f(g) \]
    Gebruik nu de definierende eigenschap van een groepsmorfisme om het volgende te bekomen.
    \[
    \begin{array}{rl}
      f(g*n) &= f(n'*g)\\
      g*n &= n'*g
    \end{array}
    \]
    \clarify{waarom moeten we hier niet eisen dat $f$ injectief is?}
    
  \end{proof}
\end{ei}

\subsection{Quotientgroepen}
\label{sec:quotientgroepen}

\begin{de}
  Zij $G,*$ een groep en $N$ een normaaldeler van $G$.
  De (kandidaat-)bewerking $\bar{*}$ op $G/N$ is gedefinieerd als volgt:
  \[ (x*N)\bar{*}(y*N) = (x'*y')*N \]
  Hierin zijn $x'\in x*N$ en $y'\in f*N$ twee willekeurige \term{representanten}.
\end{de}

\begin{pr}
  \label{pr:bewerking-quotientgroep-goed-gedefinieerd}
  De definitie van de quotientgroep $G/H,\bar{*}$ van een groep $G$ door een deelgroep $H$ van $G,*$ is enkel zinvol wanneer $H$ een normaaldeler is van $G,*$.
  't Is te zeggen. De bewerking $\bar{*}$ is enkel goed gedefinieerd, dus onafhankelijk van de representatenkeuze, wanneer $H$ een normaaldeler is.
  
  \begin{proof}
    Zij $H$ een normaaldeler van een groep $G,*$.
    Kies twee verschillende representaten $x'$, $x''$, $y'$ en $y''$ van $x$, respectievelijk $y$ in respectievelijk $x*H$ en $y*H$.
    We moeten nu bewijzen dat $(x''*y'')*N$ gelijk is aan $(x'*y')*N$.
    We weten dat $x''*N$ en $x'*N$ gelijk zijn, alsook $y''*N$ en $y'*N$ omdat ze beiden in $x*N$ zitten.\eiref{ei:nevenklasse-eigen-element-gelijk}
    Er bestaan dus elementen $n_{x}$ en $n_{y}$ in $N$ zodat het volgende geldt:
    \[ x'' = x'n_{x} \quad\text{ en }\quad y'' = y'n_{y} \]
    Hieruit volgt eenvoudigweg het volgende:
    \[ x'' * y'' = x' * n_{x} * y' * n_{y} \]
    Omdat $y'$ in $y'N$ zit, geldt ook $N*y' = y'*N$.\eiref{ei:nevenklasse-eigen-element-gelijk}
    Er bestaat dus een $n\in N$ zedat $n_{x} * y$ gelijk is aan $y' * n$.
    \[ x'' * y'' = x' * y' * n * n_{y} \]
    Omdat $n*n_{y}$ ook in $N$ zit, volgt hieruit dat de representatenkeuze niet uitmaakt.
    \[  (x'' * y'') * N = (x' * y') * N \]
  \end{proof}
\end{pr}

\begin{st}
  Zij $G,*$ een groep en $N$ een normaaldeler van $G$, dan vormt de quotientverzameling $G/N$, uitgerust met de bewerking $\bar{*}$ een groep.
  \begin{proof}
    Inderdaad, $\bar{*}$ is intern, het neutraal element van $G/N$ is $N$ en de inverse van een element $x*N$ van $G/N$ is $x^{-1}N$.
  \end{proof}
\end{st}

\begin{de}
  We noemen $G/H,\bar{*}$ de \term{quotientgroep} of het \term{quotient} van $G$ door $H$.
\end{de}

\begin{de}
  We verkorten de notatie van $\bar{*}$ soms naar $*$ als de bewerking op $G/H$.
\end{de}

\begin{st}
  Zij $G,*$ een groep en $N$ een normaaldeler van $G$, dan is de afbeelding $\pi$ een groepsmorfisme.
  \[ \pi:\ G \rightarrow G/N:\ x \mapsto x*N \]

  \begin{proof}
    We moeten dat voor elke twee elementen $x$ en $y$ uit $G$ het volgende geldt:
    \[ \pi(x*y) = \pi(x) \bar{*} \pi(y) \Rightarrow (x*N)\bar{*}(y*N) = (x*y)*N \]
    Dit is een herformulering van de definitie van $\bar{*}$.
  \end{proof}
\end{st}

\begin{de}
  We verkorten de notatie $x*N$ soms als $\bar{x}$ voor $x\in G$.
  \[ \overline{e_{G}} = e_{G}*N = N \quad\text{ en }\quad \bar{x}^{-1} = \overline{x^{-1}} \,\text{ of }\, -\bar{x} = \overline{-x} \]
\end{de}

\begin{de}
  Zij $G,*$ een groep en $N$ een normaaldeler van $G$.
  De verzameling $G/H$ van nevenklassen, uitgerust met de bewerking $\bar{*}$ noemen we de \term{quoti\"entgroep} van $G$ door $N$.
  $\bar{*}$ zullen we vanaf nu ook gewoon als $*$ schrijven.
\end{de}

\begin{de}
  Zij $G,*$ een groep en $N$ een normaaldeler van $G$.
  We gebruiken een kortere notatie $\bar{x}$ voor een element $x*N$ van de quoti\"entgroep van $G$ door $N$:
  \[ \bar{x} = x*N \]
\end{de}

\begin{ei}
  De kern van $\pi:\ G \rightarrow G/N:\ x \mapsto x*N$ is $N$.

  \begin{proof}
    Elk element $n$ in $N$ wordt onder $\pi$ afgebeeldt op $n*N$, maar vermits $n$ in $n*N$ zit, is $n*N$ gelijk aan $N$.\eiref{ei:nevenklasse-eigen-element-gelijk}
    Elk element $g$ buiten $N$ in $G$ wordt afgebeeldt op $g*N$ en vermits $g$ niet in $N$ zit, zitten er minstens $|N|+1$ elementen is $g*N$. $g*N$ kan dus niet gelijk zijn aan $N$.
  \end{proof}
\end{ei}

\begin{gev}
  Een deelgroep $N,*$ van een groep $G,*$ is een normaaldelor van $G$ als en slechts als deze de kern is van een groepsmorfisme vanuit $G$.
  
\extra{bewijs}
\end{gev}

\begin{st}
  Zij $N,*$ een deelgroep van een groep $G,*$, dan is $N$ een normaaldeler als de bewerking $\bar{*}$ van $G/N$ een groep maakt:
  \[ \bar{*}:\ G/N \times G/N \rightarrow G/N:\ (x*N)\bar{*}(y*N) = (x*y)*N \]
  
\extra{bewijs}
\end{st}

\begin{st}
  \label{st:deelgroep-centrum-quotient-cyclisch-groep-commutatief}
  Als er een deelgroep $H,*$ van het centrum $Z(G)$ van een groep $G,*$ bestaat zodat $G/H$ cyclisch is, dan is $G$ commutatief.
  
  \begin{proof}
    Stel dat $G/H$ cyclisch is, dan bewijzen we nu dat willekeurige elementen $a$ en $b$ uit $G$ commuteren.
    Zij $\bar{g}$ een generator voor $G/H$, dan bestaan er getallen $i$ en $j$ in $\mathbb{Z}$ zodat, in $G/H$ het volgende geldt:
    \[ \bar{a} = (\bar{g})^{i} = \overline{g^{i}} \quad\text{ en }\quad \bar{b} = (\bar{g})^{j} = \overline{g}^{j} \]
    Anders gezegd bestaan er $x$ en $y$ in $H$ zodat we $a$ en $b$ als volgt kunnen schrijven.
\waarom
    \[ a = g^{i}x \quad\text{ en }\quad b = g^{j}y \]
    We kunnen nu rekenen met $a$ en $b$. Merk op dat, omdat $G/H$ cyclisch is, $G/Z(G)$ ook commutatief is.\stref{st:cyclishe-groep-is-commutatief}
    \[
    \begin{array}{rll}
      ab &= g^{i}xg^{j}y &\\
         &= g^{i}g^{j}xy &\\
         &= g^{j}g^{i}yx &\\
         &= g^{j}yg^{i}x &= ba
    \end{array}
    \]
  \end{proof}
\end{st}

\begin{st}
  Zij $G,*$ een groep met $G/Z(G)$ cyclisch, dan is $G,*$ commutatief (en $G/Z(G)$ bijgevolg de triviale groep).

  \begin{proof}
    Als $G/Z(G)$ cyclisch is, dan is $G$ commutatief omdat $Z(G)$ een deelgroep is van $Z(G)$.\stref{st:deelgroep-centrum-quotient-cyclisch-groep-commutatief}
  \end{proof}
\end{st}

\begin{gev}
  Zij $p$ en $q$ verschillende priemgetallen, dan is het centrum van een niet-commutatieve groep $G,*$ van orde $pq$ triviaal.
  \[ Z(G) = \{ e_{G} \} \]

  \begin{proof}
    Omdat $G$ niet commutatief is, bestaat er geen deelgroep $H$ van het centrum $Z(G)$ zodat $G/H$ cyclisch (en dus commutatief \stref{st:cyclishe-groep-is-commutatief}) is.\stref{st:deelgroep-centrum-quotient-cyclisch-groep-commutatief}
    $Z(G)$ is dus niet cyclisch.
    De orde van $Z(G)$ moet bovendien een deler zijn van de orde van $G$ omdat $Z(G)$ een deelgroep is van $G$.\stref{st:stelling-van-lagrange} \eiref{ei:centrum-is-deelgroep}
    De orde van $Z(G)$ moet dus $p$, $q$ of $1$ zijn. $p$ en $q$ zijn echter priemgetallen, en elke groep met een priemgetal als orde is cyclisch\stref{st:priemgroep-is-cyclisch}, dus de orde van $Z(G)$ moet $1$ zijn.
    Elke groep van orde $1$ is de groep met enkel het neutraal element.
    \waarom
  \end{proof}
\end{gev}

\subsection{Enkelvoudige en oplosbare groepen}
\label{sec:enkelv-en-oplosb}

\begin{de}
  Een groep $G,*$ is enkelvoudig als de enige normaaldelers van $G$ de triviale normaaldelers $\{e_{G}\}$ en $G$ zijn.
\end{de}

\begin{st}
  We kunnen elke niet-enkelvoudige groep $G$ ontbinden in enkelvoudige groepen.
  \begin{proof}
    Kies van $G=G_{0}$ een niet-triviale normaaldeler $G_{1}$ van maximale orde, dan is $G_{0}/G_{1}$ enkelvoudig.
\waarom
    Kies nu, opnieuw, als $G_{1}$ niet enkelvoudig is, een niet-triviale normaaldeler (van $G_{1}$) van maximale orde.
    $G_{1}/G_{2}$ is dan ook enkelvoudig.
\waarom
    We kunnen zo verder groepen $G_{i}$ kiezen met $G_{i+1} \triangleleft G_{i}$ tot $G_{n}$ uiteindelijk enkelvoudig is.
  \end{proof}
\end{st}

\begin{de}
  De enkelvoudige groepen $G_{i}/G_{i+1}$ uit bovenstaande ontbinding noemen we (de) \term{compositiefactoren} van $G$.
\end{de}

\begin{st}
  Zonder bewijs\\
  De compositiefactoren van een niet-enkelvoudige groep $G$ zijn onafhankelijk (op isomorfisme na) van de keuze van de normaaldelers.\\\\
  Geen bewijs. Zie Jordan en H\"older.
\extra{ toch eens het bewijs proberen? }
\end{st}

\begin{st}
  Zonder bewijs\\
  Structuurstelling voor eindige commutatieve groepen\\
  Elke eindige commutatieve groep is isomorf met een directe som van cyclische groepen.
  \[ \bigoplus_{i=1}^{r} Z_{p_{i}^{k_{i}}} \]
  Hierin is elke $p_{i}^{k_{i}}$ een macht van een priemgetal.
\extra{ toch eens het bewijs proberen? }
\end{st}

\question{Het ziet ernaar uit dat de $p_{i}^{k_{i}}$ delers moeten zijn van de orde van $G$, waarom is dat?}

\begin{pr}
  De enkelvoudige eindige commutatieve groepen zijn de groepen $\mathbb{Z}_{p},+$ met $p$ een priemgetal en de triviale groep $\{e_{G}\},*$.
  \begin{proof}
    $\mathbb{Z}_{p}$ heeft geen niet-triviale normaaldelers. De enige directe som die mogelijk is om $\mathbb{Z}_{p}$ te bekomen is immers enkel $\mathbb{Z}_{p}$ zelf.
    \waarom
    Elke andere (niet-triviale) eindige commutatieve groep heeft een niet-triviale normaaldeler.
\waarom
  \end{proof}
\end{pr}

\begin{de}
  \label{de:oplosbaar}
  Een groep $G$ is \term{oplosbaar} als er een eindige keten van deelgroepen $H_{i}$ bestaat zodat voor elke $H_{i}$ twee eigenschappen gelden:
  \[ \{e_{G}\} = H_{k} \subset H_{k-1} \dotsb H_{2} \subset H_{1} \subset H_{0} = G \]
  \begin{itemize}
  \item $\forall i:\ H_{i+1} \triangleleft H_{i}$
  \item $\forall i:\ H_{i}/H_{i+1}$ is commutatief
  \end{itemize}
\end{de}

\begin{st}
  Zonder bewijs\\
  Een eindige groep $G,*$ is oplosbaar als en slechts als alle compositiefactoren van $G$ commutatief zijn.
\extra{ toch eens het bewijs proberen? }
\end{st}

\begin{st}
  Zonder bewijs\\
  Een eindige groep $G$ van orde $n$ is oplosbaar als \'e\'en van de volgende voorwaarden geldt:
  \begin{itemize}
  \item $n < 60$.
  \item $n$ is oneven.
  \item $n$ is een macht van een priemgetal.
  \end{itemize}
  \extra{ bewijs bekijken }
\end{st}

\section{De isomorfismestellingen}
\label{sec:isomorfismestellingen}

\begin{st}
  \label{st:eerste-isomorfismestelling}
  \term{Eerste isomorfismestelling}\\
  Zij $G.*$ en $H,\Box$ twee groepen.
  Zij $\phi: G \rightarrow H$ een groepsmorfisme, dan is $\bar{\phi}: G/Ker(\phi) \rightarrow H: gKer(\phi) \mapsto \phi(g)$ een injectief groepsmorfisme.
  Bijgevolg is $G/Ker(\phi)$ isomorf met $\phi(G)$. 
  \[ G/Ker(\phi) \cong \phi(G) \]
  Een alternatieve formulering van deze stelling gaat als volgt:
  Zij $\phi: G \rightarrow H$ een groepsmorfisme, dan is het volgende diagram commutatief en $\phi'$ een isomorfisme.
  \begin{figure}[H]
    \centering
    \begin{tikzpicture}
      \matrix (m) [matrix of math nodes,row sep=3em,column sep=4em,minimum width=2em]
      {
        G & H \\
        G/Ker(\phi) & \phi(G) \\};
      \path[-stealth]
      (m-1-1) edge node [left] {$\pi$} (m-2-1)
      edge [double] node [above] {$\phi$} (m-1-2)
      (m-2-1.east|-m-2-2) edge node [below] {$\phi'$} (m-2-2)
      (m-2-2) edge node [right] {$i$} (m-1-2)
      (m-2-1) edge [dashed] node [above] {$\bar{\phi}$} (m-1-2);
    \end{tikzpicture}
  \end{figure}
  Nog een derde formulering als volgt:
  Met dezelde notaties: een groepsmorfisme $\phi$ kan ontbonden worden als een samenstelling van een surjectie $\pi$ en de injectie $\bar{\phi} = i \circ \phi'$, of nog, een surjectie $\pi$, een bijectie $\phi'$ en een injectie (inbedding) $i$.

  \begin{proof}
    Eerst bewijzen we dat $\phi$ goed gedefinieerd is. We tonen aan dat voor twee verschillende representanten $x$ en $y$ uit $gKer(\phi)$ het beeld onder $\phi$ hetzelfde is.
    Omdat $x$ en $y$ beide in dezelfde nevenklasse van $Ker(\phi)$ zitten, bestaat er een element $h\in Ker(\phi)$ zodat $x$ gelijk is aan $y*h$.
    \[
    \begin{array}{rll}
      \phi(x) &= \phi(y*h) &\\
      &= \phi(y) \Box \phi(h) &\\
      &= \phi(y) \Box e_{H} = \phi(y)
    \end{array}
    \]
    Vervolgens bewijzen we dat $\bar{\phi}$ een groepsmorfisme is.
    Inderdaad, omdat $\phi$ een groepsmorfisme is, en \waarom.
    Tenslotte bewijzen we dat $\bar{\phi}$ een injectie is.
    Kies een $\bar{g} = gKer(\phi)$ in $G/Ker(\phi)$ zodat $\bar{\phi}$ $\bar{g}$ afbeeldt op $e_{H}$.
    We moeten nu aantonen dat daaruit volgt dat $\bar{g}$ het neutraal element ($Ker(\phi)$) is van $G/Ker(\phi)$.\stref{st:kern-triviaal-asa-morfisme-injectief}
    Omdat $\bar{\phi}$ $\bar{g}$ afbeeldt op $e_{H}$, moet $\phi$ $g$ ook afbeelden op $e_{H}$.
    \waarom
    Dit betekent dat $g$ tot $Ker(\phi)$ behoort. Bijgevolg is $\bar{g}$ gelijk aan $Ker(\phi)$, het neutraal element van $G/Ker(\phi)$.\eiref{ei:nevenklasse-eigen-element-gelijk}.
  \end{proof}
\end{st}

\begin{pr}
  Zij $G$ een groep, dan zijn de nevenklassen van het centrum van $G$ in $G$ isomorf met de groep der inwendige automorfismen van $G$.
  \[ G/Z(G) \cong Inn G \quad\text{ met }\quad Inn G = \{ \sigma_{a}: G \rightarrow G: g \mapsto aga^{-1} \ |\ a\in G \} \]

  \begin{proof}
    We passen de eerste morfismestelling\stref{st:eerste-isomorfismestelling} toe op het groepsmorfisme $\sigma$.
    \[ \sigma: G \rightarrow Aut G:\ a \mapsto \sigma_{a} \]
    $Z(G)$ is nu de kern van $\sigma$ \waarom , dus is $G/Z(G)$ isomorf met $Inn G$.
  \end{proof}
\end{pr}

\begin{st}
  De \term{factorisatiestelling} deel 1\\
  Zij $f: G\rightarrow H$ een groepsmorfisme en $N\subseteq Ker(f)$ een normaaldeler van $G$.
  Er bestaan dan een uniek groepsmorfisme $f':\ G/N \rightarrow H$ zodat $f$ kan ontbonden worden als $f'\circ \pi$.

  \begin{proof}
    Inderdaad:
    \[ f':\ G/N \rightarrow H:\ aN \mapsto f(a) \]
    Voor elke $a$ in $G$ geldt nu dat $a$ door $f= (f' \circ \pi)$ afgebeeld wordt op $f'(aN) = f(a)$.
    \clarify{waarom is $f'$ uniek?}
  \end{proof}
\end{st}

\begin{st}
  De \term{factorisatiestelling} deel 2\\
  Zij $f: G\rightarrow H$ een groepsmorfisme en $N\subseteq Ker(f)$ een normaaldeler van $G$.
  Het groepsmorfisme $f'$ met $f = f'\circ \pi$ is injectief als en slechts als $N$ gelijk is aan de kern van $f$.
  \[ f':G/N \rightarrow H \text{ is injectief } \Leftrightarrow N = Ker(f) \]

  \begin{proof}
    De nevenklassen van $N$ die door $f'$ op $e_{H}$ worden afgebeeld zijn de nevenklassen $aN$ met $a$ in de kern van $f$.
    $f'$ is dus injectief als en slechts als de verzameling $\{ aN \ |\ a \in Ker(f) \}$ gelijk is aan $\{ N \}$.
    Dit is precies wanneer $N$ de kern is van $f$.
\clarify{ik ben nog niet overtuigd}
  \end{proof}
\end{st}

\begin{de}
  Zij $G,*$ een groep en $X \subseteq G$ een deelverzameling van $G$.
  De groep $grp\{X\}$ voortgebracht door $X$ is de doorsnede van alle deelgroepen van $G$ die $X$ omvatten. 
  \[ grp\{X\} = grp<X> = <X> \]
  De groep voortgebracht door $X$ is de kleinste deelgroep van $G$ die $X$ omvat.
\end{de}

\begin{opm}
  Zij $G,*$ een groep en $x$ en $y$ twee elementen van $G$.
  De groep $<x,y>$ voortgebracht door $x$ en $y$ is niet zomaar gelijk aan de volgende verzameling (analoog met een cyclische groep).
  \[ \{ x^{k}y^{l}\ |\ k,l\in \mathbb{Z} \} \]
\end{opm}

\begin{ei}
  Zij $G,\cdot$ een groep en $X \subseteq G$ een deelverzameling van $G$, dan is de groep $<X>$ voortgebracht door $X$ gelijk aan de verzameling van alle mogelijke samenstellingen van elementen uit $X$ en hun inversen.
  \[ <X> = \left\{ \prod_{i=0}^{n}a_{i} \ \left|\ n\in \mathbb{N}, a_{i}\in X \vee a_{i}^{-1} \in X \right.\right\}\]

  \begin{proof}
    Noteer het rechterlid van bovenstaande vergelijking als $X'$.
    Elke deelgroep van $G$ die $X$ omvat moet alle elementen van $X$, hun inversen, en onderlinge samenstellingen bevatten.
    \[ X' \subseteq <X> \]
    Anderzijds is $X'$ ook een deelgroep van $G$ die $X$ omvat.
\waarom
    \[ <X> \subseteq X' \]
  \end{proof}
\end{ei}

\begin{pr}
  \[ \mathcal{S}_n = <(12),(13),(14),\dotsc, (1\, n-1), (1n)> \]
  \extra{bewijs p 34}
\end{pr}

\begin{pr}
  \[ \mathcal{S}_n = <(12),(23),(34),\dotsc, (n-2\, n-1), (n-1\,n)> \]
  \extra{bewijs p 34}
\end{pr}

\begin{pr}
  \[ \mathcal{S}_n = <(12), (123\dotsc n-1\,n)> \]
  \extra{bewijs p 34}
\end{pr}

\begin{pr}
  \[ \mathcal{A}_n = <(ab)(cd)\ |\ a\neq b \wedge c \neq d> \]
  \extra{bewijs p 34}
\end{pr}

\begin{pr}
  \[ n \ge 3 \Rightarrow \mathcal{A}_n = <(abc)\ |\ a,b,c \in \{1,2,\dotsc,n\}> \]
  \extra{bewijs p 34}
\end{pr}

\begin{de}
  Zij $H$ en $K$ deelgroepen van een groep $G,*$, dan is $HK$ als volgt gedefinieerd.
  \[ H*K = \{ h*k\ |\ h\in H, k\in K \} \]
\end{de}

\begin{opm}
  \label{opm:verm-deelgroepen-niet-noodzakelijk-deelgroep}
  Als $G$ niet commutatief is, dan zal $HK$ niet steeds de groep zijn voortgebracht door $H\cup K$.
  Het is zelfs niet steeds een deelgroep.
\end{opm}

\begin{st}
  \label{st:verm-deelgroepen-deel-van-voortbrenging-van-unie}
  Zij $H$ en $K$ deelgroepen van een groep $G,*$, dan is $HK$ een deelverzameling van de groep voortgebracht door $H\cup K$.
  \[ HK \subseteq <H\cup K> \]
  \extra{bewijs}
\end{st}

\begin{st}
  Zij $H$ en $K$ deelgroepen van een commutatieve groep $G,*$, dan is $HK$ gelijk aan de groep voortgebracht door $H\cup K$.
  \[ HK = <H\cup K> \]
  \extra{bewijs}
\end{st}

\begin{pr}
  Zij $H$ en $K$ deelgroepen van een groep $G,*$, dan zijn de volgende stellingen equivalent.
  \begin{itemize}
  \item $HK = <H\cup K>$
  \item $HK$ is een deelgroep van $G$.
  \item $HK = KH$
  \end{itemize}
  \extra{bewijs p 36}
\end{pr}

\begin{gev}
  Zij $H$ en $K$ deelgroepen van een groep $G,*$, dan geldt het volgende als $H$ of $K$ een normaaldeler is.
  \[ HK = KH = <H\cup K> \]
\end{gev}

\begin{st}
  \term{Tweede isomorfismestelling} of \term{parallellogramisomorfismestelling}\\
  Zij $G,*$ een groep. Zij $H$ een deelgroep van $G,*$ en $N$ een normaaldeler van $G,*$, dan gelden volgende beweringen.
  \[ N \triangleleft HN \]
  \[ (H \cap N) \triangleleft H \]
  \[ \frac{H}{H\cap N} \cong \frac{HN}{N} \]

  \begin{proof}
    $N$ is een normaaldeler van $G$, en $HN$ is een deelgroep van $G$, tussen $N$ en $G$, dus $N$ is ook een 'normaaldeler' van $NH$.
    \clarify{ waarom zijn we er hier plots wel zeker van dat $HN$ een deelgroep is? zie opmerking \ref{opm:verm-deelgroepen-niet-noodzakelijk-deelgroep}.}
    \[ N \triangleleft NH \]
    Beschouw nu de afbeelding $p$:
    \[ p: H \rightarrow \frac{HN}{N}:\ h \mapsto hN \]
    $p$ kunnen we beschouwen als samenstelling $\pi \circ i$ van de inbedding $i:\ H\rightarrow HN$ en het natuurlijk morfisme $\pi$:
    \[ \pi: H \rightarrow \frac{HN}{N}:\ h \mapsto hN \]
    De kern van $p$ is nu gelijk aan $H\cap N$, dus $H \cap N$ is een normaaldeler van $H$.\prref{pr:kern-is-normaaldeler}
    $p$ is bovendien surjectief.
    Kies immers een willekeurig element $x$ in $\nicefrac{HN}{N}$.
    $x$ is dan van de vorm $x=hnN$ met $h\in H$ en $n\in N$.
    $x$ is dan het beeld van $h$ onder $p$.
    \[ x = hnN = p(h) \]
    $\nicefrac{H}{H\cap N}$ is nu isomorf met $\nicefrac{HN}{N}$.\stref{st:eerste-isomorfismestelling}
    $H\cap N$ is immers de kern van $p$ en $\nicefrac{HN}{N}$ het beeld van $N$ onder $p$.
    \clarify{waarom is $p$ zeker een groepsmorfisme, en waarom moeten we surjectiviteit bewijzen?}
  \end{proof}
\end{st}

\begin{st}
  Zij $A$ en $B$ deelgroepen van een eindige groep $G$.
  \[ |AB| = \frac{|A|\cdot|B|}{|A\cap B|} \]
  \extra{bewijs, je mag gebruiken dat $B$ een normaaldeler is van $G$, maar het kan zonder.}
\end{st}

\begin{st}
  \term{Derde isomorfismestelling}\\
  Zij $G,*$ een groep en $N$ een normaaldeler van $G,*$.
  \begin{itemize}
  \item De afbeelding $H \rightarrow H/N$ bepaalt een bijectie tussen de deelgroepen van $G,*$ die $N$ omvatten en de deelgroepen van $G/N$.
  \item Onder deze bijectie corresponderen normaaldelers van $G$ met normaaldelers van $G/N$.
  \item Zij $H$ nog een normaaldeler van $G$ zodat $N$ een deelverzameling is van $H$, dan geldt het volgende:
    \[ \nicefrac{(G/N)}{(H/N)} \cong G/H\]
  \end{itemize}

  \begin{proof}
    Bewijs in delen.
    \begin{itemize}
    \item
      Zij $A$ de verzameling van alle deelgroepen $H$ van $G$ waarvan $N$ een deelverzameling is en $B$ de verzameling van deelgroepen van $G/N$.
      $A$ en $B$ zijn dan de verzamelingen waartussen we een bijectie zoeken.
      We bewijzen dat de afbeelding $p$ een (die) bijectie is.
      \[ p:\ A \rightarrow B:\ H \mapsto \pi(H) \quad\text{ met }\quad \pi: G \rightarrow G/N: g \mapsto gN \]
      \begin{itemize}
      \item $p$ is injectief.\\
        Kies een $H$ en $H$ in $A$ zodat $\pi(H)$ gelijk is aan $\pi(H')$.
        We moeten nu bewijzen dat hieruit volgt dat $H$ gelijk is aan $H'$
        \begin{itemize}
        \item $H \subseteq H'$\\
          Kies een willekeurig element $h$ uit $H$, dan is de afbeelding $\pi(h)$ van $h$ onder $\pi$ een element van $\pi(H) = \pi(H')$.
          Er bestaat dus een element $h'$ in $H'$ zodat $pi(h)$ gelijk is aan $\pi(h')$.
          Dit betekent dat er een $n$ in $N$ bestaat zodat $h$ gelijk is aan $h'n$.
          $N$ is echter een deelverzamelijk van $H'$, dus $h$ zit ook in $H'$.
        \item $H' \supseteq H$\\
          Wissel de benamingen $H$ en $H'$ om in bovenstaand puntje.
        \end{itemize}
      \item $p$ is surjectief.\\
        Kies een $H'$ uit $B$.
        Omdat $\pi$ surjectief is, \waarom geldt $H' = \pi(\pi^{-1}(H'))$.
        Er bestaat dus steeds een element $a$ uit $A$ zodat $p(a)=H'$ geldt. 
      \end{itemize}
    \item Dit volgt uit de eigenschap over beelden en inverse beelden van normaaldelers onder een groepsmorfisme. \extra{dit is me niet duidelijk}
    \item Beschouw de afbeelding $f$:
      \[ f:\ G/N \rightarrow G/H: gN \mapsto gH \]
      $f$ is een goed gedefinieerd groepsmorfisme. \waarom
      $f$ is surjectief.\\ \waarom \clarify{ en waarom is dit nodig?}
      De kern van $f$ is $H/N$.
      \[ Ker(f) = \{ gN \in G/N \ |\ gH = H \} = \{ gN \in G/N \ |\ g \in H \} = H/N \]
      Omdat $H/N$ de kern van $f$ is, en $G/H$ het beeld van $G/N$ onder $f$, $geldt \nicefrac{(G/N)}{(H/N)} \cong G/H$.\stref{st:eerste-isomorfismestelling}
    \end{itemize}
\clarify{Dit kan duidelijker en gedetailleerder!}
  \end{proof}
\end{st}

\begin{st}
  \label{st:herformulering-commutativiteit}
  Herformulering van commutativiteit\\
  Een groep is commutatief als en slechts als het volgende geldt:
  \[ \forall x,y \in G: xyx^{-1}y^{-1} = e \]
  \extra{bewijs}
\end{st}

\begin{de}
  Zij $G$ een groep, dan zijn de elementen in $G$ van de vorm $xyx^{-1}y^{-1}$ met $x,y\in G$ \term{commutatoren}
\end{de}

\begin{de}
  De \term{commutatordeelgroep} of \term{afgeleide groep} $G' = G^{(1)}$ van $G$ is de groep voortgebracht door de commutatoren van $G$.
  \[ G' = G^{(1)} = < xyx^{-1}y^{-1} \ |\ x,y\in G > \]
\end{de}

\begin{st}
  \label{st:commutatordeelgroep-is-deelgroep}
  De commutatordeelgroep $G'$ van een groep $G$ is wel degelijk een deelgroep van $G$.
  \extra{bewijs}
\end{st}

\begin{st}
  Zij $G$ een groep.
  \[ G' \triangleleft G \]

  \begin{proof}
    \hint{Probeer dit niet direct te bewijzen aan de hand van de definitie van een normaaldeler.}
    Kies een element $g$ uit $G$ en een element $h$ uit $H$.
    We bewijzen nu dat $ghg^{-1}$ in $G'$ zit.\stref{st:criteria-voor-normaaldeler}
    We schrijven $ghg^{-1}$ eerst als $ghg^{-1}h^{-1}h$
    $ghg^{-1}h^{-1}$ en $h$ zijn beide elementen van $G'$, dus $ghg^{-1}h^{-1}h$ ook.
\clarify{waarom zit $h$ in $G'$?}
  \end{proof}
\end{st}

\begin{st}
  Zij $G$ een groep en $N$ een normaaldeler van $G$, dan is $G/N$ commutatief als en slechts als $G'$ een deel is van $N$.

  \[ G/N \text{ is commutatief } \Leftrightarrow G' \subseteq N \]

  \begin{proof}
    $G/N$ is commutatief als en slechts als het volgende geldt:
    \[
    \begin{array}{rl}
      \Leftrightarrow & \forall x,y \in G:\ xN * yN * (xN)^{-1} * (yN)^{-1} = N \text{ in }G/N\\
      \Leftrightarrow & \forall x,y \in G: xyx^{-1}yN = N \text{ in } G/N\\
      \Leftrightarrow & \forall x,y \in G: xyx^{-1}y \in N\\
      \Leftrightarrow & G =\ < xyx^{-1}y^{-1} | x,y \in G >\ \subseteq N\\ 
    \end{array}
    \]
    De eerste equivalentie geldt omwille van de herformulering van commutativiteit.\stref{st:herformulering-commutativiteit}
    De tweede equivalentie geldt omdat de bewerking in een quotientgroep goed gedefinieerd is.\prref{pr:bewerking-quotientgroep-goed-gedefinieerd}
    De derde equivalentie is een eerder bewezen eigenschap.\eiref{ei:nevenklasse-eigen-element-gelijk}
    De laatste equivalentie is het resultaat van een eerder bewezen stelling.\stref{st:verm-deelgroepen-deel-van-voortbrenging-van-unie}
  \end{proof}
\end{st}

\begin{gev}
 \label{gev:afgeleide-door-groep-is-commutatief}
  Zij $G$ een groep.
  \[ G/G' \text{ is commutatief } \]

  \begin{proof}
    $G'$ is een deelverzameling van $G'$, dus dit is een speciaal geval van de stelling
  \end{proof}
\end{gev}

\begin{st}
  Zij $G$ een groep, definieer dan $p$ als volgt:
  \[ p:\ G \rightarrow G/G':\ g \mapsto gG' \]
  Zij $f$ nu een groepsmorfisme van $G$ naar een commutatieve groep $A$, dan bestaat er een uniek morfisme $f'$ als volgt:
  \[ f':\ G/G' \rightarrow A \text{ met } f = f'\circ p \]
  \extra{bewijs, hint p 39}
\end{st}

\begin{de}
  De $n$-de afgeleide groep $G^{(n)}$ van een groep $G$ als volgt.
  \[ G^{(n)} = G_{(n-1)}' \]
\end{de}

\begin{pr}
  $G$ is oplosbaar als en slechts er een $k$-de afgeleide van $G$ bestaat die enkel het neutraal element bevat.
  \[ G \text{ is oplosbaar } \Leftrightarrow \exists k\in\mathbb{N}_{0}:\ G^{(k)} = \{e_{G}\} \]

  \begin{proof}
    Bewijs van een equivalentie.
    \begin{itemize}
    \item $\Rightarrow$\\
       Zij $H_{0}=G$ tot $H_{n} = \{e_{G}\}$ een keten van deelgroepen van $G$ met $H_{i+1} \subseteq H_{i}$ en $\nicefrac{H_{i}}{H_{i+1}}$ commutatief voor elke $i$.
       \[ \{ e_{G} \} = H_{n} \triangleleft H_{n-1} \triangleleft \dotsb \triangleleft H_{i+1} \triangleleft H_{i} \triangleleft \dotsb \triangleleft H_{1} \triangleleft H_{0} = G \]
       We tonen nu per inductie op $i$ aan dat elke deelgroep $H_{i}$ een deelverzameling is van de $i$-de afgeleide groep $G^{(i)}$ van $G$.
       \begin{itemize}
       \item $G^{(1)}$ is een deelverzameling van $H_{1}$ want $G/H_{1}$ is commutatief.
\waarom
       \item Stel dat $G^{(i)} \subseteq H_{i}$ geldt, dan bewijzen we nu dat daaruit volgt dat $G^{(i+1)} \subseteq H_{i+1}$ geldt.
         Omdat $G^{(i)}$ een deelverzameling is van $H_{i}$, is $G^{(i+1)}$ een deelverzameling van $H_{i}'$
\waarom
         Omdat $H_{i}/H_{i+1}$ commutatief is, is $H_{i}'$ een deelverzameling van $H_{i+1}$.
\waarom
         $G^{(i+1)}$ is dus inderdaad een deelverzameling van $H_{i+1}$.
       \end{itemize}
    \item $\Leftarrow$\\
      De keten van deelgroepen voldoet aan de definitie van oplosbaarheid.\deref{de:oplosbaar} \gevref{gev:afgeleide-door-groep-is-commutatief} \stref{st:commutatordeelgroep-is-deelgroep}
\clarify{Iets meer uitschrijven?}
    \end{itemize}
  \end{proof}
\end{pr}

\begin{lem}
  Zij $n\in \mathbb{N}$ groter dan $5$, en $H$ een deelgroep van $\mathcal{S}_{n}$.
  Als alle $3$-cykels tot $H$ behoren, dan behoren ze ook tot $H'$.

  \begin{proof}
    Kies een willekeurige $3$-cykel $(abc)$ uit $H$.
    Neem nu twee andere elementen $d$ en $e$ uit $\{1,..,n\}$.
    \[ (abc) = (abd)(ace)(dba)(eca) = (abd)(ace)(abd)^{-1}(ace)^{-1} \in H' \]
    Elke $3$-cykel uit $H$ zit dus ook in $H'$.
  \end{proof}
\end{lem}

\begin{st}
  Voor $n$ groter dan $4$ is $\mathcal{S}_{n}$ niet oplosbaar.
\extra{bewijs}
\end{st}



\end{document}
