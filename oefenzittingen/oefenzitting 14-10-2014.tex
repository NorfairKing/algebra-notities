\documentclass[12pt,a4paper]{article}
\usepackage[latin1]{inputenc}
\usepackage[dutch]{babel}
\usepackage{amsmath}
\usepackage{amsfonts}
\usepackage{amssymb}
\usepackage{amsthm}
\usepackage{todonotes}
\usepackage{enumerate}\usepackage[left=2cm,right=2cm,top=2cm,bottom=2cm]{geometry}
\author{Kevin Teugels}
\title{Oefeningen TAI (14-10-2014)}
\date{14 oktober 2014}



\begin{document}
\maketitle
\section{oefening 1}
op R definieren we de samenstellingswet
 $a \tau b= a+b+a^2b^2$
\subsection*{a}
Deze wet heeft een neutraal element. welk?
\subsubsection*{oplossing}
het neutraal element is hier $0$ aangezien:
\[
a\tau 0=a+0+a^2*0^2=a
\]
en omgekeert:
\[
0 \tau b=0+b+0^2*b^2=b
\]

\subsection*{b}
ze is niet associatief. Ga na!
\subsubsection*{oplossing}
\[
c \tau (a \tau b) = c \tau (a+b+a^2b^2)= c+ (a+b+a^2b^2)+c^2+(a+b+a^2b^2)^2
\]
als we de haken veranderen krijgen we
\[
(c \tau a) \tau b= (c+a+c^2a^2) \tau b= (c+a+c^2a^2)+b+(c+a+c^2a^2)^2+b
\]
het is simpel te zien dat ze niet aan elkaar gelijk zijn.

\subsection*{c}
ze is commutatief. Waarom?
\subsubsection*{oplossing}
de samenstellingswet is commutatief omdat zowel de optelling als het product voor $a$ en $b$ commutatief zijn
\section{oefening 2}
Bewijs dat in $R^2$ X $R^2$ volgende relaties equivalantierelaties zijn:
\[
G= \{((a,b),(c,d)|a^2+b^2=c^2+d^2\}
\]
\[
H=\{((a,b),(c,d))|b-a=d-c\}
\]
\subsection*{partitie}
\subsubsection*{G}

\[
G_{(a,b)}=\{(x,y) \varepsilon R^2|x^2+y^2=a^2+b^2\}
\]
(cirkels met middepunt $(0,0)$ en straal $\sqrt{a^2+b^2}$)

\subsubsection*{H}

\[
b-a=d-c \rightarrow \frac{b-d}{a-c}=1
\]
\[
H_{(a,b)}=\{(x,y) \varepsilon : y=x+b-a\}
\]

\subsubsection*{J}


\[
a+b=c+d \rightarrow \frac{b-d}{a-c	=-1}
\]

\[
J_{(a,b)}=\{(x,y) \varepsilon R^2: y=-x+b+a\}
\]
welke zijn de partities die hierdoor gedefinieerd worden?

\subsection*{oplossing}
\[
(H uni J)_{(a,b)} =\{(x,y) \varepsilon R^2| x=a,y=b\}=\{(a,b)\}
\]

\section{oefening 3}
los het volgende stelsel op:

\[
3x_1-2x_2+6x_3=4 (mod7)
\]
\[
4x_1+x_2+x_3=0 (mod 7)
\]
\[
2x_1+x_2+2x_3=-1(mod 7)
\]
(let op dat we niet mogen delen.) (zie cursus pagina 85)
\subsubsection*{oplossing}
eerste vgl maal 5 (er moeten nog streepjes boven alle atomen (x en getallen)
\[
x_1-3x_2+2x_3=6
\]
\[
4x_1 +X_2+x_3=0
\]
\[
2x_1 +x_2+2x_3=-1
\]
3keer $vlg1+vgl2 \rightarrow vgl2$ 
\\
5 keer $vgl1+vgl3 \rightarrow vgl3$

\[
x_1-3x_2+2x_3=6
\]
\[
0-1x_2+0=4
\]
\[
0+0+ 5x_3=1
\]
dat geeft ons: $x_1=2$,$x_2=3$,$x_3=3$

\section{oefening 4}
Bepaal de isometrieen van een gelijkzijdige driehoek. Stel voor deze ismetrieen de bewerkingstabel op, onder de samenstellingswet o.
(zie cursus p 94-95).

\subsection*{oplossing}
\begin{itemize}
\item[1]
identieke
\item[$r_1$]
rotatie over $120$ graden
\item[$r_2$]
rotatie over $240$ graden
\item[$s_1$]
spiegeling over hoogtelijn 1
\item[$s_2$]
spiegeling over hoogtelijn 2
\item[$s_3$]
spiegeling over hoogtelijn 3
\end{itemize}
\[
\]
de bijhoordende bewerkingstabel is dan:
\\
\begin{tabular}{l| l l l l l l}

  0 & 1 & $r_1$ & $r_2$  & $s_1$ & $s_2$ & $s_3$\\
  \hline
 1&1& $r_1$ & $r_2$ & $s_1$ & $s_2$ & $s_3$\\
 $r_1$ & $r_1$ & $r_2$ & 1 &$s_3$ &$s_1$&$s_2$\\
 $r_2$ & $r_2$ & 1 & $r_1$ & $s_2$ & $s_3$ & $s_1$\\
 $s_1$ & $s_1$ & $s_2$ & $s_3$ & 1 &$r_1$ & $r_2$\\
 $s_2$ & $s_2$ & $s_3$ & $s_1$ & $r_2$ & 1 & $r_1$\\
 $s_3$ & $s_3$ & $s_1$ & $s_2$ & $r_1$ & $r_2$ & 1
 
\end{tabular}

\section{oefening 5}
een latijns vierkant in een n x n tabel waarin slechts n verschillende elementen voorkomen. In elke rij en elke kolom komt namelijk elk element juist eenmaal voor.
\begin{itemize}
\item[(a)] Bewijs dat de bewerkingstabel voor een eindige groep steeds een latijns vierkant is
\item[(b)] Is dit ook een voldoende voorwaarde om een groep te hebben? Bepaal of volgend latijns vierkant de bewerkingstabel van een groep is
\end{itemize}
tabel nog toe te voegen

\newpage
\subsection*{oplossing}
\subsubsection*{a}

Er bestaan een element a $\varepsilon$ g zodat in de a-rij van de tabel hetzelfde element x twee keer voorkomt.
\\ dus dan zijn 2 distincte elementen zeg s,t $\varepsilon$ G zodat as=x en $at=x$ Maar dan is $as=at \rightarrow s=t$ dus een element kan niet twee keer voorkomen in 1 rij dus moeten er n elementen op een rij van lengte n is.

\subsubsection*{b}
\begin{itemize}
\item T is overal bepaalt.
\item T bevat een neutraal element hier c.
\item elk element heeft een inverse
\end{itemize}
maar de tabel is niet associatief bijvoorbeeld:
\[
a \tau (a \tau b)=a \tau e=f
\]
\[
(a \tau a) \tau b)=c \tau b=b
\]
\end{document}

