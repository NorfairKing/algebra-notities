\documentclass[12pt,a4paper]{article}
\usepackage[latin1]{inputenc}
\usepackage[dutch]{babel}
\usepackage{amsmath}
\usepackage{amsfonts}
\usepackage{amssymb}
\usepackage{amsthm}
\usepackage{todonotes}
\usepackage{enumerate}\usepackage[left=2cm,right=2cm,top=2cm,bottom=2cm]{geometry}
\author{Kevin Teugels}
\title{Oefeningen TAI (21-10-2014)}
\date{21 oktober 2014}



\begin{document}
\maketitle
\section*{oefening 1}

Bewijs dat $\mathbb{R}_0 \times \mathbb{R} $ voorzien van de samenstellingswet $((a,b),(c,d)) \rightarrow (ac,bc+d)$ een groep is. Is hij abels?
\subsection*{oplossing}
\begin{proof} 
We bewijzen eerst de associativiteits vereiste:
\[
((a,b)*(c,d))\times (e,f)=(ac,bc+d)\times (e,f)=(ace,(bc+d)e+f))
\]
We verplaatsen nu de haakjes.
\[
(a,b) \times ((c,d) \times (e,f))=(a,b) \times (ce,de+f)=(ace,(bce)+(de+f))=(ace,(bc+d)e+f)
\]
Het neutraal element is $(1,0)$.
Het invers element is $(\dfrac{1}{a},\dfrac{-b}{a})$.
Aan alle voorwaarden voor een groep is voldaan dus het is een groep.
\end{proof}
Het is niet abels want de samenstellingswet is niet commutatief.
\section*{oefening 2}

$<S_n,0>$ is de groep van permutaties van een verzameling van n elementen. Stel de samenstellingstabel op voor $<S_3,0>$. Zijn er deelgroepen? Normaaldelers?
\subsection*{oplossing}
\subsubsection*{samenstellingstabel}
\[
\begin{array}{c| c c c c c c }
0 & (1,2,3) &(3,1,2) & (2,3,1) & (1,3,2) & (3,2,1)& (2,1,3)\\
\hline
(1,2,3) &(1,2,3) &(3,1,2) & (2,3,1) & (1,3,2) &(3,2,1) &(2,1,3)\\
(3,1,2)& (3,1,2) & (2,3,1) & (1,2,3) &(2,1,3) & (1,3,2) & (3,2,1)\\
(2,3,1) & (1,3,1) & (1,2,3) & (3,1,2) & (3,2,1) & (2,1,3)&(1,3,2)\\
(1,3,2) & (1,3,2) & (3,2,1) & (2,1,3) &(1,2,3) & (3,1,2)&(2,3,1)\\
(3,2,1) & (3,2,1) & (2,1,3) & (1,3,2) & (2,3,1) &(1,2,3) & (3,1,2)\\
(2,1,3) & (2,1,3) & (1,3,2) & (3,2,1) & (3,1,2) & (2,3,1) & (1,2,3)\\
\end{array}
\]
\subsubsection*{deelgroepen}
Definitie deelgroep p 94.
\begin{itemize}
\item Triviale deelgroepen: $\{(1,2,3)\},S_3$.
\item Echte deelgroepen: $\{(1,2,3),(3,1,2),(2,3,1)\}$ en $\{(1,2,3),((1,3,2)\}$ en $\{(1,2,3),(3,2,1)\}$.
\end{itemize}
\subsubsection*{normaaldelers}
Als $\forall_a \in G: a*D=D*a$ dan is $D$ een normaaldeler.
\[
D=\{e,r_1,r_2\}
\]
Rechtse nevental: $DS_1=\{s_1,s_3,s_2\},DS_2=\{s_1,s_2,s_3\}$.
\\Linkse nevental= $S_1D=\{s_1,s_2,s_3\},S_2D=\{s_1,s_2,s_3\},s_3D=\{s1,s2s,3)\}$
\\$\Rightarrow D$ is normaaldeler.
\\Op zelfde manier kunnen we aantonen dat $D=\{e,s_1\}$ en $D=\{e,s_2\}$ en $D=\{e,s_3\}$ geen normaaldelers zijn.
\section*{oefening 3}
Zoek de generatoren van de additieve cyclische groepen $\mathbb{Z}_{10},\mathbb{Z}_{11},\mathbb{Z}_{12}$.
\subsection*{oplossing}
vb: $Z=\{\bar{1},\bar{2},...,\bar{6}\} \rightarrow$:
\begin{itemize}
\item $\bar{2} \rightarrow \bar{2}$.
\item $2+\bar{2} \rightarrow \bar{4}$.
\item $2+2+\bar{2} \rightarrow\bar{6}$.
\item $2+2+2+\bar{2} \rightarrow \bar{1}$.
\item ...
\end{itemize}
\pagebreak
\section*{oefening 4}

Genereer de groep voorgebracht onder vermenigvuldiging door de matrices.
\[ \left( \begin{array}{cc}
0 & 1 \\
-1 & 0
\end{array} \right)
en
\left( \begin{array}{cc}
0 & 1 \\
1 & 0
\end{array} \right)
\]
Bewijs dat dit een niet abelse groep is van orde 8.
\subsection*{oplossing}
\[
X=
\begin{pmatrix}
0 &1\\
-1&0
\end{pmatrix}
, X^2=
\begin{pmatrix}
-1&0\\
0&-1
\end{pmatrix}
,X^3=
\begin{pmatrix}
0 & -1\\
1&0
\end{pmatrix}
,X^4=
\begin{pmatrix}
1 & 0\\
0 & 1
\end{pmatrix}
\]
\[
Y=
\begin{pmatrix}
0 & 1\\
1 & 0
\end{pmatrix}
,Y^2=
\begin{pmatrix}
1&0\\
0&1
\end{pmatrix}
\]
\[
XY=
\begin{pmatrix}
1 &0\\
0 &-1
\end{pmatrix}
=YX^3,X^2y=
\begin{pmatrix}
0 &- 1\\
-1 &0
\end{pmatrix}
=YX^2,X^3Y=
\begin{pmatrix}
-1 & 0\\
0&1
\end{pmatrix}
=YX
\]
$\rightarrow$ 8 verschillende elementen. Dus het is een groep van orde 8.
\\$\{X,Y,X^2,X^3,X^4,XY,X^2Y,X^3Y\}$
\\Het is geen abelse groep.
\section*{oefening 5}

Beschouw de groep $G=( \mathbb{Z}_7 /\{0\},.)$ van de gehele getallen modulo 7 zonder de nul en met vermenigvuldiging modulo 7.Bepaal de orde van al de elementen. Is de groep Commutatief?
\subsection*{oplossing}
Vb: $\bar{5} \rightarrow$
\begin{itemize}
\item $\bar{5}^2=\bar{25}=\bar{4}$ .
\item $\bar(5)^3=\bar{4}*\bar{5}=\bar{20}=\bar{6}$.
\item ...
\item $\bar{5}^6=\bar{3}*\bar{5}=\bar{15}=\bar{1} \rightarrow$ Orde van 5 is 6.
\end{itemize}
De groep is commutatief want de maal is commutatief.
\pagebreak
\section*{oefening 6}

Bewijs dat elke deelgroep van een cyclische groep cyclisch is.
\subsection*{oplossing}
\begin{proof}
$G=<g>$ is een cyclische groep, met $g \in G$.
\\Stel $H \subset G$.
\\Als $h\{1\}$,dan is $H$ cyclisch met generator $1$ dus we nemen aan dat $H \neq \{1\}$.
\\Bepaal generator van $H$. Omdat $G$ cyclisch is, is elk element een macht van $g$. 
\\Stel het kleinste positieve getal zodat $g^m \in H$.
\subsubsection*{te bewijzen}
$g^n$ generator $H$ maw elke $h \varepsilon H$ is een macht van $g^n \rightarrow \varepsilon H \subset G \Rightarrow h=G^n$ voor een bepaalde n. Er zijn unieke q,r zodat $r=n*q+r,0\leqslant r \leqslant n$ zodat:
\[
h=g^{nq+r}=(g^n)^q*g^r \Rightarrow g^r=h(g^n)^-q
\]
$\rightarrow$ we weten $g^n \varepsilon H \Rightarrow (g^n)^-q \varepsilon H$ dus $h(q^n)^-q \varepsilon H \rightarrow g^r \varepsilon H$
\\niet volledig

\end{proof}
\end{document}