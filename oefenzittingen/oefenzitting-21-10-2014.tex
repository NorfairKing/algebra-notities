\documentclass[12pt,a4paper]{article}
\usepackage[latin1]{inputenc}
\usepackage[dutch]{babel}
\usepackage{amsmath}
\usepackage{amsfonts}
\usepackage{amssymb}
\usepackage{amsthm}
\usepackage{todonotes}
\usepackage{enumerate}\usepackage[left=2cm,right=2cm,top=2cm,bottom=2cm]{geometry}
\author{Kevin Teugels}
\title{Oefeningen TAI (21-10-2014)}
\date{21 oktober 2014}



\begin{document}
\maketitle
\section*{Oefening 1}

Bewijs dat $\mathbb{R}_0 \times \mathbb{R}$, voorzien van de samenstellingswet $(*):\ \mathbb{R}_{0}\rightarrow \mathhbb{R}:\ ((a,b),(c,d)) \mapsto (ac,bc+d)$ een groep is. Is hij abels?
\begin{proof} 
We bewijzen elke eigenschap in het criterium voor groepen.
\begin{itemize}
  \item associativiteit
  \[
  ((a,b)*(c,d))\times (e,f)=(ac,bc+d)\times (e,f)=(ace,(bc+d)e+f))
  \]
  \[
  (a,b) \times ((c,d) \times (e,f))=(a,b) \times
  (ce,de+f)=(ace,(bce)+(de+f))=(ace,(bc+d)e+f)
  \]
  \item Het neutraal element is $(1,0)$. 
  \item Het invers element van een element $(a,b)$ is  $(\dfrac{1}{a},\dfrac{-b}{a})$.
\end{itemize}
\end{proof}
De groep is niet commutatief, tegenvoorbeeld:
\[
(1,1)*(0,0) = (0,0) \quad\text{ en }\quad (0,0)*(1,1) = (0,1)
\]

\section*{Oefening 2}
\extra{verwijzing naar $\mathcal{S}_n$.}

\section*{oefening 3}
Zoek de generatoren van de volgede additieve cyclische groepen:
\extra{verwijzing naar eigenschap over generatoren van cyclische groepen en hun orde}
\subsection*{(a)}
$\mathbb{Z}_{10}$\\
\[ \{ 1,3,7,9\} \]

\subsection*{(b)}
$\mathbb{Z}_{11}$\\
\[ \{ 1,2,3,4,5,6,7,8,9,10,11 \} \]

\subsection*{(c)}
$\mathbb{Z}_{12}$\\
\[ \{ 1,5,7,9,10,11 \} \]

\section*{oefening 4}

Genereer de groep voorgebracht onder vermenigvuldiging door de matrices.
\[ \left( \begin{array}{cc}
0 & 1 \\
-1 & 0
\end{array} \right)
en
\left( \begin{array}{cc}
0 & 1 \\
1 & 0
\end{array} \right)
\]
Bewijs dat dit een niet abelse groep is van orde 8.
\subsection*{oplossing}
\[
X=
\begin{pmatrix}
0 &1\\
-1&0
\end{pmatrix}
, X^2=
\begin{pmatrix}
-1&0\\
0&-1
\end{pmatrix}
,X^3=
\begin{pmatrix}
0 & -1\\
1&0
\end{pmatrix}
,X^4=
\begin{pmatrix}
1 & 0\\
0 & 1
\end{pmatrix}
\]
\[
Y=
\begin{pmatrix}
0 & 1\\
1 & 0
\end{pmatrix}
,Y^2=
\begin{pmatrix}
1&0\\
0&1
\end{pmatrix}
\]
\[
XY=
\begin{pmatrix}
1 &0\\
0 &-1
\end{pmatrix}
=YX^3,X^2y=
\begin{pmatrix}
0 &- 1\\
-1 &0
\end{pmatrix}
=YX^2,X^3Y=
\begin{pmatrix}
-1 & 0\\
0&1
\end{pmatrix}
=YX
\]
$\rightarrow$ 8 verschillende elementen. Dus het is een groep van orde 8.
\\$\{X,Y,X^2,X^3,X^4,XY,X^2Y,X^3Y\}$
\\Het is geen abelse groep.
\section*{oefening 5}

Beschouw de groep $G=( \mathbb{Z}_7 /\{0\},.)$ van de gehele getallen modulo 7 zonder de nul en met vermenigvuldiging modulo 7.Bepaal de orde van al de elementen. Is de groep Commutatief?
\subsection*{oplossing}
Vb: $\bar{5} \rightarrow$
\begin{itemize}
\item $\bar{5}^2=\bar{25}=\bar{4}$ .
\item $\bar(5)^3=\bar{4}*\bar{5}=\bar{20}=\bar{6}$.
\item ...
\item $\bar{5}^6=\bar{3}*\bar{5}=\bar{15}=\bar{1} \rightarrow$ Orde van 5 is 6.
\end{itemize}
De groep is commutatief want de maal is commutatief.
\pagebreak
\section*{oefening 6}

Bewijs dat elke deelgroep van een cyclische groep cyclisch is.
\subsection*{oplossing}
\begin{proof}
$G=<g>$ is een cyclische groep, met $g \in G$.
\\Stel $H \subset G$.
\\Als $h\{1\}$,dan is $H$ cyclisch met generator $1$ dus we nemen aan dat $H \neq \{1\}$.
\\Bepaal generator van $H$. Omdat $G$ cyclisch is, is elk element een macht van $g$. 
\\Stel het kleinste positieve getal zodat $g^m \in H$.
\subsubsection*{te bewijzen}
$g^n$ generator $H$ maw elke $h \varepsilon H$ is een macht van $g^n \rightarrow \varepsilon H \subset G \Rightarrow h=G^n$ voor een bepaalde n. Er zijn unieke q,r zodat $r=n*q+r,0\leqslant r \leqslant n$ zodat:
\[
h=g^{nq+r}=(g^n)^q*g^r \Rightarrow g^r=h(g^n)^-q
\]
$\rightarrow$ we weten $g^n \varepsilon H \Rightarrow (g^n)^-q \varepsilon H$ dus $h(q^n)^-q \varepsilon H \rightarrow g^r \varepsilon H$
\\niet volledig

\end{proof}
\end{document}