 \documentclass[main.tex]{subfiles}
 \begin{document}


 \chapter{Examenvragen: Algebra I}
 \label{cha:examen-a}

Disclaimer: vertrouw NIET op deze antwoorden.
Ze zijn hoogstens gedeeltelijk juist.

 \section{Snelheidsvraagjes}
 \label{sec:snelheidsvraagjes}

Deze vraagjes worden gesteld tijdens het mondelinge examen en moet u dus met gesloten boek kunnen beantwoorden.

 \begin{itemize}
 \item Waar of niet?
   \begin{center}
     Zij $G,*$ een groep en $A,*$ een normaaldeler van $G,*$ met zowel $A,*$ en $\nicefrac{G}{A},*$ commutatief, dan is $G,*$ commutatief.
   \end{center}

   Niet waar.
   Zowel $\mathcal{A}_{3}$ als $\nicefrac{\mathcal{S}_{3}}{\mathcal{A}_{3}}$ zijn commutatief, maar $\mathcal{S}_{3}$ niet.
 \item Waar of niet?
   \begin{center}
     ``Een groep is eindig als alle elementen eindige orde hebben.''
   \end{center}

   Niet waar.\\
   Tegenvoorbeeld: In $\bigoplus_{i=1}^{\infty}\mathbb{Z}_{2}$ heeft elk element orde $2$.
 \item Waar of niet?
   \begin{center}
     ``Als een groep eindig is, dan hebben al diens elementen een eindige orde.''
   \end{center}
   Waar.\\
   Als een groep eindig is met orde $n$ , hebben alle elementen een deler van $n$ als orde.\gevref{gev:orde-van-element-deelt-orde-van-groep}
 \item Waar of niet?
   \begin{center}
     Zij $R$ een ring met eenheidselement, dan hebben $u$ en $v$ elk een inverse als en slechts als $uw$ een inverse heeft.
   \end{center}
   Wat verandert er als de ring commutatief is?\\
   Waar, maar enkel als de ring commutatief is.
   \begin{proof}
     Bewijs van een equivalentie.
     \begin{itemize}
     \item $\Rightarrow$: Als $u$ en $v$ elk een inverse hebben, dan heeft $uv$ als inverse $v^{-1}u^{-1}$.
     \item $\Leftarrow$: Zij $x$ de inverse van $uv$, dan is $vx$ de inverse van $u$ en (maar dit enkel als $R,+,\cdot$ commutatief is) $ux$ de inverse van $v$.
     \end{itemize}
   \end{proof}
 \item Waar of niet?
   \begin{center}
     In een integriteitsdomein geldt $(R[X])^{\times} = R^{\times}$.
   \end{center}
   Waar.\stref{st:integriteitsdomein-veeltermring-eenheden}
 \item Waar of niet?
   \begin{center}
     Zij $E,+,\cdot$ een velduitbreiding van een veld $K,+,\cdot$. 
     Zij $a$ en $b$ transcendente elementen in $E$ over $K,+,\cdot$
     $a\cdot b$ is transcendent over $K,+,cdot$.
   \end{center}
   Niet waar. Tegenvoorbeeld: $e$ en $\frac{1}{e}$ zij beide transcendent over $\mathbb{Q}$, maar $e \cdot \frac{1}{e}=1$ is algebra\"isch over $\mathbb{Q}$.
 \item Bestaat er een algebra\"isch gesloten veld dat $\mathbb{C}$ strikt omvat?
   Nee, elke algebra\"ische uitbreiding van $\mathbb{C}$ is gelijk aan $\mathbb{C}$.
 \item Geef alle $2\times 2$ matrices oven $\mathbb{C}$ die zowel hermetisch als unitair zijn.
   \[ A = A^{*} \text{ en } A^{-1}=A^{*} \Rightarrow A=A^{-1}\]
   \[ A = 
   \begin{pmatrix}
     \cos\theta & i\sin\theta\\
     -i\sin\theta & -\cos\theta
   \end{pmatrix}
   \vee 
   A =
   \begin{pmatrix}
     \cos\theta & \sin\theta\\
     \sin\theta & -\cos\theta
   \end{pmatrix}
   \vee 
   A = I
   \vee 
   A = -I
   \]
 \item In $\mathbb{Z}$:
   \begin{itemize}
   \item Geef alle idealen.
   \item Geef alle maximale idealen.
   \item Geef alle priemidealen.
   \item Geef de priemdeelring.
   \item Geef de doorsnede van alle deelringen.
   \item Is $(4,7)$ een hoofdideaal?
   \end{itemize}

   Merk eerst op dat $\mathbb{Z}$ een HID is.\stref{st:z-is-hid}

   \begin{itemize}
   \item Elk ideaal van $\mathbb{Z}$ is een hoofdidieaal, en voor elke $n\in \mathbb{Z}$ is $n\mathbb{Z}$ een ideaal van $\mathbb{Z}$.
     $\{ n\mathbb{Z} \mid n \in \mathbb{Z} \}$ zijn dus alle idealen van $\mathbb{Z}$.
   \item De maximale idealen van $\mathbb{Z},+,\cdot$ zijn alle $n\mathbb{Z}$ waarvoor $n$ priem is.\stref{st:hoofdidiaal-van-irreducibele-veelterm-maximaal}
   \item Alle maximale idealen van $\mathbb{Z},+,\cdot$ zijn priemidealen\eiref{ei:maximaal-dan-priemideaal} alsook de nulring.
   \item De priemdeelring van $\mathbb{Z},+,\cdot$ is $\mathbb{Z},+,\cdot$ zelf.\eiref{ei:priemdeelring-voortgebracht-door-eenheidselement}
   \item De doorsnede van alle deelringen van $\mathbb{Z},+,\cdot$ is de nulring.
   \item Ja.
     \[ (4,7) = \{ 4n + 7m \ |\ n,m \in \mathbb{Z}  \} = \mathbb{Z} \]
   \end{itemize}

 \item Welke inclusies gelden er tussen $\mathbb{F}_{2}$, $\mathbb{F}_{4}$, $\mathbb{F}_{6}$ en $\mathbb{F}_{8}$

   Merk op dat $2$, $4$ en $8$ machten zijn van het priemgetal $2$ en $\mathbb{F}_{6}$ bestaat niet.\stref{st:eindig-veld-orde-macht-van-priemgetal} Er gelden dus de volgende inclusies.\prref{pr:veld-ondelinge-inclusies}
   \[ \mathbb{F}_{2} \subsetneq \mathbb{F}_{4} \subsetneq \mathbb{F}_{8} \]

 \item Wat is $\mathbb{F}_{16}\cap \mathbb{F}_{64}$?

   $64=2^{6}$ dus de strikte deelvelden van $\mathbb{F}_{64}$ zijn die met $2^{1}$, $2^{2}$ en $2^{3}$ elementen.
   $16=2^{4}$ dus de strikte deelvelden van $\mathbb{F}_{16}$ zijn die met $2^{1}$ en $2^{2}$ elementen.
   Antwoord: $\mathbb{F}_{4}$.

 \item Geef de grootst gemene deler en het kleinst gemeen veelvoud van $X$ en $Y$ in $\mathbb{R}[X,Y]$

   Antwoord: $1$ en $XY$
   Merk op dat inderdaad $1 | X$ en $1 | Y$ alsook $X | XY$ en $Y | XY$ gelden.

 \item Zij $\mathbb{F}_{q}$ een eindig veld met $q$ elementen.
   Welke groepen $\mathbb{F}_{q},+$ en $\mathbb{F}_{q}^{\times},\cdot$ zijn cyclisch?

   $\mathbb{F}_{q}^{\times},\cdot$ is steeds een cyclische groep.\stref{st:multiplicatieve-groep-van-veld-is-cyclisch}
   $\mathbb{F}_{q},+$ is enkel cyclisch als $\mathbb{F}$ isomorf is met $\mathbb{Z}_{p}$.

 \item Bespreek de inclusies tussen de volgende verzamelingen van $\mathbb{C}^{n \times n}$ matrices:
   \begin{center}
     Diagonaliseerbaar, Unitair, Normaal, Hermetisch
   \end{center}
   \begin{itemize}
   \item 
     Bijna alle matrices in $\mathbb{C}^{n\times n}$ zijn diagonaliseerbaar.
     Er zijn er ook die niet diagonaliseerbaar zijn, maar dat zijn er niet veel.
     \[
     \begin{pmatrix}
       0 & 1\\
       0 & 0
     \end{pmatrix}
     \]
   \item Alle unitaire en hermetische matrices zijn normaal.
   \item Er bestaan matrices die unitair zijn en ook hermetisch. 
   \end{itemize}
 \end{itemize}

 \section{Groepen}
 \label{sec:groepen}

 \begin{itemize}
 \item 
   Zij $p$ een priemgetal.
   \begin{itemize}
   \item Zij $H,+$ een abelse groep van orde $n$ en $p$ een priemfactor van $n$.
     Maak gebruik van de structuurstelling voor eindige groepen om aan te tonen dat $H$ een element heeft van orde $p$.
   \item Zij $G,*$ een groep van orde $p^{r}$ met $r\in \mathbb{N}_{0}$.
     \begin{itemize}
     \item Toon aan met behulp van de eerste deelvraag dat $G$ een element $g$ heeft van orde $p$ en dat de groep voortgebracht door $g$ een normaaldeler is van $G$.
       \[ g \triangleleft G \]
     \item 
       Toon aan met inductie op $r$ aan dat er voor elke $s$ met $0\le s \le r$ een normaaldeler $N$ van $G$ met orde $p^{s}$.
     \end{itemize}
   \end{itemize}
   \begin{itemize}
   \item $H,+$ is vanwege de structuurstelling isomorf met
     $\mathbb{Z}_{p^{r}} \oplus \mathbb{z}_{\nicefrac{n}{p^{r}}}$.  Er
     is in $H$ dus een element dat overeenkomt met $(1,0)$, en dat
     element heeft orde $p$:
   \item
     \begin{itemize}
     \item $p$ is een priemfactor van de orde.
       $G,*$ is een $p$-groep, dus het centrum van $G$ bevat meer dan \'e\'en element\prref{pr:orde-centrum-pgroep-groter-dan-een}
       Het centrum $Z(G)$ is een (commutatieve) deelgroep van $G$\eiref{ei:centrum-is-deelgroep}, dus heeft een orde $p^{x}$ met $1\le x\le r$.\gevref{gev:orde-van-element-deelt-orde-van-groep}
       Er bestaat in $Z(G) \subseteq G$ dus een element $g$ van orde $p$.
       Het element $g$ heeft in $G$ dus ook een orde $p$ en bijgevolg heeft $<g>$ ook orde $p$.\stref{st:orde-van-generator-is-orde-van-groep}
       $<g>$ is een deelgroep van $Z(G)$ dus zeker een normaaldeler.
     \item 
       \begin{itemize}
       \item $r = 1$: OK, zie hierboven.
       \item $r = k$: Inductiehypothese
       \item $r = k+1$:
         $G/N$ is ook een $p$-groep.
         \extra{zit vast}
       \end{itemize}
     \end{itemize}
   \end{itemize}
   
 \item 
   Zij $G$ een groep met $102 = 2 \cdot 3 \cdot 17$ elementen en $|Z(G)|=2$. 
   Toon aan:
   \begin{itemize}
   \item $|Z(\nicefrac{G}{Z(G)})| = 1$
   \item $\nicefrac{G}{Z(G)}$ heeft een deelgroep van orde $17$.
   \item $G$ heeft een deelgroep van orde $34$.
   \end{itemize}

   Antwoord:
   \begin{itemize}
   \item
     We moeten aantonen dat er maar $1$ element uit $\nicefrac{G}{Z(G)}$ commuteert met alle elementen in $\nicefrac{G}{Z(G)}$.
     Enkel de elementen uit $Z(G)$ commuteren met alle elementen in $G$, dus enkel de elementen $zZ(G)$ uit $\nicefrac{G}{Z(G)}$ met $z\in Z(G)$ commuteren met alle elementen in $\nicefrac{G}{Z(G)}$.
     Omdat $zZ(G)$ gelijk is aan $Z(G)$ voor alle $z\in Z(G)$ is er dus maar \'e\'en element in $\nicefrac{G}{Z(G)}$ dat commuteert met alle elementen in $\nicefrac{G}{Z(G)}$.
   \item 
     Omdat $|Z(G)|$ twee is, geldt $|\nicefrac{G}{Z(G)}| = 3 \cdot 17$.\stref{st:stelling-van-lagrange}
     Volgens de structuurstelling is $\nicefrac{G}{Z(G)}$ isomorf met $\mathbb{Z}_{3} \oplus \mathbb{Z}_{17}$. De groep $\{ (0,n) \mid n \in \mathbb{Z}_{17} \}$ is nu een deelgroep van $\mathbb{Z}_{3} \oplus \mathbb{Z}_{17}$ van orde $17$, dus $\nicefrac{G}{Z(G)}$ heeft een overeenkomstige deelgroep van orde $17$.

   \item Vanwege dezelfde redenering bevat $G$ een deelgroep van orde $34$.
     Deze komt dan overeen met de deelgroep $\{ (x,0,y) \mid x\in \mathbb{Z}_{2}, y \in \mathbb{Z}_{17}\}$ van $\mathbb{Z}_{2} \oplus \mathbb{Z}_{3} \oplus \mathbb{Z}_{17}$ van orde $34$.
   \end{itemize}

 \end{itemize}

 \section{Ringen}
 \label{sec:ringen}

 \begin{itemize}
 \item Zij $R,+,\cdot$ een niet-triviale commutatieve ring met eenheidselement $i$ en nulelement $e$ waarin het volgende geldt:
   \[ \forall a\in R,\ \exists b\in R:\ a^{2}b=a \]
   Toon aan dat elk priemideaal van $R,+,\cdot$ ook een maximaal ideaal van $R,+,\cdot$.

   Antwoord:
   Zij $I,+,\cdot$ een priemideaal van $R,+,\cdot$, dan is $R/I$ een integriteitsdomein.
   In $R/I$ geldt hetzelfde als in $R$:
   \[ a^{2}b=a \Rightarrow a(ab-i)=e \]
   Omdat $R/I$ een domein is moeten $a$ en $b$ inversen zijn (als $a$ niet $e$ is).
   Een integriteitsdomein $R/I$ met enkel eenheden is een veld, dus $I$ is een maximaal ideaal.\stref{st:quotientgroepen-asa-idealen-stelling}
 \item Zij $R,+,\cdot$ een HID, toon dan aan:
   \begin{itemize}
   \item Een priemideaal van $R,+,\cdot$ is een maximaal ideaal.
   \item Zij $f$ een surjectief ringmorfismme van $R$ naar $S$ met $S,\star,*$ een integriteitsdomein, dan is ofwel $f$ een isomorfisme, ofwel $S,\star,*$ een veld.
   \item Als $R[X],+,\cdot$ een HID is, dan moet $R,+,\cdot$ een veld zijn.
   \end{itemize}

   Antwoord.
   \begin{itemize}
   \item 
     Kies een willekeurig priemideaal $P$ van $R,+,\cdot$.
     Omdat $R,+,\cdot$ een HID is bestaat er dan een element $p$ dat $P$ genereert:
     \[ P = (p) \]
     Het element dat $P$ genereert moet irreducibel zijn, want $P$ is een priemideaal.\stref{st:priemideaal-gegenereert-door-priem-in-hid}
     Het hoofdideaal $(p)$ gegenereerd door $p$ is dan zeker maximaal. \stref{st:hoofdidiaal-van-irreducibele-veelterm-maximaal}
   \end{itemize}
 \end{itemize}

 \section{Velden}
 \label{sec:velden}

 \begin{itemize}
 \item Bereken $[\mathbb{Q}(\sqrt{2},\sqrt[3]{4}):\mathbb{Q}]$.
   \begin{itemize}
   \item $[\mathbb{Q}(\sqrt{2}):\mathbb{Q}] = 2$ want $X^{2}-2$ is de minimale veelterm van $\sqrt{2}$ over $\mathbb{Q}$.
   \item $[\mathbb{Q}(\sqrt[3]{4}):\mathbb{Q}] = 3$ want $X^{3}-2$ is de minimale veelterm van $\sqrt{3}$ over $\mathbb{Q}$.
   \end{itemize}
   Omdat $2$ en $3$ relatief priem zijn kunnen we deze gewoon vermenigvuldigen.
   Het antwoord is dus $6$.
 \end{itemize}


 \section{Lineaire algebra}
 \label{sec:lineaire-algebra}

 \begin{itemize}
 \item Zij $K,+,\cdot$ een veld en $A$ een nilpotente transformatie van een $K$-vectorruimte van index $k$.
   Bewijs het volgende:
   \[ \{0\} \subsetneq Ker(A) \subsetneq Ker(A)^{2} \subsetneq \dotsb \subsetneq Ker(A^{k-1}) \subsetneq Ker(A)^{k} = V \]
   \extra{examenvraag}
 \item Gegeven is de lineaire afbeelding $\mathcal{A}:\ \mathbb{C}^{6} \rightarrow \mathbb{C}^{6}$ met $A$ als matrix ten opzichte van de standaarbasis van $\mathbb{C}$.
   \[
   \begin{pmatrix}
     2 & -1 & 0 & 1 & 0 & 0\\
     0 & 2 & 0 & 0 & 0 & 0\\
     0 & -1 & 2 & 0 & 0 & 0\\
     -1 & 0 & 1 & 4 & 1 & 1\\
     1 & 0 & -1 & -1 & 1 & -1\\
     0 & 0 & 0 & 0 & 0 & 2
   \end{pmatrix}
   \]
   Bepaal de jordanvorm $J$ van $A$ en geef de $P$ zodat $J=P^{-1}AP$ geldt.
   Geef ook de karakteristieke veelterm en de minimale veelterm van $A$.

   \begin{itemize}
   \item
     \[
     \begin{array}{l}
       \begin{vmatrix}
         X-2 & 1 & 0 & -1 & 0 & 0\\
         0 & X-2 & 0 & 0 & 0 & 0\\
         0 & 1 & X-2 & 0 & 0 & 0\\
         1 & 0 & -1 & X-4 & -1 & -1\\
         -1 & 0 & 1 & 1 & X-1 & 1\\
         0 & 0 & 0 & 0 & 0 & X-2
       \end{vmatrix}\\
       =
       (X-2)
       \begin{vmatrix}
         X-2 & 1 & 0 & -1 & 0\\
         0 & X-2 & 0 & 0 & 0\\
         0 & 1 & X-2 & 0 & 0\\
         1 & 0 & -1 & X-4 & -1\\
         -1 & 0 & 1 & 1 & X-1
       \end{vmatrix}\\
       =
       (X-2)^{2}
       \begin{vmatrix}
         X-2 & 0 & -1 & 0\\
         0 & X-2 & 0 & 0\\
         1 & -1 & X-4 & -1\\
         -1 & 1 & 1 & X-1
       \end{vmatrix}\\
       =
       (X-2)^{3}
       \begin{vmatrix}
         X-2 & -1 & 0\\
         1 & X-4 & -1\\
         -1 & 1 & X-1
       \end{vmatrix}\\
       =
       (X-2)^{3}
       \left(
         (X-2)
         \begin{vmatrix}
           X-4 & -1\\
           1 & X-1
         \end{vmatrix}
         +
         \begin{vmatrix}
           1 & -1\\
           -1 & X-1
         \end{vmatrix}
       \right)\\
       =
       (X-2)^{3}
       \left(
         (X-2)\left((X-4)(X-1) +1\right)
         + ((X-1)-1)
       \right)\\
       =
       (X-2)^{3}
       \left(
         (X-2)\left(X^{2}-5X+5\right)
         + ((X-2)
       \right)\\
       =
       (X-2)^{4}
       \left(X^{2}-5X+6\right)\\
       =
       (X-2)^{5}(X-3)\\
     \end{array}
     \]
   \item De vectorruimte valt uiteen in twee componenten:
     \[ \mathbb{C} \cong Ker(A-2I)^{p_{1}} \oplus Ker(A-3I)^{p_{2}} \]
     \[ dim(V_{1}) = 5,\ dim(V_{2})=1 \]
     \begin{itemize}
     \item $V_{1}$: $p_{1}=2$.
       \[
       dim(A-2I) = dim 
       \begin{pmatrix}
         0 & -1 & 0 & 1 & 0 & 0\\
         0 & 0 & 0 & 0 & 0 & 0\\
         0 & -1 & 0 & 0 & 0 & 0\\
         -1 & 0 & 1 & 2 & 1 & 1\\
         1 & 0 & -1 & -1 & -1 & -1\\
         0 & 0 & 0 & 0 & 0 & 0\\
       \end{pmatrix}
       =3 \Rightarrow dim(A-2I) = 3
       \]
       \[ dim(A-2I)^{2} = dim
       \begin{pmatrix}
         -1 & 0 & 1 & 2 & 1 & 1\\
         0 & 0 & 0 & 0 & 0 & 0\\
         0 & 0 & 0 & 0 & 0 & 0\\
         -1 & 0 & 1 & 2 & 1 & 1\\
         0 & 0 & 0 & 0 & 0 & 0\\
         0 & 0 & 0 & 0 & 0 & 0\\
       \end{pmatrix}
       =1 \Rightarrow dim(A-2I)^{2} = 1 \Rightarrow dim(Ker(A-2I)^{2}) = 5
       \]
     \item $V_{2}$: $p=1$
       \[
       dim(A-3I) = dim
       \begin{pmatrix}
         -1 & -1 & 0 & 1 & 0 & 0\\
         0 & -1 & 0 & 0 & 0 & 0\\
         0 & -1 & -1 & 0 & 0 & 0\\
         -1 & 0 & 1 & 1 & 1 & 1\\
         1 & 0 & -1 & -1 & -2 & -1\\
         0 & 0 & 0 & 0 & -1
       \end{pmatrix}
       =5 \Rightarrow dim(A-2I) = 1
       \]
     \end{itemize}
   \item Diagram:
     \begin{itemize}
     \item $V_{1}$: $d_{1} = 3$, $d_{2} = 2$
       \[
       \begin{array}{ccc}
         \boxed{v_{2}} & \boxed{v_{4}} & \boxed{v_{5}}\\
         \boxed{v_{1}} & \boxed{v_{3}}
       \end{array}
       \]
       Kies bijvoorbeeld de vectoren als volgt:
       \[
       \begin{array}{ccc}
         \boxed{(-1,0,-1,0,0,0)} & \boxed{(0,0,0,-1,1,0)} & \boxed{(1,0,0,0,0,1)}\\
         \boxed{(0,1,0,0,0,0)} & \boxed{(1,0,-1,0,1,0)}
       \end{array}
       \]
     \item $V_{2}$: $d_{1}=1$
       \[
       \begin{array}{ccc}
         \boxed{(1,0,0,1,0,0)}
       \end{array}
       \]

     \item 
       \[ P = 
       \left(
         \begin{array}{ccccc|c}
           0 & -1 & 1 & 0 & 1 & 1\\
           1 & 0 & 0 & 0 & 0 & 0\\
           0 & -1 & -1 & 0 & 0 & 0\\
           0 & 0 & 0 & -1 & 0 & 1\\
           0 & 0 & 1 & 1 & 0 & 0 \\
           0 & 0 & 0 & 0 & 0 & 1
         \end{array}
       \right)
       \]
     \item 
       \[
       J = 
       \left(
         \begin{array}{ccccc|c}
           2 & 0 & 0 & 0 & 0 & 0\\
           1 & 2 & 0 & 0 & 0 & 0\\
           0 & 0 & 2 & 0 & 0 & 0\\
           0 & 0 & 1 & 2 & 0 & 0\\
           0 & 0 & 0 & 0 & 2 & 0\\\hline
           0 & 0 & 0 & 0 & 0 & 3
         \end{array}
       \right)
       \]
     \end{itemize}
   \end{itemize}
 \item Zij $A\in \mathbb{C}^{3\times 3}$ een diagonaliseerbare matrix met $3$ verschillende eigenwaarden $\lambda_{1}$, $\lambda_{2}$ en $\lambda_{3}$ en bijhorende eigenvectoren $v_{1}$,$v_{2}$ en $v_{3}$.
   Beschouw $C \in \mathbb{C}^{6\times 6}$.
   \[ C =
   \begin{pmatrix}
     A & I \\
     0 & A
   \end{pmatrix}
   \]
   Bepaal de jordanvorm $J$ van $A$ en geef de $P$ zodat $J=P^{-1}AP$ geldt.
 \item Zij $A\in \mathbb{C}^{n \times n}$ een matrix en zij $f_{A} = (X-i)\phi_{A}$ en $\phi_{A}^{2} = (X^{2}+1)f_{a}$.
   Bereken $f_{A}$ en $\phi_{A}$ en geef de Jordanvorm van $A$.
 \item Voor welke $n$ bestaat er een matrix $M\in \mathbb{C}^{n\times n}$ zodat de minimale veelterm van $M$ $(x-1)(x-2)^{2}(x-3)^{3}$ is?
 \item Zij $A$ een nilpotente matrix in $\mathbb{C}^{6\times 6}$ zodat $dim(Ker(A^{2}))=4$ geldt, geef alle mogelijke invariante systemen van $A$. met karakteristieke veelterm $f_{A}(X)=X^{2}+aX+b$ waarbij de discriminant $a^{2}-4b$ kleiner is dan nul.
   Toon aan dat er een basis $\beta$ van $\mathbb{R}^{2}$ bestaat zodat $A$ de volgende matrix krijgt ten opzichte van $\beta$.
 \end{itemize}


\end{document}
