\documentclass[main.tex]{subfiles}
\begin{document}


\chapter{Examenvragen: Algebra I}
\label{cha:examen-a}

\section{Snelheidsvraagjes}
\label{sec:snelheidsvraagjes}

\begin{itemize}
\item Waar of niet?
  \begin{center}
    ``Een groep is eindig als alle elementen eindige orde hebben.''
  \end{center}

  Niet waar.\\
  Tegenvoorbeeld: In $\bigoplus_{i=1}^{\infty}\mathbb{Z}_{2}$ heeft elk element orde $2$.
\item Waar of niet?
  \begin{center}
    ``Als een groep eindig is, dan hebben al diens elementen een eindige orde.''
  \end{center}
  Waar.\\
  Als een groep eindig is met orde $n$ , hebben alle elementen een deler van $n$ als orde.\gevref{gev:orde-van-element-deelt-orde-van-groep}
\item Waar of niet?
  \begin{center}
    Zij $R$ een ring met eenheidselement, dan hebben $u$ en $v$ elk een inverse als en slechts als $uw$ een inverse heeft.
  \end{center}
  Wat verandert er als de ring commutatief is?
\item Waar of niet?
  \begin{center}
    In een integriteitsdomein geldt $(R[X])^{\times} = R^{\times}$.
  \end{center}
\extra{antwoord?}
\item Waar of niet?
  \begin{center}
    Zij $E,+,\cdot$ een velduitbreiding van een veld $K,+,\cdot$. 
    Zij $a$ en $b$ transcendente elementen in $E$ over $K,+,\cdot$
    $a\cdot b$ is transcendent over $K,+,cdot$.
  \end{center}
  Niet waar. Tegenvoorbeeld: $e$ en $\frac{1}{e}$ zij beide transcendent over $\mathbb{Q}$, maar $e \cdot \frac{1}{e}=1$ is algebra\"isch over $\mathbb{Q}$.
\item Bestaat er een algebra\"isch gesloten veld dat $\mathbb{C}$ strikt omvat?
\item Geef alle $2\times 2$ matrices oven $\mathbb{C}$ die zowel hermetisch als unitair zijn.
\[ A = A^{*} \text{ en } A^{-1}=A^{*} \Rightarrow A=A^{-1}\]
\[ A = 
\begin{pmatrix}
  \cos\theta & i\sin\theta\\
  -i\sin\theta & -\cos\theta
\end{pmatrix}
\vee 
A =
\begin{pmatrix}
  \cos\theta & \sin\theta\\
  \sin\theta & -\cos\theta
\end{pmatrix}
\vee 
A = I
\vee 
A = -I
\]
\item Geef alle maximale idealen in $\mathbb{C}[X]$.
\extra{antwoord?}
\item Welke inclusies gelden er tussen $\mathbb{F}_{2}$, $\mathbb{F}_{4}$, $\mathbb{F}_{6}$ en $\mathbb{F}_{8}$
\extra{antwoord?}
\item Wat is $\mathbb{F}_{16}\cap \mathbb{F}_{64}$?
$64=2^{6}$ dus de strikte deelvelden van $\mathbb{F}_{64}$ zijn die met $2^{1}$, $2^{2}$ en $2^{3}$ elementen.
$16=2^{4}$ dus de strikte deelvelden van $\mathbb{F}_{16}$ zijn die met $2^{1}$ en $2^{2}$ elementen.
Antwoord: $\mathbb{F}_{4}$.
\item Geef de grootst gemene deler van $X$ en $Y$ in $\mathbb{R}[X,Y]$
\extra{antwoord?}
\item Zij $\mathbb{F}_{q}$ een eindig veld met $q$ elementen.
  Welke groepen $\mathbb{F}_{q},+$ en $\mathbb{F}_{q}^{\times},\cdot$ zijn cyclisch?
\extra{antwoord?}
\item Geef alle idealen van $\mathbb{Z}$.
\extra{antwoord?}
\item Als $\mathbb{Z}$ een $HID$ is, is $(4,7)$ dan ook een hoofdideaal?
\extra{antwoord?}
\item Wat is de priemdeelring van $\mathbb{Z}$?
\extra{antwoord?}
\item Wat is de doorsnede van alle deelringen van $\mathbb{Z}$?
\extra{antwoord?}
\item Wat zijn de maximale idealen in $\mathbb{Z}$?
\extra{antwoord?}
\item Bespreek de inclusies tussen de volgende verzamelingen van $\mathbb{C}^{n \times n}$ matrices:
  \begin{center}
    Diagonaliseerbaar, Unitair, Normaal, Hermetisch
  \end{center}
\extra{antwoord?}
\end{itemize}

\section{Lineaire algebra}
\label{sec:lineaire-algebra}

\begin{itemize}
\item Zij $K,+,\cdot$ een veld en $A$ een nilpotente transformatie van een $K$-vectorruimte van index $k$.
  Bewijs het volgende:
  \[ \{0\} \subsetneq Ker(A) \subsetneq Ker(A)^{2} \subsetneq \dotsb \subsetneq Ker(A^{k-1}) \subsetneq Ker(A)^{k} = V \]
\extra{examenvraag}
\item Gegeven is de lineaire afbeelding $\mathcal{A}:\ \mathbb{C}^{6} \rightarrow \mathbb{C}^{6}$ met $A$ als matrix ten opzichte van de standaarbasis van $\mathbb{C}$.
  \[
  \begin{pmatrix}
    2 & -1 & 0 & 1 & 0 & 0\\
    0 & 2 & 0 & 0 & 0 & 0\\
    0 & -1 & 2 & 0 & 0 & 0\\
    -1 & 0 & 1 & 4 & 1 & 1\\
    1 & 0 & -1 & -1 & 1 & -1\\
     0 & 0 & 0 & 0 & 0 & 2
  \end{pmatrix}
  \]
  Bepaal de jordanvorm $J$ van $A$ en geef de $P$ zodat $J=P^{-1}AP$ geldt.
  Geef ook de karakteristieke veelterm en de minimale veelterm van $A$.
\item Gegeven is de lineaire afbeelding $\mathcal{A}:\ \mathbb{C}^{6} \rightarrow \mathbb{C}^{6}$ met $A$ als matrix ten opzichte van de standaarbasis van $\mathbb{C}$.
  \[
  \begin{pmatrix}
      0 & 0 & 0 & 0 & 1\\
      2 & 1 & -1 & -1 & -1\\
      0 & 0 & 2 & 1 &0\\
      0 & 0 & 0 & 2 & 0\\
      -1 & 0 & 0 & 0 & 2
  \end{pmatrix}
  \]
  Bepaal de jordanvorm $J$ van $A$ en geef de $P$ zodat $J=P^{-1}AP$ geldt.
\item Zij $A\in \mathbb{C}^{3\times 3}$ een diagonaliseerbare matrix met $3$ verschillende eigenwaarden $\lambda_{1}$, $\lambda_{2}$ en $\lambda_{3}$ en bijhorende eigenvectoren $v_{1}$,$v_{2}$ en $v_{3}$.
  Beschouw $C \in \mathbb{C}^{6\times 6}$.
  \[ C =
  \begin{pmatrix}
    A & I \\
    0 & A
  \end{pmatrix}
  \]
  Bepaal de jordanvorm $J$ van $A$ en geef de $P$ zodat $J=P^{-1}AP$ geldt.
\item Zij $A\in \mathbb{C}^{n \times n}$ een matrix en zij $f_{A} = (X-i)\phi_{A}$ en $\phi_{A}^{2} = (X^{2}+1)f_{a}$.
  Bereken $f_{A}$ en $\phi_{A}$ en geef de Jordanvorm van $A$.
\item Gegeven is de lineaire afbeelding $\mathcal{A}:\ \mathbb{C}^{6} \rightarrow \mathbb{C}^{6}$ met $A$ als matrix ten opzichte van de standaarbasis van $\mathbb{C}$.
  \[
  \begin{pmatrix}
      1 & 0 & 0 & 0 & 0\\
      1 & 2 & -1 & 1 & 1\\
      -2 & 0 & 3 & -1 &0\\
      -1 & 0 & 1 & 1 & 0\\
      0 & 0 & 0 & 0 & 2
  \end{pmatrix}
  \]
  Bepaal de jordanvorm $J$ van $A$ en geef de $P$ zodat $J=P^{-1}AP$ geldt.
\item Gegeven is de lineaire afbeelding $\mathcal{A}:\ \mathbb{C}^{6} \rightarrow \mathbb{C}^{6}$ met $A$ als matrix ten opzichte van de standaarbasis van $\mathbb{C}$.
  \[
  \begin{pmatrix}
    i & 0 & -2\\
    0 & i+\frac{1}{2} & \frac{1}{2}\\
    0 & -\frac{1}{2} & -\frac{1}{2}+i
  \end{pmatrix}
  \]
  Bepaal de jordanvorm $J$ van $A$ en geef de $P$ zodat $J=P^{-1}AP$ geldt.
\item Voor welke $n$ bestaat er een matrix $M\in \mathbb{C}^{n\times n}$ zodat de minimale veelterm van $M$ $(x-1)(x-2)^{2}(x-3)^{3}$ is?
\item Zij $A$ een nilpotente matrix in $\mathbb{C}^{6\times 6}$ zodat $dim(Ker(A^{2}))=4$ geldt, geef alle mogelijke invariante systemen van $A$.
\item Gegeven is de lineaire afbeelding $\mathcal{A}:\ \mathbb{C}^{6} \rightarrow \mathbb{C}^{6}$ met $A$ als matrix ten opzichte van de standaarbasis van $\mathbb{C}$.
  \[
  \begin{pmatrix}
    0 & 1 & 2 & 0\\
    -2 & 4 & 1 & 0\\
    -2 & 1 & 4  &0\\
    -2 & 0 & 3 &3
  \end{pmatrix}
  \]
  Bepaal de jordanvorm $J$ van $A$ en geef de $P$ zodat $J=P^{-1}AP$ geldt.
\item Zij $A:\ \mathbb{R}^{2}\rightarrow \mathbb{R}^{2}$ een lineaire afbeelding met karakteristieke veelterm $f_{A}(X)=X^{2}+aX+b$ waarbij de discriminant $a^{2}-4b$ kleiner is dan nul.
  Toon aan dat er een basis $\beta$ van $\mathbb{R}^{2}$ bestaat zodat $A$ de volgende matrix krijgt ten opzichte van $\beta$.
\item Gegeven is de lineaire afbeelding $\mathcal{A}:\ \mathbb{C}^{6} \rightarrow \mathbb{C}^{6}$ met $A$ als matrix ten opzichte van de standaarbasis van $\mathbb{C}$.
  \[
  \begin{pmatrix}
    2 & 1 & 0 & -1\\
    0 & 3 & 0 & 0\\
    -3 & 1 & 1 &-5\\
    1 & -1 & 0 & 4
  \end{pmatrix}
  \]
  Bepaal de jordanvorm $J$ van $A$ en geef de $P$ zodat $J=P^{-1}AP$ geldt.
\end{itemize}


\end{document}
