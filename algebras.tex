\documentclass[main.tex]{subfiles}
\begin{document}

\chapter{Algebras}
\label{cha:algebras}

\begin{de}
  \label{de:algebra}
  Een \term{algebra\"ische structuur} of \term{algebra} is een verzameling $A$ waarop een aantal inwendige (en eventueel een aantal uitwendige) bewerkingen gedefinieerd zijn.
  \[ A,\top_{1},dotsc,\top_{m},\bot_{1},\dotsc,\bot_{n} \]
\end{de}

\begin{de}
  Een algebra $B$ is een \term{deelalgebra} of \term{subalgebra} van een algebra $A$ als volgende beweringen gelden:
  \begin{itemize}
  \item $B \subseteq A$: $B$ is een deelverzameling van $A$.
  \item Op $B$ zijn dezelfde bewerkingen gedefineerd.
  \item $B$ is stabiel voor de interne bewerkingen van $A$.
  \end{itemize}
\end{de}

\begin{de}
  Zij $A$ een algebra en $\sim$ een equivalentierelatie verenigbaar met de bewerkingen van $A$, dan vormt de quotientverzameling $A/R$ voorzien van de quotientbewerking de quotientalgebra van $A$ door $R$.
\end{de}

\section{Morfismen}
\label{sec:morfismen}

\begin{de}
  Twee algebras $A$ en $B$ zijn \term{homoloog} als volgende beweringen gelden:
  \begin{itemize}
  \item Met iedere inwendige bewerking op $A$ komt een inwendige bewerking op $B$ overeen. 
  \item Met iedere uitwendige bewerking op $A$ komt een uitwendige bewerking op $B$ overeen.
  \item De operatorengebieden voor $A$ zijn dezelfde als de operatorengebieden voor $B$.
  \end{itemize}
\end{de}

\begin{de}
  \label{de:morfisme}
  Zij $A,\top,\dotsc,\bot,\dotsc$ en $B,\top',\dotsc,\bot',\dotsc$ twee homologe algebras.
  Een \term{homomorfisme} tussen $A$ en $B$ is een afbeelding $f$ met de volgende eigenschappen:
  \begin{itemize}
  \item Voor elke inwendige bewerking:
    \[ \forall x,y \in A:\ f(x\top y) = f(x) \top f(y) \]  
  \item Voor elke uitwendige bewerking:
    \[ \forall x \in A,\forall \alpha\in\Omega:\ f(\alpha\bot x) = \alpha\bot f(x) \]  
  \end{itemize}
\end{de}

\begin{de}
  Een bijectief homomorfisme is een isomorfisme.
  ``$A$ is isomorf met $B$'' noteren we als volgt:
  \[ G \cong H \]
\end{de}

\begin{de}
  Een homomorfisme van een algebra met zichzelf is een endomorfisme.
\end{de}

\begin{de}
  Een isomorfisme van een algebra met zichzelf is een automorfisme.
\end{de}

\end{document}
