\documentclass[main.tex]{subfiles}
\begin{document}

\chapter{Codetheorie}
\label{cha:codetheorie}

\section{Probleemstelling}

We willen een boodschap versturen over een niet-foutloos kanaal zonder fouten.
We ontwikkelen daarvoor een encodering van de boodschap die ons toelaat om de fouten op te sporen en/of te verbeteren.

\begin{figure}[H]
  \centering
  \begin{tabular}{cl}
    Boodschap\\
    $\downarrow$\\
    encoder \\
    $\downarrow$\\
    geencodeerde boodschap\\
    $\downarrow$\\
    kanaal &$\leftarrow$ fouten\\
    $\downarrow$\\
    ontvangen boodschap\\
    $\downarrow$\\
    gedecodeerde boodschap
  \end{tabular}
  \caption{Probleemstelling}
  \label{fig:probleemstelling}
\end{figure}

\section{Begrippen}

\begin{de}
  Een \term{boodschap} van $n$ bits is een opeenvolging van $n$ elementen van een eindig veld $\mathbb{F}_{q}$ (meestal $\mathbb{Z}_{2}$).
  Het is dus een $n$-tal uit $\mathbb{F}_{q}^{n}$.
  We noemen $n$ de \term{lengte} van de boodschap
\end{de}


\begin{de}
  Een \term{blokcode} $\mathbb{C}$ is een deelverzameling van de $n$-tallen over een eindig veld $\mathbb{F}_{q}$.
  \[ \mathbb{C} \subseteq \mathbb{F}_{q}^{n} \]
\end{de}

\begin{de}
  \label{de:code}
  Een $(n,k)$-\term{code} is een afbeelding $c$ die een boodschap van $k$ bits afbeeldt op een boodschap van $n$ bits.
  \[ c:\ \mathbb{F}_{q}^{k} \rightarrow \mathbb{F}_{q}^{n}:\ b \mapsto c(b) \]

  \begin{figure}[H]
    \centering
    \begin{tikzpicture}
      \tikzset{
        elps/.style 2 args={draw,ellipse,minimum width=#1,minimum height=#2},
        node distance=3cm,
        font=\footnotesize,
        >=latex,
      }
      \node(x)[elps={1.3cm}{1cm},label={below left:$\mathbb{Z}_{2}^{k}$}]{};
      \node(y)[elps={2cm}{1.2cm},right=of x,label={below left:$\mathbb{Z}_{2}^{n}$}]{};
      \fill[gray!50]($(y.center)-(5pt,5pt)$)circle[x radius=.7cm,y radius=.3cm]coordinate(im);
      \node at (im){$\mathrm{Im}(c)$};
      \draw[->](x)to[bend left=20]node[above]{$c$}(y);
    \end{tikzpicture}
    \caption{Een code, schematisch voorgesteld}
    \label{fig:code}
  \end{figure}
  Een code is nooit surjectief, en steeds injectief.
  Dit houdt in dat $n$ groter is dan $k$.
\end{de}

\begin{opm}
  We identificeren een code vaak met de haar beeld (een blokcode).
\end{opm}

\begin{de}
  Een \term{codewoord} is een boodschap $c(b)$ in het beeld van een code $c$.
\end{de}

\begin{de}
  Een \term{ontvangen woord} $w$ van een codewoord $c(b)$ van een boodschap $b$ met een code $c$ is een (eventuele) afwijking van dat codewoord.
  \[ w = c(b) + e \]
  We noemen $e$ de \term{fout} op het ontvangen woord.
\end{de}

\begin{opm}
  Een ontvangen woord hoeft dus niet tot $Im(c)$ te behoren.
\end{opm}

\begin{de}
  Een $n$-tal uit $\mathbb{F}_{q}^{n}$ zullen we in deze contex vaak noteren als een opeenvolging van de meest eenvoudige overeenkomstige representanten in $\mathbb{Z}_{q}$.
  Het heeft op die manier wel degelijk zin om te 'rekenen' met boodschappen.
\end{de}

\section{Foutdetectie}

\begin{de}
  Het \term{Hamming gewicht} $wt(x)$ van element uit $\mathbb{F}_{q}^{n}$ is het aantal plaatsen in het $n$-tal dat niet het nulelement is.
\end{de}

\begin{de}
  De \term{Hamming afstand} $dt(x,y)$ tussen twee elementen uit $\mathbb{F}_{q}^{n}$ is het aantal plaatsen waar de $n$-tallen verschillen.
\end{de}

\begin{ei}
  De Hamming afstand tussen twee $n$-tallen is het Hamming gewicht van het verschil tussen de $n$-tallen.
  \[ \forall x,y \in \mathbb{F}_{q}^{n}:\ dt(x,y) = wt(x-y) \]
\end{ei}

\begin{de}
  De \term{complete maximum likelihood decoding} (\term{CMLD}) van een ontvangen woord $w$ resulteert in het codewoord $c$ dat ten opzichte van $w$ de kleinste Hamming afstand heeft.
\end{de}


\begin{de}
  Zij $\mathbb{C}$ een $(n,k)$-code over $\mathbb{F}_{q}$.
  Zij $c\in \mathbb{C}$ een verzonden boodschap en $w$ het ontvangen woord.
  $e= w-c$ is dan de fout.
  We zeggen dat $\mathbb{C}$ de fout $e$ \term{detecteert} als $e$ niet het verschil is tussen twee woorden uit $\mathbb{C}$:
  \[ \forall c \in \mathbb{C}:\ c+e \not\in \mathbb{C} \]
\end{de}

\begin{opm}
  Intu\"itief is dit een passende definitie, dit immers is ook wat we (als mensen) doen als we een fout zien.
  Als we bijvoorbeel het woord ``bat'' ontvangen terwijl er ``cat'' verstuurd werd, zullen we dit niet kunnen detecteren (zonder context) omdat ``bat'' een geldig woord is.
  Een fout ontvangen woord als ``cxt'' zullen we wel kunnen detecteren.
\end{opm}

\begin{de}
  De \term{afstand van een code} $\mathbb{C}$ is de kleinste afstand tussen de codewoorden.
  \[ d = \min_{c_{1}\neq c_{2}}\ dt(c_{1},c_{2}) \]
\end{de}

\begin{st}
  Een blokcode kan $s$ fouten detecteren als en slechts als de afstand ervan groter is dan $s+1$.
  
  \begin{proof}
    Bewijs van een equivalentie
    \begin{itemize}
    \item $\Rightarrow$\\
      $s$ fouten houdt in dat elke fout $e$ een gewicht kleiner of gelijk aan $s$ heeft.
      De code kan $e$ detecteren als en slechts als de minimale afstand tussun twee codewoorden groter is dan het gewicht van $e$.
      Dit is precies wanneer de afstand van de code $1$ meer is dan $s$.
    \item $\Leftarrow$\\
      Zij $e$ een foutpatroon van gewicht kleiner of gelijk aan $s$ en $d$ de afstand van de code met $d> s+1$.
      De afstand tussen $c$ en $c+e$ is precies $s$, maar $s$ is kleiner dan $d-1$, dus $\mathbb{C}$ detecteert $e$.
    \end{itemize}
  \end{proof}
\end{st}

\begin{gev}
  Een blokcode van gewicht $d$ kan $d-1$ fouten detecteren.
\end{gev}

\section{Foutverbetering}

\begin{de}
  Een blokcode $\mathbb{C}$ \term{verbetert} foutpatroon $e$ als het volgende geldt:
  \[ \exists c \in \mathbb{C},\ \forall c' \in \mathbb{C},\ c' \neq c \wedge dt(c+e,c) \le dt(c+e,c') \]
  Met andere woorden, vanuit een ontvangen woord kan afgeleid worden welk de verstuurde boodschap was.
\end{de}

\begin{de}
  We noemen een code $t$-\term{foutenverbeterend} als de code all foutpatronen met gewicht $t$ kan verbeteren, maar niet alle foutpatronen van gewicht $t+1$.
\end{de}

\begin{st}
  Een blokcode kan $t$ fouten verbeteren als en slecht als het gewicht ervan groter is dan $2t+1$.

  \begin{proof}
    Bewijs van een equivalentie
    \begin{itemize}
    \item $\Rightarrow$\\
    \item $\Leftarrow$\\
    \end{itemize}
  \end{proof}
\TODO{bewijs p 38}
\end{st}

\begin{gev}
  Een blokcode van gewicht $d$ kan $\frac{d-1}{2}$ fouten verbeteren.
\end{gev}

\section{Lineaire blokcodes}

\begin{de}
  We noemen een blokcode $\mathbb{C}$ (als deelverzameling) van $\mathbb{F}_{q}^{n}$ een \term{lineaire blokcode} als $\mathbb{C}$ een lineaire deelruimte is van de vectorruimte $\mathbb{F}_{q}^{n}$.
\end{de}

\begin{opm}
  Dit houdt in dat de som en het scalair product van codewoorden opnieuw een codewoord is.
\end{opm}

\begin{st}
  De afstand van een lineaire blokcode is het kleinste gewicht van een codewoord.
  \[ d = \min_{c\in \mathbb{C}, c\neq 0} wt(c) \]
\TODO{bewijs p 39}
\end{st}

\subsection{Matrixbeschrijving van een lineaire blokcode}

\begin{de}
  Vermits een lineaire blokcode een deelruimte is kunnen we de code beschrijven als een matrix (ten opzichte van een geschikte basis).
  De generatormatrix is is steeds een $k\times n$ matrix met lineair onafhankelijke rijen.
\end{de}

\begin{de}
  De \term{duale code} van een lineaire blokcode is het orthogonaal complement ervan.
\end{de}

\begin{ei}
  De duale code van een $k$-dimensionale lineaire blokcode heeft dimensie $n-k$.
\end{ei}

\begin{de}
  De  ($n-k\times n$) generatormatrix van een duale code noemen we de pariteitsmatrix.
\end{de}

\begin{opm}
  Met behulp van de pariteitsmatrix $H$ is het heel eenvoudig om te testen of een ontvangen boodschap $x$ een codewoord is.
  \[ x\in \mathbb{C} \Leftrightarrow xH^{T} = 0 \]
\end{opm}

\begin{st}
  Zij $H$ een pariteitsmatrix van een lineaire blokcode $\mathbb{C}$, dan heeft $\mathbb{C}$ afstand $d$ als en slechts al elke verzameling van $d-1$ kolommen van $H$ lineair onafhankelijk is, maar niet alle verzamelinggen van $d$ kolommen.
  \TODO{bewijs p 41}
\end{st}

\begin{de}
  Een code $\mathbb{C}$ is \term{equivalent} met een code $\mathbb{C}'$ als de codewoorden dezelfde zijn (op een bepaalde permutatie van de elementen na).
\end{de}

\begin{opm}
  Twee equivalente codes hebben een gelijkvormige (?) generatormatrix.
\end{opm}

\begin{de}
  Een \term{systematische code} is zo bepaal dat de eerste $k$ symbolen van elk codewoord het overeenkomstig informatiewoord vormen.
\end{de}

\begin{st}
  Elke lineaire blokcode is equivalent met een systematische code.
\TODO{bewijs p 43}
\end{st}

\subsection{Decodering van lineaire blokcodes}

\begin{de}
  De (volledige) \term{tabel van een lineaire blokcode} construeren we als volgt:
  \begin{itemize}
  \item Zet op de eerste rij alle codewoorden.
  \item Voor elke nevenklasse $x\mathbb{C}$ in $\nicefrac{\mathbb{F}_{q}^{n}}{\mathbb{C}},+$ zetten we in de eerste kolom (in volgorde) de kleinste representant.
  \item De rest van de tabel wordt opgevuld door de som te nemen van het eerste element van die rij en kolom.
  \end{itemize}
  Elke rij bevat dan de elementen van \'e\'en nevenklasse.
\end{de}

\begin{de}
  Een ontvangen woord $w$ \term{decoderen met een lineaire blokcode} gaat als volgt:
  \begin{itemize}
  \item Kijk na of $w$ een codewoord is (door het in de eerste rij te zoeken).
  \item Zo nee: zoek het ok in de tabel en neem als bijhorend codewoord het woord in dezelfde kolom op de eerste rij.
  \end{itemize}
\end{de}

\begin{st}
  Als een lineaire blokcode $t$ fouten kan verbeteren (en dus afstand $d=2t+1$ heeft), staat elk woord van gewicht kleiner of gelijk aan $t$ in de eerste kolom.
\TODO{bewijs p 45}
\end{st}

\begin{st}
  De decodering van een ontvangen woord is volgens CMLD.
\TODO{bewijs p 45}
\end{st}

\begin{opm}
  In de praktijk is deze vorm van decoderen niet toepasbaar omdat de tabel veel te groot wordt.
\end{opm}

\begin{de}
  Voor een ontvangen woord $w$ defini\"eren we het \term{syndroom} $s$ als volgt:
  \[ s = wH^{t} \]
\end{de}

\begin{st}
  Twee woorden $v$ en $w$ hebben hetzelfde syndroom als en slechts als ze in dezelfde nevenklasse zitten.
\TODO{bewijs p 45}
\end{st}

\begin{de}
  De \term{syndroomtabel} van een lineaire blokcode $\mathbb{C}$ is de tabel waarin voor elke (minimale) vertegenwoordiger van de neveklasse van de code het overeenkomstige syndroom staat.
\end{de}

\begin{de}
  Een ontvangen woord $w$ \term{decoderen met een syndroomtabel} gaat als volgt:
  \begin{itemize}
  \item Bereken het syndroom $s$ van $w$
  \item Zoek de vertegenwoordiger $e$ op met als syndroom $s$
  \item $c=w-e$ is nu het codewoord.
  \end{itemize}
\end{de}

\subsection{Hamming codes}

\begin{de}
  Een \term{Hamming code} is een lineaire blokcode die $1$ fout kan verbeteren.
\end{de}

\TODO{meer stellingen uit te tekst filteren}

\subsection{Perfecte codes}
\begin{de}
  Zij $c$ een codewoord uit een lineaire blokcode $\mathbb{C}$, dan noemen we $S_{c}(r)$ als volgt een \term{bol} met straal $r$ rond $c$.
  \[ S_{c}(r) = \{ v\in \mathbb{F}_{q}^{n} \mid dt(v,c) \le r \} \]
\end{de}

\begin{de}
  De kleinste straal $r_{p}$ zodat alle bollen van straal $r$ rond de codewoorden disjunct zijn noemen we de \term{pakkingsstraal} van de code.
\end{de}

\begin{de}
  De kleinste straal $r_{d}$ zodat de unie van alle bollen van straal $r$ heel $\mathbb{F}_{q}^{n}$ vormt noemen we de \term{dekkingsstraal} van de code.
\end{de}

\begin{de}
  We noemen een code \term{perfect} als diens dekkingsstraal gelijk is aan de pakkingsstraal.
\end{de}

\begin{st}
  Hamming codes zijn perfect.
\TODO{bewijs p 49}
\end{st}


\begin{de}
  We noemen een code \term{quasi-perfect} als de dekkingsstraal \'e\'en kleiner is dan de pakkingstraal.
\end{de}


\section{Cyclische codes}

\begin{de}
  Een $(n,k)$-code $\mathbb{C}$ is een \term{cyclische code} als het volgende geldt:
  \[ \forall (a_{0}a_{1}\dotsc a_{n-2}a_{n-1}) \in \mathbb{C} \Rightarrow (a_{n-1}a_{0}a_{1}\dotsc a_{n-2}) \in \mathbb{C} \]
\end{de}

\begin{st}
  Een lineaire deelruimte $\mathbb{C}$ van $\mathbb{F}_{q}^{n}$ is een cyclische code als en slechts als $\mathbb{C}$ isomorf is met een ideaal van $\nicefrac{\mathbb{F}_{q}[X]}{(X^{n}-1)}$
  \TODO{bewijs p 51}
\end{st}

\begin{de}
  De \term{generator} van een cyclische code is de minimale veelterm die de code voortbrengt.
\end{de}

\begin{opm}
  Elk codewoorde is dan een veelvoud van de generatorveelterm.
\end{opm}

\begin{st}
  De generator $g$ van een cyclische code $\mathbb{C}$ over $\mathbb{F}_{q}^{n}$ deelt $X^{n}-1$.
\TODO{bewijs p 51}
\end{st}

\begin{de}
  Het quotient $h$ van de deling van $X^{n}-1$ door de generatorveelterm van een cyclische code noemen we de pariteitsveelterm.
  \[ X^{n}-1 = gh \]
\end{de}







\section{wut}




\begin{de}
  Een \term{code gegenereerd door een veelterm} $p(x)\in \mathbb{Z}_{2}[x]$ van graad $n-k$ is een $(n,k)$-code $c$:
  \[ c:\ \mathbb{Z}_{2}^{k} \rightarrow \mathbb{Z}_{2}^{n}:\ b(x) \mapsto c(b) = r(x) + x^{n-k}x(x) \quad\text{ met }\quad x^{n-k}b(x) = p(x)q(x) + r(x) \]
\end{de}

\begin{opm}
  Vermits de graad $gr(r(x))$ van de rest $r(x)$ een kleiner is dan $n-k$, passen de extra $n-k$ bits netjes na de boodschap-bits.
\end{opm}

\TODO{zeker een voorbeeld maken, wss examenvraag!}

\end{document}
